\documentclass{article}

\usepackage[utf8]{inputenc}
\usepackage{svg, amsfonts, amsmath, amssymb, 
            todonotes, float, geometry, todonotes,
             amsthm, setspace, enumerate, ragged2e}
             
% docmute subfiles standalong are used to input contents of other latex files into the main.tex
\usepackage{import, standalone, docmute, subfiles}
            
\usepackage[hidelinks]{hyperref}
% \newgeometry{vmargin={15mm}, hmargin={12mm,12mm}}
\svgpath{{figures/}}

%%%%%%%%%%%%%%%%%%%%%%%%%%%%%%%%%%%%%%%%%%%%%%%%%%%%%%%%%%%%%%%%

\newcommand{\R}{\mathbb{R}}
\newcommand{\N}{\mathbb{N}}
\newcommand{\Q}{\mathbb{Q}}
\newcommand{\Z}{\mathbb{Z}}
\newcommand{\K}{\mathbb{K}}
\newcommand{\C}{\mathbb{C}}
\newcommand{\mf}[1]{\mathbf{#1}}
\newcommand{\ca}[1]{\mathcal{#1}}

% defines shortcuts for inner product, norm
\newcommand{\ip}[2]{\langle #1, #2 \rangle}
\newcommand{\norm}[1]{||#1||}



%%%%%%%%%%%%%%%%%%%%%%%%%%%%%%%%%%%%%%%%%%%%%%%%%%%%%%%%%%%
% for numbering the theorems            
\theoremstyle{plain}
%%%%%%%%%%%%%%%%%%%%%%%%%%%%%%%%%%%%%%%%%%%%%%%%%%%%%%%%%%%
\newtheorem{thm}{Theorem}[section]
\newtheorem{lemma}[thm]{Lemma}
\newtheorem{cor}[thm]{Corollary}
%%%%%%%%%%%%%%%%%%%%%%%%%%%%%%%%%%%%%%%%%%%%%%%%%%%%%%%%%%%
% the following are not in italics
\theoremstyle{definition}
\newtheorem{defn}[thm]{Definition}
\newtheorem{eg}[thm]{Example}
\newtheorem{remark}[thm]{Remark}
\newtheorem{prop}[thm]{Proposition}
\newtheorem{claim}[thm]{Claim}
%%%%%%%%%%%%%%%%%%%%%%%%%%%%%%%%%%%%%%%%%%%%%%%%%%%%%%%%%%%%
\doublespacing
% \raggedleft

\title{Functional Analysis Chapters 4 and 5}

\date{\today}

\begin{document}

\maketitle

\section{Linear Operators}  

\begin{defn}
    (Bounded Operators)  
\end{defn}   

\begin{thm}
    (Properties of Linear Operators)  
    \begin{itemize}
        \item 
    \end{itemize}
\end{thm}  

\section{Dual Spaces}  

\begin{defn}
    (Dual Spaces) The space of all continuous linear operators $\mathcal{L}(X, \R)$ is called the \textbf{dual space} of $X$ and is denoted as  $X^*$
\end{defn}  

Clearly, the dual $X^*$ is also a linear space and we denote the norm on the dual space as $||\cdot||_*$

\begin{remark}
    $X^*$ is always a Banach space and its elements are called functionals.
\end{remark}

\subsubsection{Duality in Hilbert Spaces}  
In this section, let $(H, {\langle \cdot, \cdot \rangle}_H )$ be a Hilbert space over $\R$. For $y\in H$, we define the map  

\begin{equation*}
    \Lambda_y: X \to \R, \qquad x \mapsto {\langle y, x \rangle}_H
\end{equation*}  

We note that this is an injective map from $H$ to its dual $H^*$ and we will show that this is in fact a bijective isometry.  

\begin{lemma}
    (Mapping to dual space)
    \begin{enumerate}[i)]
        \item $\Lambda_y \in H^*$
        \item The map $\Lambda: H \to H^*$ is a linear isometry with $||\Lambda_y||_{*}=||y||$
    \end{enumerate}
\end{lemma}  

\begin{proof}
    \begin{enumerate}[i)]
        \item We need to check the linearity and boundedness of the operator $\Lambda_y^*$. The former follows from that of the inner product and the latter is proven by applying Cauchy-Schwarz  
        \begin{equation*}
            ||\Lambda_y||_* = \sup_{x\in H, ||x||=1} |\ip{y}{x}_H| \leq \norm{y}_H
        \end{equation*}  
        which implies $\Lambda_y \in H^*$
        
        \item Choose $x = \frac{y}{\norm{y}_H}$ to attain the equality in the equation above , whence we have $\norm{\Lambda_y}_*=\norm{y}$.
    \end{enumerate}
\end{proof}

\begin{thm}
    (Riesz Representation)  
    For every $l \in H^*$, there is a unique $y \in H$, such that $l=\Lambda_y$
\end{thm}  

\begin{proof}
    We show the existence and uniqueness of such a linear operator.  
    \begin{itemize}
        \item \textbf{(Existence)} If $l(x)\equiv 0$, then take $y=0$. Otherwise, assume $\norm{l}_*=1$ (as we can replace...).  
        By the definition of $\norm{\cdot}_*$, there is a sequence of $(y_n)_{n\in \N} \subset H$ with  
        \begin{equation*}
            |l(y_n)| \to \norm{l}_*, \qquad \norm{y_n}=1, \forall n \in \N
        \end{equation*}
        We will show that the limit of this sequence is the desired $y$.  
        
        \textbf{Claim 1:} The sequence $(y_n)_{n \in \N}$ is Cauchy   
        
        Apply the parallelogram identity to $x = \frac{y_n}{2}$ and $y = \frac{y_m}{2}$, so we have  
        \begin{equation*}
            \forall n,m \geq 1, \qquad \norm{\frac{y_n-y_m}{2}}^2=1-\norm{\frac{y_n+y_m}{2}}^2
        \end{equation*}  
        Using linearity and boundedeness of $l$,   
        \begin{equation*}
            \frac{1}{2} |l(y_n)+l(y_m)| = |l(\frac{y_n+y_m}{2})| \leq \norm{l}_* \norm{\frac{y_n+y_m}{2}}
        \end{equation*}  
        The LHS of the equation above converges to $1$ by assumption on $(y_n)_{n \in \N}$, which implies and $(y_n)_{n \in \N}$ is Cauchy.  
        Since $H$ is complete, there is a unique $y$, such that $y_n \to y$  
        
        \textbf{Claim 2:} $l=\Lambda_y$  
        
        Since $span\{y\}$ is closed, we can consider the orthogonal decomposition $H = span\{y\} \oplus (span\{y\})^{\perp}$.  
        It suffices to show:  
        \begin{enumerate}[(1)]
            \item $l(y)=\Lambda_y(y), \forall y \in span\{y\}$
            \item $l(x)=\Lambda_y(x), \forall x \in (span\{y\})^{\perp}$
        \end{enumerate}  
        To show (1), assume wlog $\norm{y}=1$, we note by continuity of $l$  
        \begin{equation*}
            l(y) = \lim_{n \to \infty} |l(y_n)| = \norm{l}_*=1
        \end{equation*}  
        and 
        \begin{equation*}
            \norm{y}^2_H = \ip{y}{y}_H =\Lambda_y(y)=1
        \end{equation*}  
        So $\Lambda_y(y)=l(y)$.  
        
        To show (2), we need to argue that 
        \begin{equation*}
            l(x)=0, \qquad \forall x \in (span\{y\})^{\perp}
        \end{equation*}  
        Now take $y_a = \frac{y+ax}{\sqrt{1+a^2}}$ and $\norm{y_a}=1$, where $y \in span\{y\}, a\in \R$ and $x \in (span\{y\})^{\perp}$. By definition of the norm $\norm{\cdot}_*$ and (1),  
        \begin{equation*}
            l(y_a)  \leq |l(y_a)| \leq 1 = l(y)
        \end{equation*}  
        So $l(y_a)$ has a global maximum at $a=0$. Therefore,  
        \begin{equation*}
            0 = \frac{d}{da} l(y_a) \Bigr|_{a=0} = \frac{d}{da} \frac{1}{\sqrt{1+a^2}} (l(y)+al(x))=l(x)
        \end{equation*}  
        So $l(x)=\Lambda_y(x), \forall x \in (span\{y\})^{\perp}$.
        
        \item \textbf{(Uniqueness)}  
        %\todo{fill in uniqueness}
    \end{itemize}
\end{proof}

\begin{cor}
    All Hilbert spaces $H$ are isomorphic to their duals $H^*$.
\end{cor}
\begin{eg}
    Some of the examples are:  
    \begin{itemize}
        \item $(l^2)^* \cong l^2$
        \item $(L^2(\mu))^* \cong L^2(\mu)$
    \end{itemize}
\end{eg}
\subsubsection{Duality in Banach Spaces}  

\begin{thm}
    (Conjugates are duals)
\end{thm}  

\begin{defn}
    (Dual Operators)
\end{defn}  

\begin{defn}
    (Adjoint Operators)
\end{defn}  
\begin{eg}
    (Adjoint operators as matrices)
\end{eg}  

\begin{remark}
    
\end{remark}

\end{document}