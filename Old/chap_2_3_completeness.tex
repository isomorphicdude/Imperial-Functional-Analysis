\documentclass{article}

\usepackage[utf8]{inputenc}
\usepackage{svg, amsfonts, amsmath, amssymb, 
            todonotes, float, geometry, todonotes,
             amsthm, setspace, enumerate, ragged2e}
             
% docmute subfiles standalong are used to input contents of other latex files into the main.tex
\usepackage{import, standalone, docmute, subfiles}
            
\usepackage[hidelinks]{hyperref}
% \newgeometry{vmargin={15mm}, hmargin={12mm,12mm}}
\svgpath{{figures/}}

%%%%%%%%%%%%%%%%%%%%%%%%%%%%%%%%%%%%%%%%%%%%%%%%%%%%%%%%%%%%%%%%

\newcommand{\R}{\mathbb{R}}
\newcommand{\N}{\mathbb{N}}
\newcommand{\Q}{\mathbb{Q}}
\newcommand{\Z}{\mathbb{Z}}
\newcommand{\K}{\mathbb{K}}
\newcommand{\C}{\mathbb{C}}
\newcommand{\mf}[1]{\mathbf{#1}}
\newcommand{\ca}[1]{\mathcal{#1}}

% defines shortcuts for inner product, norm
\newcommand{\ip}[2]{\langle #1, #2 \rangle}
\newcommand{\norm}[1]{||#1||}



%%%%%%%%%%%%%%%%%%%%%%%%%%%%%%%%%%%%%%%%%%%%%%%%%%%%%%%%%%%
% for numbering the theorems            
\theoremstyle{plain}
%%%%%%%%%%%%%%%%%%%%%%%%%%%%%%%%%%%%%%%%%%%%%%%%%%%%%%%%%%%
\newtheorem{thm}{Theorem}[section]
\newtheorem{lemma}[thm]{Lemma}
\newtheorem{cor}[thm]{Corollary}
%%%%%%%%%%%%%%%%%%%%%%%%%%%%%%%%%%%%%%%%%%%%%%%%%%%%%%%%%%%
% the following are not in italics
\theoremstyle{definition}
\newtheorem{defn}[thm]{Definition}
\newtheorem{eg}[thm]{Example}
\newtheorem{remark}[thm]{Remark}
\newtheorem{prop}[thm]{Proposition}
\newtheorem{claim}[thm]{Claim}
%%%%%%%%%%%%%%%%%%%%%%%%%%%%%%%%%%%%%%%%%%%%%%%%%%%%%%%%%%%%
\doublespacing
% \raggedleft

\title{Functional Analysis Chapters 2 and 3}

\date{\today}

\begin{document}

\maketitle

% \input{chap_1}

\section{Completeness and Separability}  

\begin{defn}
    (Completeness)
    A metric space in which every Cauchy sequence converges to some limit in that space is called 
    \textbf{complete}.  
\end{defn}  

\begin{defn}
    (Banach Space) 
    A \textbf{Banach Space} is a complete normed vector space.
\end{defn}

\begin{thm}
    (Completion)  
    For every metric space $(X, d)$, there exists a complete metric space $(\tilde{X}, \tilde{d})$ 
    and an isometry $i: X \to \tilde{X}$.   
\end{thm}  

\begin{remark}
    The metric space $X$ is isometric to a dense subset of its completion $\tilde{X}$.  
    We construct the completeion by introducing equivalence relation of Cauchy sequences.
\end{remark}

\begin{defn}
    (Separability) 
    A metric space is \textbf{separable} if it has a countable dense subset.  
\end{defn}  

We now present some results of completeness and separability of a few common spaces.    

\subsection{Common Spaces}  

\subsubsection*{$l_p$ Spaces}    
For $p \in [1,\infty)$, the space $l_p$ is defined as the set of all sequences $(x_n)_{n\in \N}$ such that  

\begin{equation*}
    \sum_{n=1}^{\infty} |x_n|^p < \infty
\end{equation*}

the function  

\begin{equation*}
    ||x||_p = \left(\sum_{i=1}^n |x_i|^p\right)^{1/p}
\end{equation*}  

defines a norm on $l_p$.  
\begin{remark}
    $l_p \subset l_q$ when $p < q$. And $\lim_{p\to \infty} ||x||_p = ||x||_{\infty}$
\end{remark}  

\begin{prop}
    $l_p$ is a Banach space for $p \in [1,\infty)$.
\end{prop}  

\begin{prop}
    $l_p$ is separable for $p \in [1,\infty)$.
\end{prop}  


\subsubsection*{$l_\infty$ Spaces}  
The space $l_\infty$ is defined as the set of all sequences $(x_n)_{n\in \N}$ such that

\begin{equation*}
    \sup_{n\in \N} |x_n| < \infty
\end{equation*}

the function

\begin{equation*}
    ||x||_\infty = \sup_{n\in \N} |x_n|
\end{equation*}

defines a norm on $l_\infty$.  

\begin{prop}
    $l_\infty$ is a Banach space.
\end{prop}

\begin{prop}
    $l_{\infty}$ is \textbf{not} separable.  
\end{prop} 

We show the above proposition by providing an example.  

\begin{eg}
    We provide an uncountable set with each element a fixed distance to another  
    \begin{equation*}
        M = \{x = (x_1, x_2, \ldots)\in l_{\infty}: x_k \in \{0,1\}, k\in \N\}
    \end{equation*}
\end{eg}

\subsubsection*{$L^p$ spaces}  
Let $(X,A,\mu)$ be a measure space. The $L^p$ space w.r.t. $\mu$ is 
$$L^p(\mu) = L^p(X,A,\mu) = \{f:X \to \mathbb{R}/\sim, ||f||_{L^p} < \infty \}$$
where 
\begin{equation*}
    ||f||_{L^p} =\begin{cases}
    ( \int |f|^p d\mu)^{1/p} &  p<\infty \\
    esssup_{x} |f| & p=\infty
\end{cases}
\end{equation*}


See measure theory (MATH 50006) notes for more details.  
%\todo{Fill in later}

\subsubsection*{$C[a,b]$ Spaces}  

The space of continuous functions on $[a,b]$ equipped with the supremum norm is denoted by $C[a,b]$.  
The norm is defined by   

\begin{equation*}
    ||f||_{\infty}=\sup_{x \in [a,b]} |f(x)|
\end{equation*}

\begin{prop}
    $C[a,b]$ is a Banach space.
\end{prop}  
\begin{proof}
(Sketch)  
Deduce a pointwise limit of the functions for each of $x\in [a,b]$, use the completeness of $\R$, conclude by the limit of a uniformly convergent sequence of continuous functions is also continuous.  
\end{proof}

\begin{thm}
    \label{continuous sep}
    $C[a,b]$ is separable.
\end{thm}  

\begin{lemma}
    Let $\varepsilon >0$, and suppose that $f$ is a continuous function over $[a.b]$ such that  
    \begin{equation*}
        |f(x)-f(y)|<\frac{\varepsilon}{5}, \qquad \forall x,y \in [a,b]
    \end{equation*}  
    Let $h$ be a linear function on $[a,b]$, with values close to $f$ at endpoints  
    \begin{equation*}
        |h(a)-f(a)|<\frac{\varepsilon}{5} \qquad |h(b)-f(b)|<\frac{\varepsilon}{5}
    \end{equation*}  
    Then we have  
    \begin{equation*}
        |f(x)-h(x)| < \varepsilon, \forall x \in [a,b]
    \end{equation*}
\end{lemma}  

\begin{proof}

\end{proof}

Now we prove \autoref{continuous sep}  
\begin{proof}
(Sketch)  
Partition the interval into small intervals and construct piece-wise linear functions with endpoints at rational numbers, which also take rational values close to the function $f$ by density; use the above lemma to conclude.
\end{proof}

\subsubsection*{Special sequence spaces $c, c_0, c_{00}$}  
\todo{Maybe remove as it's not in the notes}
\begin{itemize}
    \item $c$ is the set of all convergent sequences in $l_{\infty}$
    \item $c_0$ is the set of all sequences converging to zero in $l_{\infty}$
    \item $c_{00}$ is the set of all sequences such that $x_n=0$ for all
          but finitely many $n$ in $l_{\infty}$
\end{itemize}  

Note that they are all equipped with the supremum norm.  

\begin{remark}
    Note that $c_{00} \subset c_{0} \subset c$
\end{remark}  

We discuss their completeness and separability.  

\begin{prop}
    $c$ and $c_0$ are both Banach spaces.
\end{prop}  

\begin{eg}
    Note that the space $c_{00}$ is not a complete space, as it is not
    closed in $c$; take the sequence $x_n = (1, \frac{1}{2}, \ldots, \frac{1}{n}, 0, \ldots)$
\end{eg}

\begin{prop}
    $c$ and $c_0$ are separable.
\end{prop}  


\subsubsection*{Sequence space $s$}  
\todo{Not in the notes, content about seminorms}

\section{Finite vs. Infinite Dimensional Spaces}  

There are some fundamental differences between finite and infinite dimensional spaces.  

\begin{defn}
    (Equivalent Norms)
    Two norms $||\cdot||_1, ||\cdot_2||$ on a vector space $X$ are \textbf{equivalent} 
    if there exists a constant $C \in [1, \infty)$ such that  
    \begin{equation*}
        \forall x \in X, \quad \frac{1}{C}||x||_1 \leq ||x||_2 \leq C||x||_1
    \end{equation*}
\end{defn}  

\subsection{Properties of Finite Dimensional Spaces}

\begin{thm}
    In finite dimensional spaces, all norms are equivalent.
\end{thm}  

However, this is not true in infinite dimensional spaces.  

\begin{eg}
    Consider $f_n(t)=t^n$ in $C[0,1]$. This sequence of functions converge to $0$ 
    in $||\cdot||_1$ but not in $||\cdot||_{\infty}$.
\end{eg}

\begin{prop}
    All finite dimensional spaces are complete; hence all finite dimensional spaces are closed.
\end{prop}  

\subsection{Compactness}
\begin{defn}
    (Compact) 
    A set $K \subset X$ is \textbf{compact} 
    if every sequence in $K$ has a convergent subsequence with limit in $K$.
\end{defn}  

\begin{remark}
    A set $K$ is compact if and only if it is bounded and closed in finite dimensional spaces.  
    Compactness implies closed and boundedness even in infinite dimensional spaces but the 
    other direction is not true.
\end{remark}  

\begin{eg}
    Take the following set in $l^1$:  
    \begin{equation*}
        K = \left\{ e_n = (0, \ldots, 0, 1, 0, \ldots), n \in \mathbb{N} \right\}
    \end{equation*}  
    This set is bounded and closed but clearly not compact as each term in the sequence
    are distance $2$ apart.  
\end{eg}  

Another example is the following  

\begin{eg}
    Consider the set $\bar{B}_1 \subset C[0,1]$,  
    \begin{equation*}
        \bar{B}_1 = \left\{ f \in C[0,1] : ||f||_{\infty} \leq 1 \right\}
    \end{equation*}  
    The sequence of functions,  
    \begin{equation*}
        f_n(t) = \sin(2^n \pi t), \quad 0 \leq t \leq 1
    \end{equation*}  
    Then $||f_n-f_m||\geq 1$ for all $n \neq m$ and hence no convergent subsequences. 
\end{eg}

\begin{thm}
    In a normed space $(X, ||\cdot||)$, the following two conditions are equivalent:  
    \begin{itemize}
        \item $\dim X < \infty$
        \item The unit ball $\bar{B}_1$ is compact 
    \end{itemize}
\end{thm}  
  
  
Before proving this, we need a lemma that allows us to construct a sequence of vectors
that are at a fixed distance to each other, hence has no convergent subsequence.  

\begin{lemma}
    (Riesz) Let $Y \subset Z \subset X$ be a \textit{proper} closed subspace of some subspace in $(X, ||\cdot||)$. Then for any $\theta \in (0,1)$, there is a $z\in Z$ on the unit ball, $||z||=1$, such that  
    \begin{equation*}
        \forall y \in Y \qquad ||y-z|| \geq \theta
    \end{equation*}
\end{lemma}  
\begin{proof}
(Sketch)  
\todo{Add alternative proof in the hand-written lecture notes. }

\begin{itemize}
    \item Choose $v \in X \setminus Y$, let $d=\inf_{y\in Y}||v-y||>0$, as $Y$ is closed and proper
    \item Choose $y_0 \in Y$, with $||v-y_0||<d/\theta$, which is greater than $d$
\end{itemize}
Set
\begin{equation*}
    z = \frac{v-y_0}{||v-y_0||}
\end{equation*}  
Compute this and show this to be greater than $\theta$.
\end{proof}  

Now to construct a sequence of vectors inductively, we take a unit vector $x_1 \in X$ and pick  

\begin{equation*}
    ||x_2-x_1|| \geq \theta=\frac{1}{2}
\end{equation*}  

Suppose a set $\{x_1, \ldots, x_m\}$ has been constructed s.t. every pair of elements satisfy the equation above, then the span of this set is closed (finite dimensional) so we can continue the construction to obtain a sequence  

\begin{equation*}
    ||x_p-x_q|| \geq \frac{1}{2}, \qquad \forall p, q \in \N
\end{equation*}

\subsection{Basis in Infinite Dimensional Spaces}  

Similar to finite dimensional vector spaces, there is something "analogous" to 
a basis in infinite dimensional spaces.  

\begin{defn}
    (Hamel)
\end{defn}  

\begin{defn}
    (Schauder)
\end{defn}   

\begin{eg}
    Schauder basis of $l^p$ is the set of all sequences of the form $e_n$ but there is no 
    Schauder basis for $l^{\infty}$.  
\end{eg}  

Sometimes the Schauder basis might look complicated.    
\begin{eg}
    A Schauder basis of $C[0,1]$ is given by the tent functions
\end{eg}  

We also provide some counter examples of Schauder bases in $C[0,1]$.  

\begin{eg}

\end{eg}

\begin{thm}
    If $(X, ||\cdot||)$ has a Schauder basis, then $X$ is separable.  
\end{thm}  


\section{Hilbert Spaces}  

\begin{defn}
    (Inner Product)
\end{defn}  

\begin{remark}
    The inner product is continuous.  
\end{remark}

\begin{defn}
    (Hilbert Space)
\end{defn}  

The inner product and norm naturally leads to the famous inequality.  

\begin{prop}
    (Cauchy-Schwarz)
\end{prop}

\begin{eg}
    $l^2$ is a Hilbert space.
\end{eg} 

\begin{prop}
    (Important Identities)
    \begin{itemize}
        \item (Parallelogram)
        \item (Polarization)
    \end{itemize}
\end{prop}  

\begin{prop}
    $l^p$ is not a Hilbert space for $p \neq 2$.
\end{prop}  

\subsection{Orthogonality}  

\begin{defn}
    (Orthogonal)
\end{defn}  

\begin{thm}
    (Projection)
\end{thm}  

\begin{thm}
    (Orthogonal Complement)
\end{thm}
\end{document}
