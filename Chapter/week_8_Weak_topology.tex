\documentclass{article}
\usepackage{geometry}
\geometry{left=3cm,right=3cm,top=2cm,bottom=2cm}
\usepackage[utf8]{inputenc}
\usepackage{amsmath, amsfonts, amssymb, amsthm}
\usepackage[framemethod=TikZ]{mdframed}
\usepackage{mathrsfs}
\usepackage{comment}
\usepackage{enumerate}
\usepackage{xcolor}
\usepackage{titlesec}
\usepackage{setspace}
\usepackage[hidelinks,backref]{hyperref}
\usepackage{cleveref}
\usepackage[most]{tcolorbox}
\usepackage{ragged2e}
\usepackage{todonotes}
\usepackage{cleveref}
\usepackage{mathtools}

%
\titleformat*{\section}{\LARGE \bfseries}
\titleformat*{\subsection}{\Large \bfseries}
\titleformat*{\subsubsection}{\Large \bfseries}
% \titleformat*{\paragraph}{\large \bfseries}
\titleformat*{\subparagraph}{\large \bfseries}



% General 
\newcommand{\nextline}{\hfill\break}
\newcommand{\nl}{\nextline\rm}
% \newcommand{\placeholder}{{\bf\color{red} NOOOOOOT COMPLEEEEEEET! COOOOOOOM BAAAAAACK!!!}}
\newcommand{\placeholder}{\todo{NOOOOOOT COMPLEEEEEEET! COOOOOOOM BAAAAAACK!!!}}

\newcommand{\defeq}{\stackrel{def.}{=}}
% FA and LA
% inner product: \inne{a}{b}
\newcommand{\inne}[2]{\left<{#1},{#2}\right>}

% norm: \norm{a}
\newcommand{\norm}[1]{\left\|{#1}\right\|}

% Curly H
\newcommand{\hbs}{$\mathscr{H}$ }
\newcommand{\hbp}{\mathscr{H}}

% Dual : \dual{x}
\newcommand{\dual}[1]{{#1}^*}

% Sequence from 1  to infty: \sequ{x_n}
\newcommand{\sequ}[1]{\left({#1}\right)_1^\infty}

% f: A-> B \func{f}{A}{B}
\newcommand{\func}[3]{${#1}:{#2}\xrightarrow{}{#3}$}

% interior
\newcommand{\interior}{\textrm{int}}

% Bounded linear funcs
\newcommand{\blf}[2]{\mathcal{L}({#1},{#2})}

\newcommand{\prf}{\textit{proof}:   }



% Fields 
\newcommand{\real}{\mathbb{R}}
\newcommand{\comp}{\mathbb{C}}
\newcommand{\inte}{\mathbb{Z}}
\newcommand{\natu}{\mathbb{N}}





% Theorems
% \newtheorem{example}{Example}[subsection]
% \newtheorem{definition}[example]{Definition}
% \newtheorem{proposition}[example]{Proposition}
% \newtheorem{remark}[example]{Remark}
% \newtheorem{theorem}[example]{Theorem}
% \newtheorem{lemma}[example]{Lemma}
% \newtheorem{corollary}[example]{Corollary}


% for numbering the theorems            
\theoremstyle{plain}
%%%%%%%%%%%%%%%%%%%%%%%%%%%%%%%%%%%%%%%%%%%%%%%%%%%%%%%%%%%
\newtheorem{theorem}{Theorem}[section]
\newtheorem{lemma}[theorem]{Lemma}
\newtheorem{corollary}[theorem]{Corollary}
\newtheorem{proposition}[theorem]{Proposition}
%%%%%%%%%%%%%%%%%%%%%%%%%%%%%%%%%%%%%%%%%%%%%%%%%%%%%%%%%%%
% the following are not in italics
\theoremstyle{definition}
\newtheorem{definition}[theorem]{Definition}
\newtheorem{example}[theorem]{Example}
\newtheorem{remark}[theorem]{Remark}
\newtheorem{claim}[theorem]{Claim}
%%%%%%%%%%%%%%%%%%%%%%%%%%%%%%%%%%%%%%%%%%%%%%%%%%%%%%%%%%%%

% proof box
\newtcbtheorem[no counter]{pf}{Proof}{
  enhanced,
  rounded corners,
  attach boxed title to top,
  colback=white,
  colframe=black!25,
  fonttitle=\bfseries,
  coltitle=black,
  boxed title style={
    rounded corners,
    size=small,
    colback=black!25,
    colframe=black!25,
  } 
}{prf}

% extra content box to put in contents not covered in the lecture notes
% use the command \begin{unexaminable}
\newmdenv[skipabove=7pt, skipbelow=7pt,
    rightline=false, leftline=false, topline=false, bottomline=false,
    backgroundcolor = gray!10,
    innerleftmargin=1in, innerrightmargin=1in, innertopmargin=5pt,
    leftmargin=-1in, rightmargin=-1in, linewidth=4pt,
    innerbottommargin=5pt]{unexamBox}
\newenvironment{unexaminable}{\begin{unexamBox}}{\end{unexamBox}}


% clever ref settings
\crefname{lemma}{lemma}{lemmas}
\Crefname{lemma}{Lemma}{Lemmas}
\crefname{theorem}{theorem}{theorems}
\Crefname{theorem}{Theorem}{Theorems}

% formatting 
% https://tex.stackexchange.com/questions/217497/aligning-stackrel-signs-beneath-each-other-using-split
\newlength{\leftstackrelawd}
\newlength{\leftstackrelbwd}
\def\leftstackrel#1#2{\settowidth{\leftstackrelawd}%
{${{}^{#1}}$}\settowidth{\leftstackrelbwd}{$#2$}%
\addtolength{\leftstackrelawd}{-\leftstackrelbwd}%
\leavevmode\ifthenelse{\lengthtest{\leftstackrelawd>0pt}}%
{\kern-.5\leftstackrelawd}{}\mathrel{\mathop{#2}\limits^{#1}}}

\doublespacing
\RaggedRight
% modify innerrightmargin if floats were lost

\usepackage{amsfonts, amsmath, amssymb, amsthm, thmtools, bm}
\usepackage{avant} % Use the Avantgarde font for headings

% Boxed/framed environments
\newtheoremstyle{royalnumbox}%
{0pt}% Space above
{0pt}% Space below
{\normalfont}% Body font
{}% Indent amount
{\small\bf\sffamily\color{royal}}% Theorem head font
{\;}% Punctuation after theorem head
{0.25em}% Space after theorem head
{\sffamily \color{royal} 
    \thmname{#1} 
    \thmnumber{#2} \thmnote{\bfseries\color{black}---\nobreakspace#3.}} % Optional theorem note
\renewcommand{\qedsymbol}{$\blacksquare$}% Optional qed square

\newtheoremstyle{blacknumex}% Theorem style name
{5pt}% Space above
{5pt}% Space below
{\normalfont}% Body font
{} % Indent amount
{\small\bf\sffamily}% Theorem head font
{\;}% Punctuation after theorem head
{0.25em}% Space after theorem head
{\sffamily
    \thmname{#1}
    \thmnumber{#2}
    \thmnote{---\nobreakspace#3.}}% Optional theorem note

\newtheorem*{notation}{Notation}
\newtheorem*{hint}{Hint}
\newtheorem*{solution}{Solution}

\newcounter{dummy} 
\numberwithin{dummy}{section}

\theoremstyle{royalnumbox}
\newtheorem{definitionT}[dummy]{Definition}
\newtheorem{theoremT}[dummy]{Theorem}
\newtheorem{lemmaT}[dummy]{Lemma}
\newtheorem{corollaryT}[dummy]{Corollary}
\newtheorem{propositionT}[dummy]{Proposition}
\newtheorem{propertyT}[dummy]{Property}
\newtheorem{remarkT}[dummy]{Remark}

\theoremstyle{blacknumex}
\newtheorem{exampleT}[dummy]{Example}
\newtheorem{exerciseT}[dummy]{Exercise}

\numberwithin{equation}{section}

\RequirePackage[framemethod=default]{mdframed}

% Definition box
\newmdenv[skipabove=7pt, skipbelow=7pt,
rightline=false, leftline=true, topline=false, bottomline=false,
backgroundcolor = reddish!10, 
linecolor=reddish,
innerleftmargin=5pt, innerrightmargin=30pt, innertopmargin=5pt,
leftmargin=0cm, rightmargin=0cm, linewidth=4pt,
innerbottommargin=5pt]{dBox}

% Main Theorem box
\newmdenv[skipabove=7pt, skipbelow=7pt,
rightline=false, leftline=true, topline=false, bottomline=false,
backgroundcolor=c0!10, 
linecolor=c0,
innerleftmargin=5pt, innerrightmargin=30pt, innertopmargin=5pt,
leftmargin=0cm, rightmargin=0cm, linewidth=4pt, innerbottommargin=5pt]{tBox}

% Lemma/Corollary/Proposition/Property box
\newmdenv[skipabove=7pt, skipbelow=7pt,
rightline=false, leftline=true, topline=false, bottomline=false,
backgroundcolor = c0!10, 
linecolor=c0!80,
innerleftmargin=5pt, innerrightmargin=30pt, innertopmargin=5pt,
leftmargin=0cm, rightmargin=0cm, linewidth=4pt,
innerbottommargin=5pt]{lBox}

% Example/Remark/Exercise box
\newmdenv[skipabove=7pt, skipbelow=7pt,
rightline=false, leftline=true, topline=false, bottomline=false,
backgroundcolor = mossgreen!10!white,
linecolor = mossgreen,
innerleftmargin=5pt, innerrightmargin=30pt, innertopmargin=5pt,
leftmargin=0cm, rightmargin=0cm, linewidth=4pt,
innerbottommargin=5pt]{exBox}

% Proof box
% \newmdenv[skipabove=7pt, skipbelow=7pt,
% rightline=false, leftline=true, topline=false, bottomline=false,
% linecolor=gray,
% innerleftmargin=5pt, innerrightmargin=30pt, innertopmargin=5pt,
% leftmargin=0cm, rightmargin=0cm, linewidth=4pt,
% innerbottommargin=5pt]{proofBox}



% Creates an environment for each type of theorem and assigns it a theorem text style from the "Theorem Styles" section above and a colored box from above
\newenvironment{definition}{\begin{dBox}\begin{definitionT}}{\end{definitionT}\end{dBox}}
\newenvironment{theorem}{\begin{tBox}\begin{theoremT}}{\end{theoremT}\end{tBox}}
\newenvironment{lemma}{\begin{lBox}\begin{lemmaT}}{\end{lemmaT}\end{lBox}}
\newenvironment{proposition}{\begin{lBox}\begin{propositionT}}{\end{propositionT}\end{lBox}}
\newenvironment{corollary}{\begin{lBox}\begin{corollaryT}}{\end{corollaryT}\end{lBox}}
\newenvironment{property}{\begin{lBox}\begin{propertyT}}{\end{propertyT}\end{lBox}}


% \newenvironment{proof*}{\begin{proofBox}\begin{proof}}{\end{proof}\end{proofBox}}
\newenvironment{exercise}{\begin{exBox}\begin{exerciseT}}{\hfill{\color{royal}}\end{exerciseT}\end{exBox}}
\newenvironment{remark}{\begin{exBox}\begin{remarkT}}{\end{remarkT}\end{exBox}}
\newenvironment{example}{\begin{exBox}\begin{exampleT}}{{}\end{exampleT}\end{exBox}}

\begin{document}  
\section{Weak Topology and weak convergence}
Let $(X, || \cdot ||_X)$ be a normed linear space with dual space $X^*$ (over $\mathbb{R}$).
\begin{definition}
\begin{itemize}
    \item A sequence $(x_n) \subset X$ converges weakly to $x\in X$, written as $x_n \xrightarrow{\text{w}} x (n \rightarrow \infty)$, if $\forall l \in X^*$, $$lim_{n\rightarrow \infty} l(x_n) = l(x);$$
    \item $(x_n)$ converges (strongly/in norm) to $x$ if $lim_{n\rightarrow \infty} ||x_n - x|| = 0$, write as: $x_n \rightarrow x (n\rightarrow \infty).$
\end{itemize}
\end{definition}

\begin{remark}
\begin{enumerate}
    \item $x_n\rightarrow x$ implies  $x_n \xrightarrow{w} x$: $|l(x_n) - l(x)| \leq ||l||_* ||x_n - x||$
    \item The converse of 1) is false. Let $x_n (= e_n) = (0,...,0,1,0,...) \in l^2$. $||x_n-x_m|| = \sqrt{2}, n\neq m$, so $x_n$ doesn't converge (strongly). But $x_n \xrightarrow{w} 0$: by Reisz Representation, $l(\cdot) = <y,\cdot>_{l^2}$ for some $y \in l^2$, Hence with $y=(y^n)_n$, $$l(x_n) = <y,x_n> = y^n \leq \sqrt{\Sigma_{k\geq n} {|y^k|}^2} \xrightarrow{n\rightarrow \infty} 0$$, since $||y||_2 < 0$.
    \item $x_n \xrightarrow{w} x, y \Rightarrow x=y$. Assume not, by Prop. p.34 (\textquotedblleft dual elements separate points"), $\exists l \in X^*: l(x) \neq l(y)$ if $x\neq y$. With this $l$: $$l(x) = lim_{n} l(x_n) = l(y) $$ 
    \item $x_n \xrightarrow{w} x \Rightarrow sup||x_n|| < \infty$. Consider $A_n \in \mathfrak{L}(X^*, \mathbb{R})$ ($=X^{**}$, the bidual) with $A_n(l) := l(x_n), l\in X^*$. $x_n \xrightarrow{w} x$ implies $sup_n |A_n(l) < \infty \forall l\in X^*$, and $X^*$ is complete so by Banach-Steinhaus: $$sup_n ||A_n||_{\mathfrak{L}(X^*, \mathbb{R})} < \infty$$ But $||A_n||_{\mathfrak{L}(X^*,\mathbb{R})} = sup_{l\in X^*, ||l||\leq 1} |l(x_n)| = ||x_n||_X$.
\end{enumerate}
\end{remark}

This naturally leads to:\begin{definition}
    $X^{**} := (X^*)^* \quad (=\mathfrak{L}(X^*,\mathbb{R}))$ is called bidual of $X$. $X$ embeds canonically into $X^{**}$ via: $$ \eta : X\rightarrow X^{**}: \eta(x)(l) := l(x) \quad \forall x \in X, l \in X^*$$ 
\end{definition}
\begin{remark}
    $\eta$ is a linear isometry: similarly as in 4) above. One has: $\forall x \in X: ||x||_X = sup_{l\in X^*, ||l|| \leq 1} |l(x)| = ||\eta(x)||_{**}$.
\end{remark}

\begin{definition}
    $X$ is reflexive if $\eta$ is surjective.
\end{definition}

\begin{eg}
\begin{enumerate}
    \item if $dimX<\infty$, $X$ is reflexive;
    \item $H$ a Hilbert space is reflexive;
    \item $L^p, 1<p<\infty$ is reflexive;
    \item $L^1, L^\infty$ are in general not reflexive.
\end{enumerate}
For 4, consider $L^\infty[-1,1]$. Define the Dirac function $\delta_{x_0}: C^0[-1,1] \rightarrow \mathbb{R}: f \mapsto \delta_{x_0}(f) = f(x_0)$.\\
$\delta_{x_0}$ is linear and continuous on $(C^0[-1,1],||\cdot ||_\infty)$. By Corollary of Hahn-Banach on p.33, there exists extension 
$$l\in (L^\infty [-1,1])^* s.t. l\mid_{C^0[0,1]} = \delta_{x_0}.$$
(with some norm) (Omit \textquotedblleft $[-1,1]$" henceforth) \\
For $g\in L^1$, $l_g(f) = \int_{-1}^1 gf dx, f\in L^\infty, l_g\in (L^\infty)^*$.
\\Claim: $\nexists g\in L^1: l=l_g$.
\\Claim implies that $\eta: L^1 \rightarrow (L^\infty)^*, g \mapsto l_g$ is not surjective, i.e., $L^1$ is not reflexive.\\
For $L^\infty$, use: $X$ reflexive $\Rightarrow$ $X^*$ reflexive (exercise).\\
Proof of Claim: Suppose $\exists g\in l^1$ s.t. $l=l_g$. For simplicity let $x_0=0$.\\
Pick a bump function $\phi \in C^\infty [-1,1]: 0 \leq \phi \leq 1, \phi \equiv 1$ on $B_{\frac{1}{2}}$, $\phi(\pm 1)=0$. For $n\leq 1: \phi(x) := \phi(nx)$. Then $0\leq \phi_n \leq 1$, $\phi_n \rightarrow 0 a.e.$.\\
This yields: $$1 = \phi_n(0)=\delta_{x_0}(\phi_n) = l(\phi_n) = l_g(\phi_n) = \int_{-1}^1 g\phi_n dx \nrightarrow 0$$
this is a contradiction, as $\delta_{x_0}(\phi_n) = 0$.
\end{eg}



\end{document}


% \usepackage{amsmath}
% \usepackage{amssymb}
% \usepackage{hyperref}
% \usepackage{graphicx}
% \usepackage{cleveref}
% \usepackage{commath}
% \usepackage{setspace}
% \usepackage[titletoc]{appendix}


% \newtheorem{theorem}{Theorem}[section]
% \newtheorem{prop}{Proposition}[section]
% \newtheorem{definition}{Definition}[section]
% \newtheorem{remark}{Remark}[section]

% \newenvironment{proof}{{\scshape Proof. }\itshape }{\hfill$\spadesuit$\par}