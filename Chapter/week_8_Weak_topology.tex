\documentclass{article}
\usepackage[utf8]{inputenc}
\input{Chapter/commands.tex}
\input{theoremstyle.tex}
\title{Week 8}

\begin{document}  
\author{\aut}
\maketitle
\section{Weak vs. Strong topologies}
Let $(X, || \cdot ||_X)$ be a normed linear space with dual space $X^*$ (over $\mathbb{R}$).

\begin{definition}[Weak Convergence]\nl
A sequence $(x_n) \subset X$ \textbf{converges weakly} to $x\in X$, written as $x_n \xrightarrow{\text{w}} x (n \rightarrow \infty)$, if $\forall l \in X^*$, 
    $$\lim_{n\rightarrow \infty} \ell(x_n) = \ell(x);$$
$(x_n)$ \textbf{converges (strongly/in norm)} to $x$ if $\lim_{n\rightarrow \infty} ||x_n - x|| = 0$, write as: $x_n \rightarrow x (n\rightarrow \infty).$
\end{definition}

\begin{remark}
\label{properties of weak convergence}
We have the following remarks about weak convergence.  

\begin{enumerate}[1)]
    \item $x_n\rightarrow x$ implies  $x_n \xrightarrow{w} x$: $|\ell(x_n) - \ell(x)| \leq ||\ell||_* ||x_n - x||$
    
    \item The converse of 1) is false. Let $x_n (= e_n) = (0,...,0,1,0,...) \in l^2$. $||x_n-x_m|| = \sqrt{2}, n\neq m$, so $x_n$ doesn't converge (strongly). But $x_n \xrightarrow{w} 0$: by Riesz Representation, $\ell(\cdot) = \inne{y}{\cdot}_{l^2}$ for some $y \in l^2$, Hence with $y=(y^n)_n$, $$\ell(x_n) = \inne{y}{x_n} = y^n \leq \sqrt{\Sigma_{k\geq n} {|y^k|}^2} \xrightarrow{n\rightarrow \infty} 0$$, since $||y||_2 < \infty$.
    
    \item If $x_n \xrightarrow{w} x, x_n \xrightarrow{w} y$, then $x=y$. Assume not, \cref{dual elements separate points}, $\exists l \in X^*: \ell(x) \neq \ell(y)$ if $x\neq y$. With this $l$: $$\ell(x) = \lim_{n \to \infty} \ell(x_n) = \ell(y) $$ 
    ,a contradiction.
    
    \item $x_n \xrightarrow{w} x \Rightarrow \sup||x_n|| < \infty$. Consider $A_n \in \mathfrak{L}(X^*, \mathbb{R})$ ($=X^{**}$, the bidual) with $A_n(\ell) := \ell(x_n), \ell \in X^*$. $x_n \xrightarrow{w} x$ implies $\sup_n |A_n(l)| < \infty,$ $\forall l\in X^*$, and $X^*$ is complete so by Banach-Steinhaus: $$\sup_n ||A_n||_{\mathfrak{L}(X^*, \mathbb{R})} < \infty$$ 
    
    But by \Cref{dual characterisation of norm},  $||A_n||_{\mathfrak{L}(X^*,\mathbb{R})} = \sup_{l\in X^*, ||\ell||\leq 1} |\ell(x_n)| = ||x_n||_X$.
\end{enumerate}
\end{remark}

This naturally leads to:

\begin{definition}[Bidual]
\label{bidual defnition}
    $X^{**} := (X^*)^* \quad (=\mathfrak{L}(X^*,\mathbb{R}))$ is called \textbf{bidual} of $X$. $X$ embeds canonically into $X^{**}$ via: $$ \iota : X\rightarrow X^{**}: \iota(x)(\ell) := \ell(x) \quad \forall x \in X, \ell \in X^*$$ 
\end{definition}
\begin{remark}
    $\iota$ is a linear isometry: similarly as in \Cref{properties of weak convergence} 4) above. One has: $$\forall x \in X: ||x||_X = \sup_{l\in X^*, ||\ell|| \leq 1} |\ell(x)| = ||\iota(x)||_{**}$$
\end{remark}

\begin{definition}[Reflexive]\nl
    The space $X$ is reflexive if $\iota$ in \Cref{bidual defnition} is surjective.
\end{definition}

\begin{example}
\label{reflexive examples}
Some examples of reflexive spaces.  
\begin{enumerate}
    \item if $\dim X<\infty$, $X$ is reflexive;
    \item $H$ a Hilbert space is reflexive;
    \item $L^p, 1<p<\infty$ is reflexive;
    \item $L^1, L^\infty$ are in general not reflexive.
\end{enumerate}
\end{example}

\begin{proposition}
\label{non-reflexive space}
    $L^1[-1,1]$ and $L^{\infty}[-1,1]$ are not reflexive.
\end{proposition}
\begin{proof}
Consider $L^\infty[-1,1]$. 

Define the Dirac function 
$$\delta_{x_0}: C^0[-1,1] \rightarrow \mathbb{R}: f \mapsto \delta_{x_0}(f) = f(x_0)$$

$\delta_{x_0}$ is linear and continuous on $(C^0[-1,1],||\cdot ||_\infty)$. By \Cref{same norm extension}, there exists extension 
$$\ell\in (L^\infty [-1,1])^* \qquad \text{and} \qquad \ell\mid_{C^0[0,1]} = \delta_{x_0}.$$
with the same norm (Omit \textquotedblleft $[-1,1]$" henceforth) \\

For $g\in L^1$, we define $\forall f\in L^\infty$

$$\ell_g(f) = \int_{-1}^1 gf dx$$
then $l_g\in (L^\infty)^*$.

% \begin{unexaminable}
% This is the canonical embedding of $L^1$ into $(L^1)^{**}=(L^{\infty})^*$, since $(L^1)^* = L^{\infty}$, every $\zeta \in (L^1)^*$ can be written as  
% $$
% \zeta(g) = \int fg d\mu
% $$
% for some $f \in L^{\infty}$.  
% So $\iota: L^1 \mapsto (L^{\infty})^*:  \iota(g)(\zeta)=\zeta(g)$
% \end{unexaminable}

\textbf{Claim:} $\nexists g\in L^1: \ell=\ell_g$.  

The claim implies that $\iota: L^1 \rightarrow (L^1)^{**}=(L^\infty)^*, g \mapsto \ell_g$ is not surjective, i.e., $L^1$ is not reflexive.  

For $L^\infty$, use: $X$ reflexive $\Rightarrow$ $X^*$ reflexive (exercise).  

\textbf{Proof of Claim:} Suppose $\exists g\in L^1$ s.t. $l=l_g$. For simplicity let $x_0=0$.\\
Pick a bump function $\phi \in C^\infty [-1,1]: 0 \leq \phi \leq 1$

$$
\phi(x) = 1, \ x \in [-\frac{1}{2}, \frac{1}{2}] \qquad \text{and} \qquad \phi(x)=0, \ x=\pm 1
$$

For $n\leq 1: \phi_n(x) := \phi(nx)$. Then $0\leq \phi_n \leq 1$, $\phi_n \rightarrow 0 \ a.e.$.\\
This yields: $$1 = \phi_n(0)=\delta_{0}(\phi_n) = \ell(\phi_n) = l_g(\phi_n) = \int_{-1}^1 g\phi_n dx \rightarrow 0$$
this is a contradiction.
\end{proof}
\begin{unexaminable}
%     An example of a bump function is given at \href{https://math.stackexchange.com/questions/101480/are-there-other-kinds-of-bump-functions-than-e-frac1x2-1}{\color{navyblue} here}.  \\
% \begin{figure}[H]
%   \centering
%   \makebox[0pt]{%
%   \includesvg[inkscapelatex=false, width=0.5\paperwidth]{Chapter/Figures/bump_function.svg}
%   }
%   \caption{A bump function}
% \end{figure} 
An example of a bump function is given below:
\begin{equation}
f_n(x)=
\left\{
\begin{aligned}
&0 &|x|<\frac{1}{2n}\\
&{\rm exp}(1-\frac{1}{1-(2n|x|-1)^2}) &\frac{1}{2n}<|x|<\frac{1}{n}\\
&1 &\frac{1}{n}<|x|\leq1\\
\end{aligned}
\right.
\end{equation}


\begin{figure}[H]
  \centering
  \makebox[0pt]{%
  \includegraphics[ width=0.5\paperwidth]{Chapter/Figures/bump_01.png}
  }
  \caption{Example of bump functions}
\end{figure} 



\end{unexaminable}

Recall that we showed unit balls in $\infty$-dimensions are never (sequentially) compact. Weak convergence allows us to restore a weak version of this . %(cf. \todo{})?
For reflexive spaces that is the whole story.  Since $X$ may not be reflexive, one must consider an even weaker topology.  

In the following, we let $(X, \norm{\cdot})$ be a normed linear space, $X^*$ its dual space, $X^{**}$ its bidual, and the isometry $\iota: X \to X^{**}$.  

\begin{definition}
    A sequence of linear functionals $(\ell_n) \subset X^*$ is \textbf{weak*-convergent} to $\ell \in X^*$ if  
    $$
    \lim_{n\to \infty} \ell_n(x) = \ell(x) \qquad \forall x\in X
    $$
    (i.e. pointwise convergence in $X$) Notation: $\ell_n \overset{w^*}{\longrightarrow} \ell$
\end{definition}  

\begin{remark}
\begin{enumerate}[1)]
    \item We now have 3 notions of convergence on $X^*$:
    \begin{enumerate}[i)]
        \item norm/strong convergence: $\norm{\ell_n-\ell}_* \overset{n\to \infty}{\longrightarrow} 0$ (i.e. $\ell_n \to \ell$)
        \item weak convergence: $\ell_n \overset{w}{\longrightarrow} \ell$, i.e.  
        \begin{equation*}
        \forall \xi \in X^{**}: \qquad \lim_{n\to \infty} \xi(\ell_n) = \xi(\ell)    \tag{**}
        \end{equation*}
        \item weak*-convergence: $\ell_n \overset{w^*}{\longrightarrow} \ell$: equivalent to asking $(**)$ for $\xi \in \iota(X)$ only. 
    \end{enumerate}
    \item If $X$ is reflexive $ii) \iff iii)$ [e.g. Hilbert space]
    \item In general, $i) \implies ii) \implies iii)$
\end{enumerate}
    
\end{remark}

\begin{theorem}[Banach-Alaoglu]\nl
\label{Banach-Alaoglu}
Let $X$ be separable. If $(\ell_n) \subset X^*$ is bounded (in $X^*$) there exists $\ell\in X^*$ and a subsequence $\Lambda \subset \natu$ s.t.  
$$
\ell_n \overset{w^*}{\longrightarrow} \ell \qquad n \to \infty, n\in \Lambda
$$
\end{theorem}

\begin{proof}
    Let $(x_j) \subset X$ be a countable, dense subset. Using boundedness, pick a subsequence $\natu \supset \Lambda_1 \supset \Lambda_2 \supset \cdots \supset \Lambda_j \supset \Lambda_{j+1}$ (inductively) such that, for all $j \in \natu$:  
    $$
    \ell_n (x_j) \to a_j \in \real \qquad (n \to \infty, n\in \Lambda_j)
    $$  
    $\Lambda \defeq$ diagonal sequence of $(\Lambda_j)_j$, so $\forall j, \ell_n(x_j) \to a_j, (n\to \infty, n\in \Lambda)$. 
    
    Define $\ell(x_j) \defeq a_j$, extend it linearly on $M = span \{x_j: j\in \natu\}$ and for all $x\in M$:  
    $$
    |\ell(x)| = \lim_{k\to \infty, k\in \Lambda} |\ell_k(x)| \leq \sup_k \norm{\ell_k}_* \norm{x}_{X}
    $$
    so $\ell \in M^*$, hence it can be extended to $\ell \in X^*$ by \Cref{same norm extension}.  

    We now show: $\ell_n \overset{w^*}{\rightarrow} \ell$ $(n \to \infty, n \in \Lambda)$. 
    
    Let $x\in X$, pick $J \subset \natu$ s.t. $x_j \to x$ $(j\to \infty, j \in J)$. For such $j$ and $n\geq 1$:  
    \begin{align*}
        |\ell_n(x)-\ell(x)| &\leq |\ell_n(x-x_j)| + |\ell(x-x_j)| + |\ell_n(x_j)-\ell(x_j)| \\
        &\leq (\sup_n \norm{\ell_n}_* + \norm{\ell}_*) \norm{x-x_j}_X + |\ell_n(x_j)-\ell(x_j)|
    \end{align*}
    Letting first $n\to \infty$ yields  
    $$
    \lim_{n\to \infty, n\in \Lambda} |\ell_n(x)-\ell(x)| \leq C \norm{x-x_j}_X, \qquad j \in J
    $$  
    Now letting $j\to \infty, j\in J$ yields the desired result.
\end{proof}  

\begin{remark}
    \begin{enumerate}[1)]
        \item If $X$ is reflexive, separability can be removed. Together with {\color{red} todo!!!!!!!!!!!}

        \item As a corollary, for $H$ Hilbert. If $(x_n) \subset H$ is bounded ($\sup_n \norm{x_n}_H < \infty$), then $(x_n)$ has a weakly convergent subsequence.  
    \end{enumerate}
\end{remark}

Unless $\dim H<\infty$, one \textbf{cannot} replace weak by strong in 2).

\begin{example}
    \begin{enumerate}[i)]
        \item $X = L^1[0,1]$ is separable, $X^* \cong L^{\infty}$.
        If $(f_n) \subset L^{\infty}$ is bounded, i.e. $\sup \norm{f_n}_\infty < \infty$, \Cref{Banach-Alaoglu} yields a subsequence $(n_k)_k \subset \natu$ and $f\in L^{\infty}$ s.t.  
        $$
        \lim_{k \to \infty} \int f_{n_k}g dx = \int fg dx, \qquad \forall g\in L^1
        $$
        \item $X = L^{\infty}(=L^{\infty}[0,1])$ is not separable (and also not reflexive). The following example shows that the conclusions of \Cref{Banach-Alaoglu} fail in this case. 
        
        For $0<\varepsilon\leq 1$ consider,  
        $$
        T_{\varepsilon}:L^{\infty} \to \real \qquad T_\varepsilon f = \frac{1}{\varepsilon} \int_0^{\varepsilon} f dx, \qquad f\in L^{\infty}
        $$  
        Then $\norm{T_\varepsilon}_{(L^{\infty})^*}\leq 1$, i.e. $T_\varepsilon \in (L^\infty)^*$. We show:  

        \textbf{\underline{Claim:}} $\{T_\varepsilon: 0< \varepsilon \leq 1\}$ is not weak*-sequentially compact.  
        \begin{proof}
            Suppose it is, i.e. $\varepsilon \overset{k  \to \infty}{\longrightarrow} 0$ and $T \in (L^{\infty})^*$ s.t. $T_{\varepsilon_k} \overset{w^*}{\longrightarrow} T$ as $k\to \infty$. By passing to a subsequence, one can assume  
            $$
            1 > \frac{\varepsilon_{k+1}}{\varepsilon_{k}} \to 0 \qquad \text{as\ } k\to \infty
            $$
            Pick $f\defeq \sum_{k=1}^{\infty}(-1)^k \mathbf{1}_{(\varepsilon_{k+1}, \varepsilon_k]} \in L^{\infty}$ with $\norm{f_n}_\infty=1$.  

            For $k\geq 1$, we have:  

            $$
            T_{\varepsilon_k}f = \frac{1}{\varepsilon_k} \sum_{j=k}^{\infty} (-1)^j (\varepsilon_j - \varepsilon_{j+1}) = (-1)^k \frac{\varepsilon_k - \varepsilon_{k+1}}{\varepsilon_{k+1}} +\frac{1}{\varepsilon_k} \int_0^{\varepsilon_{k+1}} f dx
            $$  
            Hence  
            $$
            |T_{\varepsilon_k}f-(-1)^{k}| \leq \frac{1}{\varepsilon_k} \left(\varepsilon_{k+1} + \int_0^{\varepsilon_{k+1}} |f| dx\right) \leq \frac{2\varepsilon_{k+1}}{\varepsilon_k} \overset{k\to \infty}{\longrightarrow} 0
            $$  
            so $(T_{\varepsilon_k} f)_k$ accumulates at $\pm 1$ and is thus divergent.
        \end{proof}

    \item If instead consider $X = C^0[0,1]\subset L^{\infty}$, a separable closed subspace, then \Cref{Banach-Alaoglu} applies to $T_\varepsilon \mid_X$. Indeed, one immediately sees that  
    $$
    T_\varepsilon f \overset{\varepsilon \downarrow 0}{\longrightarrow} f(0), i.e. \qquad T_\varepsilon \overset{w^*}{\rightarrow} \delta_0 \ (\varepsilon \downarrow 0)
    $$
    where $\delta_0$ is the Dirac delta functional at $0$ defined in \Cref{non-reflexive space}.
    \end{enumerate}
\end{example}
\begin{comment}
[comment by chatGPT]
This LaTeX code is a mathematical note discussing weak and strong topologies. The note defines weak convergence and strong convergence and provides remarks about weak convergence. It also defines the bidual of a space and reflexive spaces.

One possible error in this note is in Remark 1), where the inequality $|\ell(x_n) - \ell(x)| \leq ||\ell||* ||x_n - x||$ should have $\leq ||\ell||* ||x_n - x||_X$ on the right-hand side to indicate that the norm being used is that of the normed linear space.

The most difficult part of the note may be the Banach-Steinhaus theorem in Remark 1), which is a deep result in functional analysis. To improve this note, the author could consider providing more intuition and examples for the Banach-Steinhaus theorem and its application to weak convergence. The author could also provide more examples of reflexive spaces and explain why they are important. Additionally, the author could include exercises and solutions to help students practice and reinforce their understanding of the material.
\end{comment}
\end{document}