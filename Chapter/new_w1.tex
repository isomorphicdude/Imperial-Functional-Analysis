\documentclass{article}
\usepackage[utf8]{inputenc}
\usepackage{geometry}
\geometry{left=3cm,right=3cm,top=2cm,bottom=2cm}
\usepackage[utf8]{inputenc}
\usepackage{amsmath, amsfonts, amssymb, amsthm}
\usepackage[framemethod=TikZ]{mdframed}
\usepackage{mathrsfs}
\usepackage{comment}
\usepackage{enumerate}
\usepackage{xcolor}
\usepackage{titlesec}
\usepackage{setspace}
\usepackage[hidelinks,backref]{hyperref}
\usepackage{cleveref}
\usepackage[most]{tcolorbox}
\usepackage{ragged2e}
\usepackage{todonotes}
\usepackage{cleveref}
\usepackage{mathtools}

%
\titleformat*{\section}{\LARGE \bfseries}
\titleformat*{\subsection}{\Large \bfseries}
\titleformat*{\subsubsection}{\Large \bfseries}
% \titleformat*{\paragraph}{\large \bfseries}
\titleformat*{\subparagraph}{\large \bfseries}



% General 
\newcommand{\nextline}{\hfill\break}
\newcommand{\nl}{\nextline\rm}
% \newcommand{\placeholder}{{\bf\color{red} NOOOOOOT COMPLEEEEEEET! COOOOOOOM BAAAAAACK!!!}}
\newcommand{\placeholder}{\todo{NOOOOOOT COMPLEEEEEEET! COOOOOOOM BAAAAAACK!!!}}

\newcommand{\defeq}{\stackrel{def.}{=}}
% FA and LA
% inner product: \inne{a}{b}
\newcommand{\inne}[2]{\left<{#1},{#2}\right>}

% norm: \norm{a}
\newcommand{\norm}[1]{\left\|{#1}\right\|}

% Curly H
\newcommand{\hbs}{$\mathscr{H}$ }
\newcommand{\hbp}{\mathscr{H}}

% Dual : \dual{x}
\newcommand{\dual}[1]{{#1}^*}

% Sequence from 1  to infty: \sequ{x_n}
\newcommand{\sequ}[1]{\left({#1}\right)_1^\infty}

% f: A-> B \func{f}{A}{B}
\newcommand{\func}[3]{${#1}:{#2}\xrightarrow{}{#3}$}

% interior
\newcommand{\interior}{\textrm{int}}

% Bounded linear funcs
\newcommand{\blf}[2]{\mathcal{L}({#1},{#2})}

\newcommand{\prf}{\textit{proof}:   }



% Fields 
\newcommand{\real}{\mathbb{R}}
\newcommand{\comp}{\mathbb{C}}
\newcommand{\inte}{\mathbb{Z}}
\newcommand{\natu}{\mathbb{N}}





% Theorems
% \newtheorem{example}{Example}[subsection]
% \newtheorem{definition}[example]{Definition}
% \newtheorem{proposition}[example]{Proposition}
% \newtheorem{remark}[example]{Remark}
% \newtheorem{theorem}[example]{Theorem}
% \newtheorem{lemma}[example]{Lemma}
% \newtheorem{corollary}[example]{Corollary}


% for numbering the theorems            
\theoremstyle{plain}
%%%%%%%%%%%%%%%%%%%%%%%%%%%%%%%%%%%%%%%%%%%%%%%%%%%%%%%%%%%
\newtheorem{theorem}{Theorem}[section]
\newtheorem{lemma}[theorem]{Lemma}
\newtheorem{corollary}[theorem]{Corollary}
\newtheorem{proposition}[theorem]{Proposition}
%%%%%%%%%%%%%%%%%%%%%%%%%%%%%%%%%%%%%%%%%%%%%%%%%%%%%%%%%%%
% the following are not in italics
\theoremstyle{definition}
\newtheorem{definition}[theorem]{Definition}
\newtheorem{example}[theorem]{Example}
\newtheorem{remark}[theorem]{Remark}
\newtheorem{claim}[theorem]{Claim}
%%%%%%%%%%%%%%%%%%%%%%%%%%%%%%%%%%%%%%%%%%%%%%%%%%%%%%%%%%%%

% proof box
\newtcbtheorem[no counter]{pf}{Proof}{
  enhanced,
  rounded corners,
  attach boxed title to top,
  colback=white,
  colframe=black!25,
  fonttitle=\bfseries,
  coltitle=black,
  boxed title style={
    rounded corners,
    size=small,
    colback=black!25,
    colframe=black!25,
  } 
}{prf}

% extra content box to put in contents not covered in the lecture notes
% use the command \begin{unexaminable}
\newmdenv[skipabove=7pt, skipbelow=7pt,
    rightline=false, leftline=false, topline=false, bottomline=false,
    backgroundcolor = gray!10,
    innerleftmargin=1in, innerrightmargin=1in, innertopmargin=5pt,
    leftmargin=-1in, rightmargin=-1in, linewidth=4pt,
    innerbottommargin=5pt]{unexamBox}
\newenvironment{unexaminable}{\begin{unexamBox}}{\end{unexamBox}}


% clever ref settings
\crefname{lemma}{lemma}{lemmas}
\Crefname{lemma}{Lemma}{Lemmas}
\crefname{theorem}{theorem}{theorems}
\Crefname{theorem}{Theorem}{Theorems}

% formatting 
% https://tex.stackexchange.com/questions/217497/aligning-stackrel-signs-beneath-each-other-using-split
\newlength{\leftstackrelawd}
\newlength{\leftstackrelbwd}
\def\leftstackrel#1#2{\settowidth{\leftstackrelawd}%
{${{}^{#1}}$}\settowidth{\leftstackrelbwd}{$#2$}%
\addtolength{\leftstackrelawd}{-\leftstackrelbwd}%
\leavevmode\ifthenelse{\lengthtest{\leftstackrelawd>0pt}}%
{\kern-.5\leftstackrelawd}{}\mathrel{\mathop{#2}\limits^{#1}}}

\doublespacing
\RaggedRight
% modify innerrightmargin if floats were lost

\usepackage{amsfonts, amsmath, amssymb, amsthm, thmtools, bm}
\usepackage{avant} % Use the Avantgarde font for headings

% Boxed/framed environments
\newtheoremstyle{royalnumbox}%
{0pt}% Space above
{0pt}% Space below
{\normalfont}% Body font
{}% Indent amount
{\small\bf\sffamily\color{royal}}% Theorem head font
{\;}% Punctuation after theorem head
{0.25em}% Space after theorem head
{\sffamily \color{royal} 
    \thmname{#1} 
    \thmnumber{#2} \thmnote{\bfseries\color{black}---\nobreakspace#3.}} % Optional theorem note
\renewcommand{\qedsymbol}{$\blacksquare$}% Optional qed square

\newtheoremstyle{blacknumex}% Theorem style name
{5pt}% Space above
{5pt}% Space below
{\normalfont}% Body font
{} % Indent amount
{\small\bf\sffamily}% Theorem head font
{\;}% Punctuation after theorem head
{0.25em}% Space after theorem head
{\sffamily
    \thmname{#1}
    \thmnumber{#2}
    \thmnote{---\nobreakspace#3.}}% Optional theorem note

\newtheorem*{notation}{Notation}
\newtheorem*{hint}{Hint}
\newtheorem*{solution}{Solution}

\newcounter{dummy} 
\numberwithin{dummy}{section}

\theoremstyle{royalnumbox}
\newtheorem{definitionT}[dummy]{Definition}
\newtheorem{theoremT}[dummy]{Theorem}
\newtheorem{lemmaT}[dummy]{Lemma}
\newtheorem{corollaryT}[dummy]{Corollary}
\newtheorem{propositionT}[dummy]{Proposition}
\newtheorem{propertyT}[dummy]{Property}
\newtheorem{remarkT}[dummy]{Remark}

\theoremstyle{blacknumex}
\newtheorem{exampleT}[dummy]{Example}
\newtheorem{exerciseT}[dummy]{Exercise}

\numberwithin{equation}{section}

\RequirePackage[framemethod=default]{mdframed}

% Definition box
\newmdenv[skipabove=7pt, skipbelow=7pt,
rightline=false, leftline=true, topline=false, bottomline=false,
backgroundcolor = reddish!10, 
linecolor=reddish,
innerleftmargin=5pt, innerrightmargin=30pt, innertopmargin=5pt,
leftmargin=0cm, rightmargin=0cm, linewidth=4pt,
innerbottommargin=5pt]{dBox}

% Main Theorem box
\newmdenv[skipabove=7pt, skipbelow=7pt,
rightline=false, leftline=true, topline=false, bottomline=false,
backgroundcolor=c0!10, 
linecolor=c0,
innerleftmargin=5pt, innerrightmargin=30pt, innertopmargin=5pt,
leftmargin=0cm, rightmargin=0cm, linewidth=4pt, innerbottommargin=5pt]{tBox}

% Lemma/Corollary/Proposition/Property box
\newmdenv[skipabove=7pt, skipbelow=7pt,
rightline=false, leftline=true, topline=false, bottomline=false,
backgroundcolor = c0!10, 
linecolor=c0!80,
innerleftmargin=5pt, innerrightmargin=30pt, innertopmargin=5pt,
leftmargin=0cm, rightmargin=0cm, linewidth=4pt,
innerbottommargin=5pt]{lBox}

% Example/Remark/Exercise box
\newmdenv[skipabove=7pt, skipbelow=7pt,
rightline=false, leftline=true, topline=false, bottomline=false,
backgroundcolor = mossgreen!10!white,
linecolor = mossgreen,
innerleftmargin=5pt, innerrightmargin=30pt, innertopmargin=5pt,
leftmargin=0cm, rightmargin=0cm, linewidth=4pt,
innerbottommargin=5pt]{exBox}

% Proof box
% \newmdenv[skipabove=7pt, skipbelow=7pt,
% rightline=false, leftline=true, topline=false, bottomline=false,
% linecolor=gray,
% innerleftmargin=5pt, innerrightmargin=30pt, innertopmargin=5pt,
% leftmargin=0cm, rightmargin=0cm, linewidth=4pt,
% innerbottommargin=5pt]{proofBox}



% Creates an environment for each type of theorem and assigns it a theorem text style from the "Theorem Styles" section above and a colored box from above
\newenvironment{definition}{\begin{dBox}\begin{definitionT}}{\end{definitionT}\end{dBox}}
\newenvironment{theorem}{\begin{tBox}\begin{theoremT}}{\end{theoremT}\end{tBox}}
\newenvironment{lemma}{\begin{lBox}\begin{lemmaT}}{\end{lemmaT}\end{lBox}}
\newenvironment{proposition}{\begin{lBox}\begin{propositionT}}{\end{propositionT}\end{lBox}}
\newenvironment{corollary}{\begin{lBox}\begin{corollaryT}}{\end{corollaryT}\end{lBox}}
\newenvironment{property}{\begin{lBox}\begin{propertyT}}{\end{propertyT}\end{lBox}}


% \newenvironment{proof*}{\begin{proofBox}\begin{proof}}{\end{proof}\end{proofBox}}
\newenvironment{exercise}{\begin{exBox}\begin{exerciseT}}{\hfill{\color{royal}}\end{exerciseT}\end{exBox}}
\newenvironment{remark}{\begin{exBox}\begin{remarkT}}{\end{remarkT}\end{exBox}}
\newenvironment{example}{\begin{exBox}\begin{exampleT}}{{}\end{exampleT}\end{exBox}}

\title{Week 1}
\author{\aut}
\begin{document}

\maketitle
\section{Week1}
What is the course about?\\

Roughly: solving linear systems of the form
$$Ax=y$$
where \func{A}{X}{Y} is linear, $y\in Y$ given, find solutions $x\in X$. $X$ and $Y$ are linear spaces.
\begin{remark}
	When $X$ and $Y$ are finite dimensional spaces, this is linear algebra.\\
	$\infty$-dim brings into play additional structure, then it's functional analysis. (completeness, compactness, metric, norm)
\end{remark}

Example: $f\in C_0^\infty (\real^n)$, solve
$$
	\underbrace{-\Delta}_A u=f\in\real^n
$$where
$$
	C_0^\infty(\real^n)=\left\{
	f:\real^n\to\real:f\text{ infinitely differentiable }, supp(f)=\{x:f(x)\neq0\} \text{ compact }
	\right\}
$$
What space?
Can take (in PDE) $X=Y=C_\infty(\real^n)$: function spaces.
Adequate choice of space to find solutions necessary!
con vary, metric/normed linear space, locally convec topological space...

In this course: linear space is (almost always): Banach space or Hilbert space.
\begin{example}[Running example, $L^P$]
	cf. MATH50006 notes $\S2.6$ .
	$(X,\mathcal{A},\mu)$:measure space.
	$$
		L^P(\mu)\equiv l^P(X,\mathcal{A},\mu)=
		\left\{
		f:X\to\mathbb{K}/\sim:\norm{f}_{L^P}<\infty
		\right\},\quad p\in[1,\infty]
	$$
	Norm of $f\in L^p$:
	$$
		\norm{f}_p\equiv\norm{f}_{L^p}=
		\left\{
		\begin{aligned}
			 & \left(\int|f|^p{\rm d}\mu\right)^{1/p} & p<
			\infty                                               \\
			 & \esssup_X |f|                          & p=\infty
		\end{aligned}\right.
	$$
\end{example}


Choices of measure space:\\
\underline{Example 0}:\\ $X=\{1,2,....,n\}$, $\mathcal{A}=2^X$, $\mu(\{\})k=1\,\forall k=1,...,n$, counting measure on $X$, extended to measure by additivity.\\
Every function $f:X\to\real$ is simple
$$
	f(x)=\sum^n_{k=1}f(k)1_{\{k\}}(x)\qquad 1_A(x)=
	\left\{\begin{aligned}
		 & 1 & x\in A    \\
		 & 0 & x\notin A
	\end{aligned}\right.
$$
Thus
$$
	\norm{f}_p^p=\int|f|^p{\rm d}\mu=\sum_{k=1}^n|f(k)|^p
	\underbrace{\int1_{\{k\}}{\rm d}\mu}_{\mu(\{k\})=1}
	=\sum_{k=1}^n|f(k)|
$$
So $L^p(\{1,...,n\},\mu\cong\real^n$ endowed with norm
$$
	\norm{p}_p=\left(\sum_{k=1}^n|f(k)|^p\right)^{1/p}\quad f\in\real^n
$$
which are finite dimensional vector spaces over $\real$ (linear algebra).

\underline{Example 1}:\\
Same with $X=\natu=\{1,2,3,...\}\mathrel{\leadsto} \ell^p$, "Little-l-p",\\
i.e. $\mu(\{k\})=1\,\forall k\in\natu$, extended to measure by $\sigma$-additivity, i.e.
$$
	\mu(A)=\sum_{k\in A} \mu(\{k\})=|A|\quad A\subset \natu
$$
Now every $f:X\to \real $ of form

$$f(x)=\sum_{k=1}^\infty f(k)1_{\{k\}}(x)$$
Approximated by $f_n\mathrel{\leadsto}\sum_{k=1}^n$ + use of monotone convergence theorem to get
$$\norm{f}_p^p=\sum_{k=1}^\infty |f(k)|^p$$
An element $f:X\to \real$, $f=(f(1),f(2),...)\equiv(f_1,f_2,..._\equiv(f_k)_k$ is a sequence!
$$\ell^p=
	\{
	\text{all real-valued sequences }
	f=(f_k)_k
	\text{ s.t. }
	\sum_{k=1}^\infty|f_k|^p<\infty
	\}$$
Specially, when $\ell^1$ is the set of all \underline{absolutely convergent series}!.
\begin{remark}
	Adding weights (Example 10): $\ell^p(\eta)\,\eta_i\geq0$, define $\mu(\{j\})=\eta_j\,\forall j\in\natu\mathrel{\leadsto}\norm{f}_{\ell^p(\eta)}=\sum_{j=1}^\infty|f_j|^p\eta_j$.
\end{remark}

Example 2\\
$X=\real^n$ for some $n\in\natu$, $\mathcal{A}=$ Borel $\sigma$-algebra, $\mu=$Lebesgue meausre, $\mathrel{\leadsto}L^P(r\real^n)$,
$$\norm{f}_p=\left(\int|f|^p{\rm d}x\right)^\frac{1}{p}$$,
where ${\rm{d}}x$ is Lebesgue. \\
More generally, $x\subset \real^n$ open/closed $\mathrel{\leadsto} L^P(X)$, e.g. $n=1$, $X=[0,1]$.



\begin{theorem}
	Let $(X,\mathcal{A},\mu)$ be {\underline {any}} measure space. Then:
	\begin{itemize}
		\item [i)] $\norm{f}_p$ defines a \underline{norm} $\forall p\in[1,\infty]$, triangular inequality (Minkowski ineq.) holds: $\norm{f+g}_p\leq\norm{f}_p+\norm{g}_p$.
		\item [ii)] Holders inequality holds: if $\frac{1}{p}+\frac{1}{q}=1$ with $p,q\in[1,\infty]$, then $\forall f\in L^p(\mu),\,\forall g\in L^q(\mu)$, then $f\cdot g\in L^1(\mu)$ and $\norm{fg}_{L^1(\mu)}\leq\norm{f}_{L^1(\mu)}\norm{g}_{L^1(\mu)}$
		\item [iii)] $L^p(\mu)$ is complete
	\end{itemize}
\end{theorem}

\begin{definition}
	A normed linear space $(X,\norm{\cdot}$ which is complete w.r.t the induced metric is called \underline{Banach space}.\\
	Explanation:
	\begin{itemize}
		\item $\norm\cdot$ is a norm. See definition 3 in notes.
		\item the induced metric is $d(x,y)=\norm{x-y}$. It is a metric (proposition 2 in the notes) % d(x,y) is the notation used in year-2 analysis 
		\item $X$ complete w.r.t. $d$: every Cauchy sequence converges.
		\item Cauchy sequence: $(f_n)_n\subset X$ s.t. $\forall \varepsilon>0,\,\exists N\,\forall m,n\geq N,\,d(f_m,f_n)<\varepsilon$
	\end{itemize}
\end{definition}

\begin{remark}
	Theorem asserts $L^P(\mu)$ is a Banach space. Apply in the case of 1 to get immediately all of theorem 2, exercise 10-12 proposition 7 example 12-14 and much more
	!

\end{remark}

\underline{Furthur examples}\\
3. $C([a,b])=\{f:[a,b]\to\real,\,\text{continuous}\}$ with $\norm{f}_\infty=\sup_{x\in[a,b]}|f(x)|$\\
4. $C^r([a,b])$, similar to above but $f$ is set to be r-times continuously differentiable. In particular $C=C[(a,b)]=C^0[(a,b)]$.  $\norm{f}_{r,\infty}=\sup_{x\in[a,b],1\leq k\leq r}|f^{(k)}(x)|$\\
5. Sobolev space, for solving PDE(not this course).

\begin{proposition}
	$(C[0,1],\norm{\cdot}_\infty)$ is complete.
\end{proposition}
\underline{General strategy to show completeness of  $(X,\norm{\cdot})$}: For a given Cauchy sequence $(f_n)\subset X$
\begin{itemize}
	\item [1] find candidate limit $f$
	\item [2] show $\norm{f_n-f}\xrightarrow{n\to\infty}0$
	\item [3] show $f\in X$
\end{itemize}

Proof: \\
\underline{STEP I}  Let $(f_n)\subset C$ be Cauchy sequence, i.e. $\forall \varepsilon>0,\,\exists N,\,\forall m,n\geq N:\,\norm{f_m-f_n}_\infty<\varepsilon$.\\
But $\norm{f_n-f_m}_\infty\geq|f_n(x)-f_m(x),\,\forall x\in[0,1]$ so $\left(f_n(x)\right)$ is Cauchy sequence in $\real$ $\forall  x\in[0,1]$. Since $\real$ is complete, it has a limit. Call it $f(x)=\lim_{n\to\infty}f_n(x)$.\\

\underline{STEP II} 
$|f_n(x)-f_m(x)|<\varepsilon\,\forall n,m\geq N,\,\forall x\in[0,1]$, which implies
$\lim_{n\to\infty}|f_n(x)-f_m(x)|<\varepsilon$, we also have that $lim_{n\to\infty}|f_n(x)-f_m(x)|=|f(x)-f_m(x)|$ by continuity of $f$.\\

\underline{STEP III} 
To show $f\in C$, need to argue 
\[
\forall x,\,\varepsilon>0,\,\exists\delta:\,|x-y|<\delta\implies |f(x)-f(y)|<\varepsilon \tag{$C1$}\label{C1}
\]
Now we use $\varepsilon/3$ argument. Write for any $n\in\natu$ and $x,y\in[0,1]$,
\[
	|f(x)-f(y)|\leq|f(x)-f_n(x)|+|f_n(x)-f_n(y)|+|f_n(y)-f(y)|\tag{$C2$}\label{C2}
\]
First, using STEP II, pick $n$ s.t. $\norm{f_n-f}_\infty<\frac{\varepsilon}{3}$, whence
\[
	|f(x)-f_n(x)|,\,|f(y)-f_n(y)|<\frac{\varepsilon}{3},\,\,\forall x,y\in[0,1]\tag{$C3$}\label{C3}
\]
Then, with $n$ now fixed, using continuity of $f_n$, pick $\delta>0$ s.t.
\[
	|f_n(x)-f_n(y)<\frac{\varepsilon}{3}\text{ whenever } |x-y|<\delta\tag{$C4$}\label{C4}
\]
Substitute \Cref{C4},\Cref{C3} into \Cref{C2} to get \Cref{C1}.


\begin{remark}
	\begin{itemize}
		\item \cref{C3} is well-known from analysis I-II. $\lim_{n\to\infty}\norm{f_n-f}_\infty=0$ is precisely the \underline{uniform convergence} of $(f_n)$ towards $f$. so 3 asserts that "uniform limit of a sequence of continuous functions is again continuous".
		\item $f_n(x)=x^n$ is not a Cauchy sequence in $(c,\norm{\cdot}\infty$, yet $f_n(x)\xrightarrow{n\to\infty} f(x)=1_{\{1\}}(x)$.
		\item if instead consider $f_n(x)=x^n$ to be elements of $L^1([0,1])$:
		      $$\int_0^1f_n{\rm d}x=\left.\frac{x^{n+1}}{n+1}\right|^1_0=\frac{1}{n+1}<\infty$$
		      Then $(f_n)\subset L^1([0,1])$ a Cauchy sequence(Ex.), and converges by  completeness of $L^p(\mu)$. The limit in $L^1([0,1])$ is $f=0$:
		      $$\norm{f_n}_{L^1([0,1])}=\norm{f_n-0}_{L^1([0,1])}=(n+1)^{-1}\xrightarrow{n\to\infty}0$$
	\end{itemize}

\end{remark}
\end{document}

