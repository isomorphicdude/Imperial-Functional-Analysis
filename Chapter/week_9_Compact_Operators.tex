\documentclass{article}
\usepackage[utf8]{inputenc}
\usepackage{geometry}
\geometry{left=3cm,right=3cm,top=2cm,bottom=2cm}
\usepackage[utf8]{inputenc}
\usepackage{amsmath, amsfonts, amssymb, amsthm}
\usepackage[framemethod=TikZ]{mdframed}
\usepackage{mathrsfs}
\usepackage{comment}
\usepackage{enumerate}
\usepackage{xcolor}
\usepackage{titlesec}
\usepackage{setspace}
\usepackage[hidelinks,backref]{hyperref}
\usepackage{cleveref}
\usepackage[most]{tcolorbox}
\usepackage{ragged2e}
\usepackage{todonotes}
\usepackage{cleveref}
\usepackage{mathtools}

%
\titleformat*{\section}{\LARGE \bfseries}
\titleformat*{\subsection}{\Large \bfseries}
\titleformat*{\subsubsection}{\Large \bfseries}
% \titleformat*{\paragraph}{\large \bfseries}
\titleformat*{\subparagraph}{\large \bfseries}



% General 
\newcommand{\nextline}{\hfill\break}
\newcommand{\nl}{\nextline\rm}
% \newcommand{\placeholder}{{\bf\color{red} NOOOOOOT COMPLEEEEEEET! COOOOOOOM BAAAAAACK!!!}}
\newcommand{\placeholder}{\todo{NOOOOOOT COMPLEEEEEEET! COOOOOOOM BAAAAAACK!!!}}

\newcommand{\defeq}{\stackrel{def.}{=}}
% FA and LA
% inner product: \inne{a}{b}
\newcommand{\inne}[2]{\left<{#1},{#2}\right>}

% norm: \norm{a}
\newcommand{\norm}[1]{\left\|{#1}\right\|}

% Curly H
\newcommand{\hbs}{$\mathscr{H}$ }
\newcommand{\hbp}{\mathscr{H}}

% Dual : \dual{x}
\newcommand{\dual}[1]{{#1}^*}

% Sequence from 1  to infty: \sequ{x_n}
\newcommand{\sequ}[1]{\left({#1}\right)_1^\infty}

% f: A-> B \func{f}{A}{B}
\newcommand{\func}[3]{${#1}:{#2}\xrightarrow{}{#3}$}

% interior
\newcommand{\interior}{\textrm{int}}

% Bounded linear funcs
\newcommand{\blf}[2]{\mathcal{L}({#1},{#2})}

\newcommand{\prf}{\textit{proof}:   }



% Fields 
\newcommand{\real}{\mathbb{R}}
\newcommand{\comp}{\mathbb{C}}
\newcommand{\inte}{\mathbb{Z}}
\newcommand{\natu}{\mathbb{N}}





% Theorems
% \newtheorem{example}{Example}[subsection]
% \newtheorem{definition}[example]{Definition}
% \newtheorem{proposition}[example]{Proposition}
% \newtheorem{remark}[example]{Remark}
% \newtheorem{theorem}[example]{Theorem}
% \newtheorem{lemma}[example]{Lemma}
% \newtheorem{corollary}[example]{Corollary}


% for numbering the theorems            
\theoremstyle{plain}
%%%%%%%%%%%%%%%%%%%%%%%%%%%%%%%%%%%%%%%%%%%%%%%%%%%%%%%%%%%
\newtheorem{theorem}{Theorem}[section]
\newtheorem{lemma}[theorem]{Lemma}
\newtheorem{corollary}[theorem]{Corollary}
\newtheorem{proposition}[theorem]{Proposition}
%%%%%%%%%%%%%%%%%%%%%%%%%%%%%%%%%%%%%%%%%%%%%%%%%%%%%%%%%%%
% the following are not in italics
\theoremstyle{definition}
\newtheorem{definition}[theorem]{Definition}
\newtheorem{example}[theorem]{Example}
\newtheorem{remark}[theorem]{Remark}
\newtheorem{claim}[theorem]{Claim}
%%%%%%%%%%%%%%%%%%%%%%%%%%%%%%%%%%%%%%%%%%%%%%%%%%%%%%%%%%%%

% proof box
\newtcbtheorem[no counter]{pf}{Proof}{
  enhanced,
  rounded corners,
  attach boxed title to top,
  colback=white,
  colframe=black!25,
  fonttitle=\bfseries,
  coltitle=black,
  boxed title style={
    rounded corners,
    size=small,
    colback=black!25,
    colframe=black!25,
  } 
}{prf}

% extra content box to put in contents not covered in the lecture notes
% use the command \begin{unexaminable}
\newmdenv[skipabove=7pt, skipbelow=7pt,
    rightline=false, leftline=false, topline=false, bottomline=false,
    backgroundcolor = gray!10,
    innerleftmargin=1in, innerrightmargin=1in, innertopmargin=5pt,
    leftmargin=-1in, rightmargin=-1in, linewidth=4pt,
    innerbottommargin=5pt]{unexamBox}
\newenvironment{unexaminable}{\begin{unexamBox}}{\end{unexamBox}}


% clever ref settings
\crefname{lemma}{lemma}{lemmas}
\Crefname{lemma}{Lemma}{Lemmas}
\crefname{theorem}{theorem}{theorems}
\Crefname{theorem}{Theorem}{Theorems}

% formatting 
% https://tex.stackexchange.com/questions/217497/aligning-stackrel-signs-beneath-each-other-using-split
\newlength{\leftstackrelawd}
\newlength{\leftstackrelbwd}
\def\leftstackrel#1#2{\settowidth{\leftstackrelawd}%
{${{}^{#1}}$}\settowidth{\leftstackrelbwd}{$#2$}%
\addtolength{\leftstackrelawd}{-\leftstackrelbwd}%
\leavevmode\ifthenelse{\lengthtest{\leftstackrelawd>0pt}}%
{\kern-.5\leftstackrelawd}{}\mathrel{\mathop{#2}\limits^{#1}}}

\doublespacing
\RaggedRight
% modify innerrightmargin if floats were lost

\usepackage{amsfonts, amsmath, amssymb, amsthm, thmtools, bm}
\usepackage{avant} % Use the Avantgarde font for headings

% Boxed/framed environments
\newtheoremstyle{royalnumbox}%
{0pt}% Space above
{0pt}% Space below
{\normalfont}% Body font
{}% Indent amount
{\small\bf\sffamily\color{royal}}% Theorem head font
{\;}% Punctuation after theorem head
{0.25em}% Space after theorem head
{\sffamily \color{royal} 
    \thmname{#1} 
    \thmnumber{#2} \thmnote{\bfseries\color{black}---\nobreakspace#3.}} % Optional theorem note
\renewcommand{\qedsymbol}{$\blacksquare$}% Optional qed square

\newtheoremstyle{blacknumex}% Theorem style name
{5pt}% Space above
{5pt}% Space below
{\normalfont}% Body font
{} % Indent amount
{\small\bf\sffamily}% Theorem head font
{\;}% Punctuation after theorem head
{0.25em}% Space after theorem head
{\sffamily
    \thmname{#1}
    \thmnumber{#2}
    \thmnote{---\nobreakspace#3.}}% Optional theorem note

\newtheorem*{notation}{Notation}
\newtheorem*{hint}{Hint}
\newtheorem*{solution}{Solution}

\newcounter{dummy} 
\numberwithin{dummy}{section}

\theoremstyle{royalnumbox}
\newtheorem{definitionT}[dummy]{Definition}
\newtheorem{theoremT}[dummy]{Theorem}
\newtheorem{lemmaT}[dummy]{Lemma}
\newtheorem{corollaryT}[dummy]{Corollary}
\newtheorem{propositionT}[dummy]{Proposition}
\newtheorem{propertyT}[dummy]{Property}
\newtheorem{remarkT}[dummy]{Remark}

\theoremstyle{blacknumex}
\newtheorem{exampleT}[dummy]{Example}
\newtheorem{exerciseT}[dummy]{Exercise}

\numberwithin{equation}{section}

\RequirePackage[framemethod=default]{mdframed}

% Definition box
\newmdenv[skipabove=7pt, skipbelow=7pt,
rightline=false, leftline=true, topline=false, bottomline=false,
backgroundcolor = reddish!10, 
linecolor=reddish,
innerleftmargin=5pt, innerrightmargin=30pt, innertopmargin=5pt,
leftmargin=0cm, rightmargin=0cm, linewidth=4pt,
innerbottommargin=5pt]{dBox}

% Main Theorem box
\newmdenv[skipabove=7pt, skipbelow=7pt,
rightline=false, leftline=true, topline=false, bottomline=false,
backgroundcolor=c0!10, 
linecolor=c0,
innerleftmargin=5pt, innerrightmargin=30pt, innertopmargin=5pt,
leftmargin=0cm, rightmargin=0cm, linewidth=4pt, innerbottommargin=5pt]{tBox}

% Lemma/Corollary/Proposition/Property box
\newmdenv[skipabove=7pt, skipbelow=7pt,
rightline=false, leftline=true, topline=false, bottomline=false,
backgroundcolor = c0!10, 
linecolor=c0!80,
innerleftmargin=5pt, innerrightmargin=30pt, innertopmargin=5pt,
leftmargin=0cm, rightmargin=0cm, linewidth=4pt,
innerbottommargin=5pt]{lBox}

% Example/Remark/Exercise box
\newmdenv[skipabove=7pt, skipbelow=7pt,
rightline=false, leftline=true, topline=false, bottomline=false,
backgroundcolor = mossgreen!10!white,
linecolor = mossgreen,
innerleftmargin=5pt, innerrightmargin=30pt, innertopmargin=5pt,
leftmargin=0cm, rightmargin=0cm, linewidth=4pt,
innerbottommargin=5pt]{exBox}

% Proof box
% \newmdenv[skipabove=7pt, skipbelow=7pt,
% rightline=false, leftline=true, topline=false, bottomline=false,
% linecolor=gray,
% innerleftmargin=5pt, innerrightmargin=30pt, innertopmargin=5pt,
% leftmargin=0cm, rightmargin=0cm, linewidth=4pt,
% innerbottommargin=5pt]{proofBox}



% Creates an environment for each type of theorem and assigns it a theorem text style from the "Theorem Styles" section above and a colored box from above
\newenvironment{definition}{\begin{dBox}\begin{definitionT}}{\end{definitionT}\end{dBox}}
\newenvironment{theorem}{\begin{tBox}\begin{theoremT}}{\end{theoremT}\end{tBox}}
\newenvironment{lemma}{\begin{lBox}\begin{lemmaT}}{\end{lemmaT}\end{lBox}}
\newenvironment{proposition}{\begin{lBox}\begin{propositionT}}{\end{propositionT}\end{lBox}}
\newenvironment{corollary}{\begin{lBox}\begin{corollaryT}}{\end{corollaryT}\end{lBox}}
\newenvironment{property}{\begin{lBox}\begin{propertyT}}{\end{propertyT}\end{lBox}}


% \newenvironment{proof*}{\begin{proofBox}\begin{proof}}{\end{proof}\end{proofBox}}
\newenvironment{exercise}{\begin{exBox}\begin{exerciseT}}{\hfill{\color{royal}}\end{exerciseT}\end{exBox}}
\newenvironment{remark}{\begin{exBox}\begin{remarkT}}{\end{remarkT}\end{exBox}}
\newenvironment{example}{\begin{exBox}\begin{exampleT}}{{}\end{exampleT}\end{exBox}}

\title{Week 9}

% \author{\aut}
\begin{document}
\maketitle
\section{Compact operators}  

Compact operators form a very important class of bounded operators. Roughly: they are the closest thing to a matrix in infinite dimension spaces. (cf. \Cref{Spectral theory chapter}).  

\begin{definition}[Compact operator]\nl
    Let $X,Y$ be normed spaces. $T:X\to Y$ linear. $T$ is compact if for all $B\subset X$ bounded (i.e. $\sup \{\norm{x}_X: x\in B\}<\infty$). $\overline{T(B)}$ is sequentially compact, where $T(B)=\{Tx: x\in B\} \subset Y$.
\end{definition}  

\begin{lemma}
\label{equivalent characterisation of compact operators}
Let $X,Y$ be Banach spaces. The following are equivalent:  
\begin{enumerate}[i)]
    \item $T$ is compact
    \item $\overline{T(B_X(0,1))} \subset Y$ is compact
    \item $\forall (x_n) \subset X$ bounded, $(Tx_n)$ has a Cauchy subsequence
\end{enumerate}  
\end{lemma}  

\begin{remark}\nl
The above are true if $X,Y$ are normed and one replaces "Cauchy" by "convergent".
\end{remark}  

\begin{proof}
iii) $\implies$ i). Let $B\subset X$ be bounded. Consider $(y_n) \subset T(B)$ and by iii), $(y_n)$ has a Cauchy subsequence. Hence $\overline{T(B)}$ is compact. Rest is exercise.
\end{proof}

\begin{example}
\label{examples of compact operators}
\begin{enumerate}[1)]
    \item $T=id:X\to X$ is compact $\iff$ $\dim X <\infty$.  
    
    For $\dim X=\infty$, recall the closed unit ball $B=B_X(0,1)$ is not compact. 
    
    \item $T$ has \textbf{finite rank} if $\dim (im(T))<\infty$. If $T\in \mathcal{L}(X,Y)$ has finite rank, then $T$ is compact:   
    
    Using \Cref{equivalent characterisation of compact operators} iii): let $(x_n)\subset X$ be bounded. Then $\norm{Tx_n} \leq \norm{T}\norm{x_n} \leq C$ so $(Tx_n)\subset im(T)$ is bounded. Since $im(T)$ is finite dimensional, one can choose a convergent subsequence.  

    \item If $\dim X<\infty$, $T$ is compact. (apply 2)

    \item (Diagonal Operator) $1\leq p \leq \infty$, $\lambda = (\lambda_n)_{n \in \natu}, \lambda_n \in \real$ and $\sup_n |\lambda_n|<\infty$. Then  
    $$
    T_\lambda: \ell^p \to \ell^p \qquad T_\lambda x \defeq (\lambda_n x_n)_{n\in \natu} \qquad \text{for\ } x=(x_n)_{n\in \natu}
    $$
    is well-defined. If $T_\lambda$ is compact then $\lim_{n\to \infty} \lambda_n = 0$.  

    For, if $\Lambda \subset \natu$ is such that $|\lambda_n|\geq \delta$, $n\in \Lambda$, for some $\delta>0$, then the sequence $(e_n), n\in \Lambda$ is bounded but $(T_{\lambda}e_n: n\in \Lambda)$ has no Cauchy subsequence:  
    $$
    \forall n\neq m, n,m\in \Lambda: \norm{T_\lambda e_n - T_\lambda e_m}_p \geq \delta 2^{1/p}
    $$  
\end{enumerate}
\end{example}

We return to this example after the following.  

\begin{theorem}[Limit of compact operators]\nl
\label{limit of compact operators}
    Let $X, Y$ be Banach spaces. If $T_n: X\to Y$ is a sequence of compact operators and for some $T \in \mathcal{L}(X,Y)$  
    \begin{align}
    \label{converging seq of compact operators}
      \norm{T_n-T}_{\mathcal{L}(X,Y)} \to 0 \qquad n\to \infty    
    \end{align}
    Then $T$ is compact.
\end{theorem}  
\begin{remark}
    This means the space of compact operators $(\left\{T \in \mathcal{L}(X,Y): T \ \text{compact} \right\}, \norm{\cdot}_{\mathcal{L}(X,Y)}) \subset (\mathcal{L}(X,Y), \norm{\cdot}_{\mathcal{L}(X,Y)})$ is closed hence a Banach space.
\end{remark}
\begin{proof}
    We will use \Cref{equivalent characterisation of compact operators} iii) and the diagonal argument.  

    Let $(x_n)\subset X, \sup_n \norm{x_n}_X \leq C$ be bounded.  
    
    \underline{\textbf{Goal:}} Show $(Tx_n)$ has a Cauchy subsequence.  

    Since $T_1$ is compact, there exists subsequence $\Lambda_1 \subset \natu$ s.t.  
    \begin{center}
        $(T_1 x_n) \subset Y$ converges w.r.t. $\norm{\cdot}_Y$ as $n\to \infty, n\in \Lambda_1$   
    \end{center}
    By induction, one can find subsequence $\Lambda_1 \supset \Lambda_2 \supset \cdots$ s.t.  
    \begin{center}
        $\forall k \in \natu: \qquad (T_k x_n) \subset Y$ converges as $n\to \infty, n\in \Lambda_k$
    \end{center}
    Let $\Lambda$ be the diagonal subsequence of $\Lambda_1, \Lambda_2, \ldots$, then $\Lambda \subset \Lambda_k, \forall k$ so  
    \begin{align}
    \label{convergence of diagonal of compact operators}
        \forall k \in \natu: \qquad (T_k x_n) \subset Y \ \text{converges\ as} \  n\to \infty, n\in \Lambda
    \end{align}
    \underline{\textbf{Claim:}} $(Tx_n)_{n\in \Lambda}$ is Cauchy (in fact converges)  
    
    For $n,m\in \Lambda$ and $k\in \natu$ write
    \begin{align*}
        \norm{Tx_n - Tx_m}_Y &\leq \norm{(T-T_k)x_n}_Y + \norm{T_k(x_n-x_m)}_Y + \norm{(T-T_k)x_m}  \\
        &\leq \norm{T-T_k}_{\mathcal{L}(X,Y)} \cdot 2C + \norm{(T-T_k)x_n}_Y
    \end{align*}
    Let $\varepsilon >0$. First pick $k$ s.t. $\norm{T-T_k} < \frac{\varepsilon}{4C}$ (use \Cref{converging seq of compact operators}).  
    
    Then use \Cref{convergence of diagonal of compact operators} to obtain $\forall n,m \in \Lambda$, with $\min (m,n) \geq N_0(\varepsilon)$:  
    $$
    \norm{T_kx_n-T_kx_m}_Y < \frac{\varepsilon}{2}
    $$
\end{proof}
\begin{example}
    Back to \Cref{examples of compact operators} 4):  
$$
T_\lambda \ \text{compact} \iff \lim_{n\to \infty}\lambda_n=0
$$  
\end{example}

\begin{proof}
    "$\implies$": see \Cref{examples of compact operators}.  

    "$\impliedby$": use \Cref{limit of compact operators}. Define  
    $$
    T_n: \ell^p \to \ell^p \qquad x \mapsto T_nx=(\lambda_0x_0,\ldots,\lambda_nx_n,0,0,\ldots)
    $$  
    Then $\dim(im(T))\leq n$, since $im(T)\subset \{x\in \ell^p: x_i=0, \forall i>n\}$, so $T_n$ has finite rank, hence $T_n$ is compact (cf. \Cref{equivalent characterisation of compact operators}).  
    Moreover, for $x\in \ell^p$,  
    $$
    \norm{(T-T_n)x}_p = (\sum_{m>n} |\lambda_mx_m|^p)^{\frac{1}{p}} \leq \sup_{m\geq n} |\lambda_m| \norm{x}_p
    $$  
    so $\norm{T-T_n}\to 0$. By \Cref{limit of compact operators} $T$ is compact.
\end{proof}

\begin{example}[Hilbert-Schmidt integral operator]\nl
Let $X = L^2[0,1]$ and $a \in C^0[0,1]^2$. Define operator $A:X \to X$ by,  
$$
Af(x) = \int_0^1 a(x,y) f(y)dy, \qquad f\in L^2[0,1]
$$
\begin{enumerate}[1)]
    \item $A$ is well-defined and bounded:  
$$
Af(x) = \int_0^1 |Af(x)|^2 dx \overset{\text{Cauchy\ Schwartz}}{\leq} \underbrace{\int_0^1 dx \int_0^1 dy |a(x,y)|^2}_{\leq C} \norm{f}_2^2
$$
\item $A$ is compact:   

Let $(f_n) \subset X, \norm{f_n}_2 \leq M$. Check $(Af_n)$ is continuous, $\sup_n \norm{Af_n}_{\infty} < \infty$. 
Moreover, let $\varepsilon>0$, we can pick $\delta>0$ s.t. $|a(x,y)-a(x',y')|<\varepsilon$ if $|x-x'|+|y-y'|<\varepsilon$ (uniform-continuity), we have  

\begin{align*}
    |Af_n(x) - Af_n(y)| &\leq \int_0^1 \underbrace{|a(x,z)-a(y,z)|}_{<\varepsilon \ \text{if}\ |x-y|<\delta} |f_n(z)| dz \\
    & \leq \varepsilon \norm{f_n}_2 \leq M\varepsilon
\end{align*}  
so $(Af_n)$ is equicontinuous. By Arzelà–Ascoli, $(Af_n)$ has a subsequence which converges in $\norm{\cdot}_{L^{\infty}[0,1]}$ hence in $\norm{\cdot}_{L^{2}[0,1]}$.
\end{enumerate}


\end{example}
\end{document}