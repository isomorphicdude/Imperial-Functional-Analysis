\documentclass{article}
\usepackage[utf8]{inputenc}
\usepackage{geometry}
\geometry{left=3cm,right=3cm,top=2cm,bottom=2cm}
\usepackage[utf8]{inputenc}
\usepackage{amsmath, amsfonts, amssymb, amsthm}
\usepackage[framemethod=TikZ]{mdframed}
\usepackage{mathrsfs}
\usepackage{comment}
\usepackage{enumerate}
\usepackage{xcolor}
\usepackage{titlesec}
\usepackage{setspace}
\usepackage[hidelinks,backref]{hyperref}
\usepackage{cleveref}
\usepackage[most]{tcolorbox}
\usepackage{ragged2e}
\usepackage{todonotes}
\usepackage{cleveref}
\usepackage{mathtools}

%
\titleformat*{\section}{\LARGE \bfseries}
\titleformat*{\subsection}{\Large \bfseries}
\titleformat*{\subsubsection}{\Large \bfseries}
% \titleformat*{\paragraph}{\large \bfseries}
\titleformat*{\subparagraph}{\large \bfseries}



% General 
\newcommand{\nextline}{\hfill\break}
\newcommand{\nl}{\nextline\rm}
% \newcommand{\placeholder}{{\bf\color{red} NOOOOOOT COMPLEEEEEEET! COOOOOOOM BAAAAAACK!!!}}
\newcommand{\placeholder}{\todo{NOOOOOOT COMPLEEEEEEET! COOOOOOOM BAAAAAACK!!!}}

\newcommand{\defeq}{\stackrel{def.}{=}}
% FA and LA
% inner product: \inne{a}{b}
\newcommand{\inne}[2]{\left<{#1},{#2}\right>}

% norm: \norm{a}
\newcommand{\norm}[1]{\left\|{#1}\right\|}

% Curly H
\newcommand{\hbs}{$\mathscr{H}$ }
\newcommand{\hbp}{\mathscr{H}}

% Dual : \dual{x}
\newcommand{\dual}[1]{{#1}^*}

% Sequence from 1  to infty: \sequ{x_n}
\newcommand{\sequ}[1]{\left({#1}\right)_1^\infty}

% f: A-> B \func{f}{A}{B}
\newcommand{\func}[3]{${#1}:{#2}\xrightarrow{}{#3}$}

% interior
\newcommand{\interior}{\textrm{int}}

% Bounded linear funcs
\newcommand{\blf}[2]{\mathcal{L}({#1},{#2})}

\newcommand{\prf}{\textit{proof}:   }



% Fields 
\newcommand{\real}{\mathbb{R}}
\newcommand{\comp}{\mathbb{C}}
\newcommand{\inte}{\mathbb{Z}}
\newcommand{\natu}{\mathbb{N}}





% Theorems
% \newtheorem{example}{Example}[subsection]
% \newtheorem{definition}[example]{Definition}
% \newtheorem{proposition}[example]{Proposition}
% \newtheorem{remark}[example]{Remark}
% \newtheorem{theorem}[example]{Theorem}
% \newtheorem{lemma}[example]{Lemma}
% \newtheorem{corollary}[example]{Corollary}


% for numbering the theorems            
\theoremstyle{plain}
%%%%%%%%%%%%%%%%%%%%%%%%%%%%%%%%%%%%%%%%%%%%%%%%%%%%%%%%%%%
\newtheorem{theorem}{Theorem}[section]
\newtheorem{lemma}[theorem]{Lemma}
\newtheorem{corollary}[theorem]{Corollary}
\newtheorem{proposition}[theorem]{Proposition}
%%%%%%%%%%%%%%%%%%%%%%%%%%%%%%%%%%%%%%%%%%%%%%%%%%%%%%%%%%%
% the following are not in italics
\theoremstyle{definition}
\newtheorem{definition}[theorem]{Definition}
\newtheorem{example}[theorem]{Example}
\newtheorem{remark}[theorem]{Remark}
\newtheorem{claim}[theorem]{Claim}
%%%%%%%%%%%%%%%%%%%%%%%%%%%%%%%%%%%%%%%%%%%%%%%%%%%%%%%%%%%%

% proof box
\newtcbtheorem[no counter]{pf}{Proof}{
  enhanced,
  rounded corners,
  attach boxed title to top,
  colback=white,
  colframe=black!25,
  fonttitle=\bfseries,
  coltitle=black,
  boxed title style={
    rounded corners,
    size=small,
    colback=black!25,
    colframe=black!25,
  } 
}{prf}

% extra content box to put in contents not covered in the lecture notes
% use the command \begin{unexaminable}
\newmdenv[skipabove=7pt, skipbelow=7pt,
    rightline=false, leftline=false, topline=false, bottomline=false,
    backgroundcolor = gray!10,
    innerleftmargin=1in, innerrightmargin=1in, innertopmargin=5pt,
    leftmargin=-1in, rightmargin=-1in, linewidth=4pt,
    innerbottommargin=5pt]{unexamBox}
\newenvironment{unexaminable}{\begin{unexamBox}}{\end{unexamBox}}


% clever ref settings
\crefname{lemma}{lemma}{lemmas}
\Crefname{lemma}{Lemma}{Lemmas}
\crefname{theorem}{theorem}{theorems}
\Crefname{theorem}{Theorem}{Theorems}

% formatting 
% https://tex.stackexchange.com/questions/217497/aligning-stackrel-signs-beneath-each-other-using-split
\newlength{\leftstackrelawd}
\newlength{\leftstackrelbwd}
\def\leftstackrel#1#2{\settowidth{\leftstackrelawd}%
{${{}^{#1}}$}\settowidth{\leftstackrelbwd}{$#2$}%
\addtolength{\leftstackrelawd}{-\leftstackrelbwd}%
\leavevmode\ifthenelse{\lengthtest{\leftstackrelawd>0pt}}%
{\kern-.5\leftstackrelawd}{}\mathrel{\mathop{#2}\limits^{#1}}}

\doublespacing
\RaggedRight
% modify innerrightmargin if floats were lost

\usepackage{amsfonts, amsmath, amssymb, amsthm, thmtools, bm}
\usepackage{avant} % Use the Avantgarde font for headings

% Boxed/framed environments
\newtheoremstyle{royalnumbox}%
{0pt}% Space above
{0pt}% Space below
{\normalfont}% Body font
{}% Indent amount
{\small\bf\sffamily\color{royal}}% Theorem head font
{\;}% Punctuation after theorem head
{0.25em}% Space after theorem head
{\sffamily \color{royal} 
    \thmname{#1} 
    \thmnumber{#2} \thmnote{\bfseries\color{black}---\nobreakspace#3.}} % Optional theorem note
\renewcommand{\qedsymbol}{$\blacksquare$}% Optional qed square

\newtheoremstyle{blacknumex}% Theorem style name
{5pt}% Space above
{5pt}% Space below
{\normalfont}% Body font
{} % Indent amount
{\small\bf\sffamily}% Theorem head font
{\;}% Punctuation after theorem head
{0.25em}% Space after theorem head
{\sffamily
    \thmname{#1}
    \thmnumber{#2}
    \thmnote{---\nobreakspace#3.}}% Optional theorem note

\newtheorem*{notation}{Notation}
\newtheorem*{hint}{Hint}
\newtheorem*{solution}{Solution}

\newcounter{dummy} 
\numberwithin{dummy}{section}

\theoremstyle{royalnumbox}
\newtheorem{definitionT}[dummy]{Definition}
\newtheorem{theoremT}[dummy]{Theorem}
\newtheorem{lemmaT}[dummy]{Lemma}
\newtheorem{corollaryT}[dummy]{Corollary}
\newtheorem{propositionT}[dummy]{Proposition}
\newtheorem{propertyT}[dummy]{Property}
\newtheorem{remarkT}[dummy]{Remark}

\theoremstyle{blacknumex}
\newtheorem{exampleT}[dummy]{Example}
\newtheorem{exerciseT}[dummy]{Exercise}

\numberwithin{equation}{section}

\RequirePackage[framemethod=default]{mdframed}

% Definition box
\newmdenv[skipabove=7pt, skipbelow=7pt,
rightline=false, leftline=true, topline=false, bottomline=false,
backgroundcolor = reddish!10, 
linecolor=reddish,
innerleftmargin=5pt, innerrightmargin=30pt, innertopmargin=5pt,
leftmargin=0cm, rightmargin=0cm, linewidth=4pt,
innerbottommargin=5pt]{dBox}

% Main Theorem box
\newmdenv[skipabove=7pt, skipbelow=7pt,
rightline=false, leftline=true, topline=false, bottomline=false,
backgroundcolor=c0!10, 
linecolor=c0,
innerleftmargin=5pt, innerrightmargin=30pt, innertopmargin=5pt,
leftmargin=0cm, rightmargin=0cm, linewidth=4pt, innerbottommargin=5pt]{tBox}

% Lemma/Corollary/Proposition/Property box
\newmdenv[skipabove=7pt, skipbelow=7pt,
rightline=false, leftline=true, topline=false, bottomline=false,
backgroundcolor = c0!10, 
linecolor=c0!80,
innerleftmargin=5pt, innerrightmargin=30pt, innertopmargin=5pt,
leftmargin=0cm, rightmargin=0cm, linewidth=4pt,
innerbottommargin=5pt]{lBox}

% Example/Remark/Exercise box
\newmdenv[skipabove=7pt, skipbelow=7pt,
rightline=false, leftline=true, topline=false, bottomline=false,
backgroundcolor = mossgreen!10!white,
linecolor = mossgreen,
innerleftmargin=5pt, innerrightmargin=30pt, innertopmargin=5pt,
leftmargin=0cm, rightmargin=0cm, linewidth=4pt,
innerbottommargin=5pt]{exBox}

% Proof box
% \newmdenv[skipabove=7pt, skipbelow=7pt,
% rightline=false, leftline=true, topline=false, bottomline=false,
% linecolor=gray,
% innerleftmargin=5pt, innerrightmargin=30pt, innertopmargin=5pt,
% leftmargin=0cm, rightmargin=0cm, linewidth=4pt,
% innerbottommargin=5pt]{proofBox}



% Creates an environment for each type of theorem and assigns it a theorem text style from the "Theorem Styles" section above and a colored box from above
\newenvironment{definition}{\begin{dBox}\begin{definitionT}}{\end{definitionT}\end{dBox}}
\newenvironment{theorem}{\begin{tBox}\begin{theoremT}}{\end{theoremT}\end{tBox}}
\newenvironment{lemma}{\begin{lBox}\begin{lemmaT}}{\end{lemmaT}\end{lBox}}
\newenvironment{proposition}{\begin{lBox}\begin{propositionT}}{\end{propositionT}\end{lBox}}
\newenvironment{corollary}{\begin{lBox}\begin{corollaryT}}{\end{corollaryT}\end{lBox}}
\newenvironment{property}{\begin{lBox}\begin{propertyT}}{\end{propertyT}\end{lBox}}


% \newenvironment{proof*}{\begin{proofBox}\begin{proof}}{\end{proof}\end{proofBox}}
\newenvironment{exercise}{\begin{exBox}\begin{exerciseT}}{\hfill{\color{royal}}\end{exerciseT}\end{exBox}}
\newenvironment{remark}{\begin{exBox}\begin{remarkT}}{\end{remarkT}\end{exBox}}
\newenvironment{example}{\begin{exBox}\begin{exampleT}}{{}\end{exampleT}\end{exBox}}

\title{Week 2}
\author{\aut}
\begin{document}
\maketitle

\section{Finite vs. Infinite Dimensional Spaces}

In this section, let $X$ be a linear space.  

\begin{definition}[Equivalence of Norm]\label{equivalent norms}\rm\nextline
	Let $X$ be a linear space. Two norms $\norm{\cdot}_a$, $\norm{\cdot}_b$ on $X$ are {\bf equivalent} if $\exists C \in [1, \infty)$, such that:
 $$
    \forall x \in X: \frac{1}{C}\norm{x}_1 \leq \norm{x}_2 \leq C \norm{x}_1
 $$
\end{definition}

\begin{proposition}[Norms are equivalent in finite dim]\nextline
\label{Norms are equivalent in finite dim}
    If $\dim X < \infty$ (\textit{i.e.} $X \cong \mathbb{K}^n$ for some $n \geq 1$), any two norms on $X$ are equivalent.
\end{proposition}
\begin{proof}
    cf. Page 44 Notes, Theorem 9. 
\end{proof}

However, this is not true in infinite dimensional spaces.  

\begin{example}
    Take $X = C[0,1]$, with $\norm{\cdot}_1$ and $\norm{\cdot}_{\infty}$
    $$
    f_n(t) = t^n, \qquad n \geq 1, \ t\in [0,1]
    $$
    then  
    $$
    \norm{f_n}_1 = \int_0^1 t^n dt = \frac{1}{n+1} \overset{n \to \infty}{\longrightarrow} 0
    $$
    but $\norm{f_n}_{\infty}=1$.
\end{example}

\begin{proposition}
\label{finite dim is complete}
    If $(X, \norm{\cdot})$ is normed, $Y \subset X$ a finite dimensional subspace $\dim Y < \infty$, then $(Y, \norm{\cdot})$ is complete.
\end{proposition}  
\begin{proof}
    cf. Page 45 Notes, Theorem 10.
\end{proof}

\begin{remark}
In particular, if $\dim X < \infty$, then one can choose $Y=X$. 
\end{remark}  

\begin{example}
    \Cref{finite dim is complete} fails if $\dim Y =\infty$. Consider $C[0,2]=Y \subset X = L^1[0,2]$, $(f_n) \subset Y$ with  
    $$
    f_n(t) = \begin{cases}
        t^n, & 0\leq t <1 \\
        1, & t\geq 1
    \end{cases}
    $$  
    (\textit{i.e.} view $(Y, \norm{\cdot}_1)$ as a subspace of $(X, \norm{\cdot}_1)$). 
    
    Then $(f_n)$ is Cauchy in $L^1$ with $f_n \overset{L^1}{\longrightarrow} f$ where  

    $$
    f(t) = \begin{cases}
        0, & t<1 \\
        1, & t \geq 1
    \end{cases}
    $$  
    but clearly $y \notin Y$. So $(Y, \norm{\cdot}_1)$ is not complete. 
\end{example}

As a consequence of \Cref{finite dim is complete}, one also gets  

\begin{corollary}
\label{finite dim is closed}
    If $(X, \norm{\cdot})$ is normed, $Y \subset X$ a finite dimensional subspace $\dim Y < \infty$, then $(Y, \norm{\cdot})$ is closed.
\end{corollary}
\begin{proof}
    Let $(x_n)\subset Y$ be convergent, $\norm{x_n-x}\xrightarrow{n\to\infty}0$ for some $x\in X$. Need to show $x\in Y$.\\
    Since $(x_n)$ is convergent, it is Cauchy, but $x_n\in Y$ $\forall  n$, so $(x_n)$ is Cauchy in $Y$. By proposition 1.4. $(Y,\norm{\cdot})$ is complete, hence $(y_n)$ converges to a point $y\in Y$, thus $Y$ is closed.
\end{proof}

\subsection{Compactness}  

In the following, we will let $(X, \rho)$ be a metric 
% linear   // Metric space suffices here 
space.  

\begin{definition}[Compact] \nextline
    A set $K \subset X$ is \textbf{(sequentially) compact} 
    if every sequence in $(x_n) \subset K$ has a convergent subsequence with limit in $K$.
\end{definition}    

\begin{definition}[Closed]\nextline
    A set $K \subset X$ is \textbf{closed} if for $(x_n) \subset K$, $x_n \overset{\rho}{\longrightarrow} x \in X$.
\end{definition}


\begin{remark}
    From Year 2 Analysis, if $\dim X < \infty$,  
    \begin{center}
        K compact $\iff$ K closed and bounded
    \end{center}
    where $K$ is bounded if there exists $R>0$, $\forall x,y \in K$, $\rho(x,y) \leq R$ ($\rho(\cdot, \cdot)$ is the metric on $K$). 
\end{remark}  

\begin{remark}
    "$\implies$" remains true even if $\dim X = \infty$: let $K$ be a compact set.
\end{remark}

\begin{proposition}
\label{comapct implies ...}
    K compact $\implies$ K closed and bounded
\end{proposition}
\begin{proof}
Let $K\subset X$ be compact.  

\underline{\textbf{Closed:}}\nextline  
Let $(x_n) \subset K$, $x_n \overset{\rho}{\longrightarrow} x \in X$.  

We would like to show $x \in K$. By compactness, $\exists (x_{n_k})_k \subset K$ with $\rho(x_{n_k}, \tilde{x}) \overset{k\to \infty}{\longrightarrow} 0$ for some $\tilde{x} \in K$. But we know $\rho(x_{n_k}, x)\overset{k\to \infty}{\longrightarrow} 0$ [subsequence of a convergent sequence has the same limit]. So $\Tilde{x}=x$. 

\underline{\textbf{Bounded:}}\nextline
Assume that $K$ is not bounded. Fix $x_0 \in K$. By definition, $\forall n \geq 1$, we can find $x_k$ s.t. 

$$
\rho(x_k, x_0) \geq n \qquad
$$  
By compactness, $(x_n)$ has a convergent subsequence $(x_k)$ with limit $x \in K$. But then 
$$
\rho(x_{n_k}, x_0) \leq \underbrace{\rho(x_{n_k}, x)}_{\overset{k \to \infty}{\longrightarrow 0}} + \rho(x, x_0) \leq C
$$
for some $C>0$, violating (*) for large $k$.
\end{proof}

The "$\impliedby$" breaks down if $\dim X = \infty$.

\begin{example}
    Take the following set in $\ell^1$:  
    \begin{equation*}
        K = \left\{ e_n = (0, \ldots, 0, 1, 0, \ldots), n \in \mathbb{N} \right\}
    \end{equation*}  
    Then $\norm{e_n}_{\ell^1}=1$, so $K$ is closed and bounded($\norm{e_n-e_m} = 2 \times \mathbf{1}_{n\neq m}$, so any convergent sequence $(x_n) \subset K$ is eventually constant \textit{i.e. it equals $e_k$ for some $k$}). But the bounded sequence $(x_n) := (e_n)$ has no convergent subsequences, since no subsequence is Cauchy. So $K$ is \textbf{not} compact.
\end{example}  

Another (illustrative) example is the following  

\begin{example}
    Consider the set $\overline{B_1} \subset C[0,1]$,  
    \begin{equation*}
        \overline{B_1} = \left\{ f \in C[0,1] : ||f||_{\infty} \leq 1 \right\}
    \end{equation*}  
    Then $\overline{B_1}$ is closed and bounded, but \textbf{not} compact. To see this,
    consider,  
    \begin{equation*}
        f_n(t) = \sin(2^n \pi t), \quad 0 \leq t \leq 1
    \end{equation*}  
    Then $||f_n-f_m||_{\infty}\geq 1$ for all $n \neq m$, so it has no convergent subsequences. 
\end{example}

Compactness of the unit ball characterises in fact finite dimensional spaces.  

\begin{theorem}[Characterisation of finite dim. spaces]\nextline
    In a normed space $(X, \norm{\cdot})$, the following statements are equivalent:  
    \begin{enumerate}[i)]
        \item $\dim X < \infty$
        \item The unit ball $\overline{B_1}$ is compact 
    \end{enumerate}
\end{theorem}  

\begin{proof}
    i) $\implies$ ii) $\overline{B_1}$ is closed and bounded, which implies compactness since $X$ has finite dimension.
    For ii) $\implies$ i), one uses  
\begin{lemma}[Riesz]\nextline
\label{riesz unit ball lemma}
    Let $Y \subset X$, $Y\neq X$ be a \textit{proper} closed subspace of $(X, \norm{\cdot})$. Then for all $\varepsilon \in (0,1)$, $\exists x \in X\setminus Y$ such that,  
    \begin{enumerate}[i)]
        \item $\norm{x}=1$
        \item $d(x, Y) \overset{\text{def.}}{=} \inf_{y\in Y} \norm{x-y} > 1- \varepsilon$
    \end{enumerate}
\end{lemma}  
\begin{proof}
    Pick any $x^* \in X\setminus Y$. Since $Y$ is closed, $d:=d(x^*, Y)>0$.  
    By the definition of $d(x,Y)$, we can thus find $y^* \in Y$ s.t.  
    $$
    d \leq \norm{x^*-y^*} < \frac{d}{1-\varepsilon}
    $$  
    Set $x = \frac{x^*-y^*}{\norm{x^*-y^*}}$, then i) is satisfied and for all $y\in Y$ one has  
    $$
    \norm{x-y} = \norm{\frac{x^*-(y^*+\norm{x^*-y^*}y)}{\norm{x^*-y^*}}}=
    %\frac{\norm{x^*-\overbrace{(y^*+\norm{x^*-y^*}y)}^{\in Y}}}{\norm{x^*-y^*}} 
    \geq \frac{d}{\norm{x^*-y^*}} > 1-\varepsilon
    $$
\end{proof}

Returning to  ii) $\implies$ i). We show the contrapositive. 

Assume $\dim X=\infty$, let $(y_n)$ be a sequence of linearly independent vectors, define

$$Y_n = \text{span} \{y_k: 1\leq k\leq n\} $$

note $\dim Y_n < \infty$, so by \Cref{finite dim is closed}, $Y_n$ is closed.  

Pick $x_1 = \frac{y_1}{\norm{y_1}}$ and for all $n \geq 2$ using \Cref{riesz unit ball lemma} with $X=Y_n$, $Y=Y_{n-1}$, and $\varepsilon=\frac{1}{2}$, we can choose $x_n \in Y_n \setminus Y_{n-1}$ with 
$$\norm{x_n}=1 \qquad \text{and} \qquad d(x_n, Y_{n-1})>\frac{1}{2}$$

Then $\forall m > n$, one has:  
$$
\norm{x_m-x_n} \geq d(x_m, Y_n) \overset{Y_n \subset Y_m}{\geq} d(x_m, Y_{m-1}) > \frac{1}{2}
$$  
So $(x_n)$ has no convergent subsequences. Clearly $(x_n) \subset \overline{B_1} $. So $\overline{B_1} $ is not compact.  
\end{proof}

\section{Linear Operators}  
Let $(X, \norm{\cdot}_X)$, $(Y, \norm{\cdot}_Y)$ be normed spaces, $A: X \to Y$ linear.  

\begin{definition}[Bounded Operator]\nl
\label{bounded operator def}
    A linear operator $A: (X, \norm{\cdot}_X) \to (Y, \norm{\cdot}_Y)$ is bounded if $\exists C \in (0, \infty)$, such that  
    $$
    \norm{Ax}_Y \leq C \norm{x}_X \qquad \forall x \in X
    $$
\end{definition}

If $A$ is bounded, 
$$
\norm{A} \defeq \sup_{\norm{x}_X\leq 1} \norm{Ax}_Y (<\infty)
$$
is the best possible $C$.  

$\norm{A}$ is called \textbf{operator norm} (it is a norm on $\mathcal{L}(X,Y)$).  

Boundedness is the same as continuity.  

\begin{theorem}
\label{equivalent defn of bounded operators}
    The following are equivalent:  
    \begin{enumerate}[i)]
        \item $A$ is continuous at $x_0\in X$
        \item $A$ is continuous at every $x \in X$
        \item $A$ is Lipschitz continuous ($\exists L > 0: \norm{Ax-Ay}_Y \leq L \norm{x-y}_X, \forall x,y\in X$)
        \item $A$ is bounded
    \end{enumerate}
\end{theorem}
\begin{proof}
    \textbf{iv) $\implies$ iii)}\nl
    By linearity of $A$, $\forall x_1\neq x_2 \in X$,  
    \begin{align*}
        \norm{Ax_1 - Ax_2}_Y &= \norm{A(x_1-x_2)}_Y = \norm{x_1-x_2}_X \norm{A \left(\frac{x_1-x_2}{\norm{x_1-x_2}_X}\right)}_Y \\
        &\leq \norm{A} \norm{x_1-x_2}_X \qquad \text{so\ can\ take\ } L = \norm{A}
    \end{align*}
    \textbf{iii) $\implies$ ii) $\implies$ i)} is clear.  
    To show \textbf{i) $\implies$ iv)}:\nl
    Assume $\norm{A}=\infty$, so can find $(x_n) \subset X$ with  
    $$
    \norm{x_n}_X \leq 1 \qquad 0 < \norm{Ax_n}_Y \to \infty \text{\ as\ } n\to \infty
    $$
    Set $z_n \defeq \frac{x_n}{\norm{Ax_n}_Y}$,  then $\norm{z_n}_X \to 0$ as $n\to \infty$ but 
    $$
    \norm{A(x_0+z_n)-Ax_0}_Y = \norm{Az_n}_Y = 1
    $$
    which does not converge to $0$ as $n\to \infty$.
\end{proof}

\begin{corollary}
    If $\dim X < \infty$, $A: X\to Y$ linear. Then $A$ is continuous.
\end{corollary}  
\begin{proof}
    $\norm{x}_* \defeq \norm{x}_X + \norm{Ax}_Y$ defines a norm on $X$ (check this).  
    
    By \Cref{Norms are equivalent in finite dim}, $\exists C>0$ s.t. $\norm{x}_* \leq C \norm{x}_X, \forall x \in X$. But since $\norm{Ax}_Y \leq \norm{x}_X$, $A$ is bounded, hence continuous by \Cref{equivalent defn of bounded operators}.
\end{proof}

\begin{example}
    Let $X=Y=C[0,1]$, $\norm{\cdot}_X=\norm{\cdot}_1, \norm{\cdot}_Y=\norm{\cdot}_{\infty}$ and $A=id$, then $A$ is not continuous: we show it is not bounded.  

    Take 
    \begin{equation}\nonumber
f_n(x)=\left\{
\begin{aligned}
    &{n^2x} &x\in[0,\frac{1}{n}]\\
    &{-n^2x+2n} &x\in(\frac{1}{n},\frac{2}{n}]\\
    &0 &x\in(\frac{2}{n},1]\\
\end{aligned}
\right.\quad\text{then}\quad\norm{f_n}_1=1  \  \ f_n \in X, \forall n \in \natu
\end{equation}
But we have  
$$
\norm{A} \underbrace{\geq}_{\norm{f_n}_1=1} \sup_n \norm{Af_n}_Y = \sup_n \underbrace{\norm{f_n}_{\infty}}_n = \infty
$$
In fact, unboundedness is rather common, so care is needed!  
\end{example}  

A classical example is:  

\begin{example}
    Let $X=C^1[0,1]$, $Y=C[0,1]$, 
    \begin{align*}
        A: X &\to Y \\
        f &\mapsto f'
    \end{align*}
    by $"A = \frac{d}{dx}"$, which is well-defined (\textit{i.e.} if $f\in X$ then $Af \in Y$)

    Take $\norm{\cdot}_Y = \norm{\cdot}_{\infty}$ and endow $X$ with $\norm{\cdot}_X = \norm{\cdot}_Y$. Then $A$ is unbounded. Indeed, take  
    $$
    f_n(t) = \sin(nt) 
    $$
    (or $f_n(t)=t^n$) then $\norm{f_n}_X=1$ but $\norm{Af_n}_Y=n \to \infty$.

    \textbf{Note:} if instead one sets $\norm{\cdot}_X = \norm{\cdot}_{C^1}=\norm{f}_{\infty}+\norm{f'}_{\infty}$ then the above $f_n$'s are of no use (and in fact $A$ is {\bf{bounded}}.(see \cref{bounded operator def})
\end{example}  

Now set 

$$
\mathcal{L}(X, Y) = \left\{A: X \to Y: A \text{\ linear\ +\ continuous} \right\}
$$
(really we are setting $\mathcal{L}((X, \norm{\cdot}_X), (Y, \norm{\cdot}_Y))$)  
is a normed linear space with norm (check!)  
$$
\norm{A}_{\mathcal{L}(X, Y)} = \norm{A} = \sup_{\norm{x}_X\leq 1} \norm{Ax}_Y = \sup_{x \neq 0} \frac{\norm{Ax}_Y}{\norm{x}_X}
$$ 
and one has the useful inequality:  
$$
\forall x \in X: \qquad \norm{Ax}_Y \leq \norm{A} \norm{x}_X
$$
If $X=Y$ \textbf{and} $\norm{\cdot}_X = \norm{\cdot}_Y$ one sets $\mathcal{L}(X, X)=X$.  

\begin{theorem}
    If $(Y, \norm{\cdot}_Y)$ is Banach, then so is $(\mathcal{L}(X, Y), \norm{\cdot}_{\mathcal{L}(X, Y)})$.  
\end{theorem}
\begin{proof}
    Notes Page 53, Theorem 17.
\end{proof}

\begin{corollary}
    If $A:X \to Y$ is continuous and $K\subset X$ is compact, then  
    $$
    A(K) = \{Ax: x\in K\} \subset Y
    $$
    is compact
\end{corollary}
\begin{proof}
    Fix $(y_n) \subset A(K)$, we need to find a convergent subsequence $(y_{n_k})$.  

    By definition of $A(K)$,  
    $$
    y_n = Ax_n \qquad \text{\ for\ some\ }x_n \in K
    $$  
    So $(x_n) \subset K$, has a convergent subsequence $(x_{n_k})$ by compactness of $K$.  

    \textbf{Claim:} $(y_{n_k})(=Ax_{n_k})$ is convergent.  

    Indeed, let $\norm{x_{n_k}-x}_X \to 0$, $k\to \infty$. By continuity:  
    $$
    \norm{y_{n_k}-Ax}_Y = \norm{Ax_{n_k}-Ax}_Y \leq L \norm{x_{n_k}-x}_X \to 0 \text{\ as \ } k \to \infty
    $$
    so $(y_{n_k}) \subset Y$ converges and the limit is $Ax$.
\end{proof}

\end{document}