\documentclass{article}
\usepackage[utf8]{inputenc}
\input{commands}
\input{theoremstyle.tex}

\title{Week 2}

\begin{document}
\maketitle

\section{Finite vs. Infinite Dimensional Spaces}

In this section, let $X$ be a linear space.  

\begin{definition}[Equivalence of Norm]\label{equivalent norms}\rm\nextline
	Let $X$ be a linear space. Two norms $\norm{\cdot}_a$, $\norm{\cdot}_b$ on $X$ are {\bf equivalent} if $\exists C \in [1, \infty)$, such that:
 $$
    \forall x \in X: \frac{1}{C}\norm{x}_1 \leq \norm{x}_2 \leq C \norm{x}_1
 $$
\end{definition}

\begin{proposition}
    If $\dim X < \infty$ (\textit{i.e.} $X \cong \mathbb{K}^n$ for some $n \geq 1$), any two norms on $X$ are equivalent.
\end{proposition}
\begin{proof}
    cf. Page 44 Notes, Theorem 9.
\end{proof}

However, this is not true in infinite dimensional spaces.  

\begin{example}
    Take $X = C[0,1]$, with $\norm{\cdot}_1$ and $||\cdot||_{\infty}$
    $$
    f_n(t) = t^n, \qquad n \geq 1, t\in [0,1]
    $$
    then  
    $$
    \norm{f_n}_1 = \int_0^1 t^n dt = \frac{1}{n+1} \overset{n \to \infty}{\longrightarrow} 0
    $$
    but $\norm{f_n}_{\infty}=1$.
\end{example}

\begin{proposition}
\label{finite dim is complete}
    If $(X, \norm{\cdot})$ is normed, $Y \subset X$ a finite dimensional subspace $d\dim Y < \infty$, then $(Y, \norm{\cdot})$ is complete.
\end{proposition}  
\begin{proof}
    cf. Page 45 Notes, Theorem 10.
\end{proof}

\begin{remark}
In particular, if $\dim X < \infty$, then one can choose $Y=X$. 
\end{remark}  

\begin{example}
    \Cref{finite dim is complete} fails if $\dim Y =\infty$. Consider $C[0,2]=Y \subset X = L^1[0,2]$, $(f_n) \subset Y$ with  
    $$
    f_n(t) = \begin{cases}
        t^n, & 0\leq t <1 \\
        1, & t\geq 1
    \end{cases}
    $$  
    (\textit{i.e.} view $(Y, \norm{\cdot}_1)$ as a subspace of $(X, \norm{\cdot}_1)$). 
    
    Then $(f_n)$ is Cauchy in $L^1$ with $f_n \overset{L^1}{\longrightarrow} f$ where  

    $$
    f(t) = \begin{cases}
        0, & t<1 \\
        1, & t \geq 1
    \end{cases}
    $$  
    but clearly $y \notin Y$. So $(Y, \norm{\cdot}_1)$ is not complete. 
\end{example}

As a consequence of \Cref{finite dim is complete}, one also gets  

\begin{corollary}
    If $(X, \norm{\cdot})$ is normed, $Y \subset X$ a finite dimensional subspace $d\dim Y < \infty$, then $(Y, \norm{\cdot})$ is closed.
\end{corollary}
\begin{proof}
    
\end{proof}

\subsection{Compactness}  

In the following, we will let $(X, \rho)$ be a metric linear space.  

\begin{definition}[Compact] \nextline
    A set $K \subset X$ is \textbf{(sequentially) compact} 
    if every sequence in $(x_n) \subset K$ has a convergent subsequence with limit in $K$.
\end{definition}    

\begin{definition}[Closed]\nextline
    A set $K \subset X$ is \textbf{closed} if for $(x_n) \subset K$, $x_n \overset{\rho}{\longrightarrow} x \in X$.
\end{definition}


\begin{remark}
    From Year 2 Analysis, if $\dim X < \infty$,  
    \begin{center}
        K compact $\iff$ K closed and bounded
    \end{center}
    where $K$ is bounded if there exists $R>0$, $\forall x,y \in K$, $\rho(x,y) \leq R$ ($\rho(\cdot, \cdot)$ is the metric on $K$). 
\end{remark}  

\begin{remark}
    "$\implies$" remains true even if $\dim X = \infty$: let $K$ be a compact set.
\end{remark}

\begin{proposition}
    K compact $\implies$ K closed and bounded
\end{proposition}
\begin{proof}
Let $K\subset X$ be compact.  

\textbf{Closed:}\nextline  
Let $(x_n) \subset K$, $x_n \overset{\rho}{\longrightarrow} x \in X$.  

We would like to show $x \in K$. By compactness, $\exists (x_{n_k})_k \subset K$ with $\rho(x_{n_k}, \tilde{x}) \overset{k\to \infty}{\longrightarrow} 0$ for some $\tilde{x} \in K$. But we know $\rho(x_{n_k}, x)\overset{k\to \infty}{\longrightarrow} 0$ [subsequence of a convergent sequence has the same limit]. So $\Tilde{x}=x$. 

\textbf{Bounded:}\nextline
Assume that $K$ is not bounded. Fix $x_0 \in K$. By definition, $\forall n \geq 1$, we can find $x_k$ s.t. 

$$
\rho(x_k, x_0) \geq n \qquad
$$  
By compactness, $(x_n)$ has a convergent subsequence $(x_k)$ with limit $x \in K$. But then 
$$
\rho(x_{n_k}, x_0) \leq \underbrace{\rho(x_{n_k}, x)}_{\overset{k \to \infty}{\longrightarrow 0}} + \rho(x, x_0) \leq C
$$
for some $C>0$, violating (*) for large $k$.
\end{proof}

The "$\impliedby$" breaks down if $\dim X = \infty$.

\begin{example}
    Take the following set in $l^1$:  
    \begin{equation*}
        K = \left\{ e_n = (0, \ldots, 0, 1, 0, \ldots), n \in \mathbb{N} \right\}
    \end{equation*}  
    Then $\norm{e_n}_{\ell^1}=1$, so $K$ is bounded and closed
\end{example}  

Another (illustrative) example is the following  

\begin{example}
    Consider the set $\bar{B}_1 \subset C[0,1]$,  
    \begin{equation*}
        \bar{B}_1 = \left\{ f \in C[0,1] : ||f||_{\infty} \leq 1 \right\}
    \end{equation*}  
    The sequence of functions,  
    \begin{equation*}
        f_n(t) = \sin(2^n \pi t), \quad 0 \leq t \leq 1
    \end{equation*}  
    Then $||f_n-f_m||\geq 1$ for all $n \neq m$ and hence no convergent subsequences. 
\end{example}

\begin{theorem}
    In a normed space $(X, ||\cdot||)$, the following two conditions are equivalent:  
    \begin{itemize}
        \item $\dim X < \infty$
        \item The unit ball $\bar{B}_1$ is compact 
    \end{itemize}
\end{theorem}  
  
  
Before proving this, we need a lemma that allows us to construct a sequence of vectors
that are at a fixed distance to each other, hence has no convergent subsequence.  

\begin{lemma}
    (Riesz) Let $Y \subset Z \subset X$ be a \textit{proper} closed subspace of some subspace in $(X, ||\cdot||)$. Then for any $\theta \in (0,1)$, there is a $z\in Z$ on the unit ball, $||z||=1$, such that  
    \begin{equation*}
        \forall y \in Y \qquad ||y-z|| \geq \theta
    \end{equation*}
\end{lemma}  
\begin{proof}
(Sketch)  
\todo{Add alternative proof in the hand-written lecture notes. }

\begin{itemize}
    \item Choose $v \in X \setminus Y$, let $d=\inf_{y\in Y}||v-y||>0$, as $Y$ is closed and proper
    \item Choose $y_0 \in Y$, with $||v-y_0||<d/\theta$, which is greater than $d$
\end{itemize}
Set
\begin{equation*}
    z = \frac{v-y_0}{||v-y_0||}
\end{equation*}  
Compute this and show this to be greater than $\theta$.
\end{proof}  

Now to construct a sequence of vectors inductively, we take a unit vector $x_1 \in X$ and pick  

\begin{equation*}
    ||x_2-x_1|| \geq \theta=\frac{1}{2}
\end{equation*}  

Suppose a set $\{x_1, \ldots, x_m\}$ has been constructed s.t. every pair of elements satisfy the equation above, then the span of this set is closed (finite dimensional) so we can continue the construction to obtain a sequence  

\begin{equation*}
    ||x_p-x_q|| \geq \frac{1}{2}, \qquad \forall p, q \in \N
\end{equation*}

\begin{theorem}[Equivalence of finite dimensional norms]\rm\nextline
	Let $X$ be a finite dimensional vector space, then any two norms $\norm{\cdot}_a$ and $\norm{\cdot}_b$ are equivalent.\\
	\prf The proof are divided into four steps:
	\begin{itemize}
		\item Showing that equivalence of norm is transitive
		\item Showing that equivalence of equivalence on unit sphere implies equivalence on $X$
		\item Showing that any norm is continuous with respect to $\norm{\cdot}_1$
		\item Showing that any norm is equivalent to $\norm{x}_1\equiv\sum_{s=1}^n |x_s|$
	\end{itemize}
	\begin{pf}{STEP I}{}
		Let $\norm{\cdot}_a$ be equivalent to $\norm{\cdot}_b$ and $\norm{\cdot}_b$ equivalent to $\norm{\cdot}_c$.
		Then $\exists m_1,m_2,M_1,M_2>0$ with
		\begin{equation}
			\begin{split}\nonumber
				&m_1\norm{x}_a\leq \norm{x}_b\leq M_1\norm{x}_a,\,\,\forall x\in X\\
				&m_2\norm{x}_b\leq \norm{x}_c\leq M_2\norm{x}_b,\,\,\forall x\in X
			\end{split}
		\end{equation}
		Then
		\begin{equation}
			\begin{split}\nonumber
				&m_1m_2\norm{x}_a\leq m_2\norm{x}_b\leq \norm{x}_c,\,\,\forall x\in X\\
				&\norm{x}_c\leq M_2\norm{x}_b\leq M_1M_2\norm{x}_c\,\,\forall x\in X
			\end{split}
		\end{equation}
		Which gives $m_1m_2\norm{x}_a\leq\norm{c}\leq M_1M_2\norm{x}_a$ for arbitrary $x$. Hence $\norm{\cdot}_a$ and$\norm{\cdot}_c$ are equivalent.
	\end{pf}
	\begin{pf}{STEP II}{}
		Now let us assume that $\norm{\cdot}_a$ is equivalent to $\norm{\cdot}_b$ on $U_a=\{s\in X:\norm{s}_a=1\}$.
		Then let $x\in X$ be non-zero. Then we have
		$$
			m\norm{\frac{x}{\norm{x}_a}}_a\leq \norm{\frac{x}{\norm{x}_a}}_b\leq M\norm{\frac{x}{\norm{x}_a}}_a
		$$
		So
		$$
			m\norm{x}_a\frac{1}{\norm{x}_a}\leq \norm{x}_b\frac{1}{\norm{x}_a}\leq M\norm{x}_a\frac{1}{\norm{x}_a}
		$$
		And since ${\norm{x}_a}$ is non-zero,we have that
		$$
			m{\norm{x}_a}\leq \norm{x}_b\leq M\norm{x}_a
		$$
		Now since $x$ is arbitrary, the proof is completed.	
	\end{pf}
	\begin{pf}{STEP III}{}
		Now we shall proof continuity of any norm under $\norm\cdot _1$. This can be done by showing for a sequence $(x_n)$ converging to $x$ under the metric induced by $\norm{\cdot}_1$, the norm if its terms under $\norm{\cdot}_a$ converges to $\norm{x}_a$. So let $(x_n)$ be a sequence in $X$ with $x_n\xrightarrow[]{n\to \infty}x$. We have
		\begin{equation}
			\begin{split}\nonumber
				|\norm{x_n}_a-\norm{x}_a|&\leq\norm{x_n-x}_a\leq M\norm{x_n-x}_1\to0\quad \text{when}\,\,n\to\infty
			\end{split}
		\end{equation}
		Thus
		$$
			\lim_{n\to\infty}|\norm{x_n}_a-\norm{x}_a|=0
		$$
	\end{pf}
\begin{pf}{STEP IV}{}
	Now we shall apply extreme value theorem to obtain our final result here. Using the theorem requires unit sphere to be a compact set. Proof of this fact is given later, one should realise that the proof does not depend on equivalence of norms, as we only require compactness in $X,\norm{\cdot}_1$. However, it is true that unit sphere is compact under any norm in finite dimensional cases. So we have that $U_1=\{s\in X:\norm{s}_1=1\}$ is compact with  a  function $\norm{\cdot}_a$ continuous on it, so by extreme value theorem it attains maximum $M_U=max\{\norm{x}_a:x\in U_1\}$ and minimum $m_u=min\{\norm{x}_a:x\in U_1\}$ on $U_1$, thus for any $x\in U_1$ we have
	$$
		m_u \norm{x}_1=m_u\leq\norm{x}_a\leq M_u=M_u \norm{x}_1
	$$
	Hence we show that any norm is equivalent to $\norm{\cdot}_1$ on unit sphere.
\end{pf}
By combining the results of the four steps, we finish the proof of the theorem.
\end{theorem}

\begin{remark}\rm\nextline
	The proof is not unique. We can also choose other norms to be the "bridging" norm, say supremum norm which in finite dimensional case becomes the max norm: $\norm{x}\equiv max\{|x_i|\}$. About the meaning of equivalence here, in fact, equivalent norms are equivalent in the sense that they induces same topology, so it is also called "topologically"equivalent. Generally speaking, this means that topological properties such as open, close, compact, convergence, continuity which holds for one norm will hold in its equivalent norms.
\end{remark}
Following results are simple exercises to check statements above.

\begin{proposition}[Equivalence of openness]\rm\nextline
	Open sets in $(X,\norm{\cdot}_a)$ are open in $(X,\norm{\cdot}_b)$ (Following notation in definition of equivalent norms).
	\begin{pf}{}{}
	It suffices to check open balls. Let $B^a_x(r)\equiv\{s\in X:\norm{x-s}_a<r\}$ be open balls with radius $r>0$ centered at $x$, which is an open ball in $(X,\norm{\cdot}_a)$. Choose $p\in B^a_x(r)$, we should show that $\exists \varepsilon>0$ with $B^b_p(\varepsilon)\equiv\{s\in X:\norm{p-s}_b<\varepsilon\}\subset B^a_x(r)$.\\
	By openness of $B^a_x(r)$, we have that $\exists \varepsilon_a>0$ with
	\begin{equation}
		\begin{split}\nonumber
			B^a_p(\varepsilon_a)&\equiv\{s\in X:\norm{p-s}_a<\varepsilon_a\}\\
			&=\{s\in X:m\norm{p-s}_a<m\varepsilon_a\}\\
			&\supseteq \{s\in X:\norm{p-s}_b<m\varepsilon_a\}\\
			&=B^b_p(m\varepsilon_a)
		\end{split}
	\end{equation}
	Note that $B^b_p(m\varepsilon_a)\subseteq B^a_p(\varepsilon_a)\subset B^a_x(r)$, hence $\varepsilon=m\varepsilon_a$.
	\end{pf}
	
\end{proposition}



\end{document}