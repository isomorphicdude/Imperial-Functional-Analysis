\documentclass{article}
\usepackage[utf8]{inputenc}
\input{commands}
\input{theoremstyle.tex}

\title{Week 2}

\begin{document}
\maketitle

\section{Finite vs. Infinite Dimensional Spaces}

In this section, let $X$ be a linear space.  

\begin{definition}[Equivalence of Norm]\label{equivalent norms}\rm\nextline
	Let $X$ be a linear space. Two norms $\norm{\cdot}_a$, $\norm{\cdot}_b$ on $X$ are {\bf equivalent} if $\exists C \in [1, \infty)$, such that:
 $$
    \forall x \in X: \frac{1}{C}\norm{x}_1 \leq \norm{x}_2 \leq C \norm{x}_1
 $$
\end{definition}

\begin{proposition}[Norms are equivalent in finite dim]\nextline
    If $\dim X < \infty$ (\textit{i.e.} $X \cong \mathbb{K}^n$ for some $n \geq 1$), any two norms on $X$ are equivalent.
\end{proposition}
\begin{proof}
    cf. Page 44 Notes, Theorem 9.
\end{proof}

However, this is not true in infinite dimensional spaces.  

\begin{example}
    Take $X = C[0,1]$, with $\norm{\cdot}_1$ and $||\cdot||_{\infty}$
    $$
    f_n(t) = t^n, \qquad n \geq 1, t\in [0,1]
    $$
    then  
    $$
    \norm{f_n}_1 = \int_0^1 t^n dt = \frac{1}{n+1} \overset{n \to \infty}{\longrightarrow} 0
    $$
    but $\norm{f_n}_{\infty}=1$.
\end{example}

\begin{proposition}
\label{finite dim is complete}
    If $(X, \norm{\cdot})$ is normed, $Y \subset X$ a finite dimensional subspace $d\dim Y < \infty$, then $(Y, \norm{\cdot})$ is complete.
\end{proposition}  
\begin{proof}
    cf. Page 45 Notes, Theorem 10.
\end{proof}

\begin{remark}
In particular, if $\dim X < \infty$, then one can choose $Y=X$. 
\end{remark}  

\begin{example}
    \Cref{finite dim is complete} fails if $\dim Y =\infty$. Consider $C[0,2]=Y \subset X = L^1[0,2]$, $(f_n) \subset Y$ with  
    $$
    f_n(t) = \begin{cases}
        t^n, & 0\leq t <1 \\
        1, & t\geq 1
    \end{cases}
    $$  
    (\textit{i.e.} view $(Y, \norm{\cdot}_1)$ as a subspace of $(X, \norm{\cdot}_1)$). 
    
    Then $(f_n)$ is Cauchy in $L^1$ with $f_n \overset{L^1}{\longrightarrow} f$ where  

    $$
    f(t) = \begin{cases}
        0, & t<1 \\
        1, & t \geq 1
    \end{cases}
    $$  
    but clearly $y \notin Y$. So $(Y, \norm{\cdot}_1)$ is not complete. 
\end{example}

As a consequence of \Cref{finite dim is complete}, one also gets  

\begin{corollary}
\label{finite dim is closed}
    If $(X, \norm{\cdot})$ is normed, $Y \subset X$ a finite dimensional subspace $\dim Y < \infty$, then $(Y, \norm{\cdot})$ is closed.
\end{corollary}
\begin{proof}
    
\end{proof}

\subsection{Compactness}  

In the following, we will let $(X, \rho)$ be a metric linear space.  

\begin{definition}[Compact] \nextline
    A set $K \subset X$ is \textbf{(sequentially) compact} 
    if every sequence in $(x_n) \subset K$ has a convergent subsequence with limit in $K$.
\end{definition}    

\begin{definition}[Closed]\nextline
    A set $K \subset X$ is \textbf{closed} if for $(x_n) \subset K$, $x_n \overset{\rho}{\longrightarrow} x \in X$.
\end{definition}


\begin{remark}
    From Year 2 Analysis, if $\dim X < \infty$,  
    \begin{center}
        K compact $\iff$ K closed and bounded
    \end{center}
    where $K$ is bounded if there exists $R>0$, $\forall x,y \in K$, $\rho(x,y) \leq R$ ($\rho(\cdot, \cdot)$ is the metric on $K$). 
\end{remark}  

\begin{remark}
    "$\implies$" remains true even if $\dim X = \infty$: let $K$ be a compact set.
\end{remark}

\begin{proposition}
\label{comapct implies ...}
    K compact $\implies$ K closed and bounded
\end{proposition}
\begin{proof}
Let $K\subset X$ be compact.  

\textbf{Closed:}\nextline  
Let $(x_n) \subset K$, $x_n \overset{\rho}{\longrightarrow} x \in X$.  

We would like to show $x \in K$. By compactness, $\exists (x_{n_k})_k \subset K$ with $\rho(x_{n_k}, \tilde{x}) \overset{k\to \infty}{\longrightarrow} 0$ for some $\tilde{x} \in K$. But we know $\rho(x_{n_k}, x)\overset{k\to \infty}{\longrightarrow} 0$ [subsequence of a convergent sequence has the same limit]. So $\Tilde{x}=x$. 

\textbf{Bounded:}\nextline
Assume that $K$ is not bounded. Fix $x_0 \in K$. By definition, $\forall n \geq 1$, we can find $x_k$ s.t. 

$$
\rho(x_k, x_0) \geq n \qquad
$$  
By compactness, $(x_n)$ has a convergent subsequence $(x_k)$ with limit $x \in K$. But then 
$$
\rho(x_{n_k}, x_0) \leq \underbrace{\rho(x_{n_k}, x)}_{\overset{k \to \infty}{\longrightarrow 0}} + \rho(x, x_0) \leq C
$$
for some $C>0$, violating (*) for large $k$.
\end{proof}

The "$\impliedby$" breaks down if $\dim X = \infty$.

\begin{example}
    Take the following set in $\ell^1$:  
    \begin{equation*}
        K = \left\{ e_n = (0, \ldots, 0, 1, 0, \ldots), n \in \mathbb{N} \right\}
    \end{equation*}  
    Then $\norm{e_n}_{\ell^1}=1$, so $K$ is closed and bounded($\norm{e_n-e_m} = 2 \times \mathbf{1}_{n\neq m}$, so any convergent sequence $(x_n) \subset K$ is eventually constant \textit{i.e. it equals $e_k$ for some $k$}). But the bounded sequence $(x_n) := (e_n)$ has no convergent subsequences, since no subsequence is Cauchy. So $K$ is \textbf{not} compact.
\end{example}  

Another (illustrative) example is the following  

\begin{example}
    Consider the set $\overline{B_1} \subset C[0,1]$,  
    \begin{equation*}
        \overline{B_1} = \left\{ f \in C[0,1] : ||f||_{\infty} \leq 1 \right\}
    \end{equation*}  
    Then $\overline{B_1}$ is closed and bounded, but \textbf{not} compact. To see this,
    consider,  
    \begin{equation*}
        f_n(t) = \sin(2^n \pi t), \quad 0 \leq t \leq 1
    \end{equation*}  
    Then $||f_n-f_m||_{\infty}\geq 1$ for all $n \neq m$, so it has no convergent subsequences. 
\end{example}

Compactness of the unit ball characterizes in fact finite dimensional spaces.  

\begin{theorem}[Characterization of finite dim. spaces]\nextline
    In a normed space $(X, \norm{\cdot})$, the following statements are equivalent:  
    \begin{enumerate}[i)]
        \item $\dim X < \infty$
        \item The unit ball $\overline{B_1}$ is compact 
    \end{enumerate}
\end{theorem}  

\begin{proof}
    i) $\implies$ ii) $\overline{B_1}$ is closed + bounded, so we use \Cref{comapct implies ...}  
    
    For ii) $\implies$ i), one uses  
\begin{lemma}[Riesz]\nextline
\label{riesz unit ball lemma}
    Let $Y \subset X$, $Y\neq X$ be a \textit{proper} closed subspace of $(X, \norm{\cdot})$. Then for all $\varepsilon \in (0,1)$, $\exists x \in X\setminus Y$ such that,  
    \begin{enumerate}[i)]
        \item $\norm{x}=1$
        \item $d(x, Y) \overset{\text{def.}}{=} \inf_{y\in Y} \norm{x-y} > 1- \varepsilon$
    \end{enumerate}
\end{lemma}  
\begin{proof}
    Pick any $x^* \in X\setminus Y$. Since $Y$ is closed, $d:=d(x^*, Y)>0$.  
    By the definition of $d(x,Y)$, we can thus find $y^* \in Y$ s.t.  
    $$
    d \leq \norm{x^*-y^*} < \frac{d}{1-\varepsilon}
    $$  
    Set $x = \frac{x^*-y^*}{\norm{x^*-y^*}}$, then i) is satisfied and for all $y\in Y$ one has  
    $$
    \norm{x-y} = \frac{x^*-y^*-\norm{x^*-y^*}y}{\norm{x^*-y^*}} \geq \frac{d}{\norm{x^*-y^*}} > 1-\varepsilon
    $$
\end{proof}

Returning to  ii) $\implies$ i). We show the contrapositive. 

Assume $\dim X=\infty$, let $(y_n)$ be a sequence of linearly independent vectors, define

$$Y_n = \text{span} \{y_k: 1\leq k\leq n\} $$

note $\dim Y_n < \infty$, so by \Cref{finite dim is closed}, $Y_n$ is closed.  

Pick $x_1 = \frac{y_1}{\norm{y_1}}$ and for all $n \geq 2$ using \Cref{riesz unit ball lemma} with $X=Y_n$, $Y=Y_{n-1}$, and $\varepsilon=\frac{1}{2}$, we can choose $x_n \in Y_n \setminus Y_{n-1}$ with 
$$\norm{x_n}=1 \qquad \text{and} \qquad d(x_n, Y_{n-1})>\frac{1}{2}$$

Then $\forall m > n$, one has:  
$$
\norm{x_m-x_n} \geq d(x_m, Y_n) \overset{Y_n \subset Y_m}{\geq} d(x_n, Y_{n-1}) > \frac{1}{2}
$$  
So $(x_n)$ has no convergent subsequences. Clearly $(x_n) \subset \overline{B_1} $. So $\overline{B_1} $ is not compact.  
\end{proof}

\todo{linear operators}
\end{document}