\documentclass{article}
\usepackage[utf8]{inputenc}
\usepackage{geometry}
\geometry{left=3cm,right=3cm,top=2cm,bottom=2cm}
\usepackage[utf8]{inputenc}
\usepackage{amsmath, amsfonts, amssymb, amsthm}
\usepackage[framemethod=TikZ]{mdframed}
\usepackage{mathrsfs}
\usepackage{comment}
\usepackage{enumerate}
\usepackage{xcolor}
\usepackage{titlesec}
\usepackage{setspace}
\usepackage[hidelinks,backref]{hyperref}
\usepackage{cleveref}
\usepackage[most]{tcolorbox}
\usepackage{ragged2e}
\usepackage{todonotes}
\usepackage{cleveref}
\usepackage{mathtools}

%
\titleformat*{\section}{\LARGE \bfseries}
\titleformat*{\subsection}{\Large \bfseries}
\titleformat*{\subsubsection}{\Large \bfseries}
% \titleformat*{\paragraph}{\large \bfseries}
\titleformat*{\subparagraph}{\large \bfseries}



% General 
\newcommand{\nextline}{\hfill\break}
\newcommand{\nl}{\nextline\rm}
% \newcommand{\placeholder}{{\bf\color{red} NOOOOOOT COMPLEEEEEEET! COOOOOOOM BAAAAAACK!!!}}
\newcommand{\placeholder}{\todo{NOOOOOOT COMPLEEEEEEET! COOOOOOOM BAAAAAACK!!!}}

\newcommand{\defeq}{\stackrel{def.}{=}}
% FA and LA
% inner product: \inne{a}{b}
\newcommand{\inne}[2]{\left<{#1},{#2}\right>}

% norm: \norm{a}
\newcommand{\norm}[1]{\left\|{#1}\right\|}

% Curly H
\newcommand{\hbs}{$\mathscr{H}$ }
\newcommand{\hbp}{\mathscr{H}}

% Dual : \dual{x}
\newcommand{\dual}[1]{{#1}^*}

% Sequence from 1  to infty: \sequ{x_n}
\newcommand{\sequ}[1]{\left({#1}\right)_1^\infty}

% f: A-> B \func{f}{A}{B}
\newcommand{\func}[3]{${#1}:{#2}\xrightarrow{}{#3}$}

% interior
\newcommand{\interior}{\textrm{int}}

% Bounded linear funcs
\newcommand{\blf}[2]{\mathcal{L}({#1},{#2})}

\newcommand{\prf}{\textit{proof}:   }



% Fields 
\newcommand{\real}{\mathbb{R}}
\newcommand{\comp}{\mathbb{C}}
\newcommand{\inte}{\mathbb{Z}}
\newcommand{\natu}{\mathbb{N}}





% Theorems
% \newtheorem{example}{Example}[subsection]
% \newtheorem{definition}[example]{Definition}
% \newtheorem{proposition}[example]{Proposition}
% \newtheorem{remark}[example]{Remark}
% \newtheorem{theorem}[example]{Theorem}
% \newtheorem{lemma}[example]{Lemma}
% \newtheorem{corollary}[example]{Corollary}


% for numbering the theorems            
\theoremstyle{plain}
%%%%%%%%%%%%%%%%%%%%%%%%%%%%%%%%%%%%%%%%%%%%%%%%%%%%%%%%%%%
\newtheorem{theorem}{Theorem}[section]
\newtheorem{lemma}[theorem]{Lemma}
\newtheorem{corollary}[theorem]{Corollary}
\newtheorem{proposition}[theorem]{Proposition}
%%%%%%%%%%%%%%%%%%%%%%%%%%%%%%%%%%%%%%%%%%%%%%%%%%%%%%%%%%%
% the following are not in italics
\theoremstyle{definition}
\newtheorem{definition}[theorem]{Definition}
\newtheorem{example}[theorem]{Example}
\newtheorem{remark}[theorem]{Remark}
\newtheorem{claim}[theorem]{Claim}
%%%%%%%%%%%%%%%%%%%%%%%%%%%%%%%%%%%%%%%%%%%%%%%%%%%%%%%%%%%%

% proof box
\newtcbtheorem[no counter]{pf}{Proof}{
  enhanced,
  rounded corners,
  attach boxed title to top,
  colback=white,
  colframe=black!25,
  fonttitle=\bfseries,
  coltitle=black,
  boxed title style={
    rounded corners,
    size=small,
    colback=black!25,
    colframe=black!25,
  } 
}{prf}

% extra content box to put in contents not covered in the lecture notes
% use the command \begin{unexaminable}
\newmdenv[skipabove=7pt, skipbelow=7pt,
    rightline=false, leftline=false, topline=false, bottomline=false,
    backgroundcolor = gray!10,
    innerleftmargin=1in, innerrightmargin=1in, innertopmargin=5pt,
    leftmargin=-1in, rightmargin=-1in, linewidth=4pt,
    innerbottommargin=5pt]{unexamBox}
\newenvironment{unexaminable}{\begin{unexamBox}}{\end{unexamBox}}


% clever ref settings
\crefname{lemma}{lemma}{lemmas}
\Crefname{lemma}{Lemma}{Lemmas}
\crefname{theorem}{theorem}{theorems}
\Crefname{theorem}{Theorem}{Theorems}

% formatting 
% https://tex.stackexchange.com/questions/217497/aligning-stackrel-signs-beneath-each-other-using-split
\newlength{\leftstackrelawd}
\newlength{\leftstackrelbwd}
\def\leftstackrel#1#2{\settowidth{\leftstackrelawd}%
{${{}^{#1}}$}\settowidth{\leftstackrelbwd}{$#2$}%
\addtolength{\leftstackrelawd}{-\leftstackrelbwd}%
\leavevmode\ifthenelse{\lengthtest{\leftstackrelawd>0pt}}%
{\kern-.5\leftstackrelawd}{}\mathrel{\mathop{#2}\limits^{#1}}}

\doublespacing
\RaggedRight
% modify innerrightmargin if floats were lost

\usepackage{amsfonts, amsmath, amssymb, amsthm, thmtools, bm}
\usepackage{avant} % Use the Avantgarde font for headings

% Boxed/framed environments
\newtheoremstyle{royalnumbox}%
{0pt}% Space above
{0pt}% Space below
{\normalfont}% Body font
{}% Indent amount
{\small\bf\sffamily\color{royal}}% Theorem head font
{\;}% Punctuation after theorem head
{0.25em}% Space after theorem head
{\sffamily \color{royal} 
    \thmname{#1} 
    \thmnumber{#2} \thmnote{\bfseries\color{black}---\nobreakspace#3.}} % Optional theorem note
\renewcommand{\qedsymbol}{$\blacksquare$}% Optional qed square

\newtheoremstyle{blacknumex}% Theorem style name
{5pt}% Space above
{5pt}% Space below
{\normalfont}% Body font
{} % Indent amount
{\small\bf\sffamily}% Theorem head font
{\;}% Punctuation after theorem head
{0.25em}% Space after theorem head
{\sffamily
    \thmname{#1}
    \thmnumber{#2}
    \thmnote{---\nobreakspace#3.}}% Optional theorem note

\newtheorem*{notation}{Notation}
\newtheorem*{hint}{Hint}
\newtheorem*{solution}{Solution}

\newcounter{dummy} 
\numberwithin{dummy}{section}

\theoremstyle{royalnumbox}
\newtheorem{definitionT}[dummy]{Definition}
\newtheorem{theoremT}[dummy]{Theorem}
\newtheorem{lemmaT}[dummy]{Lemma}
\newtheorem{corollaryT}[dummy]{Corollary}
\newtheorem{propositionT}[dummy]{Proposition}
\newtheorem{propertyT}[dummy]{Property}
\newtheorem{remarkT}[dummy]{Remark}

\theoremstyle{blacknumex}
\newtheorem{exampleT}[dummy]{Example}
\newtheorem{exerciseT}[dummy]{Exercise}

\numberwithin{equation}{section}

\RequirePackage[framemethod=default]{mdframed}

% Definition box
\newmdenv[skipabove=7pt, skipbelow=7pt,
rightline=false, leftline=true, topline=false, bottomline=false,
backgroundcolor = reddish!10, 
linecolor=reddish,
innerleftmargin=5pt, innerrightmargin=30pt, innertopmargin=5pt,
leftmargin=0cm, rightmargin=0cm, linewidth=4pt,
innerbottommargin=5pt]{dBox}

% Main Theorem box
\newmdenv[skipabove=7pt, skipbelow=7pt,
rightline=false, leftline=true, topline=false, bottomline=false,
backgroundcolor=c0!10, 
linecolor=c0,
innerleftmargin=5pt, innerrightmargin=30pt, innertopmargin=5pt,
leftmargin=0cm, rightmargin=0cm, linewidth=4pt, innerbottommargin=5pt]{tBox}

% Lemma/Corollary/Proposition/Property box
\newmdenv[skipabove=7pt, skipbelow=7pt,
rightline=false, leftline=true, topline=false, bottomline=false,
backgroundcolor = c0!10, 
linecolor=c0!80,
innerleftmargin=5pt, innerrightmargin=30pt, innertopmargin=5pt,
leftmargin=0cm, rightmargin=0cm, linewidth=4pt,
innerbottommargin=5pt]{lBox}

% Example/Remark/Exercise box
\newmdenv[skipabove=7pt, skipbelow=7pt,
rightline=false, leftline=true, topline=false, bottomline=false,
backgroundcolor = mossgreen!10!white,
linecolor = mossgreen,
innerleftmargin=5pt, innerrightmargin=30pt, innertopmargin=5pt,
leftmargin=0cm, rightmargin=0cm, linewidth=4pt,
innerbottommargin=5pt]{exBox}

% Proof box
% \newmdenv[skipabove=7pt, skipbelow=7pt,
% rightline=false, leftline=true, topline=false, bottomline=false,
% linecolor=gray,
% innerleftmargin=5pt, innerrightmargin=30pt, innertopmargin=5pt,
% leftmargin=0cm, rightmargin=0cm, linewidth=4pt,
% innerbottommargin=5pt]{proofBox}



% Creates an environment for each type of theorem and assigns it a theorem text style from the "Theorem Styles" section above and a colored box from above
\newenvironment{definition}{\begin{dBox}\begin{definitionT}}{\end{definitionT}\end{dBox}}
\newenvironment{theorem}{\begin{tBox}\begin{theoremT}}{\end{theoremT}\end{tBox}}
\newenvironment{lemma}{\begin{lBox}\begin{lemmaT}}{\end{lemmaT}\end{lBox}}
\newenvironment{proposition}{\begin{lBox}\begin{propositionT}}{\end{propositionT}\end{lBox}}
\newenvironment{corollary}{\begin{lBox}\begin{corollaryT}}{\end{corollaryT}\end{lBox}}
\newenvironment{property}{\begin{lBox}\begin{propertyT}}{\end{propertyT}\end{lBox}}


% \newenvironment{proof*}{\begin{proofBox}\begin{proof}}{\end{proof}\end{proofBox}}
\newenvironment{exercise}{\begin{exBox}\begin{exerciseT}}{\hfill{\color{royal}}\end{exerciseT}\end{exBox}}
\newenvironment{remark}{\begin{exBox}\begin{remarkT}}{\end{remarkT}\end{exBox}}
\newenvironment{example}{\begin{exBox}\begin{exampleT}}{{}\end{exampleT}\end{exBox}}

\title{Week 8}

\begin{document}
\maketitle
% This file is made purely according to Written note
\section{Open mapping theorem}
\begin{definition}[Open Ball]\nl
An open ball in normed linear space $X$ with radius $r>0$ centered at $x\in X$ is
$$
B_X(x,r)=\{y\in X:\norm{y-x}_X<r\}
$$
Also, when $x=0$ we write 
$$
B_X(0,r)\equiv B_x(r)
$$
\end{definition}

\begin{definition}[Open map]\nl
	Let $X$, $Y$ be linear spaces. \func{A}{X}{Y} is {\underline{open}} if $A(U)\subset Y $ is open.
\end{definition}
\begin{remark}\hfill

\begin{itemize}
    \item $A$ being continuous means $A^{-1}(V)\subset{X}$ open $\forall V\subset Y$ open.
    \item $A$ being continuous need not be open. e.g. $Ax\stackrel{def}{=}0\in Y$
\end{itemize}
\end{remark}

\begin{theorem}[Open Mapping Theorem]\nl
	Let $X,Y$ be Banach, $A\subset\mathcal{L}(X,Y)$. Then:
	\begin{itemize}
	    \item[i)] if $A$ is surjective, $A$ is open.
	    \item[ii)] if $A$ is bijective, then $A^{-1}\in \mathcal{L}(X,Y)$. (Inverse operator theorem)
	\end{itemize}
\end{theorem}

\begin{remark}\nl
ii) important in application. If $A\in\blf{X}{Y}$ is bijective then \func{A^{-1}}{X}{Y} liner is easy (why?). The point is $A^{-1}$ is also bounded, or equivalently continuous.
\end{remark}

The main step of the proof is the following:
\begin{lemma}[$A$ as in i)]\nl
$\exists r>0$ s.t. $B_Y(r)\subset \overline{A(B_x(1))}$
\begin{pf}{}{}\rm
Since $A$ is suerjective 
$$
Y=\bigcup_{k=1}^\infty A(B_X(k))
$$
Since $Y$ is complete, by  Baire Category theorem, $\exists k_0$ s.t. 
$$ int(\overline{A(B_X(1))})\neq\emptyset$$
So by surjectivity of $A$, one can find $y_0=Ax_0\in Y$, $r_0>0$ s.t. 
$$ \underbrace{B_Y(y_0,r_0)}_{=Ax_0+B_Y(r_0)}\subset \overline{A(B_X(k_0))}$$
By linearity of $A$,
\begin{equation}\nonumber
    \begin{split}
        B_Y(r_0)&\subset\overline{A(B_X(k_0))}-Ax_0 \\ &=\overline{A(B_X(k_0)-x_0)}\\
        &\subset \overline{A(B_X(k_0+M))}   \\
        &=(k_0+M)\overline{A(B_X(1))}\\
    \end{split}
\end{equation}
Where $M\stackrel{def}{=}\norm{x_0}_X$. So pick $r=\frac{r_0}{k_0+M}$.
\end{pf}

\end{lemma}
Proof of theorem:
\begin{pf}{}{}
i) Pick $r$ as in Lemma.\\
Claim: $B_Y(r/2)\subset A(B_X(1)))$.\\
If claim holds, then for $U\subset X$ open, pick $x_0\in U$, $s>0$ small so that $B_X(x_0,s)\subset U$. Letting $y_0\stackrel{def}{=}Ax_0$, get 
$$
B_Y(y_0,rs/2)=y_0+sB_Y(r/2)\stackrel{claim}{\subset}Ax_0+sA(B_X(1)\stackrel{lin.}{=}A(B_X(x_0,s))\subset A(U)
$$
which proves i). To see i) $\implies$ ii), it's enough to show that $B=A^{-1}:Y\to X$ is continuous; but for any $U\subset X$ open, $B^{-1}(U)=(A^{-1})^{-1}(U)=A(U)$
which is open by i). $\square$
\end{pf}
Proof of claim:
\begin{pf}{}{}
Fix $y\in B_Y(r/2)$. Need to show: $y=Ax$ for some $x\in X$ with $\norm{x}_X<1.$\\
We construct a sequence $(x_k)\subset X$ with 
$$
\sum_{k=1}^\infty \norm{x_k}_X<1\,\,\text{and}\,\,\sum_{k=1}^\infty Ax_k\stackrel{wrt\norm{\cdot}_Y}{\longrightarrow{}}y,\,n\to\infty
$$
By completeness of $X$, $\sum_{k=1}^\infty x_k\stackrel{def.}{=}x$ exists, $x\in B_X(1)$ and by continuity of $A$,
$$Ax=\sum_{k=1}^\infty Ax_k=y$$
By lemma above, 
$$\forall s>0, B_Y(sr)\subset \overline{A(B_X(s))}\,\,(*)$$
$s=1/2$. Pick $x_1\in B_X(1/2)$ s.t. $\norm{Ax_1-y}<r/2$. Now set $y_1=y-Ax(\in B_X(r/2)$. Iterate. Assume that for some $\geq 1$ have $x_1,......x_k,y_1,......y_k$ s.t.
$$
\forall 1\leq\tilde{k}\leq k:\,\,\norm{\tilde{x_k}}_X<2^{-k},\,y_{\tilde{k}}=y_{\tilde{k}-1}-Ax_{\tilde{k}}\in B_Y(2^{-\tilde{k}}r
$$
Then using $(*)$ with $s=2^{-(k+1)}$ find $x_{k+1}\in B_X(2^{-(k+1)})$ such that
$$
y_{k+1}\stackrel{def}{=}y_k-Ax_{k+1}\in B_Y(2^{-(k+1)}r
$$
This yields $\sum{k=1}^{\infty}\norm{x_k}_X<1$ and 
$$
y-\sum_{k=1}^n Ax_k=y_1-\sum_{k=2}^n Ax_k=...=y_n\to0\,(n\to \infty)\quad\square
$$
\end{pf}

\begin{example}[Equivalence of Norm]\nl
Let $X=Y$, with norms $\norm{\cdot}_1$ and $\norm{\cdot}_2$ and assume $\exists C>0$ s.t. $$\norm{x}_2\leq C\norm{x}_1,\,\forall x\in X\quad(1)$$
If $X$ is complete, with respect to both $\norm{\cdot}_1$ and $\norm{\cdot}_2$ then consider $A=id:(X,\norm{\cdot}_1)\to(X,\norm{\cdot}_2)$ is open by Theorem (indeed thm applies $b/c$ $A$ is bounded by $(1)$. Since $A$ is bijective, ii) gives that $A^{-1}=id:(X,\norm{\cdot}_2)\to(X,\norm{\cdot}_1)$ is bounded, i.e.
$$
\exists C': \norm{A^{-1}}_1=\norm{x}_1\leq C'\norm{x}_2
$$
so $\norm{\cdot}_1$ and $\norm{\cdot}_2$ are actually equivalent.
\end{example}

\begin{example}[Completeness of $Y$]\nl
Consider $X=C(=C^0[0,1])$ with $\norm{\cdot}_1=\norm{\cdot}_\infty$, $\norm{\cdot}_2=\norm{\cdot}_{L^1}$. Then $A=id:(X,\norm{\cdot}_1)\to(X,\norm{\cdot}_2)$ is continuous:
$$
\norm{Af}_2
=\norm{f}_2
=\int_0^1|f(t)|dt
\leq\norm{f}_\infty
=\norm{f}_1
$$
but not open. Else by 1),  $\norm{\cdot}_1$ and $\norm{\cdot}_2$ would be equivalent. However consider counter example:
\begin{equation}\nonumber
f_n(x)=\left\{
\begin{split}
    &{2n^2x} &x\in[0,\frac{1}{2n}]\\
    &{-2n^2x+2n} &x\in(\frac{1}{2n},\frac{1}{n}]\\
    &0 &x\in(\frac{1}{n},1]\\
\end{split}
\right.\quad\text{satisfy}\quad\norm{f_n}_2=1,\norm{f_n}_1=n\to\infty
\end{equation}
This shows $Y$ needs to be complete in theorem.
\end{example}
\begin{example}[Completeness of $X$]\nl
This example shows completness of $X$ is also required.
Take 
$$
X=Y=\{(x_n)\in\ell^\infty:\exists N:x_n=0\,\forall m\geq N\}\subset\ell^\infty
$$
with norm $\norm{\cdot}_X=\norm{\cdot}_Y=\norm{\cdot}_\infty$. This is a linear normed space. It's not complete (Exercise: show directly $\overline{X}=c_0$). Another way:
Define \func{A}{X}{X}, 
$$
Ax=(x_1,\frac{x_2}{2},\frac{x_3}{3}\underbrace{......}_{0\,eventually})\quad if \,\,x=(x_1,x_2......)
$$
Then $A$ is linear, bijective with 
$$
A^{-1}:X\to ,\,\,\,A^{-1}x=(x_1,2x_2,3x_3\underbrace{......}_{0\,eventually})
$$
and $A$ is bounded. 
$$
\norm{Ax}_\infty=\sup_{n\geq1}\frac{|x_n|}{n}\leq\sup_{n\geq1}|x_n|=\norm{x}_\infty
$$
so $\norm{A}\leq1$. But $A^{-1}$ is unbounded. 
Pick $x^{(n)}=(\overbrace{1,1,1,1}^{n},0,......)$ then $\norm{x^{(n)}}_\infty=1$ but $\norm{A^{-1}x^{(n)}}=n$. Hence $A^{-1}\not\in\mathcal{L}(X)$ and $X$ cannot be complete, else by theorem i), $A^{-1}$ would be bounded.
\end{example}

\section{Closed Graph Theorem}
Consider $X$, $Y$ normed spaces. Often an operator $A$ not defined on all of $A$ but on a "domain" $D(A)$. So we assume that 
$D(A)\subset X$ is a linear subspace on which $A:D(A)(\subset X)\to Y$, linear is defined.
\begin{example}{Running Example}\nl
$Y=X=C=C^0[0,1]$ with $\norm{\cdot}_X=\norm{\cdot}_\infty$ and $A=\frac{d}{dt}$, with $D(A)\stackrel{eg}{=}C^1[0,1]\subset X$ or subspaces thereof. Prime example of  (\underline{unbounded}) operator with dense domain $D(A)$: indeed $C^1[0,1]=C$ using e.g. Weierstrass Approximation Theorem (Polynomials are already $\norm{\cdot}_\infty$-dense in C)
\end{example}
\begin{definition}[Graph]\nl
Let $X$, $Y$ be normed space, $A:D(A)(\subset X)\to Y$ . Graph of $A$ (really  of $(A,D(A))$) is the linear (!) space 
$$
\Gamma_A=\{(x,Ax):x\in D(A)\}\subset X\times Y
$$
We endowed $X\times Y$ with the norm $\norm{(x,y)}_{X\times Y}=\norm{x}_X+\norm{y}_Y$, for all $x\in X$, $y\in Y$.
    
\end{definition}
\begin{definition}[Closed Operator]
$A$ is called \underline{closed} if $\Gamma_A$ is closed in  $(X\times Y,\norm{\cdot}_{X\times Y})$
\end{definition}

\begin{example}
Let $A\in\mathcal{L}(X,Y)$ with $D(A)=X$. Then $A$ is closed.
\begin{pf}{}{}
	Let $(x_k,y_k)_k\subset \Gamma_A$  with $\norm{(x_k,y_k)-(x,y)}_{X\times Y}\xrightarrow{k\to\infty}0$ for some $(x,y)\in X\times Y$\\
	NTS: $(x,y)\in \Gamma_A$ i.e. $y=Ax$. 
	Know $ y_k=Ax_k$ and $\norm{x_k-x}_X\xrightarrow{k\to\infty}0$, $\norm{Ax-y}_Y\xrightarrow{k\to\infty}0$
	But $\forall k\geq 1$
	$$\norm{y-Ax}_Y\leq\norm{y-Ax}_Y+\norm{Ax_k-ax}_Y
	\leq \norm{y-Ax}_Y\norm{A}\norm{x_k-x}_X$$
	Thus 
	$$\lim_{k\to\infty}\norm{y-Ax}_Y\leq\lim_{k\to\infty}\norm{y-Ax}_Y\norm{A}\norm{x_k-x}_X=0$$
\end{pf}
\end{example}

\begin{theorem}[Closed Graph]\nl
Let $X$, $Y$ be Banach $A:X\to Y$ linear. The following are equivalent:
\begin{itemize}
    \item [i)] $A\in\mathcal{L}(X,Y)$
    \item [ii)] $A$ is closed
\end{itemize} 
\begin{pf}{}{}
    i) $\implies$ ii): see example\\
    ii) $\implies$ i): If $X$, $Y$ complete, then so is $(X\times Y,\norm{\cdot}_{X\times Y})$ (exercise). A closed meas $\Gamma_A$ is closed in $(X\times Y,\norm{\cdot}_{X\times Y})$, so $(\Gamma_A,\norm{\cdot}_{X\times Y})$ is complete. Consider:
    \begin{equation}
        \begin{aligned}
            \Pi_X:\,\,\Gamma_A&\to X \qquad\qquad& \Pi_Y:\Gamma_A&\to Y\\
        (x,Ax)&\mapsto x  & (x,Ax)&\mapsto Ax\\
        \end{aligned}    
    \end{equation}
$\Pi_X$, $\Pi_Y$ are continuous with $\norm{\Pi_X},\norm{\Pi_Y}\leq 1$, $\Pi_X$ is injective, and surjective. By OMT, ii), ${\Pi_X}^{-1}\in\mathcal{L}{(X,\Gamma_A)}$ and so
$$
A=\Pi_Y\circ \Pi_X^{-1}\in\mathcal{L}(X,Y)
$$
\end{pf}
\end{theorem}

\begin{remark}
	ii) is simpler than i), but equivalent.\\
	i) says A is continuous, i.e. if $(x_n)\subset X$, $x\in X$
	$$\norm{x_n-x}_X\rightarrow{n\to\infty}0\implies\norm{Ax_n-Ax}_Y\rightarrow{n\to\infty}0$$
	This contains two things to check: $(Ax_n)$ converges and limit is $Ax$.\\
	ii) says $A$ is closed, i.e.
	\begin{equation}
		\left\{
		\begin{aligned}
			&\norm{x_n-x}_X\rightarrow{n\to\infty}0\\
			&\norm{Ax_n-y}_Y\rightarrow{n\to\infty}0\\
		\end{aligned}
		\right.\implies
		Ax=y
	\end{equation}
	Which is only one condition to check.
\end{remark}

\begin{example}[running example continues]
	$(D(A),\norm{\cdot}_\infty)$ with $D(A)=C^1[0,1]$ is NOT Banach, and $A:D(A)\to C$ is an example of an operator which is:\\
	claim: \\
	i) closed, but\\
	ii) not continuous\\
	For ii), take $f_n(t)=t^n\in D(A)$, $Af_n=nf_{n-1}$ so $\norm{f_n}_\infty=1$, $\norm{Af_n}\infty=n\norm{f_{n-1}}\infty=n$. So 
	$$ \sup_{f\in D(A),\norm{f}\infty\leq1}\norm{Af}_\infty=\infty$$
	For i), if $(f_n,f_n')\to(f,g)$ in $(D(A)\times C)$ then $\norm{f-f_n}_\infty\to 0$, $\norm{f_n'-g}_\infty\to0$ but
	$$
	\forall t\in(0,1],\,\underbrace{f_n(t)}_{\rightarrow{n\to\infty}f(t)}
	=\underbrace{\int_0^t f'_n(x) dx}_{\rightarrow{DCT}\int_0^t g(x) dx}
	+f_n(0)
	$$
	so $f'=g$ by fundamental theorem of calculus(FTC), i.e. $(f,g)=(f,f')\in\Gamma_A$.
\end{example}


\begin{corollary}[Continuous Inverse]\nl
	$X$, $Y$ Banach, $A:(DA)\subset X\to Y$ linear, closed and bijective. Then $\exists B=A^{-1}\in\mathcal{L}(Y,X)$ with $AB=id_Y$ and $BA=id_{D(A)}$.
	Proof is left as an exercise. Hint: similar to CGT, consider $\Pi_Y:\Gamma_A\to Y$, $B\stackrel{def.}{=}\Pi_X\circ \Pi_Y^{-1}$
	
\end{corollary}

\begin{example}[???]\nl
	A is surjective: for $g\in C$ define $f(t)=\int_0^t g(s) ds$. Then by FTC, $Af=g$.\\
	A is not injective: $Af=A\tilde{f}\implies f=\tilde{f}+c,c\in\real$. 
	Let $D(A)\defeq C_0^1[0,1]=\{f\in C^1[0,1]:f(0)=0\}$
	Then $A:D(A)\to C$ is bijective and has continuous inverse $B=A^{-1}$ by corollary. In fact, $Bf(t)=\int_0^tf(s)ds$ with $Bf\in D(A)$.
\end{example}



\end{document}

