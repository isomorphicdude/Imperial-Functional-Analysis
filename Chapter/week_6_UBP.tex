\documentclass{article}
\usepackage[utf8]{inputenc}
\usepackage{geometry}
\geometry{left=3cm,right=3cm,top=2cm,bottom=2cm}
\usepackage[utf8]{inputenc}
\usepackage{amsmath, amsfonts, amssymb, amsthm}
\usepackage[framemethod=TikZ]{mdframed}
\usepackage{mathrsfs}
\usepackage{comment}
\usepackage{enumerate}
\usepackage{xcolor}
\usepackage{titlesec}
\usepackage{setspace}
\usepackage[hidelinks,backref]{hyperref}
\usepackage{cleveref}
\usepackage[most]{tcolorbox}
\usepackage{ragged2e}
\usepackage{todonotes}
\usepackage{cleveref}
\usepackage{mathtools}

%
\titleformat*{\section}{\LARGE \bfseries}
\titleformat*{\subsection}{\Large \bfseries}
\titleformat*{\subsubsection}{\Large \bfseries}
% \titleformat*{\paragraph}{\large \bfseries}
\titleformat*{\subparagraph}{\large \bfseries}



% General 
\newcommand{\nextline}{\hfill\break}
\newcommand{\nl}{\nextline\rm}
% \newcommand{\placeholder}{{\bf\color{red} NOOOOOOT COMPLEEEEEEET! COOOOOOOM BAAAAAACK!!!}}
\newcommand{\placeholder}{\todo{NOOOOOOT COMPLEEEEEEET! COOOOOOOM BAAAAAACK!!!}}

\newcommand{\defeq}{\stackrel{def.}{=}}
% FA and LA
% inner product: \inne{a}{b}
\newcommand{\inne}[2]{\left<{#1},{#2}\right>}

% norm: \norm{a}
\newcommand{\norm}[1]{\left\|{#1}\right\|}

% Curly H
\newcommand{\hbs}{$\mathscr{H}$ }
\newcommand{\hbp}{\mathscr{H}}

% Dual : \dual{x}
\newcommand{\dual}[1]{{#1}^*}

% Sequence from 1  to infty: \sequ{x_n}
\newcommand{\sequ}[1]{\left({#1}\right)_1^\infty}

% f: A-> B \func{f}{A}{B}
\newcommand{\func}[3]{${#1}:{#2}\xrightarrow{}{#3}$}

% interior
\newcommand{\interior}{\textrm{int}}

% Bounded linear funcs
\newcommand{\blf}[2]{\mathcal{L}({#1},{#2})}

\newcommand{\prf}{\textit{proof}:   }



% Fields 
\newcommand{\real}{\mathbb{R}}
\newcommand{\comp}{\mathbb{C}}
\newcommand{\inte}{\mathbb{Z}}
\newcommand{\natu}{\mathbb{N}}





% Theorems
% \newtheorem{example}{Example}[subsection]
% \newtheorem{definition}[example]{Definition}
% \newtheorem{proposition}[example]{Proposition}
% \newtheorem{remark}[example]{Remark}
% \newtheorem{theorem}[example]{Theorem}
% \newtheorem{lemma}[example]{Lemma}
% \newtheorem{corollary}[example]{Corollary}


% for numbering the theorems            
\theoremstyle{plain}
%%%%%%%%%%%%%%%%%%%%%%%%%%%%%%%%%%%%%%%%%%%%%%%%%%%%%%%%%%%
\newtheorem{theorem}{Theorem}[section]
\newtheorem{lemma}[theorem]{Lemma}
\newtheorem{corollary}[theorem]{Corollary}
\newtheorem{proposition}[theorem]{Proposition}
%%%%%%%%%%%%%%%%%%%%%%%%%%%%%%%%%%%%%%%%%%%%%%%%%%%%%%%%%%%
% the following are not in italics
\theoremstyle{definition}
\newtheorem{definition}[theorem]{Definition}
\newtheorem{example}[theorem]{Example}
\newtheorem{remark}[theorem]{Remark}
\newtheorem{claim}[theorem]{Claim}
%%%%%%%%%%%%%%%%%%%%%%%%%%%%%%%%%%%%%%%%%%%%%%%%%%%%%%%%%%%%

% proof box
\newtcbtheorem[no counter]{pf}{Proof}{
  enhanced,
  rounded corners,
  attach boxed title to top,
  colback=white,
  colframe=black!25,
  fonttitle=\bfseries,
  coltitle=black,
  boxed title style={
    rounded corners,
    size=small,
    colback=black!25,
    colframe=black!25,
  } 
}{prf}

% extra content box to put in contents not covered in the lecture notes
% use the command \begin{unexaminable}
\newmdenv[skipabove=7pt, skipbelow=7pt,
    rightline=false, leftline=false, topline=false, bottomline=false,
    backgroundcolor = gray!10,
    innerleftmargin=1in, innerrightmargin=1in, innertopmargin=5pt,
    leftmargin=-1in, rightmargin=-1in, linewidth=4pt,
    innerbottommargin=5pt]{unexamBox}
\newenvironment{unexaminable}{\begin{unexamBox}}{\end{unexamBox}}


% clever ref settings
\crefname{lemma}{lemma}{lemmas}
\Crefname{lemma}{Lemma}{Lemmas}
\crefname{theorem}{theorem}{theorems}
\Crefname{theorem}{Theorem}{Theorems}

% formatting 
% https://tex.stackexchange.com/questions/217497/aligning-stackrel-signs-beneath-each-other-using-split
\newlength{\leftstackrelawd}
\newlength{\leftstackrelbwd}
\def\leftstackrel#1#2{\settowidth{\leftstackrelawd}%
{${{}^{#1}}$}\settowidth{\leftstackrelbwd}{$#2$}%
\addtolength{\leftstackrelawd}{-\leftstackrelbwd}%
\leavevmode\ifthenelse{\lengthtest{\leftstackrelawd>0pt}}%
{\kern-.5\leftstackrelawd}{}\mathrel{\mathop{#2}\limits^{#1}}}

\doublespacing
\RaggedRight

\title{Week 6}

\begin{document}
\maketitle

\section{Baire Category and UBP}

\subsection{Baire Category}

We seek a topological characterization of the "size" of the sets. First recall some relevant topological terms.  

\begin{definition}[Clousure and interior]
    Let $(X,d)$ be a metric space, $A\subset X$.
    The \textbf{interior} of $A$, written as $\interior(A)$ or sometimes $A^o$:
    $$ \interior(A)=\bigcup_{G\subset A,\,open}G$$
    The closure of A, written as $\bar{A}$ or $cl(A)$:
    $$ \bar{A}=\bigcap_{U\supset A\,closed}U$$

\end{definition}  

\begin{definition}[nowhere dense]
    A set $A$ is \textbf{nowhere dense} if $\interior{\bar{A}}=\emptyset$
\end{definition}

\begin{proposition}
\label{complement of nowhere}
    $G \subset X$ is open and dense in $X$ if and only if $X \setminus G$ is closed and nowhere dense
\end{proposition}   

\begin{proof}
    This is left as an exercise.
\end{proof}


We now present a key lemma.   


\begin{lemma}%\rm\nextline
\label{baire lemma}
    Let $(X,d)$ be a complete metric space over $\mathbb{R}$, $X\neq\emptyset$.    
    If $X=\bigcup_{k=1}^\infty A_k$, where each $A_k$ is closed (i.e. $\bar{A_k}=A_k$), then at least one of the $A_k$ contains an open ball. ($\exists k \in \mathbb{N}$ such that $\interior{A_k}\neq\emptyset$)  
\end{lemma}    

    \begin{proof}
        Assume for the sake of contradiction, that

        \begin{equation*}
            \interior(A_k)=\emptyset \tag{$\star$}, \qquad \forall k \in \mathbb{N}
        \end{equation*}
        
        We pick $x_1\in X\setminus A_1$, by \Cref{complement of nowhere}.
        
        Since $X\setminus A_1=X\cap A_1^c$ is open, we can find $r_1 \in (0,2^{-1})$ such that
        
        $$B(x_1,r_1)\subset X\setminus A_1$$  
        
        We can repeat the construction above and find $x_2\in B(x_1,r_1/2)\setminus A_2$ and $r_2 \in (0, 2^{-2} r_1)$, such that 
        
        $$B(x_2,r_2)\subset X\setminus A_2$$
        
        \textbf{Claim:} $\forall n \in \mathbb{N}$, $\exists x_n \in X$ and $0<r_n<2^{-n}r_1$ such that,  
        \begin{equation*}
            B(x_{n+1},r_{n+1})\subset B(x_n,r_n/2)\subset B(x_n,r_n)\subset X\setminus A_n
        \end{equation*}
            
        We prove this claim by induction on $n$.   
        
        The base case $n=1$ is shown above.    
        
        For $n=k$, we have by the inductive hypothesis, $B(x_k, r_k) \subset X \setminus A_k$, which is open and dense, so we can choose $x_{k+1} \in B(x_k, r_k) \setminus A_{k}$ and $B(x_{k+1}, r_{k+1}) \subset B(x_k, r_k) \setminus A_{k}$.  
        
        The sequence $(x_n)_{n \geq 1}$ is Cauchy. By (1)
        
        $$d(x_m,x_n)\stackrel{x_m\in B(x_n,r_n/2)}{\leq}r_n/2<2^{-(k+1)}r_1, \qquad \forall m\geq n\geq1$$
        
        By completeness, $\exists x_*\in X$, such that $d(x_n,x_*)\to0$ as $n\to \infty$.  
    
        \textbf{Claim:} $x_*\in B(x_n,r_n),\,\,\forall k\geq 1$  
        
        If the claim is true then,  $x_*\in X\setminus A_k\,\forall k \in \mathbb{N}$ i,e, 
        $$x_*\in\bigcap_{k\geq1}(x\setminus A_k)=X\setminus\bigcup_{k\geq1}A_k=\emptyset$$
        which is a contradiction. 
        To get claim, note that for all $b\geq0$:
        $$
        d(x_*,x_n)\leq \underbrace{{d(x_*,x_{n+1})}}_{\to0\,\,n\to\infty}+\underbrace{d(x_{n+1},x_n)}_{\leq r_n/2}
        $$
        so by choosing $n$ large enough, the sum is less than $r_n$
    \end{proof} 

\begin{flushleft}
The lemma motivates the following.  
\end{flushleft}


\begin{definition}
    (Category)  
    Let $(X,d)$ be a metric space.  
\end{definition}
\begin{enumerate}[i)]
    \item $A \subset X$ is called \textbf{meager} (or of the $1^{\textrm{st}}$ Baire Category) if $A = \cup_{k=1}^{\infty} A_k$ with nowhere dense sets $A_k$; denoted $\textrm{cat}(A)=1$.  
    \item $A \subset X$ is called \textbf{fat} (or of the $2^{\textrm{nd}}$ Baire Category) if it is not meager; denoted $\textrm{cat}(A)=2$.  
\end{enumerate}  

In this language, \cref{baire lemma} becomes  

\begin{theorem}
(Baire category)  
Let  $(X,d)$ be a complete, non-empty metric space. Then $\textrm{cat}(A)=2$.  
\end{theorem}  

\begin{remark}

\end{remark}

\subsection{UBP}
\begin{flushleft}
Uniform boundedness principle is sometimes called Banach–Steinhaus theorem. In its basic form, it asserts that for a family of bounded linear operators  whose domain is a Banach space, pointwise boundedness is equivalent to uniform boundedness in operator norm.
\end{flushleft}


\begin{theorem}[Uniform Boundedness principle] %\rm\nextline
	Let $X$ be a Banach space, $Y$ a normed vector space. Let $F$ be a collection of bounded linear operators from $X$ to $Y$. If
	$$
		\sup_{f\in F}\norm{fx}<\infty,,\,\forall x\in X
	$$
	Then
	$$
		\sup_{f\in F}\norm{f}<\infty
	$$
\end{theorem}

\begin{proof}
Let $A_n=\{
	x\in X, \norm{fx}\leq n,\,\forall f\in F
	\},\,n\in\mathbb{N}$. By assumption we have $\bigcap_{n=1}^\infty A_n=X$.\\
We claim that there exists some $j\in \mathbb{N}$ such that $A_j$ is non-empty and closed. To see this, first by by Baire category theorem, there is some $A_j$ such that $\overline{A_j}\strut^\mathrm{o}\not=\emptyset$. Then let $\{x_m\}$ be a Cauchy sequence in $A_j$ with $x_n\xrightarrow{}x$, then by continuity of $f$, $\norm{fx}=\lim_{n\to\infty}\norm{fx_m}\leq n,\,\forall f\in F$. So $x\in A_j$, hence $A_j$ is closed, thus $\overline{A_j}=A_j$, $A_j\strut^\mathrm{o}=\overline{A_j}\strut^\mathrm{o}\not=\emptyset$. So we can choose a point $p$ from interior of $A_j$, and $\varepsilon>0$ such that open ball $B_{\varepsilon}(p)\subseteq A_j$.\\
Now for any $x<\norm{\varepsilon}$ with any $T\in F$ we have
$$
	\norm{T(x)}=\norm{T(x+p-p)}=\norm{T(x+p)-T(p)}\leq\norm{T(x+p)}+\norm{T(p)}\leq n+n=2n
$$
So for any non-zero vector $x\in X$, we have
$$
	\norm{T(x)}=\frac{\norm{x}}{\varepsilon}\norm{T(\varepsilon \frac{x}{\norm{x}})}\leq\frac{2n}{\varepsilon}\norm{x}
$$
This holds for any $T\in F$, thus
$$
	\sup_{f\in F}\norm{f}\leq\frac{2n}{\varepsilon}<\infty
$$
\end{proof}  

A simple corollary of the theorem is Banach limit.
\begin{corollary}[Banach Limit]\rm\nextline
	Let $T_n:X\xrightarrow{}Y$ be a sequence of operators, where $X$ and $Y$ are Banach spaces. Suppose $\{T_n\}$ converges pointwise,
	then these pointwise limits define a bounded linear operator $T$.
\end{corollary}
\end{document}