\documentclass{article}
\usepackage[utf8]{inputenc}
\usepackage{geometry}
\geometry{left=3cm,right=3cm,top=2cm,bottom=2cm}
\usepackage[utf8]{inputenc}
\usepackage{amsmath, amsfonts, amssymb, amsthm}
\usepackage[framemethod=TikZ]{mdframed}
\usepackage{mathrsfs}
\usepackage{comment}
\usepackage{enumerate}
\usepackage{xcolor}
\usepackage{titlesec}
\usepackage{setspace}
\usepackage[hidelinks,backref]{hyperref}
\usepackage{cleveref}
\usepackage[most]{tcolorbox}
\usepackage{ragged2e}
\usepackage{todonotes}
\usepackage{cleveref}
\usepackage{mathtools}

%
\titleformat*{\section}{\LARGE \bfseries}
\titleformat*{\subsection}{\Large \bfseries}
\titleformat*{\subsubsection}{\Large \bfseries}
% \titleformat*{\paragraph}{\large \bfseries}
\titleformat*{\subparagraph}{\large \bfseries}



% General 
\newcommand{\nextline}{\hfill\break}
\newcommand{\nl}{\nextline\rm}
% \newcommand{\placeholder}{{\bf\color{red} NOOOOOOT COMPLEEEEEEET! COOOOOOOM BAAAAAACK!!!}}
\newcommand{\placeholder}{\todo{NOOOOOOT COMPLEEEEEEET! COOOOOOOM BAAAAAACK!!!}}

\newcommand{\defeq}{\stackrel{def.}{=}}
% FA and LA
% inner product: \inne{a}{b}
\newcommand{\inne}[2]{\left<{#1},{#2}\right>}

% norm: \norm{a}
\newcommand{\norm}[1]{\left\|{#1}\right\|}

% Curly H
\newcommand{\hbs}{$\mathscr{H}$ }
\newcommand{\hbp}{\mathscr{H}}

% Dual : \dual{x}
\newcommand{\dual}[1]{{#1}^*}

% Sequence from 1  to infty: \sequ{x_n}
\newcommand{\sequ}[1]{\left({#1}\right)_1^\infty}

% f: A-> B \func{f}{A}{B}
\newcommand{\func}[3]{${#1}:{#2}\xrightarrow{}{#3}$}

% interior
\newcommand{\interior}{\textrm{int}}

% Bounded linear funcs
\newcommand{\blf}[2]{\mathcal{L}({#1},{#2})}

\newcommand{\prf}{\textit{proof}:   }



% Fields 
\newcommand{\real}{\mathbb{R}}
\newcommand{\comp}{\mathbb{C}}
\newcommand{\inte}{\mathbb{Z}}
\newcommand{\natu}{\mathbb{N}}





% Theorems
% \newtheorem{example}{Example}[subsection]
% \newtheorem{definition}[example]{Definition}
% \newtheorem{proposition}[example]{Proposition}
% \newtheorem{remark}[example]{Remark}
% \newtheorem{theorem}[example]{Theorem}
% \newtheorem{lemma}[example]{Lemma}
% \newtheorem{corollary}[example]{Corollary}


% for numbering the theorems            
\theoremstyle{plain}
%%%%%%%%%%%%%%%%%%%%%%%%%%%%%%%%%%%%%%%%%%%%%%%%%%%%%%%%%%%
\newtheorem{theorem}{Theorem}[section]
\newtheorem{lemma}[theorem]{Lemma}
\newtheorem{corollary}[theorem]{Corollary}
\newtheorem{proposition}[theorem]{Proposition}
%%%%%%%%%%%%%%%%%%%%%%%%%%%%%%%%%%%%%%%%%%%%%%%%%%%%%%%%%%%
% the following are not in italics
\theoremstyle{definition}
\newtheorem{definition}[theorem]{Definition}
\newtheorem{example}[theorem]{Example}
\newtheorem{remark}[theorem]{Remark}
\newtheorem{claim}[theorem]{Claim}
%%%%%%%%%%%%%%%%%%%%%%%%%%%%%%%%%%%%%%%%%%%%%%%%%%%%%%%%%%%%

% proof box
\newtcbtheorem[no counter]{pf}{Proof}{
  enhanced,
  rounded corners,
  attach boxed title to top,
  colback=white,
  colframe=black!25,
  fonttitle=\bfseries,
  coltitle=black,
  boxed title style={
    rounded corners,
    size=small,
    colback=black!25,
    colframe=black!25,
  } 
}{prf}

% extra content box to put in contents not covered in the lecture notes
% use the command \begin{unexaminable}
\newmdenv[skipabove=7pt, skipbelow=7pt,
    rightline=false, leftline=false, topline=false, bottomline=false,
    backgroundcolor = gray!10,
    innerleftmargin=1in, innerrightmargin=1in, innertopmargin=5pt,
    leftmargin=-1in, rightmargin=-1in, linewidth=4pt,
    innerbottommargin=5pt]{unexamBox}
\newenvironment{unexaminable}{\begin{unexamBox}}{\end{unexamBox}}


% clever ref settings
\crefname{lemma}{lemma}{lemmas}
\Crefname{lemma}{Lemma}{Lemmas}
\crefname{theorem}{theorem}{theorems}
\Crefname{theorem}{Theorem}{Theorems}

% formatting 
% https://tex.stackexchange.com/questions/217497/aligning-stackrel-signs-beneath-each-other-using-split
\newlength{\leftstackrelawd}
\newlength{\leftstackrelbwd}
\def\leftstackrel#1#2{\settowidth{\leftstackrelawd}%
{${{}^{#1}}$}\settowidth{\leftstackrelbwd}{$#2$}%
\addtolength{\leftstackrelawd}{-\leftstackrelbwd}%
\leavevmode\ifthenelse{\lengthtest{\leftstackrelawd>0pt}}%
{\kern-.5\leftstackrelawd}{}\mathrel{\mathop{#2}\limits^{#1}}}

\doublespacing
\RaggedRight
% modify innerrightmargin if floats were lost

\usepackage{amsfonts, amsmath, amssymb, amsthm, thmtools, bm}
\usepackage{avant} % Use the Avantgarde font for headings

% Boxed/framed environments
\newtheoremstyle{royalnumbox}%
{0pt}% Space above
{0pt}% Space below
{\normalfont}% Body font
{}% Indent amount
{\small\bf\sffamily\color{royal}}% Theorem head font
{\;}% Punctuation after theorem head
{0.25em}% Space after theorem head
{\sffamily \color{royal} 
    \thmname{#1} 
    \thmnumber{#2} \thmnote{\bfseries\color{black}---\nobreakspace#3.}} % Optional theorem note
\renewcommand{\qedsymbol}{$\blacksquare$}% Optional qed square

\newtheoremstyle{blacknumex}% Theorem style name
{5pt}% Space above
{5pt}% Space below
{\normalfont}% Body font
{} % Indent amount
{\small\bf\sffamily}% Theorem head font
{\;}% Punctuation after theorem head
{0.25em}% Space after theorem head
{\sffamily
    \thmname{#1}
    \thmnumber{#2}
    \thmnote{---\nobreakspace#3.}}% Optional theorem note

\newtheorem*{notation}{Notation}
\newtheorem*{hint}{Hint}
\newtheorem*{solution}{Solution}

\newcounter{dummy} 
\numberwithin{dummy}{section}

\theoremstyle{royalnumbox}
\newtheorem{definitionT}[dummy]{Definition}
\newtheorem{theoremT}[dummy]{Theorem}
\newtheorem{lemmaT}[dummy]{Lemma}
\newtheorem{corollaryT}[dummy]{Corollary}
\newtheorem{propositionT}[dummy]{Proposition}
\newtheorem{propertyT}[dummy]{Property}
\newtheorem{remarkT}[dummy]{Remark}

\theoremstyle{blacknumex}
\newtheorem{exampleT}[dummy]{Example}
\newtheorem{exerciseT}[dummy]{Exercise}

\numberwithin{equation}{section}

\RequirePackage[framemethod=default]{mdframed}

% Definition box
\newmdenv[skipabove=7pt, skipbelow=7pt,
rightline=false, leftline=true, topline=false, bottomline=false,
backgroundcolor = reddish!10, 
linecolor=reddish,
innerleftmargin=5pt, innerrightmargin=30pt, innertopmargin=5pt,
leftmargin=0cm, rightmargin=0cm, linewidth=4pt,
innerbottommargin=5pt]{dBox}

% Main Theorem box
\newmdenv[skipabove=7pt, skipbelow=7pt,
rightline=false, leftline=true, topline=false, bottomline=false,
backgroundcolor=c0!10, 
linecolor=c0,
innerleftmargin=5pt, innerrightmargin=30pt, innertopmargin=5pt,
leftmargin=0cm, rightmargin=0cm, linewidth=4pt, innerbottommargin=5pt]{tBox}

% Lemma/Corollary/Proposition/Property box
\newmdenv[skipabove=7pt, skipbelow=7pt,
rightline=false, leftline=true, topline=false, bottomline=false,
backgroundcolor = c0!10, 
linecolor=c0!80,
innerleftmargin=5pt, innerrightmargin=30pt, innertopmargin=5pt,
leftmargin=0cm, rightmargin=0cm, linewidth=4pt,
innerbottommargin=5pt]{lBox}

% Example/Remark/Exercise box
\newmdenv[skipabove=7pt, skipbelow=7pt,
rightline=false, leftline=true, topline=false, bottomline=false,
backgroundcolor = mossgreen!10!white,
linecolor = mossgreen,
innerleftmargin=5pt, innerrightmargin=30pt, innertopmargin=5pt,
leftmargin=0cm, rightmargin=0cm, linewidth=4pt,
innerbottommargin=5pt]{exBox}

% Proof box
% \newmdenv[skipabove=7pt, skipbelow=7pt,
% rightline=false, leftline=true, topline=false, bottomline=false,
% linecolor=gray,
% innerleftmargin=5pt, innerrightmargin=30pt, innertopmargin=5pt,
% leftmargin=0cm, rightmargin=0cm, linewidth=4pt,
% innerbottommargin=5pt]{proofBox}



% Creates an environment for each type of theorem and assigns it a theorem text style from the "Theorem Styles" section above and a colored box from above
\newenvironment{definition}{\begin{dBox}\begin{definitionT}}{\end{definitionT}\end{dBox}}
\newenvironment{theorem}{\begin{tBox}\begin{theoremT}}{\end{theoremT}\end{tBox}}
\newenvironment{lemma}{\begin{lBox}\begin{lemmaT}}{\end{lemmaT}\end{lBox}}
\newenvironment{proposition}{\begin{lBox}\begin{propositionT}}{\end{propositionT}\end{lBox}}
\newenvironment{corollary}{\begin{lBox}\begin{corollaryT}}{\end{corollaryT}\end{lBox}}
\newenvironment{property}{\begin{lBox}\begin{propertyT}}{\end{propertyT}\end{lBox}}


% \newenvironment{proof*}{\begin{proofBox}\begin{proof}}{\end{proof}\end{proofBox}}
\newenvironment{exercise}{\begin{exBox}\begin{exerciseT}}{\hfill{\color{royal}}\end{exerciseT}\end{exBox}}
\newenvironment{remark}{\begin{exBox}\begin{remarkT}}{\end{remarkT}\end{exBox}}
\newenvironment{example}{\begin{exBox}\begin{exampleT}}{{}\end{exampleT}\end{exBox}}

\title{Week 6}

\begin{document}
% \author{\aut}
\maketitle

\section{Baire Category and UBP}

\subsection{Baire Category}

We seek a topological characterisation of the "size" of the sets. First recall some relevant topological terms.  

\begin{definition}[Closure and interior]
    Let $(X,d)$ be a metric space, $A\subset X$.
    The \textbf{interior} of $A$, written as $\interior(A)$ or sometimes $A^o$:
    $$ \interior(A)=\bigcup_{G\subset A,\,open}G$$
    The closure of A, written as $\bar{A}$ or $cl(A)$:
    $$ \bar{A}=\bigcap_{U\supset A\,closed}U$$

\end{definition}  

\begin{definition}[nowhere dense]
    A set $A$ is \textbf{nowhere dense} if $\interior{\bar{A}}=\emptyset$
\end{definition}

\begin{proposition}
\label{complement of nowhere}
    $G \subset X$ is open and dense in $X$ if and only if $X \setminus G$ is closed and nowhere dense
\end{proposition}   

\begin{proof}
    This is left as an exercise.
\end{proof}


We now present a key lemma.   


\begin{lemma}%\rm\nextline
\label{baire lemma}
    Let $(X,d)$ be a complete metric space over $\mathbb{R}$, $X\neq\emptyset$.    
    If $X=\bigcup_{k=1}^\infty A_k$, where each $A_k$ is closed (i.e. $\bar{A_k}=A_k$), then at least one of the $A_k$ contains an open ball. ($\exists k \in \mathbb{N}$ such that $\interior{A_k}\neq\emptyset$)  
\end{lemma}    

    \begin{proof}
        Assume for the sake of contradiction, that

        \begin{equation*}
            \interior(A_k)=\emptyset \tag{$\star$}, \qquad \forall k \in \mathbb{N}
        \end{equation*}
        
        We pick $x_1\in X\setminus A_1$, by \Cref{complement of nowhere}.
        
        Since $X\setminus A_1=X\cap A_1^c$ is open, we can find $r_1 \in (0,2^{-1})$ such that
        
        $$B(x_1,r_1)\subset X\setminus A_1$$  
        
        We can repeat the construction above and find $x_2\in B(x_1,2^{-1}r_1)\setminus A_2$ and $r_2 \in (0, 2^{-2} r_1)$, such that 
        
        $$B(x_2,r_2)\subset X\setminus A_2$$
        
        \textbf{Claim:} $\forall n \in \mathbb{N}$, $\exists x_n \in X$ and $0<r_n<2^{-n}r_1$ such that,  
        \begin{equation*}
            B(x_{n+1},r_{n+1})\subset B(x_n,r_n/2)\subset B(x_n,r_n)\subset X\setminus A_n
        \end{equation*}
            
        We prove this claim by induction on $n$.   
        
        The base case $n=1$ is shown above.    
        
        For $n=k$, we have by the inductive hypothesis, $B(x_k, r_k) \subset X \setminus A_k$, which is open and dense, so we can choose $x_{k+1} \in B(x_k, r_k) \setminus A_{k}$ and $B(x_{k+1}, r_{k+1}) \subset B(x_k, r_k) \setminus A_{k}$.  
        
        The sequence $(x_n)_{n \geq 1}$ is Cauchy. By (1)
        
        $$d(x_m,x_n)\stackrel{x_m\in B(x_n,r_n/2)}{\leq}r_n/2<2^{-(k+1)}r_1, \qquad \forall m\geq n\geq1$$
        
        By completeness, $\exists x_*\in X$, such that $d(x_n,x_*)\to0$ as $n\to \infty$.  
    
        \textbf{Claim:} $x_*\in B(x_n,r_n),\,\,\forall k\geq 1$  
        
        If the claim is true then,  $x_*\in X\setminus A_k\,\forall k \in \mathbb{N}$ i,e, 
        $$x_*\in\bigcap_{k\geq1}(X\setminus A_k)=X\setminus\bigcup_{k\geq1}A_k=\emptyset$$
        which is a contradiction. 
        To get claim, note that for all $b\geq0$:
        $$
        d(x_*,x_n)\leq \underbrace{{d(x_*,x_{n+1})}}_{\to0\,\,n\to\infty}+\underbrace{d(x_{n+1},x_n)}_{\leq r_n/2}
        $$
        so by choosing $n$ large enough, the sum is less than $r_n$
    \end{proof} 

\begin{flushleft}
The lemma motivates the following.  
\end{flushleft}


\begin{definition}  
    Let $(X,d)$ be a metric space.  
    \begin{enumerate}[i)]
    \item $A \subset X$ is called \textbf{meager} (or of the $1^{\textrm{st}}$ Baire Category) if $A = \cup_{k=1}^{\infty} A_k$ with nowhere dense sets $A_k$; denoted $\textrm{cat}(A)=1$. ($A_k$ is nowhere dense if $\interior(\overline{A_k})=\emptyset$)
    \item $A \subset X$ is called \textbf{fat} (or of the $2^{\textrm{nd}}$ Baire Category) if it is not meager; denoted $\textrm{cat}(A)=2$.  
\end{enumerate} 
\end{definition} 


In this language, \Cref{baire lemma} becomes  

\begin{theorem}
[Baire category theorem]\nl
\label{baire category}
Let  $(X,d)$ be a complete, non-empty metric space. Then $\textrm{cat}(X)=2$.  
\end{theorem}  
Note that the "completeness" assumption depends and by conclusion is purely topological; thus, the theorem extends to $(X,d)$ not complete if one can find a metric $\Tilde{d}$ such that $(X, \Tilde{d})$ is complete and $d$ induces the same topology as $d$.

\begin{remark}
    Every subset of a meager set $A$ is meager (in particular, \Cref{baire category} implies the \textbf{existence} of fat sets, examples are $(\real, |\cdot|), (C, \norm{\cdot}_{\infty})$)
\end{remark}

\begin{remark}
\label{Baire remark}
\begin{enumerate}[1)]

    \item $\qq = \bigcup_{x\in \qq} \{x\} \subset \real (=X)$ is meager. What about $\real \setminus \qq$? (this is fat. Why?)

    
    \item Under the assumption of \Cref{baire category}, if $A\subset X$  
    $$\text{cat}(A)=1 \implies \text{cat}(X \setminus A)=2 \ \text{and} \ X\setminus A \ \text{is} \ \text{dense}$$
    use (exercise): $U \subset X$ open and dense $\iff$ $A = X\setminus U$ closed and nowhere dense, hence:  
    \begin{equation*}
        \text{\Cref{baire lemma}} \iff U_k \subset X, k\geq 1, \text{open,\ dense}. \ \text{Then} \ U = \cap_{k=1}^{\infty} U_k \ \text{dense}
    \end{equation*}

    
    \item If $\emptyset \neq U \subset X$ open $\implies$ $\text{cat}(U)=2$ (under the same assumption in 2) )
    \begin{proof}
        If $\text{cat}(U)=1$, by $2)$, $X\setminus U$ is dense, \textit{i.e.}  
        $$
        X = \overline{X\setminus U}\overset{X\setminus U \ closed}{=} X\setminus U
        $$
        in other words, $U=\emptyset$, which is also closed, a contradiction.
    \end{proof}

    
    \item (Topological vs. measure-theoretic size)  $X=\real$ with $\lambda$ the Lebesgue measure. Does $\lambda(A)=0 \implies A $ meager? Does $A \subset \real$ meager $\implies \lambda(A)=0$?  

    The answer to both questions are NO.  
    
    Take $\qq = \{q_1, q_2, \ldots\}$, for $j\geq 1$:  
    $$
    U_j = \cup_{k\geq 1} (q_k - \frac{1}{2^{j+k+1}}, q_k + \frac{1}{2^{j+k+1}})
    $$ 
    which is decreasing in $j$  
    $$
    \lambda(U_j) \leq \sum_{k\geq 1} 2 \cdot \frac{1}{2^{j+k+1}} = 2^{-j}
    $$
    $U_j$ is open and $\overline{U_j} \supset \overline{\qq} = \real$, so $\overline{U_j} = \real$, \textit{i.e.} $U_j$ is dense.  
    By $2)$ (see exercise), $A_j \defeq X \setminus U_j$ is nowhere dense, $A \defeq \cup_{j=1}^{\infty} A_j$ is meager and hence $U = X\setminus A=\cap_{j=1}^{\infty} U_j$ is fat. But  
    $$
    \lambda(U) = \lim_{j\to \infty} \lambda(U_j) = 0 \qquad \lambda(A)=\infty
    $$
\end{enumerate}
\end{remark}

\subsection{Uniform boundednness principle (UBP)}

Baire's category theorem leads to UBP. The first instance of this is not a 'linear' property at all.

\begin{theorem}
\label{uniform bounded but not linear}
    Let $(X,d)$ be a complete metric space and $(f_\lambda)_{\lambda \in \Lambda}$ be a family of continuous functions $f_{\lambda}:X \to \real$. If $(f_\lambda)_{\lambda \in \Lambda}$ is bounded pointwise, \textit{i.e.}  
    $$
    \sup_{\lambda \in \Lambda} |f_\lambda(x)| < \infty \qquad \forall x \in X
    $$
    then $\exists B\subset X$ an open ball s.t. 
    $$
    \sup_{\lambda \in \Lambda, x\in B} |f_{\lambda}(x)|<\infty
    $$
    \textit{i.e.} $(f_{\lambda})_{\lambda \in \Lambda}$ uniformly bounded on $B$.
\end{theorem}
\begin{remark}
    The $(f_\lambda)_{\lambda \in \Lambda}$ need NOT to be linear.
\end{remark}
\begin{proof}
    For $k\geq 1$, consider the closed set  
    $$
    A_k = \left\{x \in X: \forall \lambda \in \Lambda: |f_{\lambda}(x)|\leq k = \bigcap_{\lambda\in \Lambda} \underbrace{\{|f_{\lambda}|\leq k\}}_{\text{closed\ as\ $f_{\lambda}$\ continuous\ }}\right\}
    $$
    Clearly $\cup_{k=1}^{\infty}A_k=X$. Since $X$ is complete,  
    by \Cref{baire lemma}, $\exists k_0 \in \natu$, s.t. $\interior (A_{k_0})\neq \emptyset$. Pick $B\subset A_{k_0}$.
\end{proof}



Incorporating the linear structure, this leads to the following.  

\begin{corollary}[Banach-Steinhaus]\nl
\label{banach steinhaus}
Let $X,Y$ be normed vector spaces and $X$ is complete.  

Let $(A_\lambda)_{\lambda \in \Lambda} \subset \mathcal{L}(X,Y)$. If $(A_{\lambda})_{\lambda \in \Lambda}$ are bounded pointwise \textit{i.e.}  
$$
\sup_{\lambda \in \Lambda} \norm{A_\lambda x}_Y < \infty \qquad \forall x\in X
$$  
then $(A_{\lambda})_{\lambda \in \Lambda}$ is bounded uniformly, \textit{i.e.}  
$$
\sup_{\lambda \in \Lambda} \norm{A_\lambda}_{\mathcal{L}(X,Y)} < \infty
$$
\end{corollary}

\begin{proof}
For $\lambda \in \Lambda$ define the continuous (check this!) map $f_\lambda: X \to \real$ by  
$$
f_{\lambda}(x) = \norm{A_{\lambda}x}_Y, \qquad x\in X
$$  
By assumption on $A_\lambda$, \Cref{uniform bounded but not linear} applies and yields $B=B_r(x_0) \subset X$ s.t.  
$$
\sup_{\lambda \in \Lambda, x\in B} |f_{\lambda}(x)| < \infty
$$  
This gives for all $\norm{x}_X<1$ and $\lambda \in \Lambda$:
\begin{align*}
    \norm{A_\lambda x}_Y &= \frac{1}{r} \norm{A_\lambda(x_0+rx)-A_\lambda(x_0)}_Y \\
    &\leq \frac{1}{r} \norm{A_\lambda(x_0+rx)}_Y + \frac{1}{r} \norm{A_\lambda(x_0)}_Y \\
    &\leq M
\end{align*}
\end{proof}  

An application of \Cref{banach steinhaus} is the following.  

\begin{proposition}
    With $X,Y$ normed and $X$ \textbf{complete}, let $A_k \in \mathcal{L}(X,Y)$, $(A_k)$ converges point-wise to $A: X\to Y$, \textit{i.e.}  
    $$
    \forall x \in X: \qquad \norm{A_k x - Ax}_Y \overset{k\to \infty}{\longrightarrow} 0 
    $$
    Then $A$ is linear and continuous, \textit{i.e.} $A \in \mathcal{L}(X,Y)$ ,and  
    $$
    \norm{A}_{\mathcal{L}(X,Y)} \leq \liminf_{k \to \infty} \norm{A_k}_{\mathcal{L}(X,Y)} <\infty
    $$
\end{proposition}

\begin{proof}
    $(A_k x)\subset Y$ is convergent hence bounded, so \Cref{banach steinhaus} applies and yields $\sup_k \norm{A_k}_{\mathcal{L}(X,Y)} < \infty$. This shows $\liminf_{k \to \infty} \norm{A_k}_{\mathcal{L}(X,Y)}$ is finite hence well-defined. Pick subsequence $k_j$ s.t. 
    $$\norm{A_{k_j}}\overset{j \to \infty}{\longrightarrow} \liminf_k \norm{A_k}_{\mathcal{L}(X,Y)} \defeq M$$
    $(A_{k_j})_j$ converges pointwise, $A$ is linear (check!) and $\forall x \in X$:  
    $$
    \norm{Ax}_Y = \lim_{j} \norm{A_{k_j}x}_Y \leq \lim_j \norm{A_{k_j}}_{\mathcal{L}(X,Y)} \norm{x}_X \leq M \norm{x}_X
    $$
\end{proof}

\begin{remark}
    \textbf{Completeness} of $X$ is important (as for Baire)  
    
    Take $X=C(=C^0[0,1])$, $\norm{\cdot}_X=\norm{\cdot}_1$. Let  
    $$
    A_k f = k\int_{1-1/k}^1 f(t) dt \qquad k\geq 1
    $$  
    Clearly, $|A_k f|\leq k \norm{f}_1$, so $A_k: X\to \real$ continuous with $\norm{A_k}_{\mathcal{L}(X,Y)} \leq k, \forall k \in \natu$. Moreover,  
    $$
    \forall f \in X: \qquad A_k f \overset{k\to \infty}{\longrightarrow} Af \defeq f(1)
    $$
    But $A:X\to \real$ is not continuous: take for examples, $f_n(t)=t^n$ then $\norm{f_n}=\frac{1}{n+1}$ so $f_n \overset{L^1}{\longrightarrow} 0 $ but  
    $$
    Af_n = f_n(1) = 1 \nrightarrow 0
    $$  
    Of course $(X, \norm{\cdot}_1)$ is not complete (why?), so there is no contradiction.  
\end{remark}

%\todo{(Lebesgue) derivatives definition here?}


\begin{remark}
    If instead we consider $(C, \norm{\cdot}_\infty)$, then  
    $$
    |A_k f| \leq \norm{f}_{\infty}
    $$
    and $Af=f(1)$ is continuous as it's bounded $\norm{A}_{\mathcal{L}(X,Y)}\leq 1$ as $\norm{Af}_{\infty} \leq \norm{f}_\infty$
\end{remark}

The following section is not examinable.  


\subsection{Baire's Original Problem}
Let $(f_n)_{n\geq 1}$ be a sequence of continuous functions $f_n: [0,1] \to \real$ and converges point-wise, \textit{i.e.}  
$$
f(x) = \lim_{n\to\infty} f_n(x) \ \text{exists} \ \forall x \in [0,1]
$$  
\textbf{Question:} Does $f$ have points of continuity?  

The answer is given by the following theorem.  

\begin{theorem}[Baire]\nl
    Let $(X,d)$ be a complete metric space and $(f_n)$ a sequence of continuous functions $f_n: X \to \real$, and for each $x\in X$, the point-wise limit 
    $$
    \lim_{n\to \infty} f_n(x) \defeq f(x) \in \real
    $$
    exists. Then 
    $$
    R \defeq \left\{ x\in \real: f \ \text{is\ continuous\ at}\ x\right\}
    $$
    is \textbf{dense} in $X$.
\end{theorem}
\begin{proof}
    For $\varepsilon > 0, n\geq 1$ set  
    $$
    P_{n, \varepsilon} = \{x\in \real: |f_n(x)-f(x)|\leq \varepsilon\}
    $$
    and
    $$
    R_\varepsilon = \bigcup_{n=1}^\infty \interior(P_{n, \varepsilon})
    $$
    We have the following two claims.   
    
    \underline{\textbf{Claim 1:}} $\bigcap_{n=1}^\infty R_{\frac{1}{n}}=R$  
    \begin{proof}
        This can be checked by writing out explicitly what $x \in \bigcap_{n=1}^\infty R_{\frac{1}{n}}=\bigcap_{n=1}^\infty \bigcup_{m=1}^{\infty} \interior(P_{m,\frac{1}{n}})$ means, namely $\forall n\in \natu, \exists M \in \natu, \forall m \geq M, |f_m(x)-f(x)|<\frac{1}{n}$, as the sets $\interior(P_{m,\frac{1}{n}})$ are increasing in $n$. By the usual '$3\varepsilon$' argument, for any $\varepsilon>0$, we can find $|x-y|<\delta$, so that $ |f_n(x)-f_n(y)|<\frac{\varepsilon}{3}$ and we choose $n$ accordingly to let $\frac{1}{n}<\frac{\varepsilon}{3}$ such that
        \begin{align*}
            |f(x)-f(y)|&\leq|f(x)-f_n(x)|+ |f_n(x)-f_n(y)|+|f_n(y)-f(y)| < \varepsilon
        \end{align*}
        which shows the continuity.
    \end{proof}
    
    \underline{\textbf{Claim 2:}} $R_\varepsilon$ is open and dense for all $\varepsilon>0$.  
    \begin{proof}
        $R_\varepsilon$ is open, as it is a countable union of open sets.  

        For density, consider $F_{n,\varepsilon}=\bigcap_{k\geq 1} \left\{ x: |f_n(x)-f_{n+k}(x)|\leq \varepsilon\right\}$, which is closed.  

        Since $f_{n+k}(x) \overset{k\to \infty}{\longrightarrow} f(x)$, we have $F_{n,\varepsilon}\subseteq P_{n, \varepsilon}$. % This needs to be checked 
        %\todo{check the direction of inclusion}
        Also, since $f_n(x)\overset{n\to \infty}{\longrightarrow} f(x)$, we have 
        $$
        \bigcup_{n=1}^\infty P_{n, \varepsilon} = X \qquad  \forall \varepsilon >0
        $$
        Let $A_{n,\varepsilon} = \partial F_{n,\varepsilon}$, then $A_{n,\varepsilon}$ is closed and 
        $$
        \interior(A_{n,\varepsilon}) = \interior(\bar{A}_{n,\varepsilon} \setminus \interior(A_{n,\varepsilon})) = \emptyset
        $$  
        \textit{i.e.} $A_{n,\varepsilon}$ is nowhere dense. Hence the set 
        $$
        A_\varepsilon = \bigcup_{n\geq 1} A_{n,\varepsilon}
        $$
        is meager, also we have
        \begin{align*}
        A_\varepsilon &= \bigcup_{n\geq 1} A_{n,\varepsilon} \\
        &\supseteq \left(\bigcup_{n\geq 1} F_{n, \varepsilon}\right) \setminus \left(\bigcup_{n\geq 1} \interior (F_{n, \varepsilon}) \right) \\
        &\supseteq X \setminus \left(\bigcup_{n\geq 1} \interior (P_{n, \varepsilon}) \right) \\
        &= X \setminus R_\varepsilon
        \end{align*}
        In other words, $R_\varepsilon \supseteq X\setminus A_\varepsilon$, which is dense by (Baire Category Thm. \Cref{Baire remark}).  
        
    \end{proof}
    By Claim 2, $R_{\frac{1}{n}}$ is open and dense, so by Claim 1 $R$ is dense, as the countable intersection of open and dense sets is dense (Baire Category Thm.).
\end{proof}
\begin{remark}
    As a corollary, the Dirichlet function $\mathbf{1}_\mathbb{Q}$ is nowhere continuous; hence there does not exist any continuous function $f_n: \real \to \real$  with $f_n \to \mathbf{1}_\mathbb{Q}$ point-wise.
\end{remark}
\end{document}