\documentclass{article}
\usepackage[utf8]{inputenc}
\usepackage{geometry}
\geometry{left=3cm,right=3cm,top=2cm,bottom=2cm}
\usepackage[utf8]{inputenc}
\usepackage{amsmath, amsfonts, amssymb, amsthm}
\usepackage[framemethod=TikZ]{mdframed}
\usepackage{mathrsfs}
\usepackage{comment}
\usepackage{enumerate}
\usepackage{xcolor}
\usepackage{titlesec}
\usepackage{setspace}
\usepackage[hidelinks,backref]{hyperref}
\usepackage{cleveref}
\usepackage[most]{tcolorbox}
\usepackage{ragged2e}
\usepackage{todonotes}
\usepackage{cleveref}
\usepackage{mathtools}

%
\titleformat*{\section}{\LARGE \bfseries}
\titleformat*{\subsection}{\Large \bfseries}
\titleformat*{\subsubsection}{\Large \bfseries}
% \titleformat*{\paragraph}{\large \bfseries}
\titleformat*{\subparagraph}{\large \bfseries}



% General 
\newcommand{\nextline}{\hfill\break}
\newcommand{\nl}{\nextline\rm}
% \newcommand{\placeholder}{{\bf\color{red} NOOOOOOT COMPLEEEEEEET! COOOOOOOM BAAAAAACK!!!}}
\newcommand{\placeholder}{\todo{NOOOOOOT COMPLEEEEEEET! COOOOOOOM BAAAAAACK!!!}}

\newcommand{\defeq}{\stackrel{def.}{=}}
% FA and LA
% inner product: \inne{a}{b}
\newcommand{\inne}[2]{\left<{#1},{#2}\right>}

% norm: \norm{a}
\newcommand{\norm}[1]{\left\|{#1}\right\|}

% Curly H
\newcommand{\hbs}{$\mathscr{H}$ }
\newcommand{\hbp}{\mathscr{H}}

% Dual : \dual{x}
\newcommand{\dual}[1]{{#1}^*}

% Sequence from 1  to infty: \sequ{x_n}
\newcommand{\sequ}[1]{\left({#1}\right)_1^\infty}

% f: A-> B \func{f}{A}{B}
\newcommand{\func}[3]{${#1}:{#2}\xrightarrow{}{#3}$}

% interior
\newcommand{\interior}{\textrm{int}}

% Bounded linear funcs
\newcommand{\blf}[2]{\mathcal{L}({#1},{#2})}

\newcommand{\prf}{\textit{proof}:   }



% Fields 
\newcommand{\real}{\mathbb{R}}
\newcommand{\comp}{\mathbb{C}}
\newcommand{\inte}{\mathbb{Z}}
\newcommand{\natu}{\mathbb{N}}





% Theorems
% \newtheorem{example}{Example}[subsection]
% \newtheorem{definition}[example]{Definition}
% \newtheorem{proposition}[example]{Proposition}
% \newtheorem{remark}[example]{Remark}
% \newtheorem{theorem}[example]{Theorem}
% \newtheorem{lemma}[example]{Lemma}
% \newtheorem{corollary}[example]{Corollary}


% for numbering the theorems            
\theoremstyle{plain}
%%%%%%%%%%%%%%%%%%%%%%%%%%%%%%%%%%%%%%%%%%%%%%%%%%%%%%%%%%%
\newtheorem{theorem}{Theorem}[section]
\newtheorem{lemma}[theorem]{Lemma}
\newtheorem{corollary}[theorem]{Corollary}
\newtheorem{proposition}[theorem]{Proposition}
%%%%%%%%%%%%%%%%%%%%%%%%%%%%%%%%%%%%%%%%%%%%%%%%%%%%%%%%%%%
% the following are not in italics
\theoremstyle{definition}
\newtheorem{definition}[theorem]{Definition}
\newtheorem{example}[theorem]{Example}
\newtheorem{remark}[theorem]{Remark}
\newtheorem{claim}[theorem]{Claim}
%%%%%%%%%%%%%%%%%%%%%%%%%%%%%%%%%%%%%%%%%%%%%%%%%%%%%%%%%%%%

% proof box
\newtcbtheorem[no counter]{pf}{Proof}{
  enhanced,
  rounded corners,
  attach boxed title to top,
  colback=white,
  colframe=black!25,
  fonttitle=\bfseries,
  coltitle=black,
  boxed title style={
    rounded corners,
    size=small,
    colback=black!25,
    colframe=black!25,
  } 
}{prf}

% extra content box to put in contents not covered in the lecture notes
% use the command \begin{unexaminable}
\newmdenv[skipabove=7pt, skipbelow=7pt,
    rightline=false, leftline=false, topline=false, bottomline=false,
    backgroundcolor = gray!10,
    innerleftmargin=1in, innerrightmargin=1in, innertopmargin=5pt,
    leftmargin=-1in, rightmargin=-1in, linewidth=4pt,
    innerbottommargin=5pt]{unexamBox}
\newenvironment{unexaminable}{\begin{unexamBox}}{\end{unexamBox}}


% clever ref settings
\crefname{lemma}{lemma}{lemmas}
\Crefname{lemma}{Lemma}{Lemmas}
\crefname{theorem}{theorem}{theorems}
\Crefname{theorem}{Theorem}{Theorems}

% formatting 
% https://tex.stackexchange.com/questions/217497/aligning-stackrel-signs-beneath-each-other-using-split
\newlength{\leftstackrelawd}
\newlength{\leftstackrelbwd}
\def\leftstackrel#1#2{\settowidth{\leftstackrelawd}%
{${{}^{#1}}$}\settowidth{\leftstackrelbwd}{$#2$}%
\addtolength{\leftstackrelawd}{-\leftstackrelbwd}%
\leavevmode\ifthenelse{\lengthtest{\leftstackrelawd>0pt}}%
{\kern-.5\leftstackrelawd}{}\mathrel{\mathop{#2}\limits^{#1}}}

\doublespacing
\RaggedRight

\title{Week 6}

\begin{document}
\maketitle

\section{Uniform Boundedness principle}
Uniform boundedness principle is sometimes called Banach–Steinhaus theorem. In its basic form, it asserts that for a family of bounded linear operators  whose domain is a Banach space, pointwise boundedness is equivalent to uniform boundedness in operator norm.

\begin{theorem}[Uniform Boundedness principle]\rm\nextline
	Let $X$ be a Banach space, $Y$ a normed vector space. Let $F$ be a collection of bounded linear operators from $X$ to $Y$. Then if
	$$
		\sup_{f\in F}\norm{fx}<\infty,,\,\forall x\in X
	$$
	Then
	$$
		\sup_{f\in F}\norm{f}<\infty
	$$
\end{theorem}
To prove this theorem we shall use {\textbf{Baire category theorem}}.

\begin{definition}[Nowhere dense]\rm\nextline
	A set $S$ in metric space $X$ is {\bf nowhere dense} if its closure has empty interior. i.e. $\overline{S}\strut^\mathrm{o}=\emptyset$.
\end{definition}

\begin{theorem}[Baire Category Theorem]\rm\nextline
	A complete metric space is not countable union of nowhere dense set.
\begin{pf}{Baire Category Theorem}{}
	The idea of proof is to construct a Cauchy sequence in the space with no limit point, giving contradiction. First let $M$ be a complete metric space. Assume
	$$M=\bigcup_{n=1}^\infty A_n$$
	We know that $A_1$ is nowhere dense, which means $\overline{A_1}\strut^\mathrm{o}=\emptyset$. Since $\overline{A_1}$ is closed, $M\backslash \overline{A_1}$ is open, we may find open ball $B_1$ with radius $r_1<1$ such that $B_1\cap\overline{A_1}=\emptyset$. Clearly we have that $B_1\not\subset\overline{A_2}$ otherwise $\overline{A_2}$ has non-empty interior. So we have that $(M\backslash\overline{A_2})\cap B_1$ is open and non empty. Now we may choose open ball $B_2\subset((M\backslash\overline{A_2})\cap B_1)\subset B_1$ with radius $r_2<1/2$. \\
	We repeat this process, so that $B_n\subset((M\backslash\overline{A_n})\cap B_{n-1})\subset B_{n-1}$ is an  open ball with radius $r_n<2^{1-n}$. and name the center of the open ball $B_i$ to be $x_i$. Clearly $\{x_i\}$ gives a Cauchy sequence (why?). Thus it converges to a point $x$. Since $x$ is a limit point in open ball $B_j$, it has  the property that
	$$x\in \overline{B_{j+1}}\subset B_j\subset (M\backslash\overline{A_j})\subset(M\backslash A_j),\,\forall j\in\mathbb{N}$$
	So we have that $x\not\in A_j,\,\forall j\in\mathbb{N}$.
	So
	$$
		x\not\in \bigcup_{j=1}^{\infty}A_j=M
	$$
	Which leads to contradiction.
	\end{pf}
\end{theorem}

\begin{pf}{Uniform Boundedness principle}{}
Let $A_n=\{
	x\in X, \norm{fx}\leq n,\,\forall f\in F
	\},\,n\in\mathbb{N}$. By assumption we have $\bigcap_{n=1}^\infty A_n=X$.\\
We claim that there exists some $j\in \mathbb{N}$ such that $A_j$ is non-empty and closed. To see this, first by by Baire category theorem, there is some $A_j$ such that $\overline{A_j}\strut^\mathrm{o}\not=\emptyset$. Then let $\{x_m\}$ be a Cauchy sequence in $A_j$ with $x_n\xrightarrow{}x$, then by continuity of $f$, $\norm{fx}=\lim_{n\to\infty}\norm{fx_m}\leq n,\,\forall f\in F$. So $x\in A_j$, hence $A_j$ is closed, thus $\overline{A_j}=A_j$, $A_j\strut^\mathrm{o}=\overline{A_j}\strut^\mathrm{o}\not=\emptyset$. So we can choose a point $p$ from interior of $A_j$, and $\varepsilon>0$ such that open ball $B_{\varepsilon}(p)\subseteq A_j$.\\
Now for any $x<\norm{\varepsilon}$ with any $T\in F$ we have
$$
	\norm{T(x)}=\norm{T(x+p-p)}=\norm{T(x+p)-T(p)}\leq\norm{T(x+p)}+\norm{T(p)}\leq n+n=2n
$$
So for any non-zero vector $x\in X$, we have
$$
	\norm{T(x)}=\frac{\norm{x}}{\varepsilon}\norm{T(\varepsilon \frac{x}{\norm{x}})}\leq\frac{2n}{\varepsilon}\norm{x}
$$
This holds for any $T\in F$, thus
$$
	\sup_{f\in F}\norm{f}\leq\frac{2n}{\varepsilon}<\infty
$$
\end{pf}
A simple corollary of the theorem is Banach limit.
\begin{corollary}[Banach Limit]\rm\nextline
	Let $T_n:X\xrightarrow{}Y$ be a sequence of operators, where $X$ and $Y$ are Banach spaces. Suppose $\{T_n\}$ converges pointwise,
	then these pointwise limits define a bounded linear operator $T$.
\end{corollary}

\end{document}