\documentclass{article}

\usepackage{geometry}
\geometry{left=3cm,right=3cm,top=2cm,bottom=2cm}
\usepackage[utf8]{inputenc}
\usepackage{amsmath, amsfonts, amssymb, amsthm}
\usepackage[framemethod=TikZ]{mdframed}
\usepackage{mathrsfs}
\usepackage{comment}
\usepackage{enumerate}
\usepackage{xcolor}
\usepackage{titlesec}
\usepackage{setspace}
\usepackage[hidelinks,backref]{hyperref}
\usepackage{cleveref}
\usepackage[most]{tcolorbox}
\usepackage{ragged2e}
\usepackage{todonotes}
\usepackage{cleveref}
\usepackage{mathtools}

%
\titleformat*{\section}{\LARGE \bfseries}
\titleformat*{\subsection}{\Large \bfseries}
\titleformat*{\subsubsection}{\Large \bfseries}
% \titleformat*{\paragraph}{\large \bfseries}
\titleformat*{\subparagraph}{\large \bfseries}



% General 
\newcommand{\nextline}{\hfill\break}
\newcommand{\nl}{\nextline\rm}
% \newcommand{\placeholder}{{\bf\color{red} NOOOOOOT COMPLEEEEEEET! COOOOOOOM BAAAAAACK!!!}}
\newcommand{\placeholder}{\todo{NOOOOOOT COMPLEEEEEEET! COOOOOOOM BAAAAAACK!!!}}

\newcommand{\defeq}{\stackrel{def.}{=}}
% FA and LA
% inner product: \inne{a}{b}
\newcommand{\inne}[2]{\left<{#1},{#2}\right>}

% norm: \norm{a}
\newcommand{\norm}[1]{\left\|{#1}\right\|}

% Curly H
\newcommand{\hbs}{$\mathscr{H}$ }
\newcommand{\hbp}{\mathscr{H}}

% Dual : \dual{x}
\newcommand{\dual}[1]{{#1}^*}

% Sequence from 1  to infty: \sequ{x_n}
\newcommand{\sequ}[1]{\left({#1}\right)_1^\infty}

% f: A-> B \func{f}{A}{B}
\newcommand{\func}[3]{${#1}:{#2}\xrightarrow{}{#3}$}

% interior
\newcommand{\interior}{\textrm{int}}

% Bounded linear funcs
\newcommand{\blf}[2]{\mathcal{L}({#1},{#2})}

\newcommand{\prf}{\textit{proof}:   }



% Fields 
\newcommand{\real}{\mathbb{R}}
\newcommand{\comp}{\mathbb{C}}
\newcommand{\inte}{\mathbb{Z}}
\newcommand{\natu}{\mathbb{N}}





% Theorems
% \newtheorem{example}{Example}[subsection]
% \newtheorem{definition}[example]{Definition}
% \newtheorem{proposition}[example]{Proposition}
% \newtheorem{remark}[example]{Remark}
% \newtheorem{theorem}[example]{Theorem}
% \newtheorem{lemma}[example]{Lemma}
% \newtheorem{corollary}[example]{Corollary}


% for numbering the theorems            
\theoremstyle{plain}
%%%%%%%%%%%%%%%%%%%%%%%%%%%%%%%%%%%%%%%%%%%%%%%%%%%%%%%%%%%
\newtheorem{theorem}{Theorem}[section]
\newtheorem{lemma}[theorem]{Lemma}
\newtheorem{corollary}[theorem]{Corollary}
\newtheorem{proposition}[theorem]{Proposition}
%%%%%%%%%%%%%%%%%%%%%%%%%%%%%%%%%%%%%%%%%%%%%%%%%%%%%%%%%%%
% the following are not in italics
\theoremstyle{definition}
\newtheorem{definition}[theorem]{Definition}
\newtheorem{example}[theorem]{Example}
\newtheorem{remark}[theorem]{Remark}
\newtheorem{claim}[theorem]{Claim}
%%%%%%%%%%%%%%%%%%%%%%%%%%%%%%%%%%%%%%%%%%%%%%%%%%%%%%%%%%%%

% proof box
\newtcbtheorem[no counter]{pf}{Proof}{
  enhanced,
  rounded corners,
  attach boxed title to top,
  colback=white,
  colframe=black!25,
  fonttitle=\bfseries,
  coltitle=black,
  boxed title style={
    rounded corners,
    size=small,
    colback=black!25,
    colframe=black!25,
  } 
}{prf}

% extra content box to put in contents not covered in the lecture notes
% use the command \begin{unexaminable}
\newmdenv[skipabove=7pt, skipbelow=7pt,
    rightline=false, leftline=false, topline=false, bottomline=false,
    backgroundcolor = gray!10,
    innerleftmargin=1in, innerrightmargin=1in, innertopmargin=5pt,
    leftmargin=-1in, rightmargin=-1in, linewidth=4pt,
    innerbottommargin=5pt]{unexamBox}
\newenvironment{unexaminable}{\begin{unexamBox}}{\end{unexamBox}}


% clever ref settings
\crefname{lemma}{lemma}{lemmas}
\Crefname{lemma}{Lemma}{Lemmas}
\crefname{theorem}{theorem}{theorems}
\Crefname{theorem}{Theorem}{Theorems}

% formatting 
% https://tex.stackexchange.com/questions/217497/aligning-stackrel-signs-beneath-each-other-using-split
\newlength{\leftstackrelawd}
\newlength{\leftstackrelbwd}
\def\leftstackrel#1#2{\settowidth{\leftstackrelawd}%
{${{}^{#1}}$}\settowidth{\leftstackrelbwd}{$#2$}%
\addtolength{\leftstackrelawd}{-\leftstackrelbwd}%
\leavevmode\ifthenelse{\lengthtest{\leftstackrelawd>0pt}}%
{\kern-.5\leftstackrelawd}{}\mathrel{\mathop{#2}\limits^{#1}}}

\doublespacing
\RaggedRight
% modify innerrightmargin if floats were lost

\usepackage{amsfonts, amsmath, amssymb, amsthm, thmtools, bm}
\usepackage{avant} % Use the Avantgarde font for headings

% Boxed/framed environments
\newtheoremstyle{royalnumbox}%
{0pt}% Space above
{0pt}% Space below
{\normalfont}% Body font
{}% Indent amount
{\small\bf\sffamily\color{royal}}% Theorem head font
{\;}% Punctuation after theorem head
{0.25em}% Space after theorem head
{\sffamily \color{royal} 
    \thmname{#1} 
    \thmnumber{#2} \thmnote{\bfseries\color{black}---\nobreakspace#3.}} % Optional theorem note
\renewcommand{\qedsymbol}{$\blacksquare$}% Optional qed square

\newtheoremstyle{blacknumex}% Theorem style name
{5pt}% Space above
{5pt}% Space below
{\normalfont}% Body font
{} % Indent amount
{\small\bf\sffamily}% Theorem head font
{\;}% Punctuation after theorem head
{0.25em}% Space after theorem head
{\sffamily
    \thmname{#1}
    \thmnumber{#2}
    \thmnote{---\nobreakspace#3.}}% Optional theorem note

\newtheorem*{notation}{Notation}
\newtheorem*{hint}{Hint}
\newtheorem*{solution}{Solution}

\newcounter{dummy} 
\numberwithin{dummy}{section}

\theoremstyle{royalnumbox}
\newtheorem{definitionT}[dummy]{Definition}
\newtheorem{theoremT}[dummy]{Theorem}
\newtheorem{lemmaT}[dummy]{Lemma}
\newtheorem{corollaryT}[dummy]{Corollary}
\newtheorem{propositionT}[dummy]{Proposition}
\newtheorem{propertyT}[dummy]{Property}
\newtheorem{remarkT}[dummy]{Remark}

\theoremstyle{blacknumex}
\newtheorem{exampleT}[dummy]{Example}
\newtheorem{exerciseT}[dummy]{Exercise}

\numberwithin{equation}{section}

\RequirePackage[framemethod=default]{mdframed}

% Definition box
\newmdenv[skipabove=7pt, skipbelow=7pt,
rightline=false, leftline=true, topline=false, bottomline=false,
backgroundcolor = reddish!10, 
linecolor=reddish,
innerleftmargin=5pt, innerrightmargin=30pt, innertopmargin=5pt,
leftmargin=0cm, rightmargin=0cm, linewidth=4pt,
innerbottommargin=5pt]{dBox}

% Main Theorem box
\newmdenv[skipabove=7pt, skipbelow=7pt,
rightline=false, leftline=true, topline=false, bottomline=false,
backgroundcolor=c0!10, 
linecolor=c0,
innerleftmargin=5pt, innerrightmargin=30pt, innertopmargin=5pt,
leftmargin=0cm, rightmargin=0cm, linewidth=4pt, innerbottommargin=5pt]{tBox}

% Lemma/Corollary/Proposition/Property box
\newmdenv[skipabove=7pt, skipbelow=7pt,
rightline=false, leftline=true, topline=false, bottomline=false,
backgroundcolor = c0!10, 
linecolor=c0!80,
innerleftmargin=5pt, innerrightmargin=30pt, innertopmargin=5pt,
leftmargin=0cm, rightmargin=0cm, linewidth=4pt,
innerbottommargin=5pt]{lBox}

% Example/Remark/Exercise box
\newmdenv[skipabove=7pt, skipbelow=7pt,
rightline=false, leftline=true, topline=false, bottomline=false,
backgroundcolor = mossgreen!10!white,
linecolor = mossgreen,
innerleftmargin=5pt, innerrightmargin=30pt, innertopmargin=5pt,
leftmargin=0cm, rightmargin=0cm, linewidth=4pt,
innerbottommargin=5pt]{exBox}

% Proof box
% \newmdenv[skipabove=7pt, skipbelow=7pt,
% rightline=false, leftline=true, topline=false, bottomline=false,
% linecolor=gray,
% innerleftmargin=5pt, innerrightmargin=30pt, innertopmargin=5pt,
% leftmargin=0cm, rightmargin=0cm, linewidth=4pt,
% innerbottommargin=5pt]{proofBox}



% Creates an environment for each type of theorem and assigns it a theorem text style from the "Theorem Styles" section above and a colored box from above
\newenvironment{definition}{\begin{dBox}\begin{definitionT}}{\end{definitionT}\end{dBox}}
\newenvironment{theorem}{\begin{tBox}\begin{theoremT}}{\end{theoremT}\end{tBox}}
\newenvironment{lemma}{\begin{lBox}\begin{lemmaT}}{\end{lemmaT}\end{lBox}}
\newenvironment{proposition}{\begin{lBox}\begin{propositionT}}{\end{propositionT}\end{lBox}}
\newenvironment{corollary}{\begin{lBox}\begin{corollaryT}}{\end{corollaryT}\end{lBox}}
\newenvironment{property}{\begin{lBox}\begin{propertyT}}{\end{propertyT}\end{lBox}}


% \newenvironment{proof*}{\begin{proofBox}\begin{proof}}{\end{proof}\end{proofBox}}
\newenvironment{exercise}{\begin{exBox}\begin{exerciseT}}{\hfill{\color{royal}}\end{exerciseT}\end{exBox}}
\newenvironment{remark}{\begin{exBox}\begin{remarkT}}{\end{remarkT}\end{exBox}}
\newenvironment{example}{\begin{exBox}\begin{exampleT}}{{}\end{exampleT}\end{exBox}}

\title{Functional Analysis Chapters 4 and 5}

\date{\today}

\begin{document}

\maketitle

\section{Duality}  

Recall that the space of all bounded linear operators is defined as

\[
\mathcal{L}(X,Y) = \{A: X \to Y, A \textrm{ bounded, linear}\}
\] 
 
\(\mathcal{L}(X,Y)\) is Banach if  \(Y\) is Banach and it has norm  

\[
\norm{A} = \norm{A}_{\mathcal{L}(X,Y)} = \sup_{\norm{x}_X \leq 1} \norm{Ax}
\]  

An Important special case is  

\[
X^* \overset{\textrm{def}}{=} \mathcal{L}(X,\real)
\]  

which is the \textit{dual space} of  \(X\).  

\begin{definition}[Dual Spaces]\nextline
    The space of all continuous linear operators  \(\mathcal{L}(X, \real)\) is called the \textbf{dual space} of  \(X\) and is denoted as   \(X^*\)
\end{definition}  

\begin{remark}
     \(X^*\) is always Banach (eventhough  \(X\) may not be). We often abbreviate  \(\norm{\cdot}_* = \norm{\cdot}_{X^*}\). The elements of  \(X^*\) are called (bounded, linear) \textbf{functionals}.
\end{remark}

Dual spaces play a central role in functional analysis. They are easiest to grasp in the following contexts.  

\subsection{Duality in Hilbert Spaces}  
In this section, let  \((H, {\langle \cdot, \cdot \rangle}_H )\) be a Hilbert space over  \(\real\). For  \(y\in H\), we define the map  

\begin{equation*}
    \Lambda_y: X \to \real, \qquad x \mapsto {\langle y, x \rangle}_H
\end{equation*}  

We note that this is an injective map from  \(H\) to its dual  \(H^*\) and we will show that this is in fact a bijective isometry.  

\begin{lemma}
    (Mapping to dual space)
    \begin{enumerate}[i)]
        \item  \(\Lambda_y \in H^*\)
        \item The map  \(\Lambda: H \to H^*\) is a linear isometry with  \(||\Lambda_y||_{*}=||y||\)
    \end{enumerate}
\end{lemma}  

\begin{proof}
    \begin{enumerate}[i)]
        \item We need to check the linearity and boundedness of the operator  \(\Lambda_y^*\). The former follows from that of the inner product and the latter is proven by applying Cauchy-Schwarz  
        \begin{equation*}
            ||\Lambda_y||_* = \sup_{x\in H, ||x|| \leq 1} |\inne{y}{x}_H| \leq \norm{y}_H
        \end{equation*}  
        which implies  \(\Lambda_y \in H^*\)
        
        \item Choose  \(x = \frac{y}{\norm{y}_H}\) to attain the equality in the equation above , whence we have  \(\norm{\Lambda_y}_*=\norm{y}\).
    \end{enumerate}
\end{proof}

\begin{theorem}
    (Riesz Representation)  
    For every  \(\ell \in H^*\), there is a unique  \(y \in H\), such that  \(\ell=\Lambda_y\)
\end{theorem}  

\begin{proof}
    We show the existence and uniqueness of such a linear operator.  
    \begin{itemize}
        \item \textbf{(Existence)} If  \(\ell(x)\equiv 0\), then take  \(y=0\). Otherwise, assume  \(\norm{\ell}_*=1\) (as we can replace  \(\ell(\cdot)\) by  \(\frac{\ell(\cdot)}{\norm{\ell}_*}\)).  
        By the definition of  \(\norm{\cdot}_*\), there is a sequence of  \((y_n)_{n\in \natu} \subset H\) with  
        \begin{equation*}
            |\ell(y_n)| \to \norm{\ell}_*, \qquad \norm{y_n}=1, \forall n \in \natu
        \end{equation*}
        We will show that the limit of this sequence is the desired  \(y\).  
        
        \textbf{Claim 1:} The sequence  \((y_n)_{n \in \natu}\) is Cauchy   
        
        Apply the parallelogram identity to  \(x = \frac{y_n}{2}\) and  \(y = \frac{y_m}{2}\), so we have  
        \begin{equation*}
            \forall n,m \geq 1, \qquad \norm{\frac{y_n-y_m}{2}}^2=1-\norm{\frac{y_n+y_m}{2}}^2
        \end{equation*}  
        Using linearity and boundedeness of  \(\ell\),   
        \begin{equation*}
            \frac{1}{2} |l(y_n)+l(y_m)| = |l(\frac{y_n+y_m}{2})| \leq \norm{l}_* \norm{\frac{y_n+y_m}{2}}
        \end{equation*}  
        The LHS of the equation above converges to  \(1\) by assumption on  \((y_n)_{n \in \natu}\), which implies and  \((y_n)_{n \in \natu}\) is Cauchy.  
        Since  \(H\) is complete, there is a unique  \(y\), such that  \(y_n \to y\)  
        
        \textbf{Claim 2:}  \(\ell=\Lambda_y\)  
        
        Since  \(\textrm{span}\{y\}\) is closed, we can consider the orthogonal decomposition  \(H = \textrm{span}\{y\} \oplus (\textrm{span}\{y\})^{\perp}\).  
        It suffices to show:  
        \begin{enumerate}[(1)]
            \item  \(\ell(y)=\Lambda_y(y), \forall y \in \textrm{span}\{y\}\)
            \item  \(\ell(x)=\Lambda_y(x), \forall x \in (\textrm{span}\{y\})^{\perp}\)
        \end{enumerate}  
        To show (1), assume wlog  \(\norm{y}=1\), we note by continuity of  \(\ell\)  
        \begin{equation*}
            \ell(y) = \lim_{n \to \infty} |\ell(y_n)| = \norm{\ell}_*=1
        \end{equation*}  
        and 
        \begin{equation*}
            \norm{y}^2_H = \inne{y}{y}_H =\Lambda_y(y)=1
        \end{equation*}  
        So  \(\Lambda_y(y)=\ell(y)\).  
        
        To show (2), we need to argue that 
        \begin{equation*}
            \ell(x)=0, \qquad \forall x \in (\textrm{span}\{y\})^{\perp}
        \end{equation*}  
        Now take  \(y_a = \frac{y+ax}{\sqrt{1+a^2}}\) and  \(\norm{y_a}=1\), where  \(y \in \textrm{span}\{y\}, a\in \real\) and  \(x \in (\textrm{span}\{y\})^{\perp}\). By definition of the norm  \(\norm{\cdot}_*\) and (1),  
        \begin{equation*}
            \ell(y_a)  \leq |\ell(y_a)| \leq 1 = \ell(y)
        \end{equation*}  
        So  \(\ell(y_a)\) has a global maximum at  \(a=0\)  (\(y_0=y\)). Therefore,  
        \begin{equation*}
            0 = \frac{d}{da} \ell(y_a) \Bigr|_{a=0} = \frac{d}{da} \frac{1}{\sqrt{1+a^2}} (\ell(y)+a\ell(x))=\ell(x)
        \end{equation*}  
        So  \(\ell(x)=\Lambda_y(x), \forall x \in (\textrm{span}\{y\})^{\perp}\).
        
        \item \textbf{(Uniqueness)}  
        If  \(\ell=\Lambda_y=\Lambda_z\) for some  \(y,z \in H\), then  
        \[
        \forall x \in H \qquad \Lambda_y(x) = \inne{y}{x}_H = \inne{z}{x}_H = \Lambda_z(x)
        \]  
        Pick  \(x=y-z\), then  \(\inne{y}{z-x}_H = \inne{z}{z-x}_H \implies \norm{y-z}_H^2=0\). Hence  \(y=z\).
    \end{itemize}
\end{proof}

\begin{corollary}
    All Hilbert spaces  \(H\) are isomorphic to their duals  \(H^*\).
\end{corollary}  

\begin{unexaminable}
\begin{example}
    Some of the examples are:  
    \begin{itemize}
        \item  \((l^2)^* \cong l^2\)
        \item  \((L^2(\mu))^* \cong L^2(\mu)\)
    \end{itemize}
\end{example}
\end{unexaminable}

\subsection{Duality in Banach Spaces}  

\begin{theorem}
    (Conjugates are duals) For all  \(p \in (1, \infty)\),  \((\ell^p)^* \cong \ell^q\), where  \(\frac{1}{p}+\frac{1}{q}=1\)
\end{theorem}  

\begin{proof}
    For  \(y \in \ell^q\) define  
    \begin{align*}
        \Lambda_y: \ell^p &\to \real \\
        x &\mapsto \Lambda_y(x) = \sum_{n \in \natu} y_n x_n
    \end{align*}
\begin{lemma} (cf. \todo{what lemma})
    \label{isometry in lp dual}
\begin{enumerate}[i)]
    \item  \(\Lambda_y \in (\ell^p)^*\)
    \item  \(\Lambda: \ell^q \to (\ell^p)^*, y \mapsto \Lambda_y\) is a linear isometry
\end{enumerate}  
\end{lemma}
\begin{enumerate}[i)]
    \item By Hölder's inequality,  
    	\begin{equation*}
		\left|\Lambda_y(x)\right|
		\leq\sum_{n=1}^{\infty}|y_n x_n|
		\leq\norm{y}_q\norm{x}_p
	\end{equation*}
	in particular  \(\Lambda_y\) is well-defined (i.e. maps to  \(\real\)). The inequality implies  \(\norm{\Lambda_y}_* \leq \norm{y}_q\).   
	In fact, one has equality. Let  \(x=(x_n)_n\) with  
	\[x_n = \textrm{sign}(y_n) |y_n|^{q-1}\]
	where the sign function is  
	\begin{equation*}
	    \textrm{sign}(t) = \begin{cases}
	    1 & t \geq 0 \\
	    -1 & t<0
	\end{cases}
	\end{equation*}
	with  \(\textrm{sign}(t)t=|t|, \forall t \in \real\).  
	Then  \(x \in \ell^p\) as  \(|x_n|^p=|y_n|^{p(q-1)}=|y_n|^q\) so  
	\[
	\norm{x}_p = \left(\sum_{n \in \natu} |y_n|^q\right)^{\frac{1}{p}} = \norm{y}_q^{\frac{q}{p}} = \norm{y}_q^{q-1}. 
	\]
	and
	\[
	|\Lambda_y(x)| = |\sum_{n \in \natu} x_ny_n| = \sum_{n \in \natu} |y_n|^q = \norm{y}^q_q =\norm{y}_q\norm{x}_p 
	\]
	which implies  
	\[
        \norm{\Lambda_y}_* \geq \frac{|\Lambda_y(x)|}{\norm{x}_p} = \frac{\Lambda_y(x)}{\norm{x}_p} = \norm{y}_q
	\]
        from which ii) follows.
\end{enumerate}
To complete the proof of the theorem, we have to show the following analogue of Riesz representation theorem (which applies only when  \(p=q=2\)).  

\begin{lemma}
     \(\Lambda: \ell^q \to (\ell^p)^*\) given by  \(y \mapsto \Lambda_y\) is surjective (onto).
\end{lemma}  
Let  \(e_n = (0, \ldots, 0, 1, 0, \ldots) \in \ell^p\) and define  \(y_n = \ell(e_n)\) for  \(\ell \in (\ell^p)^*\)

\textbf{Claim:} 
\begin{enumerate}[i)]
    \item  \(y = (y_n) \in \ell^q\)
    \item  \(\ell = \Lambda_y\) for some  \(y\)
\end{enumerate}  

Consider the "truncated y":  \(y^{(n)}=(y_1, \ldots, y_n, 0, \ldots) = \sum_{i=1}^n y_i e_i \in \ell^q\) and 
\[x^{(n)}=\sum_{i=1}^{n}|y_{i}|^{q-1} \textrm{sign}(y_i) e_i \in \ell^p\] 

Then as before:  \(\norm{x^{(n)}}_p = \norm{y^{(n)}}_q^{q-1}\)  
with 
\[
\ell(x^{(n)})=\sum_{i=1}^{n}|y_{i}|^{q-1} \textrm{sign}(y_i) \ell(e_i) = \sum_{i=1}^n |y_i|^q=\norm{y^{(n)}}^q_q  
\]
where  \(y_i = \ell(e_i)\). Hence
\[
 \left(\sum_{i=1}^n |y_i|^q \right)^{\frac{1}{q}} = \norm{y^{(n)}}_q = \frac{\ell(x^{(n)})}{\norm{y^{(n)}}_q^{q-1}} = \frac{\ell(x^{(n)})}{\norm{x^{(n)}}_p} \leq \norm{\ell}_* < \infty
\]
and letting  \(n \to \infty\), Claim (i) follows.  

For ii), let  \(x \in \ell^p\) and  \(\varepsilon > 0\). Since  \(e_n\)'s form a Schauder basis, we know that  

\[\norm{x^{(n)}-x}_p \overset{n \to \infty}{\rightarrow} 0\]  

(since by definition of Schauder basis, we have unique rep. of  \(x = \sum_{n \in \natu} x_n e_n\))

By choosing  \(n\) large, we can ensure that  

\[\norm{l(x)-l(x^{(n)})}<\varepsilon/2, \qquad |\Lambda_y(x)-\Lambda_y(x^{(n)})| < \frac{\varepsilon}{2}\]  

using continuity of  \(\ell(\cdot)\) and  \(\Lambda_y(\cdot)\), where the latter is a consequence of \Cref{isometry in lp dual}.  

But writing  

\begin{equation}
|\ell(x)-\Lambda_y(x)| \leq |\ell(x)- \ell(x^{(n)})| + |\ell(x^{(n)}) - \Lambda_y(x^{(n)})|+|\Lambda_y(x^{(n)})-\Lambda_y(x)|
\end{equation}  
and observing that  \(\ell(x^{(n)})=\sum_{i=1}^n x_i \ell(e_i) = \sum_{i=1}^n x_i y_i = \Lambda_y(x^{(n)})\), it follows that  

\[
|\ell(x)-\Lambda_y(x)| \leq \varepsilon \quad \rm{and} \quad \ell=\Lambda_y
\]  
by letting  \(\varepsilon \downarrow 0\)
\end{proof}

\begin{remark}
    \begin{enumerate}[1)]
        \item The proof extends to  \(p=1\), so  \((\ell^1)^* = \ell^{\infty}\). In fact, for  \((X, \mathcal{A}, \mu)\), one has  
        \[L^p(\mu)^* \cong L^q(\mu) \qquad \forall p \in [1, \infty)\]
        \item For  \(p=\infty\), one can still define  
        \[\Lambda: \ell^1 \to (\ell^{\infty})^* \qquad y \mapsto \Lambda_y\]
        as before and check that  \(\Lambda\) is a linear isometry between Banach spaces.
        
        However, it is \textbf{not} surjective.  
        
        To see this, consider 
        \[c_0 = \{(x_n): \lim_{n \to \infty} = 0\} \subset \ell^{\infty}\]
    \end{enumerate}

\textbf{Claim:} \((c_0)^* \cong \ell^1\ \)
\begin{proof}
    (Sketch) We show the following statements are true:
    \begin{enumerate}[1)]
        \item \(\Lambda_y: c_0 \to \real, x \mapsto \Lambda_y(x)\) is well-defined and bounded for any 
         \(y \in \ell^1\).  
        \[|\Lambda_y(x)| \leq \norm{x}_{\infty} \norm{y}_1\]

        \item The map \( \ell^1 \to c_0^*, y \mapsto \Lambda_y\) is a linear isometry. To check 
        \(\norm{\Lambda_y}_* \geq \norm{y}_1\), use \(x = \sum_{i=1}^n \rm{sign}(y_i)e_i\), as before, where
        \(n \geq 1\).  

        Clearly, \(x \in c_0\) and  
        \[\Lambda_y(x) = \sum_{i=1}^n \rm{sign}(y_i) \Lambda_y(e_i)=\sum_{i=1}^n |y_i|\]
        so for \(y \neq 0\)  
        \[\norm{\Lambda_y}_* \geq \frac{|\Lambda_y(x)|}{\norm{x}_{\infty}}=|\Lambda_y(x)|=\Lambda_y(x)=\sum_{i=1}^n |y_i|\]

        and letting \(n \to \infty\) gives the result.  

        \item \(y \mapsto \Lambda_y\) is onto: similar as before (exercise),
    \end{enumerate}
\end{proof}
\end{remark}  

\begin{definition}
    (Dual Operators) 
    \((X, \norm{\cdot}_X)\), \((Y, \norm{\cdot}_Y)\) are normed spaces over \(\real\), and \(A: X \to Y\) is a bounded linear operator.  
    Then the dual operator 
    \[A^*: Y^* \to X^*\]
    is defined by  
    \[A^* y^* \overset{\rm{def}}{=} y^* \circ A \qquad \forall y^* \in Y^* \]
    where \(A^* y^* : X \to \real\)
\end{definition}  

A note on the notation: For \(\ell \in X^*\), instead of \(\ell(x)\), we write \(\inne{\ell}{x}\). Then the above is equivalent to
\[\inne{A^* y^*}{x} = \inne{y^*}{A x} \qquad \forall x \in X, y^* \in Y^*\]  

Later we will show that if \(A\) is a bounded linear operator, then \(A^*\) is also bounded and \(\norm{A^*}=\norm{A}\)
using Hahn-Banach; if \(X, Y\) are Hilbert spaces, then we can use the Riesz representation theorem instead.

\begin{example}
    \begin{enumerate}[1)]
        \item \(A \in \real^{m \times n}\) a real matrix, which induces linear map \(L_A: \real^n \to \real^m, L_Ax=Ax\).
              If \(A^T\) is the transpose of \(A\) and \(i_k: \real^k \to (\real^k)^* \) is the canonical isomorphisim, then
              \[(L_A)^* \circ i_m = i_n \circ L_{A^T}: \qquad \real^m \to (\real^n)^*\]
        \item More generally, if \(H\) is a Hilbert space, \(A: H \to H\) is a bounded linear operator with \(\Lambda: H\to H^*\) as the
                canonical isomorphism, the operator
                \[\tilde{A^*} \overset{\rm{def}}{=} \Lambda^{-1} \circ A^* \Lambda: H \to H\]
            is called the adjoint of \(A\) (and one writes \(A^*\) for \(\tilde{A^*}\) with abuse of notation). Thus,
            \[\inne{\tilde{A^*}y}{x} = \inne{y}{Ax}, \qquad x,y \in H\]
            where \(\inne{\cdot}{\cdot}\) is the inner product on \(H\). If \(A=\tilde{A^*}\), then \(A\) is \textbf{self-adjoint}.
    \end{enumerate}
\end{example}  

\end{document}