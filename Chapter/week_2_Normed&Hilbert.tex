\documentclass{article}
\usepackage[utf8]{inputenc}
\usepackage{geometry}
\geometry{left=3cm,right=3cm,top=2cm,bottom=2cm}
\usepackage[utf8]{inputenc}
\usepackage{amsmath, amsfonts, amssymb, amsthm}
\usepackage[framemethod=TikZ]{mdframed}
\usepackage{mathrsfs}
\usepackage{comment}
\usepackage{enumerate}
\usepackage{xcolor}
\usepackage{titlesec}
\usepackage{setspace}
\usepackage[hidelinks,backref]{hyperref}
\usepackage{cleveref}
\usepackage[most]{tcolorbox}
\usepackage{ragged2e}
\usepackage{todonotes}
\usepackage{cleveref}
\usepackage{mathtools}

%
\titleformat*{\section}{\LARGE \bfseries}
\titleformat*{\subsection}{\Large \bfseries}
\titleformat*{\subsubsection}{\Large \bfseries}
% \titleformat*{\paragraph}{\large \bfseries}
\titleformat*{\subparagraph}{\large \bfseries}



% General 
\newcommand{\nextline}{\hfill\break}
\newcommand{\nl}{\nextline\rm}
% \newcommand{\placeholder}{{\bf\color{red} NOOOOOOT COMPLEEEEEEET! COOOOOOOM BAAAAAACK!!!}}
\newcommand{\placeholder}{\todo{NOOOOOOT COMPLEEEEEEET! COOOOOOOM BAAAAAACK!!!}}

\newcommand{\defeq}{\stackrel{def.}{=}}
% FA and LA
% inner product: \inne{a}{b}
\newcommand{\inne}[2]{\left<{#1},{#2}\right>}

% norm: \norm{a}
\newcommand{\norm}[1]{\left\|{#1}\right\|}

% Curly H
\newcommand{\hbs}{$\mathscr{H}$ }
\newcommand{\hbp}{\mathscr{H}}

% Dual : \dual{x}
\newcommand{\dual}[1]{{#1}^*}

% Sequence from 1  to infty: \sequ{x_n}
\newcommand{\sequ}[1]{\left({#1}\right)_1^\infty}

% f: A-> B \func{f}{A}{B}
\newcommand{\func}[3]{${#1}:{#2}\xrightarrow{}{#3}$}

% interior
\newcommand{\interior}{\textrm{int}}

% Bounded linear funcs
\newcommand{\blf}[2]{\mathcal{L}({#1},{#2})}

\newcommand{\prf}{\textit{proof}:   }



% Fields 
\newcommand{\real}{\mathbb{R}}
\newcommand{\comp}{\mathbb{C}}
\newcommand{\inte}{\mathbb{Z}}
\newcommand{\natu}{\mathbb{N}}





% Theorems
% \newtheorem{example}{Example}[subsection]
% \newtheorem{definition}[example]{Definition}
% \newtheorem{proposition}[example]{Proposition}
% \newtheorem{remark}[example]{Remark}
% \newtheorem{theorem}[example]{Theorem}
% \newtheorem{lemma}[example]{Lemma}
% \newtheorem{corollary}[example]{Corollary}


% for numbering the theorems            
\theoremstyle{plain}
%%%%%%%%%%%%%%%%%%%%%%%%%%%%%%%%%%%%%%%%%%%%%%%%%%%%%%%%%%%
\newtheorem{theorem}{Theorem}[section]
\newtheorem{lemma}[theorem]{Lemma}
\newtheorem{corollary}[theorem]{Corollary}
\newtheorem{proposition}[theorem]{Proposition}
%%%%%%%%%%%%%%%%%%%%%%%%%%%%%%%%%%%%%%%%%%%%%%%%%%%%%%%%%%%
% the following are not in italics
\theoremstyle{definition}
\newtheorem{definition}[theorem]{Definition}
\newtheorem{example}[theorem]{Example}
\newtheorem{remark}[theorem]{Remark}
\newtheorem{claim}[theorem]{Claim}
%%%%%%%%%%%%%%%%%%%%%%%%%%%%%%%%%%%%%%%%%%%%%%%%%%%%%%%%%%%%

% proof box
\newtcbtheorem[no counter]{pf}{Proof}{
  enhanced,
  rounded corners,
  attach boxed title to top,
  colback=white,
  colframe=black!25,
  fonttitle=\bfseries,
  coltitle=black,
  boxed title style={
    rounded corners,
    size=small,
    colback=black!25,
    colframe=black!25,
  } 
}{prf}

% extra content box to put in contents not covered in the lecture notes
% use the command \begin{unexaminable}
\newmdenv[skipabove=7pt, skipbelow=7pt,
    rightline=false, leftline=false, topline=false, bottomline=false,
    backgroundcolor = gray!10,
    innerleftmargin=1in, innerrightmargin=1in, innertopmargin=5pt,
    leftmargin=-1in, rightmargin=-1in, linewidth=4pt,
    innerbottommargin=5pt]{unexamBox}
\newenvironment{unexaminable}{\begin{unexamBox}}{\end{unexamBox}}


% clever ref settings
\crefname{lemma}{lemma}{lemmas}
\Crefname{lemma}{Lemma}{Lemmas}
\crefname{theorem}{theorem}{theorems}
\Crefname{theorem}{Theorem}{Theorems}

% formatting 
% https://tex.stackexchange.com/questions/217497/aligning-stackrel-signs-beneath-each-other-using-split
\newlength{\leftstackrelawd}
\newlength{\leftstackrelbwd}
\def\leftstackrel#1#2{\settowidth{\leftstackrelawd}%
{${{}^{#1}}$}\settowidth{\leftstackrelbwd}{$#2$}%
\addtolength{\leftstackrelawd}{-\leftstackrelbwd}%
\leavevmode\ifthenelse{\lengthtest{\leftstackrelawd>0pt}}%
{\kern-.5\leftstackrelawd}{}\mathrel{\mathop{#2}\limits^{#1}}}

\doublespacing
\RaggedRight
% modify innerrightmargin if floats were lost

\usepackage{amsfonts, amsmath, amssymb, amsthm, thmtools, bm}
\usepackage{avant} % Use the Avantgarde font for headings

% Boxed/framed environments
\newtheoremstyle{royalnumbox}%
{0pt}% Space above
{0pt}% Space below
{\normalfont}% Body font
{}% Indent amount
{\small\bf\sffamily\color{royal}}% Theorem head font
{\;}% Punctuation after theorem head
{0.25em}% Space after theorem head
{\sffamily \color{royal} 
    \thmname{#1} 
    \thmnumber{#2} \thmnote{\bfseries\color{black}---\nobreakspace#3.}} % Optional theorem note
\renewcommand{\qedsymbol}{$\blacksquare$}% Optional qed square

\newtheoremstyle{blacknumex}% Theorem style name
{5pt}% Space above
{5pt}% Space below
{\normalfont}% Body font
{} % Indent amount
{\small\bf\sffamily}% Theorem head font
{\;}% Punctuation after theorem head
{0.25em}% Space after theorem head
{\sffamily
    \thmname{#1}
    \thmnumber{#2}
    \thmnote{---\nobreakspace#3.}}% Optional theorem note

\newtheorem*{notation}{Notation}
\newtheorem*{hint}{Hint}
\newtheorem*{solution}{Solution}

\newcounter{dummy} 
\numberwithin{dummy}{section}

\theoremstyle{royalnumbox}
\newtheorem{definitionT}[dummy]{Definition}
\newtheorem{theoremT}[dummy]{Theorem}
\newtheorem{lemmaT}[dummy]{Lemma}
\newtheorem{corollaryT}[dummy]{Corollary}
\newtheorem{propositionT}[dummy]{Proposition}
\newtheorem{propertyT}[dummy]{Property}
\newtheorem{remarkT}[dummy]{Remark}

\theoremstyle{blacknumex}
\newtheorem{exampleT}[dummy]{Example}
\newtheorem{exerciseT}[dummy]{Exercise}

\numberwithin{equation}{section}

\RequirePackage[framemethod=default]{mdframed}

% Definition box
\newmdenv[skipabove=7pt, skipbelow=7pt,
rightline=false, leftline=true, topline=false, bottomline=false,
backgroundcolor = reddish!10, 
linecolor=reddish,
innerleftmargin=5pt, innerrightmargin=30pt, innertopmargin=5pt,
leftmargin=0cm, rightmargin=0cm, linewidth=4pt,
innerbottommargin=5pt]{dBox}

% Main Theorem box
\newmdenv[skipabove=7pt, skipbelow=7pt,
rightline=false, leftline=true, topline=false, bottomline=false,
backgroundcolor=c0!10, 
linecolor=c0,
innerleftmargin=5pt, innerrightmargin=30pt, innertopmargin=5pt,
leftmargin=0cm, rightmargin=0cm, linewidth=4pt, innerbottommargin=5pt]{tBox}

% Lemma/Corollary/Proposition/Property box
\newmdenv[skipabove=7pt, skipbelow=7pt,
rightline=false, leftline=true, topline=false, bottomline=false,
backgroundcolor = c0!10, 
linecolor=c0!80,
innerleftmargin=5pt, innerrightmargin=30pt, innertopmargin=5pt,
leftmargin=0cm, rightmargin=0cm, linewidth=4pt,
innerbottommargin=5pt]{lBox}

% Example/Remark/Exercise box
\newmdenv[skipabove=7pt, skipbelow=7pt,
rightline=false, leftline=true, topline=false, bottomline=false,
backgroundcolor = mossgreen!10!white,
linecolor = mossgreen,
innerleftmargin=5pt, innerrightmargin=30pt, innertopmargin=5pt,
leftmargin=0cm, rightmargin=0cm, linewidth=4pt,
innerbottommargin=5pt]{exBox}

% Proof box
% \newmdenv[skipabove=7pt, skipbelow=7pt,
% rightline=false, leftline=true, topline=false, bottomline=false,
% linecolor=gray,
% innerleftmargin=5pt, innerrightmargin=30pt, innertopmargin=5pt,
% leftmargin=0cm, rightmargin=0cm, linewidth=4pt,
% innerbottommargin=5pt]{proofBox}



% Creates an environment for each type of theorem and assigns it a theorem text style from the "Theorem Styles" section above and a colored box from above
\newenvironment{definition}{\begin{dBox}\begin{definitionT}}{\end{definitionT}\end{dBox}}
\newenvironment{theorem}{\begin{tBox}\begin{theoremT}}{\end{theoremT}\end{tBox}}
\newenvironment{lemma}{\begin{lBox}\begin{lemmaT}}{\end{lemmaT}\end{lBox}}
\newenvironment{proposition}{\begin{lBox}\begin{propositionT}}{\end{propositionT}\end{lBox}}
\newenvironment{corollary}{\begin{lBox}\begin{corollaryT}}{\end{corollaryT}\end{lBox}}
\newenvironment{property}{\begin{lBox}\begin{propertyT}}{\end{propertyT}\end{lBox}}


% \newenvironment{proof*}{\begin{proofBox}\begin{proof}}{\end{proof}\end{proofBox}}
\newenvironment{exercise}{\begin{exBox}\begin{exerciseT}}{\hfill{\color{royal}}\end{exerciseT}\end{exBox}}
\newenvironment{remark}{\begin{exBox}\begin{remarkT}}{\end{remarkT}\end{exBox}}
\newenvironment{example}{\begin{exBox}\begin{exampleT}}{{}\end{exampleT}\end{exBox}}

\title{Week 2}

\begin{document}
\maketitle


\section{Separability}  

In this section, we will be working with metric spaces \((V, \rho)\).

\begin{definition}[Separable]
	A metric space $(V,\rho)$ is \textbf{separable} if \(\exists D \subset V\) countable, 
	such that \(B_{\rho}(x, \varepsilon) \cap D \neq \empty, \forall x \in V, \forall \varepsilon >0\) 
	(\textit{i.e.} D is dense in \(V\))
\end{definition}

Here \(B_{\rho}(x, \varepsilon)=\{y \in V: \rho (y,x) < \varepsilon\}\) is an open ball of radius \(\varepsilon\) 
centered at \(x\).

\begin{proposition}
	\(\ell^p\) space is separable for \(p \in [1. \infty)\)
\end{proposition}  
(Here it is understood that \(\ell^p, \rho\), where \(\rho\) is the metric indcued by the norm \(\norm{\cdot}_p\), 
\textit{i.e.} \(\rho(x,y)=\norm{x-y}_p\))  

\begin{proof}
	Consider \(D = \cup_{n\geq 1} D_n\), where  
	\[ D_n = \{x=(x_n): x_n \in \mathbb{Q}, \forall n \in \natu, \ \text{and} \ x_k=0, \forall k >n\}\]
 Clearly $D_n \subset \ell^p$, hence $D\subset \ell^p$ and $D_n \cong \mathbb{Q}^n$ is countable, hence $D$ is also countable.   

\textbf{Claim:} $D$ is dense in $\ell^p$.  

Let $x = (x_n) \in \ell^p$ and $\varepsilon >0$ . First we build a $\norm{\cdot}_p-$close sequence $\Tilde{x}=(\Tilde{x}_n)$ with values in $\mathbb{Q}$. Since $\mathbb{Q} \subset \real$ is dense, we find for every $n$ a number $\tilde{x}_n \in \qq$ s.t.  

$$
|x_n-\Tilde{x}_n| \leq \left(\frac{\varepsilon}{2}\right) (2^{-\frac{n}{p}})
$$  

This implies  

\begin{equation}
\label{1}
    \norm{x-\Tilde{x}}_p^p = \sum_{n=1}^{\infty} |x_n-\Tilde{x}_n|^p \leq \left(\frac{\varepsilon}{2}\right)^p \sum_{n=1}^{\infty} 2^{-n}=\left(\frac{\varepsilon}{2}\right)^p
\end{equation}

Note that this also implies $\Tilde{x} \in \ell^p$, since $\norm{\Tilde{x}}_p \leq \underbrace{\norm{\Tilde{x}-x}_p}_{<\infty}+\underbrace{\norm{x}_p}_{<\infty}$.  

Since $\Tilde{x} \in \ell^p$, we have $\sum_n |\Tilde{x}_n|^p < \infty$ hence we can pick an $n$ s.t.  

\begin{equation}
\label{2}
    \sum_{k\geq n} |\Tilde{x}_n|^p < \left(\frac{\varepsilon}{2}\right)^p
\end{equation}

Now define $y = (\Tilde{x}_1, \ldots, \Tilde{x}_n, 0, 0, \ldots,)$. Clearly, $y \in D_n$. Moreover, \cref{2} asserts that $\norm{\Tilde{x}-y} < \frac{\varepsilon}{2}$. Combining with \cref{1} and using the triangle inequality yields $\norm{x-y}_p < \varepsilon$, \textit{i.e.} $x\in B(y, \varepsilon)$.  
\end{proof}  

\begin{proposition}
    $L^p(\real^n), p \in [1, \infty)$ is separable
\end{proposition}  

\begin{proof}
    (Sketch)  Consider  
    $$
    C = \left\{Q\: \text{\ dyadic\ cube}, i.e. \ Q = x + [0, 2^{-l}) \text{\ for\ some\ } x \in 2^{-l} \inte^n(\subset \real^n) \text{\ and\ } l\in \natu \cup \{0\}\right\}
    $$   
    Define  
    $$
    D = \left\{g = \sum_{k=1}^n a_k \mathbf{1}_{Q_k}: n\in \natu, a_k \in \qq, Q_k \in C \right\}
    $$
    \textbf{Claim:} D is dense in $L^p(\real^n), p \in [1, \infty)$  

    Let $f \in L^p(\real^n)$. Assume $f \geq 0$ (else split into $f=f^+-f^-$)  

    \textbf{Step 1} By approximation of simple functions, we can find $\Tilde{g}$ simple, s.t. $0 \leq \Tilde{g} \leq f$ and  
    $$
    \norm{f-\Tilde{g}} < \frac{\varepsilon}{3}
    $$  
    with $\Tilde{g}=\sum_{k=1}^m a_k \mathbf{1}_{A_k}$ for suitable $A_k \in \mathcal{B}(\real^n)$.  

    \textbf{Step 2} We can find a sequence of simple functions $\hat{g}$ with coefficients $a_l \to a_k$ where $a_l\in \qq$ such that  
    $$
    \norm{\hat{g}-\Tilde{g}} < \frac{\varepsilon}{3}
    $$  

    \textbf{Step 3} For each $A_k$, we can find $O_k$ open s.t.  
    $$
    \lambda(O_k\setminus A_k) < \frac{\varepsilon}{6} 2^{-k}
    $$
    And we can approximate $O_k$ using dyadic cubes with precision $\frac{\varepsilon}{6} 2^{-k}$
\end{proof}  

It is crucial that $\lambda(A_k) < \infty$, as  

$$
\forall k \in \natu: |a_k|^p \lambda(A_k) \leq \norm{\Tilde{g}}_p \leq \norm{f}_p < \infty
$$  

see also MATH50006 proof of (4.13). 

\begin{definition}[Schauder basis]

Let $(X,\norm{\cdot})$ be a normed linear space. A \textbf{Schauder basis} of $X$ is a sequence of linearly independent $(e_i)_{i\in \natu}, e_i \in X$, such that $\forall x \in X$,  there is a \textit{unique} sequence $(a_n)_{n\in \natu}, a_n \in \real$ with
$$
\norm{x - \lim_{n\to\infty}\sum_{i=1}^n a_i e_i} \overset{n\to \infty}{\longrightarrow} 0
$$
\end{definition}


\begin{proposition}[Schauder implies separability]
\label{Schauder implies separability}
	If $(X, \norm{\cdot})$ has a Schauder basis, then it is separable.
\end{proposition}  

\begin{proof}
    Define the set $D \subset X$ as,  
    $$
    D = \left\{\sum_{i=1}^n q_i e_i: q_i \in \qq \right\}
    $$  
    where $(e_i)$ is a Schauder basis. (if $X$ is over $\mathbb{C}$, then use $q \in \qq + i \qq$)
    
    Then by definition, one can find $n$ and $x_i$ such that  
    \begin{equation}
        \label{3}
        \norm{x - \sum_{i=1}^n x_i e_i} \leq \frac{\varepsilon}{2}  
    \end{equation}
    Choose $q_i \in \qq$ such that $|q_i - x_i|<\frac{\varepsilon}{2n \sum_{i=1}^n \norm{e_i}}$, we have  
    \begin{equation}
    \label{4}
        \norm{\sum_{i=1}^n x_i e_i - \sum_{i=1}^n q_i e_i} \leq \sum_{i=1}^n |x_i-q_i| \norm{e_i} \leq \frac{\varepsilon}{2}
    \end{equation}
    Using triangle inequality and \cref{3}, \cref{4} above, we see
    $$
    \norm{x-\sum_{i=1}^n q_i e_i} < \frac{\varepsilon}{2}
    $$
\end{proof}

\begin{remark}
    The converse of \cref{Schauder implies separability} is not true, Per Enflo constructed a counter example that is Banach in \href{https://projecteuclid.org/download/pdf_1/euclid.acta/1485889774}{this paper}.
\end{remark}

\begin{example}
    A Schauder basis of $\ell^p, p\in [1, \infty)$ is $e_n=(0,\ldots, 0,1,0, \ldots,0,\ldots), n\in \natu$ (the $n^{th}$ entry is $0$).  
    Take $x=(x_n)\in \ell^p$  
    $$
    \norm{x-\sum_{i=1}^n x_ie_i}_p^p = \sum_{i=n+1}^{\infty} |x_i|^p \overset{n\to \infty}{\longrightarrow} 0
    $$  
    since $\norm{x}_0 < \infty$.  
\end{example}


\newpage
\section{Hilbert Space}  

\begin{unexaminable}
Hilbert space is a special class of Banach space. 
Apart from completeness and norm, it is also equipped with an additional structure, 
{\bf inner product}. This allow us to explore nice geometric properties of the space, 
like orthogonality and angle. We'll see later that this structure resemble Euclidean space in many ways. A Hilbert space is naturally Banach, while the reverse may not be true.
\end{unexaminable}

In this section we work with linear space $H$ over $\mathbb{K} = \real \ \textrm{or} \ \mathbb{ C}$

\begin{definition}
[Bilinear Map]\nextline
	Let $X$ be a vector space over $\mathbb C$. An {\bf bilinear map} is a function $\left<\cdot,\cdot\right>:H \times H \xrightarrow{}{\mathbb C}$ satisfying following: $\forall x,y,z\in H,\alpha$ a scalar,
	\begin{itemize}
		\item[1] $\left<x,y\right>={\overline{\left<y,x\right>}},\forall x,y\in H$
		\item[2] $\left<x,x\right>\geq0$
		\item[3] $\left<x,x\right>=0$ iff $x=0$
		\item[4] $\left<x+y,z\right>=\left<x,z\right>+\left<y,z\right>$
		\item[5] $\left<ax,z\right>=a\left<x,z\right>$
	\end{itemize}
\end{definition}
1 is complex conjugation. 2 and 3 is positive-definiteness. 4 and 5 is left-linearity.


\begin{theorem}
	If $\left<\cdot,\cdot\right>$ is an inner product on $X$, define $\norm{x}\overset{\textrm{def}}{=}\sqrt{\inne{x}{x}}$.  
	\begin{enumerate}[i)]
	    \item (Cauchy-Schwarz) $\ \forall x,y\in X$,
	$$|\inne{x}{y}|^2\leq \norm{x} \norm{y}$$
	\item $\norm{x}$ is a norm
\end{enumerate}  
\end{theorem}

\begin{proof}
\begin{enumerate}[i)]
    \item If $x=0$ or $y=0$, the inequality holds. Else, let $\xi = \frac{x}{\norm{x}}, \eta = \frac{y}{\norm{y}}$, so $\norm{\xi}=\norm{\eta}=1$. Hence 
    $$0 \leq \norm{\eta-\inne{\xi}{\eta}\xi}^2=\norm{\eta}^2 - |\inne{\xi}{\eta}|^2=1-|\inne{\xi}{\eta}|^2$$  
    so $|\inne{\xi}{\eta}| \leq 1$
    \item Positivity and homogeneity follows from definition of $\inne{\cdot}{\cdot}$; and triangle inequality follows from i)  
    $$
    \norm{x+y}^2=\norm{x}^2 + 2 \inne{x}{y} + \norm{y}^2 \leq (\norm{x}+\norm{y})^2
    $$
\end{enumerate}
\end{proof}

\begin{definition}[Hilbert space]
    An inner product space $(H, \inne{\cdot}{\cdot})$ which is complete w.r.t. the metric induced by $\norm{\cdot}=\sqrt{\inne{\cdot}{\cdot}}$ is called a \textbf{Hilbert space}
\end{definition}

\begin{example}[$L^2-$spaces]
The space $L^2(\mu)$ for all measures $\mu$ is a Hilbert space with inner product $\inne{f}{g}=\int fg d\mu$ and $\inne{f}{f}=\norm{f}_2^2$    
\end{example}

\begin{example}[$l^2-$spaces]
The sequence space
	$\ell^2=
		\left\{\{x_k\}_{k\in \natu}:\sum_{k=1}^\infty |x_k|^2<\infty \right\}$
	is a Hilbert space with inner product defined by 
	$
		\inne{x}{y}=\sum_{k=1}^\infty \,{x_k \overline{y_k}}
	$
\end{example}

\begin{unexaminable}
Structure of inner product allows discussion for nice geometric property of Hilbert spaces. This includes orthogonality, angles and nearest distance etc.
\end{unexaminable}
\begin{unexaminable}
Orthogonality is the generalization of two lines being perpendicular. In euclidean geometry, we have Pythagorean theorem closely related to such property. Results on orthogonality in Hilbert spaces in many ways resemble their Euclidean version.
\end{unexaminable}

\begin{theorem}
[Nearest Point Property]\nextline
\label{nearest point}
	Let $H$ be a Hilbert space, $K\subset H$ be a closed, convex subset, then $\forall y \in H$ there exists a {\bf{unique}} $x_0\in K$ such that
	$$
		\delta \overset{\rm{def}}{=} \inf_{x\in K} \norm{x-y} = \norm{x_0-y}
	$$
\end{theorem}

\begin{proof} 
By considering the set $K-y = \{x-y: x\in K\}$ (still closed and convex), we can assume $y=0$.  

\underline{\textbf{Existence:}}\nl
By definition of $\delta$, $\exists (x_n)_{n \in \natu}$, $x_n\in K$ such that $\lim_{n \to \infty} \norm{x_n}=\delta$. We show that $(x_n)_{n \in \natu}$ is a Cauchy sequence. Let $\varepsilon >0$. Pick $N \in \natu$ such that  

$$
\forall n \geq N \qquad \norm{x_n}^2 < \delta^2 + \frac{\varepsilon^2}{4}
$$  

$K$ being convex implies that $\frac{x_n+x_n}{2} \in K, \forall n,m \in \natu$, which implies by definition of $\delta$, $\norm{x_n+x_m} \geq 2\delta$.  

It follows that for all $n,m \geq N$,  

\begin{equation*}
    \norm{x_n-x_m}^2 = \underbrace{2(\norm{x_n}^2+\norm{x_m}^2)}_{< 2\delta^2 + \varepsilon^2 /2} \underbrace{-\norm{x_n+x_m}^2}_{\leq 4\delta^2} < \varepsilon^2
\end{equation*}  

where we have used the Parallelogram law (\Cref{parallelogram}).  

By completeness, $\exists x_0 \ $ s.t. $x_k \to x_0$ as $k \to \infty$. Since $K$ is closed, the limit $x_0\in K$ and $\norm{x_0}=\delta$ by continuity of the norm $\norm{\cdot}$.  

\underline{\textbf{Uniqueness:}}\nl
Take $x_0, x_1 \in K$ with $\norm{x_0}=\norm{x_1}=\delta$ and assume $x_0\neq x_1$, then $\frac{1}{2}(x_0+x_1) \in K$ by convexity and so $\norm{x_0+x_1}\geq 2\delta$. By the Parallelogram law,  
$$
\norm{x_0-x_1}^2 = 2(\norm{x_0}^2 + \norm{x_1}^2) - \norm{x_0+x_1}^2 \leq 4\delta^2-4\delta^2= 0
$$  
So $x_0=x_1$, a contradiction.
\end{proof}  

\begin{remark}
    A good example of $K$ is $K \subset H$ a subspace.
\end{remark}  

\begin{proposition}[Parallelogram law]\rm\label{parallelogram}\nextline
	Let $x,y\in H$, then
        $$
		\norm{x+y}^2+\norm{x-y}^2=2\norm{x}^2+2\norm{y}^2
	$$
\end{proposition}
\begin{unexaminable}
    \begin{proof}
    Then
	\begin{equation}
		\begin{split}
			\norm{x+y}^2&=\inne{x+y}{x+y}\\
			&=\inne{x}{x}+\inne{x}{y}+\inne{y}{x}+\inne{y}{y}\\
			&=\inne{x}{x}+\inne{x}{y}+{\overline{\inne{x}{y}}}+\inne{y}{y}\\
			&=\norm{x}^2+2Re(\inne{x}{y})+\norm{y}^2
		\end{split}
	\end{equation}

	Similarly,
	\begin{equation}
		\begin{split}
			\norm{x-y}^2&=\inne{x-y}{x-y}\\
			&=\inne{x}{x}-\inne{x}{y}-\inne{y}{x}+\inne{y}{y}\\
			&=\inne{x}{x}-\inne{x}{y}-{\overline{\inne{x}{y}}}+\inne{y}{y}\\
			&=\norm{x}^2-2Re(\inne{x}{y})+\norm{y}^2
		\end{split}
	\end{equation}
 Adding up, we obtain the identity.  
\end{proof}
\end{unexaminable}


If $H$ is Hilbert space,  a lot of geometric intuition from linear algebra prevails. For instance, call $x \perp y$ if $\inne{x}{y}=0$.  

Then if $K\subset H$ is a closed subspace, $y \in H$, then \Cref{nearest point} applies with $x_0$ being a nearest point in $K$ to $y$ and  

$$
z = y-x_0 \perp K
$$  
in other words, $z\perp x, \forall x \in K$.  

Assume this is not true, then $\exists x \in K: \inne{z}{x} \neq 0$, then  
$$
\norm{z - \frac{\inne{z}{x}}{\norm{x}^2}x}^2 = \underbrace{\norm{z}^2}_{=\delta} - \frac{|\inne{z}{x}|^2}{\norm{x}^2}<\delta
$$  
which violates the minimality of $x_0$.  

More generally, one has  


\begin{definition}[Orthogonal Complement]\nextline
    For $S\subset H$, $H$ a Hilbert space, we define the \textbf{orthogonal complement} 
    $$
    S^{\perp} \overset{\text{def.}}{=} \left\{y \in H: \inne{x}{y}=0, \forall x \in S \right\}
    $$
    We can also check that $S^{\perp}$ is closed.  
\end{definition}  


\begin{corollary}[Orthogonal Decomposition] \nextline
Let $H$ be a Hilbert space and $E \subset H$ be a closed subspace. Then  
$$
H = E \oplus E^{\perp}
$$  
(\textit{i.e.} $E \cap E^{\perp} = \{0\}$ and $H = E + E^{\perp}$, that is $\forall x \in H, x=e+ e^{\perp}$ for some $e \in E, e^{\perp} \in E^{\perp}$)
\end{corollary}

\begin{proof}
If $x \in E\cap E^{\perp}$, then $\inne{x}{x}=0$, so $\norm{x}=0$, $x=0$.  

For all $x\in H$, the subspace $K\overset{\text{def.}}{=}x+E$ is closed and convex. 
Thus, by \Cref{nearest point}, $\exists x_0 \in E$, s.t. $\norm{x-x_0} \leq \norm{x-\eta}, \forall \eta \in E$.
\begin{unexaminable}
    We show that every $x\in H$ can be written as $x=x_0+(x-x_0)$.
\end{unexaminable}

We have, $\forall \eta \in E$:  
\begin{equation*}
    t=0 \text{\ is\ a\ minimum\ of\ } t\in [0,1] \mapsto \frac{1}{2} \norm{x-x_0+t\eta}^2
\end{equation*} 

which is a quadratic function of $t$, therefore  

\begin{equation*}
    0=\frac{d}{dt}\frac{1}{2} \norm{x-x_0+t\eta}^2 \Big|_{t=0} = t\norm{\eta}^2 + \inne{x-x_0}{\eta} \Big|_{t=0}=\inne{x-x_0}{\eta}
\end{equation*}

\textit{i.e.} $(x-x_0) \in E^{\perp}$ and $x=x_0+(x-x_0) \in E+E^{\perp}$
\end{proof}




\end{document}