\documentclass{article}
\usepackage[utf8]{inputenc}
\usepackage{geometry}
\geometry{left=3cm,right=3cm,top=2cm,bottom=2cm}
\usepackage[utf8]{inputenc}
\usepackage{amsmath, amsfonts, amssymb, amsthm}
\usepackage[framemethod=TikZ]{mdframed}
\usepackage{mathrsfs}
\usepackage{comment}
\usepackage{enumerate}
\usepackage{xcolor}
\usepackage{titlesec}
\usepackage{setspace}
\usepackage[hidelinks,backref]{hyperref}
\usepackage{cleveref}
\usepackage[most]{tcolorbox}
\usepackage{ragged2e}
\usepackage{todonotes}
\usepackage{cleveref}
\usepackage{mathtools}

%
\titleformat*{\section}{\LARGE \bfseries}
\titleformat*{\subsection}{\Large \bfseries}
\titleformat*{\subsubsection}{\Large \bfseries}
% \titleformat*{\paragraph}{\large \bfseries}
\titleformat*{\subparagraph}{\large \bfseries}



% General 
\newcommand{\nextline}{\hfill\break}
\newcommand{\nl}{\nextline\rm}
% \newcommand{\placeholder}{{\bf\color{red} NOOOOOOT COMPLEEEEEEET! COOOOOOOM BAAAAAACK!!!}}
\newcommand{\placeholder}{\todo{NOOOOOOT COMPLEEEEEEET! COOOOOOOM BAAAAAACK!!!}}

\newcommand{\defeq}{\stackrel{def.}{=}}
% FA and LA
% inner product: \inne{a}{b}
\newcommand{\inne}[2]{\left<{#1},{#2}\right>}

% norm: \norm{a}
\newcommand{\norm}[1]{\left\|{#1}\right\|}

% Curly H
\newcommand{\hbs}{$\mathscr{H}$ }
\newcommand{\hbp}{\mathscr{H}}

% Dual : \dual{x}
\newcommand{\dual}[1]{{#1}^*}

% Sequence from 1  to infty: \sequ{x_n}
\newcommand{\sequ}[1]{\left({#1}\right)_1^\infty}

% f: A-> B \func{f}{A}{B}
\newcommand{\func}[3]{${#1}:{#2}\xrightarrow{}{#3}$}

% interior
\newcommand{\interior}{\textrm{int}}

% Bounded linear funcs
\newcommand{\blf}[2]{\mathcal{L}({#1},{#2})}

\newcommand{\prf}{\textit{proof}:   }



% Fields 
\newcommand{\real}{\mathbb{R}}
\newcommand{\comp}{\mathbb{C}}
\newcommand{\inte}{\mathbb{Z}}
\newcommand{\natu}{\mathbb{N}}





% Theorems
% \newtheorem{example}{Example}[subsection]
% \newtheorem{definition}[example]{Definition}
% \newtheorem{proposition}[example]{Proposition}
% \newtheorem{remark}[example]{Remark}
% \newtheorem{theorem}[example]{Theorem}
% \newtheorem{lemma}[example]{Lemma}
% \newtheorem{corollary}[example]{Corollary}


% for numbering the theorems            
\theoremstyle{plain}
%%%%%%%%%%%%%%%%%%%%%%%%%%%%%%%%%%%%%%%%%%%%%%%%%%%%%%%%%%%
\newtheorem{theorem}{Theorem}[section]
\newtheorem{lemma}[theorem]{Lemma}
\newtheorem{corollary}[theorem]{Corollary}
\newtheorem{proposition}[theorem]{Proposition}
%%%%%%%%%%%%%%%%%%%%%%%%%%%%%%%%%%%%%%%%%%%%%%%%%%%%%%%%%%%
% the following are not in italics
\theoremstyle{definition}
\newtheorem{definition}[theorem]{Definition}
\newtheorem{example}[theorem]{Example}
\newtheorem{remark}[theorem]{Remark}
\newtheorem{claim}[theorem]{Claim}
%%%%%%%%%%%%%%%%%%%%%%%%%%%%%%%%%%%%%%%%%%%%%%%%%%%%%%%%%%%%

% proof box
\newtcbtheorem[no counter]{pf}{Proof}{
  enhanced,
  rounded corners,
  attach boxed title to top,
  colback=white,
  colframe=black!25,
  fonttitle=\bfseries,
  coltitle=black,
  boxed title style={
    rounded corners,
    size=small,
    colback=black!25,
    colframe=black!25,
  } 
}{prf}

% extra content box to put in contents not covered in the lecture notes
% use the command \begin{unexaminable}
\newmdenv[skipabove=7pt, skipbelow=7pt,
    rightline=false, leftline=false, topline=false, bottomline=false,
    backgroundcolor = gray!10,
    innerleftmargin=1in, innerrightmargin=1in, innertopmargin=5pt,
    leftmargin=-1in, rightmargin=-1in, linewidth=4pt,
    innerbottommargin=5pt]{unexamBox}
\newenvironment{unexaminable}{\begin{unexamBox}}{\end{unexamBox}}


% clever ref settings
\crefname{lemma}{lemma}{lemmas}
\Crefname{lemma}{Lemma}{Lemmas}
\crefname{theorem}{theorem}{theorems}
\Crefname{theorem}{Theorem}{Theorems}

% formatting 
% https://tex.stackexchange.com/questions/217497/aligning-stackrel-signs-beneath-each-other-using-split
\newlength{\leftstackrelawd}
\newlength{\leftstackrelbwd}
\def\leftstackrel#1#2{\settowidth{\leftstackrelawd}%
{${{}^{#1}}$}\settowidth{\leftstackrelbwd}{$#2$}%
\addtolength{\leftstackrelawd}{-\leftstackrelbwd}%
\leavevmode\ifthenelse{\lengthtest{\leftstackrelawd>0pt}}%
{\kern-.5\leftstackrelawd}{}\mathrel{\mathop{#2}\limits^{#1}}}

\doublespacing
\RaggedRight
% modify innerrightmargin if floats were lost

\usepackage{amsfonts, amsmath, amssymb, amsthm, thmtools, bm}
\usepackage{avant} % Use the Avantgarde font for headings

% Boxed/framed environments
\newtheoremstyle{royalnumbox}%
{0pt}% Space above
{0pt}% Space below
{\normalfont}% Body font
{}% Indent amount
{\small\bf\sffamily\color{royal}}% Theorem head font
{\;}% Punctuation after theorem head
{0.25em}% Space after theorem head
{\sffamily \color{royal} 
    \thmname{#1} 
    \thmnumber{#2} \thmnote{\bfseries\color{black}---\nobreakspace#3.}} % Optional theorem note
\renewcommand{\qedsymbol}{$\blacksquare$}% Optional qed square

\newtheoremstyle{blacknumex}% Theorem style name
{5pt}% Space above
{5pt}% Space below
{\normalfont}% Body font
{} % Indent amount
{\small\bf\sffamily}% Theorem head font
{\;}% Punctuation after theorem head
{0.25em}% Space after theorem head
{\sffamily
    \thmname{#1}
    \thmnumber{#2}
    \thmnote{---\nobreakspace#3.}}% Optional theorem note

\newtheorem*{notation}{Notation}
\newtheorem*{hint}{Hint}
\newtheorem*{solution}{Solution}

\newcounter{dummy} 
\numberwithin{dummy}{section}

\theoremstyle{royalnumbox}
\newtheorem{definitionT}[dummy]{Definition}
\newtheorem{theoremT}[dummy]{Theorem}
\newtheorem{lemmaT}[dummy]{Lemma}
\newtheorem{corollaryT}[dummy]{Corollary}
\newtheorem{propositionT}[dummy]{Proposition}
\newtheorem{propertyT}[dummy]{Property}
\newtheorem{remarkT}[dummy]{Remark}

\theoremstyle{blacknumex}
\newtheorem{exampleT}[dummy]{Example}
\newtheorem{exerciseT}[dummy]{Exercise}

\numberwithin{equation}{section}

\RequirePackage[framemethod=default]{mdframed}

% Definition box
\newmdenv[skipabove=7pt, skipbelow=7pt,
rightline=false, leftline=true, topline=false, bottomline=false,
backgroundcolor = reddish!10, 
linecolor=reddish,
innerleftmargin=5pt, innerrightmargin=30pt, innertopmargin=5pt,
leftmargin=0cm, rightmargin=0cm, linewidth=4pt,
innerbottommargin=5pt]{dBox}

% Main Theorem box
\newmdenv[skipabove=7pt, skipbelow=7pt,
rightline=false, leftline=true, topline=false, bottomline=false,
backgroundcolor=c0!10, 
linecolor=c0,
innerleftmargin=5pt, innerrightmargin=30pt, innertopmargin=5pt,
leftmargin=0cm, rightmargin=0cm, linewidth=4pt, innerbottommargin=5pt]{tBox}

% Lemma/Corollary/Proposition/Property box
\newmdenv[skipabove=7pt, skipbelow=7pt,
rightline=false, leftline=true, topline=false, bottomline=false,
backgroundcolor = c0!10, 
linecolor=c0!80,
innerleftmargin=5pt, innerrightmargin=30pt, innertopmargin=5pt,
leftmargin=0cm, rightmargin=0cm, linewidth=4pt,
innerbottommargin=5pt]{lBox}

% Example/Remark/Exercise box
\newmdenv[skipabove=7pt, skipbelow=7pt,
rightline=false, leftline=true, topline=false, bottomline=false,
backgroundcolor = mossgreen!10!white,
linecolor = mossgreen,
innerleftmargin=5pt, innerrightmargin=30pt, innertopmargin=5pt,
leftmargin=0cm, rightmargin=0cm, linewidth=4pt,
innerbottommargin=5pt]{exBox}

% Proof box
% \newmdenv[skipabove=7pt, skipbelow=7pt,
% rightline=false, leftline=true, topline=false, bottomline=false,
% linecolor=gray,
% innerleftmargin=5pt, innerrightmargin=30pt, innertopmargin=5pt,
% leftmargin=0cm, rightmargin=0cm, linewidth=4pt,
% innerbottommargin=5pt]{proofBox}



% Creates an environment for each type of theorem and assigns it a theorem text style from the "Theorem Styles" section above and a colored box from above
\newenvironment{definition}{\begin{dBox}\begin{definitionT}}{\end{definitionT}\end{dBox}}
\newenvironment{theorem}{\begin{tBox}\begin{theoremT}}{\end{theoremT}\end{tBox}}
\newenvironment{lemma}{\begin{lBox}\begin{lemmaT}}{\end{lemmaT}\end{lBox}}
\newenvironment{proposition}{\begin{lBox}\begin{propositionT}}{\end{propositionT}\end{lBox}}
\newenvironment{corollary}{\begin{lBox}\begin{corollaryT}}{\end{corollaryT}\end{lBox}}
\newenvironment{property}{\begin{lBox}\begin{propertyT}}{\end{propertyT}\end{lBox}}


% \newenvironment{proof*}{\begin{proofBox}\begin{proof}}{\end{proof}\end{proofBox}}
\newenvironment{exercise}{\begin{exBox}\begin{exerciseT}}{\hfill{\color{royal}}\end{exerciseT}\end{exBox}}
\newenvironment{remark}{\begin{exBox}\begin{remarkT}}{\end{remarkT}\end{exBox}}
\newenvironment{example}{\begin{exBox}\begin{exampleT}}{{}\end{exampleT}\end{exBox}}

\title{Week 8}

\begin{document}
\maketitle


\section{Spectrum}
In this section, we consider normed linear spaces over $\comp$, for the reason that it provides a simpler picture of the spectrum problem. In linear algebra, we know that Hermitian matrices over $\comp$ has real eigenvalues. This is the spectral theory of finite dimensional space. In infinite dimensional spaces, the settings are different, however, are extensions of finite dimensional cases.

\begin{definition}[Resolvent \& Spectral]\nl
Let $X$ be Banach. \func{T}{D_T\subset X}{X} be a linear operator, its {\underline{resolvent set}} is
$$
\varrho({T})=
\{
\lambda\in\comp:(\lambda I-T) \text{ is bijective and } \exists (xI-T)^{-1}\in \mathcal{L}(X)
\}
$$
The {\underline{spectrum}} is defined as the complement:
$$
\sigma(T)=\comp\backslash\varrho(T)
$$
The {\underline{resolvent}} of A is the map \func{R}{\varrho(A)}{\mathcal{L}(X)},$\varrho(A)\ni\lambda\mapsto \in\mathcal{L}(X)$
\end{definition}

\begin{remark}\nl
    In this section, we define $\lambda-A=\lambda id -A=\lambda I-A$, where $id$ is the identity map and the third expression uses first-year linear algebra notation.
\end{remark}

% Below are in accordance with written notes
\begin{example}
    Consider $X=\comp$, $A\in \mathcal{L}(X)$, i.e. $D_A=X$, $\lambda\in\comp$, $(\lambda-A)$ invertible $\iff p(\lambda)\stackrel{def.}{=}det(\lambda-A)\neq0$. Since  $p(\cdot)$ has at least $1$ and at most $n$ (distinct) solutions, one get $\sigma(A)\neq \emptyset$, $\sigma(A)$ contains at most $n$ points. Hence $\varrho(A)\neq\emptyset$ and $\varrho(A)\subset\comp$ is dense.
\end{example}

\begin{lemma}\nl
If $\sigma_0\in\varrho(A)$, then 
$$D\stackrel{def.}{=}
\{
\varrho\in\comp:|\varrho-\varrho_0|<\frac{1}{\norm{R(\varrho_0)}_{\mathcal{L}(X)}}
\}
\subset\varrho(A)$$
Hence $\varrho(A)$ is open, and $\sigma(A)$ is closed.
\begin{pf}{}{}
Write 
$$
\zeta-A=
( \zeta-\zeta_0)+(\zeta_0-A)=(1+(\zeta-\zeta_0)R_{\zeta_0})(\zeta_0-A)
$$
If $\zeta\in D$ then $1+(\zeta-\zeta_0)R_{\zeta_0}$ is invertible with $\circ{1}$:
$$
(1+(\zeta-\zeta_0)R_{\zeta_0}^{-1}=\sum_{n\geq0}(\zeta0-\zeta)^n {R_{\zeta_0}}^n
$$
hence also
$$
R_\zeta=(\zeta-A)^{-1}\stackrel{(*)}{=}R_{\zeta_0}(1+(\zeta-\zeta_0)R_{\zeta_0})^{-1}\in\mathcal{L}(X)
$$
For $\circ{1}$ use: if $A\in\mathcal{L}(X)$, $\norm{A}<1$ then with $A^0=Id=1$, 
$$
\sum_{n=0}^\infty A^n\in\mathcal{L}(X)
$$
i.e. the sequence
$$
\left(
S_n=\sum_{k=0}n A^k
\right)\subset\mathcal{L}(X)
$$
converges, and
$$
\sum_{n=0}^\infty A^n=(1-A^{-1})^{-1}
$$
\end{pf}
\end{lemma}

\begin{example}\nl
    Diagonal operator $T=T_\lambda$ continued. Claim: $\sigma(T)=\overline{\{\lambda_k:k\in\natu\}}$.
    \begin{pf}{$\sigma(T)\supset\overline{\{\lambda_k:k\in\natu\}}$}{}
    $Tx=\lambda_k x$ so $\lambda_k-T$ is not injective, so $\{\lambda_k:k\in\natu\}\subset\sigma(T)$, hence by lemma $\sigma(T)=\overline{\{\lambda_k:k\in\natu\}}$.
    \end{pf}

    \begin{pf}{$\sigma(T)\subset\overline{\{\lambda_k:k\in\natu\}}$}{}
    If $\Tilde{\lambda}\notin\overline{\{\lambda_k:k\in\natu\}}$, then $\exists\delta>0$ s.t. $|\Tilde{\lambda}-\lambda_k|>\delta$ $\forall k\in\natu$. Let $x\in\ell^2$, $y\stackrel{def.}{=}(\Tilde{\lambda}-T)x=((\Tilde{\lambda}-\lambda_k)x_k)_k$. So $x_k=(\Tilde{\lambda}-\lambda_k)^{-1}y_k$ and $\norm{x}_{\ell^2}\leq\delta^{-1}\norm{y}_{\ell^2}$ which implies $(\Tilde{\lambda}-T)^{-1}\in\mathcal{L}(H)$ and $\tilde{\lambda}\in\varrho(T)$
    \end{pf}
\end{example}

\begin{remark}\nl
    In finite dimension, $(\lambda-A)$ not inverttible $\iff(\lambda-A)$ not injective by rank formula. One may wonder if "lack of injectivity"is the only reason for $\lambda\notin \sigma(A)$.
\end{remark}

\begin{definition}\nl
    Consider linear operatorn $A$ with colsed grapgh and specturm $\sigma(A)$.
    \begin{itemize}
        \item [point spectrum] $\sigma_p(A)\stackrel{def.}{=}
        \left\{
        \lambda\in \comp :\lambda-A \text{ not  injective}
        \right\}$
        \item [continuous specturm] $\sigma_c(A)\stackrel{def.}{=}
        \left\{
        \lambda\in\comp\setminus\varrho(A):(\lambda-A) \text{ injective }, Im(\lambda-A) \text{ dense }        
        \right\}$
        \item [residual specturm] $\sigma_r(A)\stackrel{def.}{=}\sigma(A)\setminus(
        \sigma_p(A)\cup\sigma_c(A))$         Elements of point specturm are called {\underline Eigenvalues of $A$} with {\underline eigenspaces} $ker(\lambda-A)=\{x\in D_A:Ax=\lambda x\}\neq\{0\}$
    \end{itemize}
\end{definition}

\begin{example}[Shift operator]\nl
    \func{S}{\ell^2}{\ell^2}, $S(x_1,x_2...)=(0,x_1,x_2...)$. Then $0\in\sigma(S)$: indeed $S$ is not invertible since it's not surjective: $\forall y\in\ell^2$ with $y_1\neq0$, $ y\notin Im(S)$, but $0\notin\sigma_p(S)$: $S$ is injective.\\
    In fact $Sx=\lambda x\implies0=\lambda x_1$, $x_n=\lambda x_(n+1)\,\,\forall n\in\natu\implies x_k=0$ for all k. So $\sigma_p(S)=\emptyset$. In fact, $\sigma(S)=\overline{D}=\{\zeta\in\comp:|\zeta|\leq1\}$, $\sigma_p(S)=D$, $\sigma_c(S)=\partial D=S^1$
\end{example} 
\begin{example}\nl
    $X=\comp^n$, $\sigma(\cdot)=\sigma_p(\cdot)$
\end{example}
\begin{example}\nl
    $T_\lambda$ is indicative of a certain class : $ X$ Hilbert, $T\in\mathcal{L}(X)$ compact, self-adjoint, then by Riesz-Schauder, $\sigma(T)\setminus\{0\}=\sigma_p(T)\setminus\{0\}
    $
\end{example}
% Below are content from written lecture note 9.2-10

\subsection{Spectral Theory in Hilbert space}
Consider $(H,\inne{\cdot}{\cdot}$, Hilbert space over $\comp$, \func{A}{{D_A\subset H}}{H} linear, with adjoint \func{\dual{A}}{D_\dual{A}\subset H}{H}. Recall $\dual{A}$ characterised by $\forall x\in D_A$, $y\in D_\dual{A}$: $\inne{\dual{A}y}{x}=\inne{y}{Ax}$ and $D_\dual{A}=\{y\in H : l_y:D_A\to \comp,\,x\mapsto \inne{y}{Ax}\,\text{ is continuous }\}$.
In sequel write $A\subset B$, reads $B$ is extension of $A$ if $D_A\subset D_B$ and $B|_{D_A}=A$.


\begin{definition}\nl
	i)  $A$ is {\underline{symmetric}} if $A\subset \dual{A}$, i.e. $D_A\subset D_\dual{A}$ and $\inne{Ax}{y}=\inne{x}{Ay}$ for all $x,y\in D_A$.\\
	ii) $A$ is {\underline{self-adjoint}} if $A=\dual{A}$, i.e. A symmetric with $D_\dual{A}=D_A$.
\end{definition}

What can we say about spectrum $\sigma(A)$ for such $A$?

\begin{lemma}
	IF $A$ is symmetric, $\sigma_p(A)\subset\real$.
	\begin{pf}{}{}
		Let $\lambda\in\sigma_p(A)$ with non-zero eigenvector $x\in ker(\lambda-A)$. Then $\lambda\norm{X}_H^2=\inne{Ax}{x}\stackrel{symm.}{=}\inne{x}{Ax}=\overline{\inne{Ax}{x}}=\overline{\lambda}\norm{x}_H^1$. So $\lambda=\overline{\lambda}\implies \lambda\in\real$.
	\end{pf}
\end{lemma}
Is this true for all the spectrum of $\mathcal{H}$ (cf. $\comp^n$)?
%question part----------------------------------
\begin{example}\nl
	$H$=$L^2(0,1)$, $\inne{f}{g}=\int_0^1f\bar{g} \mathrm{d}t$. $A\in\frac{\mathrm{d}}{\mathrm{d}t}$. More precisely, $f\in\mathcal{H}$ is said to have a \underline{weak derivative} $f'$ if $f'\defeq{v}$ for some $v\in \mathcal{H}$, $\int_0^1 fg'\mathrm{d}t=-\int_0^1 vg\mathrm{d}t$, $\forall g\in C^\infty_c(0,1)$.\\
	Consider
	$$
		A_\infty =i\frac{\mathrm{d}}{\mathrm{d}t}:C^\infty_c(0,1)\subset H\to H
	$$
	and extensions $A_1$.....$A_3$ with
	\begin{equation}
		\begin{aligned}
			 & D_{A_1}=H^1\defeq\{f\in H: f \text{ has a weak derivative  f'}\} \\
			 & \cup                                                            \\
			 & D_{A_2}=\{f\in H^1:f(0)=f(1)\}\text{ periodic b.c.}             \\
			 & \cup                                                            \\
			 & D_{A_3}=\{f\in H^1: f(0)=0=f(1)\}\text{ Dirichlet b.c.}         \\
		\end{aligned}
	\end{equation}
	Set $A_k(f)=if'\,\,\forall f\in D_{A_k}$. Evidently $ A_\infty\subsetneq A_3\subsetneq A_2 \subsetneq A_1$.\\
	One can show that
	$$
		A_3\subset\dual{A_1}\subset\dual{A_2}=A_2\subset\dual{A_3}
	$$
	So: $A_3$ is symmetric but because $A_3\subsetneq A_2\subset \dual{A_3}$ not self-adjoint, $A_2$ is self-adjoint.\\
	Claim:
	\begin{itemize}
		\item [i)] $\sigma(A_1)=\sigma_p(A_1)=\comp$,$ \varrho(A_1)=\emptyset$.
		\item [ii)] $\sigma(A_2)-\sigma_p(A_2)=a\pi\inte$, $\varrho(A_2)=\comp$.
		\item [iii)] $\sigma(A_3)=\comp$, $\sigma_p(A_3)=\emptyset$, $\varrho(A_3)=\emptyset$
	\end{itemize}
	So symmetric operators can have imaginary spectrum!
	\begin{pf}{}{}
		i) For $\lambda\in\comp$, pick $f(t)=e^{-i\lambda t}\in ker(\lambda-A_1)$.\\
		ii) For $k\in\inte$, $f(t)=e^{-2\pi ikt}\in D_{A_2}\cap ker(a\pi k-A_2$,
		so $2\pi \inte\subset \sigma_p(A_2)\subset\sigma(A_2)$.
		Let $\lambda\in\comp\setminus a\pi\inte$.
		Need to show: $\lambda\in\varrho(A_2)$ i.e. \func{\lambda-A_2}{D_{A_2}}{H} is invertible and $(\lambda-A_2)^{-1}\in\mathcal{H}$.\\
		For $g\in H$, the general sd. of $\lambda f-if'=g$ can be obtained via variation of constant formula:
		$$
			(*)\rightsquigarrow f(t)=ae^{-i\lambda t}+i \int_0^t e^{i\lambda}g(s)\mathrm{d}s
		$$
		for some $a\in\comp$. But if $\lambda\notin2\pi\inte$, the b.b. determines $a$ uniquely:
		$$
			a=f(0)=f(1)=ae^{-i\lambda}+i\int_0^1e^{i\lambda(s-1)}g(s)\mathrm{d}s$$
		so
		$$
			a=(1-e^{-i\lambda})^{-1}i\int_0^1e^{i\lambda(s-1)}g(s)\mathrm{d}s
		$$
		so $\lambda-A_2$ is invertible and
		$$
        \norm{f}_{L^2}\leq |A|+\norm{g}_{L^2}\leq(|1-e^{-i\lambda}|^{-1})\norm{g}_{L^2}
        $$
		which shows $(\lambda-A_2)^{-1}\in\mathcal{L}(H)$\\
		iii) If $A_3f=if'=\lambda f$ for some $\lambda\in\comp$, then by (*) $f(t)=ae^{-i\lambda t}$ and $a=0$ since $f(0)=0$, so $\sigma_p(A_3)=\emptyset$.\\
		On the other hand $(\lambda-A_3)$ for $\lambda\in\comp$ is never surjective ($\implies \lambda\in\varrho(A_3)$). Indeed, consider $g(s)=e^{i\lambda s}$, then using $(*)$ and b.c. get $a=0$ and
		$$
        f(t)=ie^{-i\lambda t}\int_0^1e^{i\lambda s} \mathrm{d}s=ite^{-i\lambda t}
        $$
		but $f(1)\neq0$ so $f\notin D_{A_3}$.
	\end{pf}
\end{example}
%question part-------------------------------------
We have seen: symmetric operators can have imaginary spectrum, but:
\begin{lemma}
	Let $A\subset \dual{A}$. Then $\forall\zeta\in\comp\,\forall u\in D_A$: $\norm
		{(\zeta-A)u}_H\geq |Im(\zeta)|\norm{u}_H$.\\
	(So for $\zeta\notin\real\implies(\zeta-A)$ injective, i.e. $\zeta\notin\sigma_p(A)$)
	\begin{pf}{}{}
		For $u\in D_A:$ $\inne{u}{Au}\stackrel{A\subset\dual{A}}{=}\inne{Au}{u}=\overline{\inne{u}{Au}}\in\real$. Hence,
		$$
			|Im(\zeta)|\norm{u}^2_H=|Im(\inne{u}(\zeta-A)u)|\leq|\inne
			{u}{(\zeta-A)u}\leq\norm{u}_H\norm{(\zeta-A)u}_H
		$$
	\end{pf}
\end{lemma}
% mind Im and im, Im is imaginary part and im is image. Pay attention  to the difference.
The example also nicely illustrates:
\begin{proposition}
	If $A=\dual{A}$, then $\sigma(A)\subset \real$
	\begin{pf}{}{}
		Let $\zeta\in\comp\setminus\real$. Want to show $\zeta\in\varrho(A)$, i.e. \func{zeta-A}{D_A}{H} is bijective with $(\zeta-A)^{-1}\in\mathcal{L}(H)$.\\
		We will show:\\
		(*) $\zeta-A$ is surjective.\\
		Once  (*) holds, we are done: by previous lemma, $\zeta-A$ is injective hence bijective, and surjectivity + same lemma also yields
		$$
			\norm{(\zeta-A)^{-1}}_{\mathcal{L}(H)}\leq\frac{1}{|Im(\zeta)}
		$$
		proof of (*): we first show
		$$
			(**)\qquad im(\zeta-A)(\subset H) \text{  is closed}
		$$
		Assume $v_k=(\zeta-A)u_k\xrightarrow{k\to\infty}$. BY Lemma p.65,
		$$
			\norm{u_k-v_k}_H\leq\frac{1}{|Im(\zeta)|}\norm{v_k-v_l}_H\xrightarrow{k,l\to\infty}$$Hence $(u_k)$ is Cauchy and $u_k\to u$ for some $u\in H$. But $A=\dual{A}$ has closed graph so $v=(\zeta-A)u$, i.e. (**) holds.\\
		Back to (*):\\
		Due to (**), $M\defeq Im(\zeta-A)$ is closed. Assume $M\neq H$. Pick $v\in M^\perp\setminus\{0\}$. Then
		$$
			\forall u\in D_A:\inne{v}{(\zeta-A)u}=0\text{  or  }\inne{v}{Au}=\overline{\zeta}\inne{v}{u}$$
		Hence,
		$
			D_A\ni u\mapsto \inne{v}{Au}
		$
		is continuous, $v\in D_\dual{A}=D_A$ and $Av=\dual{A}v=\overline{\zeta}v$
		but by lemma p.65,
		$$|Im(\zeta)|\norm{v}_H\leq\norm{(\overline{\zeta}-A)v}_H=0$$
		which yields $v=0 $
	\end{pf}
\end{proposition}
% Below are content for week 10
\subsection{Spectral theorem for compact self-adjoint operators}
$H$: Hilbert space over $\comp$, inner product $\inne{\cdot}{\cdot}$, with $\norm{x}^2_H=\inne{x}{x}$. Following is an extension (!) of the familiar result from linear algebra concerning diagonalization of symmetric matrices.
\begin{theorem}[Reisz-Schauder]\nl
	Let  \func{T}{H}{H} be compact and self-adjoint, then:
	\begin{itemize}
		\item [i)] $\sigma(T)\subset\real$
		\item [ii)] $\sigma_p(T)$ contains at most countably many eigenvalues $\lambda_k\in\real\setminus\{0\}$, which accumulate at most at $\lambda=0$
		\item [iii)] One can choose $e_k$  corresponding to $\lambda_k$ such that $e_k\perp e_l$ $\forall k\neq l$ and one has $\forall x\in H:Tx=\sum_k \lambda_k e_k\inne{x}{e_k}$
	\end{itemize}
\end{theorem}
\begin{example}\nl
	Shift operator \func{T_\lambda}{\ell^2}{\ell^2} continued.\\
	\begin{itemize}
	    \item $T_\lambda$ is compact $\iff \lim_{k\to\infty}\lambda_k=0$
	    \item $T_\lambda$ is self-adjoint $\iff \lambda_k\in\real\,,\forall k$
	\end{itemize}
	 and we know $\sigma_p(T_\lambda)=\overline{\{\lambda_k:k\in\natu\}}$.
\end{example}
We start with the following lemma:
\begin{lemma}[Lemma 1]
	$T\in\mathcal{L}(H)$, self-adjoint. If $\lambda_1\neq\lambda_2$, $\lambda_1,\lambda_2\in\sigma_p(T)$ with eigenvectors $e_1,e_2$, i.e. $\lambda_1 e_1=Te_1$ and $\lambda_2e_2=Te_2$, then $\inne{e_1}{e_2}=0$
	\begin{pf}{}{}
		\begin{equation}
			\begin{split}
			\lambda_1\inne{e_1}{e_2}&=\inne{\lambda_1e_1}{e_2}=\inne{Te_1}{e_2}\stackrel{SA.}{=}\inne{e_1}{Te_2}\\
			&=\inne{e_1}{\lambda_2e_2}\stackrel{\lambda_2=\overline{\lambda_2}}{=}\lambda_2\inne{e_1}{e_2}
			\end{split}
		\end{equation}
		Since $\lambda_1\neq\lambda_2$, $\inne{e_1}{e_2}=0$.
	\end{pf}
\end{lemma}
Henceforth, we always assume \func{T}{H}{H} is compact and self-adjoint. In particular, Lemma 1 is in force.\\
Define, for $\lambda\in\sigma_p(T)\setminus\{0\}$, $X_\lambda=ker(\lambda-T)\neq\{0\}$. By lemma 1:
$$X_\lambda\perp X_\lambda\,\,\forall\lambda\neq\lambda',\lambda\lambda'\in\sigma_p(T)\setminus\{0\}$$
\begin{lemma}[Lemma 2]
	$\lambda\in\sigma_p(T)\setminus\{0\}$.
	\begin{itemize}
		\item $dim(X_\lambda)<\infty$
		\item $\forall r>0$: $\sigma_p(T)\setminus B_r(0)$ is finite.
	\end{itemize}
	\begin{pf}{}{}
		i) Let $B_r^{X_\lambda}(0)=\{x\in X_\lambda:\norm{x}_H<r\}$, which is a bounded set. By compactness of $T$, $T(B_r^{X_\lambda}(0))$ is compact. But since $Tx=\lambda x\,\forall x\in X_\lambda$, $T(B_r^{X_\lambda}(0))=\lambda B_1^{X_\lambda}(0)$. So $\lambda\overline{(B_1^{X_\lambda}(0))}$ is compact $\implies \overline{B_1^{X_\lambda}(0)}$ is compact $\implies dim(X_\lambda)<\infty$.\\
		ii) Suppose not, then $\exists r>0$: $\sigma_p(T)\setminus B_r(0)$ is infinite (one can show $\sup_{\lambda\in\sigma_p(T)}|\lambda|<\infty$). Then one can pick sequence $(\lambda_k)\subset\sigma_p(T)$ with $\lambda_k\neq\lambda_l\,\forall k\neq l$ and $\lambda_k|>r\,\forall k$.\\
		Let $e_k\neq0$ be eigenvector for $\lambda_k$: $Te_k=\lambda_ke_k\,\forall k$.
		By compactness of $T$, $\exists\Lambda\subset\natu$ such that for some $y\in H$, $$Te_k\xrightarrow[k\to\infty,k\in\Lambda]{wrt.\norm{\cdot}_H}y\in H$$
		Hence,
		$$(\lambda_ke_k)\xrightarrow[k\to\infty,k\in\Lambda]{wrt.\norm{\cdot}_H}y\in H$$
		In particular, $(\lambda_ke_k)){k\in\Lambda}$ is Cauchy. But for $k\neq l$,
		$$\norm{\lambda_ke_k-\lambda_le_l}^2_H=
			+\inne{\lambda_le_l}{\lambda_le_l}
			+\underbrace{\inne{\lambda_ke_k}{\lambda_le_l}}_{=\lambda_k\overline{\lambda_l}\inne{e_k}{e_l}\stackrel{Lemma1}{=}0}
			+\inne{\lambda_le_l}{\lambda_ke_k}
		$$
		Assume, $\norm{e_k}_H=\norm{e_l}_H=1$, otherwise replace $e_k$ by $e_k/\norm{e_k}_H$,
		$$
			\norm{\lambda_ke_k-\lambda_le_l}^2_H=|\lambda_k|^2+|\lambda_l|^2>2r
		$$
	\end{pf}
\end{lemma}
\begin{pf}{Riesz-Schauder}{}
	Proof of i) see last week.\\
	ii) By Lemma 2 ii),
	$$A_n=\sigma_p(T)\cap\{z:\frac{1}{n+1}\leq|z|\leq\frac{1}{n}\}(\subset \sigma_p(T)\setminus B_{1/(n+1)}(0))$$
	is finite and $\sigma_p(T)\setminus\{0\}=\bigcup_n A_n$ is thus countable. This also implies $\sigma_p(T)\setminus\{0\}$ has no accumulation point.\\
	iii) By applying Gram-Schmidt to the (finite-dimensional) space $X_\lambda$, $\lambda\in\sigma_p(T)\setminus\{0\}$, can ensure that eigenvectors $e_k$, $e_l$ to eigenvalues $\lambda_k=\lambda_l$ are orthonormal. If they belong
	% remark: Lemma 2 gives important structural information on the spectrum. In particular, ii) implies that $\sigma_p(T)\setminus\{0\}$ is countable with no accumulation point and by I) each eigenvalue has finite multiplicity.
	to distinct eigenvalues $\lambda+k\neq\lambda_l$ this is automatic after normalizing them to have norm 1 by lemma 1.\\
	So let $(\lambda_k)$ be the elements of $\sigma_p(T)\setminus\{0\}$, counted with multiplicities( i.e. $dim(X_{\lambda_k})$ copies of $\lambda_k$. Let
	$$
		X\defeq\overline{span\{e_k\}}=\overline{\oplus_{\lambda\in\sigma_p(T)\setminus\{0\}}X_\lambda}
	$$
	Claim 1 $\forall x\in X$: $x=\sum_k \inne{x}{e_k}e_k$\\
	proof: Let $x_n=\sum_k\leq n \inne{x}{e_k}e_k$.
	Then $\forall n\geq0$,
	$$
		\norm{x_n}_H^2=\sum_{k\leq n}|\inne{x,e_k}|^2=\inne{x_n}{x}\leq\norm{x_n}_H\norm{x}_H$$
	so $\norm{x_n}_H\leq\norm{x}_H$ unless $x_n=0$. Hence
	$$\sum_{k}|\inne{x,e_k}|^2=\lim_{n\to\infty}\norm{x_n}_H^2\leq\norm{x}_H^2<\infty$$
	and $\forall n\geq m\geq0$:
	$$\norm{x_n-x_m}^2_H=\sum_{m\leq k\leq n}|\inne{x,e_k}|^2\xrightarrow{n,m\to\infty}0$$
	Thus $x_n\xrightarrow[\norm{\cdot}_H]{n\to\infty}y\in X$. Moreover, $\forall k\geq 0$,
	$$\inne{x-y}{e_k}=\lim_{n\to\infty}\inne{x-x_k}{e_k}=\inne{x}{e_k}-\lim_{n\to\infty}\inne{x_n}{e_k}=0$$
	so $x=y$. With claim 1 and continuity of $T$ we have
	$$\forall x\in X\,: Tx=\sum_k \inne{x}{e_k}Te_k=\sum_k \inne{x}{e_k}\lambda_ke_k$$
	It remains to argue:\\
	{\bf Claim 2}: $Y\defeq X^\perp=ker(T)$, which concludes the proof.
	Proof of claim2:\\
	If $Y=\{0\}$, it's trivial. So one can assume $Y\neq\{0\}$. First, note that
	\begin{equation}T(y)\subset Y\tag{$*$}\end{equation}
	For, if $y\in Y$ then $\forall k$:
	\[\inne{e_k}{Ty}=\inne{Te_k}{y}=\lambda\inne{e_k}{y}\stackrel{Y\perp\{e_k\}}{=}0\]
	By $(*)$,
	\[T_Y=T|_Y:Y\to Y\] is well defined. $T_Y$ inherits compactness and self-adjointness from $T$ (exercise). We want to show that \func{T_Y}{Y}{Y}: $y\mapsto0 $ is the  "0-map" on $Y$, or equivalently, $\norm{T_Y}_{\mathcal{L}(X)}=0$. If not, one can show (no proof) that
	\[sigma_p(T)\setminus\{0\}\neq\emptyset\]
	(in fact $+\lambda$ or $-\lambda$ is an eigenvalue where $\lambda=\norm{T_Y}_{\mathcal{L}(X)}$) But this can't be because if $e\in Y$ is an eigenvector for $\lambda\in]sigma_p(T)$, $\lambda\neq 0$, then
	\[Te\stackrel{e\in Y}{=}T_Y e=\lambda e\]
	So $e\in X_\lambda\subset X=Y^\perp$. But $Y^\perp\cap Y=\{0\}$
\end{pf}
\begin{remark}
	We have actually shown that $H$ admits the orthogonal decomposition:
	\[H=ker(T)\oplus
		\overline{\oplus_{\lambda\in\sigma_p(T)\setminus\{0\}X_\lambda}}
	\]
	where $\sigma_p(T)\setminus\{0\}X_\lambda$ is countable.
\end{remark}



\end{document}

