\documentclass{article}
\usepackage[utf8]{inputenc}
\input{commands}
\input{theoremstyle.tex}

\title{Week 10}
% \author{\aut}
\begin{document}
\maketitle


\section{Spectral Theory}\label{Spectral theory chapter}
In this section, we consider Banach space over $\comp$. 

\begin{definition}[Resolvent \& Spectral]\nl
Let $X$ be Banach. \func{A}{D_A\subset X}{X}  linear operator.\\
The {\textbf{resolvent set}} of $A$ is
$$
\varrho({A})=
\{
\lambda\in\comp:(\lambda I-A) \text{ is bijective with } \exists (\lambda I-A)^{-1}\in \mathcal{L}(X)
\}
$$
The {\textbf{spectrum}} is defined as the complement:
$$
\sigma(A)=\comp\backslash\varrho(A)
$$
The {\textbf{resolvent}} of A is the map \func{R}{\varrho(A)}{\mathcal{L}(X)}, $\varrho(A)\ni\lambda\mapsto R_\lambda= (\lambda-A)^{-1}\in\mathcal{L}(X)$
\end{definition}

\begin{remark}\nl
    In this section, we define $\lambda-A=\lambda Id -A=\lambda I-A$, where $id$ is the identity map and the third expression uses first-year linear algebra notation.
\end{remark}

% Below are in accordance with written notes
\begin{example}\nl
    Consider $X=\comp$, $A\in \mathcal{L}(X)$, i.e. $D_A=X$, $\lambda\in\comp$.\\
    \begin{center}
        $(\lambda-A)$ invertible $\iff p(\lambda)\stackrel{\text{def.}}{=}\det(\lambda-A)\neq0$.  
    \end{center}  
    
    Since  $p(\cdot)$ has at least $1$ and at most $n$ (distinct) solutions, one get $\sigma(A)\neq \emptyset$, $\sigma(A)$ contains at most $n$ points. Hence $\varrho(A)\neq\emptyset$ and $\varrho(A)\subset\comp$ is dense.
\end{example}

\begin{lemma}\nl
\label{spectrum is closed}
If $z_0\in\varrho(A)$, then 
$$D\defeq
\{
z\in\comp:|z-z_0|<\frac{1}{\norm{R_{z_0}}_{\mathcal{L}(X)}}
\}
\subset\varrho(A)$$
Hence $\varrho(A)$ is open, and $\sigma(A)$ is closed.
\end{lemma}

\begin{proof}
Write 
\[
z-A=
( z-z_0)+(z_0-A)=(1+(z-z_0)R_{z_0})(z_0-A) \tag{$*$}
\]
If $z\in D$ then $1+(z-z_0)R_{z_0}$ is invertible with:
\[
(1+(z-z_0)R_{z_0})^{-1}=\sum_{n\geq0}(z_0-z)^n R_{z_0}^n \tag{1}
\]
hence also
$$
R_z=(z-A)^{-1}\stackrel{(*)}{=}R_{z_0}(1+(z-z_0)R_{z_0})^{-1}\in\mathcal{L}(X)
$$
For (1) use: if $A\in\mathcal{L}(X)$, $\norm{A}<1$ then with $A^0=Id=1$, 
$$
\sum_{n=0}^\infty A^n\in\mathcal{L}(X)
$$
i.e. the sequence
$$
\left(
S_n=\sum_{k=0}^n A^k
\right)\subset\mathcal{L}(X)
$$
converges, and
$$
\sum_{n=0}^\infty A^n=(1-A^{-1})^{-1}
$$
\end{proof}


\begin{example}\nl
cf. \Cref{examples of compact operators}
% {\color{red}!!!!!!!{add ref to diagonal op}}\\
    Diagonal operator $T=T_\lambda$ continued. Claim: $\sigma(T)=\overline{\{\lambda_k:k\in\natu\}}$.
\end{example}
\begin{proof}
\begin{itemize}
    \item $\sigma(T)\supset\overline{\{\lambda_k:k\in\natu\}}$  

    If $x=e_k$, then $Tx=\lambda_k x$, so $\lambda_k-T$ is not injective, so $\{\lambda_k:k\in\natu\}\subset\sigma(T)$, hence by \Cref{spectrum is closed} $\overline{\{\lambda_k:k\in\natu\}} \subset \sigma(T)$.
    
      

    \item $\sigma(T)\subset\overline{\{\lambda_k:k\in\natu\}}$  
    
    If $\widetilde{\lambda}\notin\overline{\{\lambda_k:k\in\natu\}}$, then $\exists\delta>0$ s.t. $|\widetilde{\lambda}-\lambda_k|>\delta$, $\forall k\in\natu$. \\
    Let $x\in\ell^2$, 
    $$
    y\defeq(\widetilde{\lambda}-T)x=((\widetilde{\lambda}-\lambda_k)x_k)_k
    $$
    So $x_k=(\widetilde{\lambda}-\lambda_k)^{-1}y_k$ and $\norm{x}_{\ell^2}\leq\delta^{-1}\norm{y}_{\ell^2}$ which implies $(\widetilde{\lambda}-T)^{-1}\in\mathcal{L}(H)$ and $\widetilde{\lambda}\in\varrho(T)$
\end{itemize}
\end{proof}

\begin{remark}\nl
    In finite dimension, $(\lambda-A)$ not invertible $\iff(\lambda-A)$ not injective by rank formula. One may wonder if "lack of injectivity"is the only reason for $\lambda\notin \sigma(A)$.
\end{remark}

\begin{definition}\nl
    Consider linear operator $A$ with closed graph and spectrum $\sigma(A)$.
    \begin{itemize}
        \item Point spectrum: $\sigma_p(A)\defeq
        \left\{
        \lambda\in \comp :\lambda-A \text{ not  injective}
        \right\}$
        \item Continuous spectrum: $\sigma_c(A)\defeq
        \left\{
        \lambda\in\comp\setminus\varrho(A):(\lambda-A) \text{ injective }, im(\lambda-A) \text{\ not\ dense }        
        \right\}$
        \item Residual spectrum: $\sigma_r(A)\defeq\sigma(A)\setminus(
        \sigma_p(A)\cup\sigma_c(A))$         
    \end{itemize}
    Elements of point specturm are called {\textbf{eigenvalues} of $A$} with {\textbf{eigenspaces}}(null spaces) $$ker(\lambda-A)=\{x\in D_A:Ax=\lambda x\}\neq\{0\}$$
\end{definition}

\begin{example}[Shift operator]
\label{example: shift operator}
\begin{enumerate}[1)]
    \item \func{S}{\ell^2}{\ell^2}, $S(x_1,x_2...)=(0,x_1,x_2...)$. Then $0\in\sigma(S)$:   
    
    Indeed $S$ is not invertible since it's not surjective: $\forall y\in\ell^2$ with $y_1\neq0$, $ y\notin im(S)$, but $0\notin\sigma_p(S)$: $S$ is injective.\\
    In fact $Sx=\lambda x\implies0=\lambda x_1$, $x_n=\lambda x_{(n+1)},\,\,\forall n\in\natu\implies x_k=0$ for all $k\in\natu$. So $\sigma_p(S)=\emptyset$. (In fact, $\sigma(S)=\overline{D}=\{\xi\in\comp:|\xi|\leq1\}$ the closed unit disk, $\sigma_r(S)=D$, $\sigma_c(S)=\partial D=S^1$, the unit circle.)
    % can be seen by considering \xi>1, and show injectivity and surjectivity

    \item $X=\comp^n$, $\sigma(\cdot)=\sigma_p(\cdot)$

    \item  $T_\lambda$ is indicative of a certain class: $ X$ Hilbert, $T\in\mathcal{L}(X)$ compact, self-adjoint. \\
    Then by Riesz-Schauder, $\sigma(T)\setminus\{0\}=\sigma_p(T)\setminus\{0\}$
\end{enumerate}
    
\end{example} 

% Below are content from written lecture note 9.2-10

\subsection{Spectral Theory in Hilbert space}
%\todo{should mention $D_A$ is a dense subspace as in notes?}
Consider $(H,\inne{\cdot}{\cdot}$, Hilbert space over $\comp$, \func{A}{{D_A\subset H}}{H} linear, with adjoint \func{\dual{A}}{D_{A^*}\subset H}{H}.
Recall $\dual{A}$ characterised by $$\forall x\in D_A, y\in D_{A^*} \qquad \inne{\dual{A}y}{x}=\inne{y}{Ax}$$ 
and 
$$D_{A^*}=\{y\in H : \ell_y:D_A\to \comp,\,x\mapsto \inne{y}{Ax}\,\text{ is continuous }\}$$
In sequel write $A\subset B$, reads $B$ is extension of $A$ if $D_A\subset D_B$ and $B|_{D_A}=A$.


\begin{definition}\nl
	i)  $A$ is {\textbf{symmetric}} if $A\subset \dual{A}$, i.e. $D_A\subset D_{A^*}$ and $\inne{Ax}{y}=\inne{x}{Ay}$ for all $x,y\in D_A$.\\
	ii) $A$ is {\textbf{self-adjoint}} if $A=\dual{A}$, i.e. A symmetric with $D_{A^*}=D_A$.
\end{definition}

What can we say about spectrum $\sigma(A)$ for such $A$?

\begin{lemma}
	If $A$ is symmetric, $\sigma_p(A)\subset\real$.

\end{lemma}
	\begin{proof}
		Let $\lambda\in\sigma_p(A)$ with non-zero eigenvector $x\in ker(\lambda-A)$. \\
		Then $\lambda\norm{x}_H^2=\inne{Ax}{x}\stackrel{symm.}{=}\inne{x}{Ax}=\overline{\inne{Ax}{x}}=\overline{\lambda}\norm{x}_H$.   
  
  So $\lambda=\overline{\lambda}\implies \lambda\in\real$.
	\end{proof}
Is this true for all the spectrum of $  H$ (cf. $\comp^n$)?
%question part----------------------------------

% the support is denoted as C_c(0,1), outside of which the function takes zero value

\begin{example}[Weak derivative]\nl
	$H$=$L^2(0,1)$, $\inne{f}{g}=\int_0^1f\bar{g} \mathrm{d}t$. $A\in\frac{\mathrm{d}}{\mathrm{d}t}$.\\
	More precisely, $f\in  H$ is said to have a \textbf{weak derivative} $f'$ if $f'\defeq{v}$ for some $v\in   H$, 
 $$\int_0^1 fg'\mathrm{d}t=-\int_0^1 vg\mathrm{d}t \qquad \forall g\in C^\infty_c(0,1)$$
	Consider
	$$
		A_\infty =i\frac{\mathrm{d}}{\mathrm{d}t}:C^\infty_c(0,1)\subset H\to H
	$$
	and extensions $A_1, A_2, A_3$ with
	\begin{equation}
		\begin{aligned}
			 & D_{A_1}=H^1\defeq\{f\in H: f \text{ has a weak derivative  f'}\} \\
			 & \bigcup                                                            \\
			 & D_{A_2}=\{f\in H^1:f(0)=f(1)\}\text{ (periodic boundary condition)}             \\
			 & \bigcup                                                            \\
			 & D_{A_3}=\{f\in H^1: f(0)=0=f(1)\}\text{ (Dirichlet boundary condition)}         \\
		\end{aligned}
	\end{equation}
	Set $A_k(f)=if',\,\,\forall f\in D_{A_k}$. Evidently $ A_\infty\subsetneq A_3\subsetneq A_2 \subsetneq A_1$.\\
	One can show that
	$$
		A_3\subset\dual{A_1}\subset\dual{A_2}=A_2\subset\dual{A_3}
	$$
	So: $A_3$ is symmetric but because $A_3\subsetneq A_2\subset \dual{A_3}$ not self-adjoint, $A_2$ is self-adjoint.\\
	Claim:
	\begin{itemize}
		\item [i)] $\sigma(A_1)=\sigma_p(A_1)=\comp$, $ \varrho(A_1)=\emptyset$.
		\item [ii)] $\sigma(A_2)-\sigma_p(A_2)=a\pi\inte$, $\varrho(A_2)=\comp$.
		\item [iii)] $\sigma(A_3)=\comp$, $\sigma_p(A_3)=\emptyset$, $\varrho(A_3)=\emptyset$
	\end{itemize}
	So symmetric operators can have imaginary spectrum!
	\begin{proof}
		i) For $\lambda\in\comp$, pick $f(t)=e^{-i\lambda t}\in ker(\lambda-A_1)$.\\
		ii) For $k\in\inte$, $f(t)=e^{-2\pi ikt}\in D_{A_2}\cap ker(a\pi k-A_2)$,
		so $2\pi \inte\subset \sigma_p(A_2)\subset\sigma(A_2)$.
		Let $\lambda\in\comp\setminus a\pi\inte$.
		Need to show: $\lambda\in\varrho(A_2)$ i.e. \func{\lambda-A_2}{D_{A_2}}{H} is invertible and $(\lambda-A_2)^{-1}\in{H}$.\\
		For $g\in H$, the general sd. of $\lambda f-if'=g$ can be obtained via variation of constant formula:
		$$
			(*)\rightsquigarrow f(t)=ae^{-i\lambda t}+i \int_0^t e^{i\lambda}g(s)\mathrm{d}s
		$$
		for some $a\in\comp$. But if $\lambda\notin2\pi\inte$, the b.b. determines $a$ uniquely:
		$$
			a=f(0)=f(1)=ae^{-i\lambda}+i\int_0^1e^{i\lambda(s-1)}g(s)\mathrm{d}s$$
		so
		$$
			a=(1-e^{-i\lambda})^{-1}i\int_0^1e^{i\lambda(s-1)}g(s)\mathrm{d}s
		$$
		so $\lambda-A_2$ is invertible and
		$$
        \norm{f}_{L^2}\leq |A|+\norm{g}_{L^2}\leq(|1-e^{-i\lambda}|^{-1})\norm{g}_{L^2}
        $$
		which shows $(\lambda-A_2)^{-1}\in\mathcal{L}(H)$\\
		iii) If $A_3f=if'=\lambda f$ for some $\lambda\in\comp$, then by (*) $f(t)=ae^{-i\lambda t}$ and $a=0$ since $f(0)=0$, so $\sigma_p(A_3)=\emptyset$.\\
		On the other hand $(\lambda-A_3)$ for $\lambda\in\comp$ is never surjective ($\implies \lambda\in\varrho(A_3)$). Indeed, consider $g(s)=e^{i\lambda s}$, then using $(*)$ and b.c. get $a=0$ and
		$$
        f(t)=ie^{-i\lambda t}\int_0^1e^{i\lambda s} \mathrm{d}s=ite^{-i\lambda t}
        $$
		but $f(1)\neq0$ so $f\notin D_{A_3}$.
	\end{proof}
\end{example}
%question part-------------------------------------
We have seen: symmetric operators can have imaginary spectrum, but:
\begin{lemma}
\label{images of symmetric operators bounded below by imaginary part}

Let $A\subset \dual{A}$ (\textit{i.e.} $A$ is symmetric). Then 
 $$
 \forall \xi \in \comp \, \forall u\in D_A: \qquad  \norm{(\xi-A)u}_H\geq |Im(\xi)|\norm{u}_H
  $$
	(So for $\xi\notin\real\implies(\xi-A)$ injective, i.e. $\xi\notin\sigma_p(A)$)
\end{lemma}
\begin{proof}
		For $u\in D_A:$ $\inne{u}{Au}\stackrel{A\subset\dual{A}}{=}\inne{Au}{u}=\overline{\inne{u}{Au}}\in\real$. Hence,
		$$
			|Im(\xi)|\norm{u}^2_H=|Im(\inne{u}{(\xi-A)u)}|\leq|\inne
			{u}{(\xi-A)u}|\leq\norm{u}_H\norm{(\xi-A)u}_H
		$$
\end{proof}
% mind Im and im, Im is imaginary part and im is image. Pay attention  to the difference.
The example also nicely illustrates:
\begin{proposition}
\label{proposition: self adjoint operators have real spectrum}
	If $A=\dual{A}$, then $\sigma(A)\subset \real$
\end{proposition}
\begin{proof}
		Let $\xi\in\comp\setminus\real$. Want to show $\xi\in\varrho(A)$, i.e. \func{\xi-A}{D_A}{H} is bijective with $(\xi-A)^{-1}\in\mathcal{L}(H)$.\\
		We will show:\\
  \begin{center}
     		$(*)$ $\xi-A$ is surjective 
  \end{center}
		Once  $(*)$ holds, we are done: by previous lemma, $\xi-A$ is injective hence bijective, and surjectivity + same lemma also yields
		$$
			\norm{(\xi-A)^{-1}}_{\mathcal{L}(H)}\leq\frac{1}{|Im(\xi)|}
		$$
		\textbf{proof of (*):} we first show
		$$
			(**)\qquad im(\xi-A)(\subset H) \text{  is closed}
		$$
		Assume $v_k=(\xi-A)u_k\xrightarrow{k\to\infty}v$. By \Cref{images of symmetric operators bounded below by imaginary part},
		$$
			\norm{u_k-u_l}_H\leq\frac{1}{|Im(\xi)|}\norm{v_k-v_l}_H\xrightarrow{k,l\to\infty}0$$
   Hence $(u_k)$ is Cauchy and $u_k\to u$ for some $u\in H$. But $A=\dual{A}$ has a closed graph so $v=(\xi-A)u$, i.e. (**) holds.\\ 
   \begin{unexaminable}
    every self-adjoint operator is bounded \href{https://math.stackexchange.com/questions/2602548/is-every-self-adjoint-operator-bounded}{link here}.
   \end{unexaminable}
   % \todo{Is A a bounded operator?(unless A is defined everywhere)}
   % Short answer: we don't need boundedness. consider operator on $\ell^2$, that multiplies n-th term by n. This is unbounded, but for any $e_k$ it gives finite result.  
   
		Back to (*): Due to (**), $M\defeq Im(\xi-A)$ is closed. Assume $M\neq H$. Pick $v\in M^\perp\setminus\{0\}$. Then
		$$
			\forall u\in D_A:\inne{v}{(\xi-A)u}=0\text{  or  }\inne{v}{Au}=\overline{\xi}\inne{v}{u}$$
		Hence,
		$
			D_A\ni u\mapsto \inne{v}{Au}
		$
		is continuous, $v\in D_{A^*}=D_A$ and $Av=\dual{A}v=\overline{\xi}v$
		but by \Cref{images of symmetric operators bounded below by imaginary part},
		$$|Im(\xi)|\norm{v}_H\leq\norm{(\overline{\xi}-A)v}_H=0$$
		which yields $v=0$. Contradiction.
  \end{proof}
 
% Below are content for week 10
\subsection{Spectral theorem for compact self-adjoint operators}
$H$: Hilbert space over $\comp$, inner product $\inne{\cdot}{\cdot}$, with $\norm{x}^2_H=\inne{x}{x}$. \\
Following is an extension (!) of the familiar result from linear algebra concerning diagonalization of symmetric matrices.
\begin{theorem}[Riesz-Schauder]\nl
\label{riesz schauder}
	Let  \func{T}{H}{H} be compact and self-adjoint, then:
	\begin{itemize}
		\item [i)] $\sigma(T)\subset\real$
		\item [ii)] $\sigma_p(T)$ contains at most countably many eigenvalues $\lambda_k\in\real\setminus\{0\}$, which accumulate at most at $\lambda=0$
		\item [iii)] One can choose $e_k$  corresponding to $\lambda_k$ such that $e_k\perp e_l$ $\forall k\neq l$ and one has $\forall x\in H:Tx=\sum_k \lambda_k e_k\inne{x}{e_k}$
	\end{itemize}
\end{theorem}
\begin{example}\nl
	Diagonal operator (\Cref{examples of compact operators}) \func{T_\lambda}{\ell^2}{\ell^2} continued.\\
	\begin{itemize}
	    \item $T_\lambda$ is compact $\iff \lim_{k\to\infty}\lambda_k=0$
	    \item $T_\lambda$ is self-adjoint $\iff \lambda_k\in\real\,,\forall k$
	\end{itemize}
	 and we know $\sigma_p(T_\lambda)=\overline{\{\lambda_k:k\in\natu\}}$.
\end{example}
We start with the following lemma:
\begin{lemma}[Lemma 1]
\label{lemma: eigenvectors are orthogonal for T self-adjoint}
	$T\in\mathcal{L}(H)$, self-adjoint. If $\lambda_1\neq\lambda_2$, $\lambda_1,\lambda_2\in\sigma_p(T)$ with eigenvectors $e_1,e_2$, i.e. $\lambda_1 e_1=Te_1$ and $\lambda_2e_2=Te_2$, then $\inne{e_1}{e_2}=0$
\end{lemma}
\begin{proof}
   		\begin{equation}
			\begin{split}
			\lambda_1\inne{e_1}{e_2}&=\inne{\lambda_1e_1}{e_2}=\inne{Te_1}{e_2}\stackrel{SA.}{=}\inne{e_1}{Te_2}\\
			&=\inne{e_1}{\lambda_2e_2}\stackrel{\lambda_2=\overline{\lambda_2}}{=}\lambda_2\inne{e_1}{e_2}
			\end{split}
		\end{equation}
		Since $\lambda_1\neq\lambda_2$, $\inne{e_1}{e_2}=0$. 
\end{proof}

Henceforth, we always assume \func{T}{H}{H} is compact and self-adjoint. In particular, Lemma 1(\Cref{lemma: eigenvectors are orthogonal for T self-adjoint}) is in force.\\
Define, for $\lambda\in\sigma_p(T)\setminus\{0\}$, 
$$X_\lambda=ker(\lambda-T)\neq\{0\}$$
% the kernel is a closed subspace hence can take orthogonal complement
By  Lemma 1(\Cref{lemma: eigenvectors are orthogonal for T self-adjoint}):
$$X_{\lambda} \perp X_{\lambda '} \qquad \forall \lambda \neq {\lambda '} \ \text{and} \ \lambda , {\lambda '} \in \sigma_p(T)\setminus\{0\}$$


\begin{lemma}[Lemma 2]\nl
\label{lemma: eigenspaces have finite dimensions for compact self-adjoint operators}
	Let $\lambda\in\sigma_p(T)\setminus\{0\}$.
	\begin{enumerate}[i)]
		\item $\dim (X_\lambda)<\infty$
		\item $\forall r>0$: $\sigma_p(T)\setminus B_r(0)$ is finite.
	\end{enumerate}
\end{lemma}
\begin{proof}
		i) Let $B_r^{X_\lambda}(0)=\{x\in X_\lambda:\norm{x}_H<r\}$, which is a bounded set. By compactness of $T$, $\overline{T(B_r^{X_\lambda}(0))}$ is compact. 
  But since $Tx=\lambda x\, \ \forall x\in X_\lambda$, taking $r=1$
  
  $$T(B_1^{X_\lambda}(0))=\lambda B_1^{X_\lambda}(0)$$
  
  So
  \begin{center}
        $\lambda\overline{(B_1^{X_\lambda}(0))}$ is compact $\implies \overline{B_1^{X_\lambda}(0)}$ is compact $\implies \dim(X_\lambda)<\infty$
  \end{center}
\begin{unexaminable}
    By \Cref{thm: unit ball not compact in infty dim}, the unit ball is compact if and only if the space is finite dimensional
\end{unexaminable}
ii) Suppose not, then $\exists r>0$: $\sigma_p(T)\setminus B_r(0)$ is infinite (one can show $\sup_{\lambda\in\sigma_p(T)}|\lambda|<\infty$, no proof provided here).  
\begin{unexaminable}
    A proof of $\sup \sigma(A)=\sup_{\norm{x}=1}\inne{x}{Ax}$ can be found on Pg.231(243 in pdf) in THEOREM 5.3.16 of \href{https://people.math.ethz.ch/~salamon/PREPRINTS/funcana-ams.pdf}{\textcolor{blue}{the notes provided on Blackboard}}.
\end{unexaminable}

Then one can pick sequence $(\lambda_k)\subset\sigma_p(T)$ with 
\begin{center}
$\lambda_k\neq\lambda_l, \ \forall k\neq l$ and $|\lambda_k|>r, \ \forall k \in \natu$    
\end{center}

Let $e_k\neq0$ be eigenvector for $\lambda_k$: 

$$Te_k=\lambda_ke_k\qquad \forall k\in \natu$$ 
% {\color{red}{Are those normalised eigenvectors?}!!!!!!!!!!!!!!!!!!!}
By compactness of $T$, $\exists\Lambda\subset\natu$ such that for some $y\in H$, $$Te_k\xrightarrow[k\to\infty,k\in\Lambda]{\norm{\cdot}_H}y\in H$$
		Hence,
		$$(\lambda_ke_k)\xrightarrow[k\to\infty,k\in\Lambda]{\norm{\cdot}_H}y\in H$$
		In particular, $(\lambda_ke_k)_{k\in\Lambda}$ is Cauchy. But for $k\neq l$,
		$$\norm{\lambda_ke_k-\lambda_le_l}^2_H=
                \inne{\lambda_ke_k}{\lambda_ke_k}
			+\inne{\lambda_le_l}{\lambda_le_l}
			+\underbrace{\inne{\lambda_ke_k}{\lambda_le_l}}_{=\lambda_k\overline{\lambda_l}\inne{e_k}{e_l}\stackrel{Lemma1}{=}0}
			+\inne{\lambda_le_l}{\lambda_ke_k}
		$$
		Assume, $\norm{e_k}_H=\norm{e_l}_H=1$, otherwise replace $e_k$ by $e_k/\norm{e_k}_H$,
		$$
			\norm{\lambda_ke_k-\lambda_le_l}^2_H=|\lambda_k|^2+|\lambda_l|^2>2r
		$$
\end{proof}  

Now return to the proof of \Cref{riesz schauder}.  

\begin{proof}
        i) $\sigma(T) \subset \real$ by \Cref{proposition: self adjoint operators have real spectrum}.  
        
	ii) By Lemma 2 ii)(\Cref{lemma: eigenspaces have finite dimensions for compact self-adjoint operators}),
	$$A_n=\sigma_p(T)\cap\{z:\frac{1}{n+1}\leq|z|\leq\frac{1}{n}\}(\subset \sigma_p(T)\setminus B_{1/(n+1)}(0))$$
	is finite and $\sigma_p(T)\setminus\{0\}=\bigcup_n A_n$ is thus countable. This also implies $\sigma_p(T)\setminus\{0\}$ has no accumulation point.\\
	iii) By applying Gram-Schmidt to the (finite-dimensional) space $X_\lambda$, $\lambda\in\sigma_p(T)\setminus\{0\}$, can ensure that eigenvectors $e_k$, $e_l$ to eigenvalues $\lambda_k=\lambda_l$ are orthonormal (that is $e_k \perp e_l$ and $\norm{e_k}=\norm{e_l}=1$). If they belong to distinct eigenvalues,  this is automatic after normalizing them to have norm $1$ by Lemma 1(\Cref{lemma: eigenvectors are orthogonal for T self-adjoint}).\\  

 \begin{remark}
      Lemma 2 gives important structural information on the spectrum. In particular, ii) implies that $\sigma_p(T)\setminus\{0\}$ is countable with no accumulation point and by i) each eigenvalue has "finite multiplicity".
 \end{remark}  
 
	So let $(\lambda_k)$ be the elements of $\sigma_p(T)\setminus\{0\}$, counted with multiplicities( i.e. $\dim(X_{\lambda_k})$ copies of $\lambda_k$. Let
	$$
		X\defeq\overline{span\{e_k\}}=\overline{\bigoplus_{\lambda\in\sigma_p(T)\setminus\{0\}}X_\lambda}
	$$
	\textbf{Claim 1:} $\forall x\in X$: $x=\sum_k \inne{x}{e_k}e_k$\\
	Proof of Claim1:   
 
        Let $x_n=\sum_{k \leq n} \inne{x}{e_k}e_k$.
	Then $\forall n\geq0$,
	$$
	\norm{x_n}_H^2=\sum_{k\leq n}|\inne{x}{e_k}|^2=\inne{x_n}{x}\leq\norm{x_n}_H\norm{x}_H$$
	so $\norm{x_n}_H\leq\norm{x}_H$ unless $x_n=0$. Hence
	$$\sum_{k}|\inne{x}{e_k}|^2=\lim_{n\to\infty}\norm{x_n}_H^2\leq\norm{x}_H^2<\infty$$
	and $\forall n\geq m\geq0$:
	$$\norm{x_n-x_m}^2_H=\sum_{m\leq k\leq n}|\inne{x}{e_k}|^2\xrightarrow{n,m\to\infty}0$$
	Thus $x_n\xrightarrow[\norm{\cdot}_H]{n\to\infty}y\in X$. Moreover, $\forall k\geq 0$,
	$$\inne{x-y}{e_k}=\lim_{n\to\infty}\inne{x-x_n}{e_k}=\inne{x}{e_k}-\lim_{n\to\infty}\inne{x_n}{e_k}=0$$
	so $x=y$. With Claim 1 and continuity of $T$ we have
	$$\forall x\in X: \qquad Tx=\sum_k \inne{x}{e_k}Te_k=\sum_k \inne{x}{e_k}\lambda_ke_k$$
	It remains to argue:\\
	{\bf Claim 2}: $Y\defeq X^\perp=ker(T)$, which concludes the proof.  
 
	Proof of Claim2:\\
	If $Y=\{0\}$, it's trivial. So one can assume $Y\neq\{0\}$. First, note that
	\begin{equation}T(Y)\subset Y\tag{$*$}\end{equation}
	For, if $y\in Y$ then $\forall k$:
	\[\inne{e_k}{Ty}=\inne{Te_k}{y}=\lambda\inne{e_k}{y}\stackrel{Y\perp\{e_k\}}{=}0\]
	By $(*)$,
	\[T_Y=T|_Y:Y\to Y\] is well defined. $T_Y$ inherits compactness and self-adjointness from $T$ (exercise). We want to show that \func{T_Y}{Y}{Y}: $y\mapsto0 $ is the  "0-map" on $Y$, or equivalently, $\norm{T_Y}_{\mathcal{L}(X)}=0$. If not, one can show (no proof) that
	\[\sigma_p(T_Y)\setminus\{0\}\neq\emptyset\]
	(in fact $+\lambda$ or $-\lambda$ is an eigenvalue where $\lambda=\norm{T_Y}_{\mathcal{L}(X)}$)   
 % \begin{unexaminable}
     
 % \end{unexaminable}
 But this can't be because if $e\in Y$ is an eigenvector for $\lambda\in \sigma_p(T_Y)$, $\lambda\neq 0$, then
	\[Te\stackrel{e\in Y}{=}T_Y e=\lambda e\]
	So $e\in X_\lambda\subset X=Y^\perp$. But $Y^\perp\cap Y=\{0\}$.
\end{proof}
\begin{remark}
	We have actually shown that $H$ admits the orthogonal decomposition:
	\[H=ker(T)\oplus
		\overline{\bigoplus_{\lambda\in\sigma_p(T)\setminus\{0\}}X_\lambda}
	\]
	where there are countably many $X_\lambda$ and each is of finite dimension.
\end{remark}



\end{document}