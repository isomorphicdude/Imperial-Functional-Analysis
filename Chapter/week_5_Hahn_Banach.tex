\documentclass{article}
\usepackage[utf8]{inputenc}
\usepackage{geometry}
\geometry{left=3cm,right=3cm,top=2cm,bottom=2cm}
\usepackage[utf8]{inputenc}
\usepackage{amsmath, amsfonts, amssymb, amsthm}
\usepackage[framemethod=TikZ]{mdframed}
\usepackage{mathrsfs}
\usepackage{comment}
\usepackage{enumerate}
\usepackage{xcolor}
\usepackage{titlesec}
\usepackage{setspace}
\usepackage[hidelinks,backref]{hyperref}
\usepackage{cleveref}
\usepackage[most]{tcolorbox}
\usepackage{ragged2e}
\usepackage{todonotes}
\usepackage{cleveref}
\usepackage{mathtools}

%
\titleformat*{\section}{\LARGE \bfseries}
\titleformat*{\subsection}{\Large \bfseries}
\titleformat*{\subsubsection}{\Large \bfseries}
% \titleformat*{\paragraph}{\large \bfseries}
\titleformat*{\subparagraph}{\large \bfseries}



% General 
\newcommand{\nextline}{\hfill\break}
\newcommand{\nl}{\nextline\rm}
% \newcommand{\placeholder}{{\bf\color{red} NOOOOOOT COMPLEEEEEEET! COOOOOOOM BAAAAAACK!!!}}
\newcommand{\placeholder}{\todo{NOOOOOOT COMPLEEEEEEET! COOOOOOOM BAAAAAACK!!!}}

\newcommand{\defeq}{\stackrel{def.}{=}}
% FA and LA
% inner product: \inne{a}{b}
\newcommand{\inne}[2]{\left<{#1},{#2}\right>}

% norm: \norm{a}
\newcommand{\norm}[1]{\left\|{#1}\right\|}

% Curly H
\newcommand{\hbs}{$\mathscr{H}$ }
\newcommand{\hbp}{\mathscr{H}}

% Dual : \dual{x}
\newcommand{\dual}[1]{{#1}^*}

% Sequence from 1  to infty: \sequ{x_n}
\newcommand{\sequ}[1]{\left({#1}\right)_1^\infty}

% f: A-> B \func{f}{A}{B}
\newcommand{\func}[3]{${#1}:{#2}\xrightarrow{}{#3}$}

% interior
\newcommand{\interior}{\textrm{int}}

% Bounded linear funcs
\newcommand{\blf}[2]{\mathcal{L}({#1},{#2})}

\newcommand{\prf}{\textit{proof}:   }



% Fields 
\newcommand{\real}{\mathbb{R}}
\newcommand{\comp}{\mathbb{C}}
\newcommand{\inte}{\mathbb{Z}}
\newcommand{\natu}{\mathbb{N}}





% Theorems
% \newtheorem{example}{Example}[subsection]
% \newtheorem{definition}[example]{Definition}
% \newtheorem{proposition}[example]{Proposition}
% \newtheorem{remark}[example]{Remark}
% \newtheorem{theorem}[example]{Theorem}
% \newtheorem{lemma}[example]{Lemma}
% \newtheorem{corollary}[example]{Corollary}


% for numbering the theorems            
\theoremstyle{plain}
%%%%%%%%%%%%%%%%%%%%%%%%%%%%%%%%%%%%%%%%%%%%%%%%%%%%%%%%%%%
\newtheorem{theorem}{Theorem}[section]
\newtheorem{lemma}[theorem]{Lemma}
\newtheorem{corollary}[theorem]{Corollary}
\newtheorem{proposition}[theorem]{Proposition}
%%%%%%%%%%%%%%%%%%%%%%%%%%%%%%%%%%%%%%%%%%%%%%%%%%%%%%%%%%%
% the following are not in italics
\theoremstyle{definition}
\newtheorem{definition}[theorem]{Definition}
\newtheorem{example}[theorem]{Example}
\newtheorem{remark}[theorem]{Remark}
\newtheorem{claim}[theorem]{Claim}
%%%%%%%%%%%%%%%%%%%%%%%%%%%%%%%%%%%%%%%%%%%%%%%%%%%%%%%%%%%%

% proof box
\newtcbtheorem[no counter]{pf}{Proof}{
  enhanced,
  rounded corners,
  attach boxed title to top,
  colback=white,
  colframe=black!25,
  fonttitle=\bfseries,
  coltitle=black,
  boxed title style={
    rounded corners,
    size=small,
    colback=black!25,
    colframe=black!25,
  } 
}{prf}

% extra content box to put in contents not covered in the lecture notes
% use the command \begin{unexaminable}
\newmdenv[skipabove=7pt, skipbelow=7pt,
    rightline=false, leftline=false, topline=false, bottomline=false,
    backgroundcolor = gray!10,
    innerleftmargin=1in, innerrightmargin=1in, innertopmargin=5pt,
    leftmargin=-1in, rightmargin=-1in, linewidth=4pt,
    innerbottommargin=5pt]{unexamBox}
\newenvironment{unexaminable}{\begin{unexamBox}}{\end{unexamBox}}


% clever ref settings
\crefname{lemma}{lemma}{lemmas}
\Crefname{lemma}{Lemma}{Lemmas}
\crefname{theorem}{theorem}{theorems}
\Crefname{theorem}{Theorem}{Theorems}

% formatting 
% https://tex.stackexchange.com/questions/217497/aligning-stackrel-signs-beneath-each-other-using-split
\newlength{\leftstackrelawd}
\newlength{\leftstackrelbwd}
\def\leftstackrel#1#2{\settowidth{\leftstackrelawd}%
{${{}^{#1}}$}\settowidth{\leftstackrelbwd}{$#2$}%
\addtolength{\leftstackrelawd}{-\leftstackrelbwd}%
\leavevmode\ifthenelse{\lengthtest{\leftstackrelawd>0pt}}%
{\kern-.5\leftstackrelawd}{}\mathrel{\mathop{#2}\limits^{#1}}}

\doublespacing
\RaggedRight
% modify innerrightmargin if floats were lost

\usepackage{amsfonts, amsmath, amssymb, amsthm, thmtools, bm}
\usepackage{avant} % Use the Avantgarde font for headings

% Boxed/framed environments
\newtheoremstyle{royalnumbox}%
{0pt}% Space above
{0pt}% Space below
{\normalfont}% Body font
{}% Indent amount
{\small\bf\sffamily\color{royal}}% Theorem head font
{\;}% Punctuation after theorem head
{0.25em}% Space after theorem head
{\sffamily \color{royal} 
    \thmname{#1} 
    \thmnumber{#2} \thmnote{\bfseries\color{black}---\nobreakspace#3.}} % Optional theorem note
\renewcommand{\qedsymbol}{$\blacksquare$}% Optional qed square

\newtheoremstyle{blacknumex}% Theorem style name
{5pt}% Space above
{5pt}% Space below
{\normalfont}% Body font
{} % Indent amount
{\small\bf\sffamily}% Theorem head font
{\;}% Punctuation after theorem head
{0.25em}% Space after theorem head
{\sffamily
    \thmname{#1}
    \thmnumber{#2}
    \thmnote{---\nobreakspace#3.}}% Optional theorem note

\newtheorem*{notation}{Notation}
\newtheorem*{hint}{Hint}
\newtheorem*{solution}{Solution}

\newcounter{dummy} 
\numberwithin{dummy}{section}

\theoremstyle{royalnumbox}
\newtheorem{definitionT}[dummy]{Definition}
\newtheorem{theoremT}[dummy]{Theorem}
\newtheorem{lemmaT}[dummy]{Lemma}
\newtheorem{corollaryT}[dummy]{Corollary}
\newtheorem{propositionT}[dummy]{Proposition}
\newtheorem{propertyT}[dummy]{Property}
\newtheorem{remarkT}[dummy]{Remark}

\theoremstyle{blacknumex}
\newtheorem{exampleT}[dummy]{Example}
\newtheorem{exerciseT}[dummy]{Exercise}

\numberwithin{equation}{section}

\RequirePackage[framemethod=default]{mdframed}

% Definition box
\newmdenv[skipabove=7pt, skipbelow=7pt,
rightline=false, leftline=true, topline=false, bottomline=false,
backgroundcolor = reddish!10, 
linecolor=reddish,
innerleftmargin=5pt, innerrightmargin=30pt, innertopmargin=5pt,
leftmargin=0cm, rightmargin=0cm, linewidth=4pt,
innerbottommargin=5pt]{dBox}

% Main Theorem box
\newmdenv[skipabove=7pt, skipbelow=7pt,
rightline=false, leftline=true, topline=false, bottomline=false,
backgroundcolor=c0!10, 
linecolor=c0,
innerleftmargin=5pt, innerrightmargin=30pt, innertopmargin=5pt,
leftmargin=0cm, rightmargin=0cm, linewidth=4pt, innerbottommargin=5pt]{tBox}

% Lemma/Corollary/Proposition/Property box
\newmdenv[skipabove=7pt, skipbelow=7pt,
rightline=false, leftline=true, topline=false, bottomline=false,
backgroundcolor = c0!10, 
linecolor=c0!80,
innerleftmargin=5pt, innerrightmargin=30pt, innertopmargin=5pt,
leftmargin=0cm, rightmargin=0cm, linewidth=4pt,
innerbottommargin=5pt]{lBox}

% Example/Remark/Exercise box
\newmdenv[skipabove=7pt, skipbelow=7pt,
rightline=false, leftline=true, topline=false, bottomline=false,
backgroundcolor = mossgreen!10!white,
linecolor = mossgreen,
innerleftmargin=5pt, innerrightmargin=30pt, innertopmargin=5pt,
leftmargin=0cm, rightmargin=0cm, linewidth=4pt,
innerbottommargin=5pt]{exBox}

% Proof box
% \newmdenv[skipabove=7pt, skipbelow=7pt,
% rightline=false, leftline=true, topline=false, bottomline=false,
% linecolor=gray,
% innerleftmargin=5pt, innerrightmargin=30pt, innertopmargin=5pt,
% leftmargin=0cm, rightmargin=0cm, linewidth=4pt,
% innerbottommargin=5pt]{proofBox}



% Creates an environment for each type of theorem and assigns it a theorem text style from the "Theorem Styles" section above and a colored box from above
\newenvironment{definition}{\begin{dBox}\begin{definitionT}}{\end{definitionT}\end{dBox}}
\newenvironment{theorem}{\begin{tBox}\begin{theoremT}}{\end{theoremT}\end{tBox}}
\newenvironment{lemma}{\begin{lBox}\begin{lemmaT}}{\end{lemmaT}\end{lBox}}
\newenvironment{proposition}{\begin{lBox}\begin{propositionT}}{\end{propositionT}\end{lBox}}
\newenvironment{corollary}{\begin{lBox}\begin{corollaryT}}{\end{corollaryT}\end{lBox}}
\newenvironment{property}{\begin{lBox}\begin{propertyT}}{\end{propertyT}\end{lBox}}


% \newenvironment{proof*}{\begin{proofBox}\begin{proof}}{\end{proof}\end{proofBox}}
\newenvironment{exercise}{\begin{exBox}\begin{exerciseT}}{\hfill{\color{royal}}\end{exerciseT}\end{exBox}}
\newenvironment{remark}{\begin{exBox}\begin{remarkT}}{\end{remarkT}\end{exBox}}
\newenvironment{example}{\begin{exBox}\begin{exampleT}}{{}\end{exampleT}\end{exBox}}

\title{Week 5 Hahn-Banach and Consequences}

\begin{document}
\maketitle

\section{Hahn-Banach Theorem} 

\begin{definition}
\label{sublinear map}
(Sublinear functional)
	Let $X$ be a vector space. \func{p}{X}{\real} is called \textbf{sublinear} if the following holds
	\begin{enumerate}[i)]
		\item $p( \alpha x)= \alpha p(x),$ $\forall x\in X$ and $\alpha\geq0$
		\item $p(x+y) \leq p(x)+p(y),\,\,\forall x,y\in X$
	\end{enumerate}
\end{definition}

\begin{example}
    Any linear functional is also sublinear. Also, the norms on $X$ are sublinear.
\end{example}  

\begin{theorem}
(Hahn-Banach)
	Let $M\subset X$ be a linear subspace, \func{p}{X}{\real} is sublinear, and \func{f}{M}{\real} is linear with 
	
	$$
	f(x)\leq p(x) \qquad \forall x\in M
	$$
	
	Then, there exists a linear map \func{F}{X}{\real} with $F|_M=f$ and
	
	$$F(x)\leq p(x) \qquad \forall x\in X$$
\end{theorem}

\begin{lemma}[one-dimensional extension]
\label{ODEX}
	Let $X$ be a normed vector space over $\mathbb F$ ($\mathbb{C}$ or $\mathbb{R}$), $Y_n$ is a proper subspace of $X$. Let $v\in X\backslash Y$, $X_{n+1}=\{x+hv:x\in X_n,\,h\in\mathbb{C}\}$
	. If $T_n:Y\xrightarrow{}\mathbb{F}$ is a bounded linear functional, then there exists a bounded linear functional $T_{n+1}:X_{n+1}\xrightarrow{}\mathbb{F}$ satisfying:
	\begin{itemize}
		\item $T_{n+1}(x)=T_n(x)$ for all $x\in X_n$
		\item $\norm{T_{n+1}}=\norm{T_n}$
	\end{itemize}
\end{lemma}

\begin{pf}{One-dimensional extension}{}
		Define linear functional $P:X_{n+1}\xrightarrow{}\mathbb{R}$ by
		$$
			P(x+kv)=T_n(x)-Ck,\,\forall\,x\in X_n,\,k\in\mathbb{R}
		$$
		where C is a constant to be determined.
		First we shall check linearity, which is left as an exercise.
		Then we shall show that we can find a proper constant $C$ so that $\norm{P}=\norm{T_n}$. Note that $X_n\subset X_{n+1}$, so we have
		\begin{equation}
			\begin{split}
				\norm{P}&=\sup_{x\in X_{n+1}}(\{|Px|:\norm{x}=1\})\\
				&\geq\sup_{x\in X_{n}}(\{|Px|:\norm{x}=1\})\\
				&=\sup_{x\in X_{n}}(\{|T_nx|:\norm{x}=1\})\\
				&=1
			\end{split}
		\end{equation}
		So by choosing $C$ such that $P(x+kv)\leq \norm{x+kv}$ for any $x\in X_n$ and $k\in \mathbb{R}$, we will have that $\norm{P}\leq 1$, giving $\norm{P}=1$. Thus it remains to show that we can find such a constant $C$.\\
		We aim to find $C$ such that
		$$
			|P(x+kv)|=|T_n(x)-Ck|\leq \norm{x+kv},\,\forall x\in X_n,\,\forall k\in\mathbb{R}
		$$
		Hence,
		$$
			T_n(x)-\norm{x+kv}\leq Ck\leq T_n(x)+\norm{x+kv},\,\forall x\in X_n,\,\forall k\in\mathbb{R}
		$$
		Note that for all $x,y\in X_n$ we have:
		\begin{equation}
			\begin{split}
				T_nx-T_ny&=T_n(x-y)\\
				&\leq \norm{x-y}\\
				&=\norm{(x+kv)-(kv+y)}\\
				&\leq\norm{x+kv}+\norm{y+kv}
			\end{split}
		\end{equation}
		Thus
		$$
			l^-=\sup_{x\in X_n,k\in\mathbb{R}}(T_n(x)-\norm{x+kv})\leq  \inf_{x\in X_n,k\in\mathbb{R}}(T_n(x)+\norm{x+kv})=l^+
		$$
		Hence we can always find a $C$ such that
		$$
			T_n(x)-\norm{x+kv}\leq l^-\leq Ck\leq l^+ \leq T_n(x)+\norm{x+kv},\,\forall x\in X_n,\,\forall k\in\mathbb{R}
		$$
		Which finishes the proof.
	\end{pf}

\begin{remark}

\end{remark}  

\begin{proof}

\end{proof}

\begin{definition}
(Partial Order)
	A partial order on set $X$, is a binary relation, written generically $\leq$, satisfying following property.
	\begin{itemize}
		\item transitivity: if $a\leq b$ and $b\leq c$ then $a\leq c$
		\item reflexivity: $a\leq a$
		\item anti-symmetry: if $a\leq b$ and $b\leq a$ then $a=b$

	\end{itemize}
	If we also have that for any $a$ and $b$, either $a\leq b$ or $b\leq a$, then we say $\leq$ is a total order.

\end{definition}

\begin{definition}
(Upper bound)
	Let $X$ be a set partially ordered by $\leq$ and $Y\subset X$, we say an element $x\in X$ is an {\bf upper bound} of $Y$ if $y\leq x$ for all $y \in Y$

\end{definition}

\begin{definition}
(Maximal element)
	Let $X$ be a set partially ordered by $\leq$ and $Y\subset X$. say $x\in X$ is a \textbf{maximal element} of $X$ if $x\leq m$ implies $m=x$.

\end{definition}
\begin{lemma}
(Zorn's lemma)
\label{Zorn's Lemma}
	If $X$ is a nonempty partially ordered set with the
	property that every totally ordered subset of $X$ has an upper bound in $X$, then $X$ has
	a maximal element.
\end{lemma}


\subsection{Applications of Hahn-Banach (H-B)}  

Let $(X, \norm{\cdot}_X)$ be a normed vector space, we have the following corollaries.  

\begin{corollary}
\label{same norm extension}
(Extending a linear functional)  
Let $M \subset X$ be a linear subspace. Then $\exists F \in X^*$ with 
$$F |_M = f \ \rm{and} \ \norm{F}_{X^*}=\norm{f}_{M^*}$$
\end{corollary}

\begin{proof}
Define \func{p}{X}{\real} via  

$$p(x) = \norm{x}_X \norm{f}_{M^*}$$  

Note that $p$ is sublinear and $\forall x \in M$ and,  

$$f(x) \leq |f(x)|=\norm{x}_X \frac{|f(x)|}{\hphantom{x} \norm{x}_X} \leq \norm{x}_X \norm{f}_{M^*}=p(x)$$  
Now apply Hahn-Banach to obtain $F: X \to \real$, with  
\begin{unexaminable}
$$\norm{F}_{X^*}(x) \leq \norm{x}_X \norm{f}_{M^*} \implies \norm{F}_{X^*} \leq \norm{f}_{X^*}$$
and the other direction of the inequality follows as $F |_M = f$
\end{unexaminable}
\end{proof}  

\begin{theorem}
(Dual functional)
\label{dual charaterization of norm}
	Let $X$ be a normed linear space. $\forall x\in X,\,\exists x^*\in \dual{X}$ s.t. $$\inne{\dual{x}}{x}\equiv\dual{x}(x)=\norm{x}^2_X=\norm{\dual{x}}^2_{\dual{X}}$$
\end{theorem}

\begin{proof}
	Let $M=\text{span}(x)$. Define $f: M \to \real$
	
	$$f(tx)=t\norm{x}^2_X \qquad \forall t\in \real$$
	
	Then $f$ is linear, and 
	
	$$\norm{f}_{\dual{M}} = \sup_{\norm{tx}_X \leq 1} |f(tx)| = \norm{x}_X$$
	
	Then we apply \Cref{same norm extension} to extend $f$ to $\dual{x}=F\in \dual{X}$, with $\norm{\dual{x}}_{\dual{X}}=\norm{f}_M=\norm{x}_X$ and $\inne{\dual{x}|_{M}}{x}=f(x)=\norm{x}_X^2$
\end{proof}

\begin{remark}
    \Cref{dual charaterization of norm} gives dual characterisation of the norm later.  
    \begin{unexaminable}
        When the space is a Hilbert space, this theorem becomes Riesz representation theorem. (without changing notation! That's why bracket is a good notation here) In short, this theorem says that you can always find a linear functional such that for its value for a chosen element is precisely the norm of this element.
    \end{unexaminable}
\end{remark}  

Using Hahn-Banach, one can "separate" all sorts of things. Two examples:  

\paragraph{1) Separating points}  

\begin{proposition}
$\forall x,y \in X$, $x\neq y$, there exists $\ell \in \dual{X}$, such that $\ell(x) \neq \ell(y)$
\end{proposition}  

\begin{proof}
    Choose $\ell \in \dual{X}$ according to \Cref{same norm extension} with $y-x$ in place of $x$.  
    Then  
    $$\ell (x-y) = \ell(x) - \ell(y) = \norm{y-x}^2_X >0$$ 
    Thus, $\ell(x) \neq \ell(y)$.
\end{proof}

\begin{unexaminable}
	(Question here is do  we need X to be normed? or even Banach?)
\end{unexaminable}

\paragraph{2) Separating points from closed subspaces (Urysohn-type result)}

\begin{theorem}
$M \subset X$ linear, closed. Assume $x_0 \in M$, such that  
$$d = \rm{dist} (x_0, M) = \inf_{x \in M} \norm{x_0-x}_M >0$$
Then $\exists \ell \in \dual{X}$ with $\ell|_M=0$ and  
$$\norm{\ell}_{\dual{X}}=1, \  \ell(x_0)=d$$
\end{theorem}

\begin{proof}
	Let $M_0=\{x+t x_0: x\in M\}$. Define a linear functional \func{f}{M_0}{\real}, $f(x+tx_0)=td$.
\end{proof}

From \Cref{dual charaterization of norm}, one gets a dual characterization of the norm:  

\begin{corollary}
\label{dualilty of norm}
\end{corollary}
\begin{enumerate}[i)]
    \item $\forall x \in X$: $\norm{x}_X = \underset{{\norm{x^*}_{\dual{X}} \leq 1}}{\sup} |\inne{\dual{x}}{x}|$
    \item $\forall \dual{x} \in \dual{X}$: $\norm{\dual{x}}_{\dual{X}} = \underset{{\norm{x}_X \leq 1}}{\sup} |\inne{\dual{x}}{x}|$
\end{enumerate}
The supremum in i) is always achieved.  

\begin{proof}
For $x=0$, the RHS of i) is $0$ by linearity. Let $x \neq 0$, we show two directions of inequality.  
\begin{itemize}
    \item "$\geq$": By homogeneity, we can assume $\norm{x}_X=1$. If $\dual{x} \in \dual{X}$ satisfies $\norm{x^*}_{\dual{X}} \leq 1$, then  
    $$|\inne{\dual{x}}{x}| \leq \norm{x^*}_{\dual{X}} \norm{x}_X \leq \norm{x}_X$$
    \item "$\leq$": By \Cref{dual charaterization of norm}, $\exists \dual{x} \in \dual{X}$, such that $|\inne{\dual{x}}{x}|=\norm{x}_X^2=1$. So the supremum is achieved.
\end{itemize}   

For ii), note that this is the definition of operator norm.   
\end{proof}

Another consequence \placeholder

\begin{theorem}
Let $X,Y$ be normed linear spaces and $A \in \blf{X}{Y}$. The dual operator \func{A^*}{\dual{Y}}{\dual{X}} is bounded and $\norm{\dual{A}}_{\blf{\dual{Y}}{\dual{X}}}=\norm{A}_{\blf{X}{Y}}$
\end{theorem}  

\begin{proof}
\begin{align*}
    \norm{\dual{A}} &\stackrel{\textrm{def \ of} \ \norm{\cdot}}{=} \sup_{\norm{\dual{y}}_{\dual{Y}}} \norm{\dual{A} \dual{y}}_{\dual{X}} \\
    &\stackrel{\textrm{def \ of} \  \norm{\cdot}_{\dual{X}}}{=} \sup_{\norm{\dual{y}}_{\dual{Y}}} \sup_{\norm{x}_X=1} |\inne{\dual{A}\dual{y}}{x}| \\
    &\stackrel{\textrm{def \ of} \ \dual{A}}{=} \sup_{\norm{x}_X=1}  \sup_{\norm{\dual{y}}_{\dual{Y}}}  |\inne{\dual{y}}{Ax}| \\
    &\stackrel{\phantom{\textrm{def \ of} \ \dual{A}}}{=} \sup_{\norm{x}_X=1} \norm{Ax}_Y
\end{align*}
where in the last step we used the "$\leq$" direction in the proof of \Cref{dualilty of norm} holds and the supremum over $\dual{y}$ is attained.  
\end{proof}

\end{document}