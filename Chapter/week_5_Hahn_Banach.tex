\documentclass{article}
\usepackage[utf8]{inputenc}
\usepackage{geometry}
\geometry{left=3cm,right=3cm,top=2cm,bottom=2cm}
\usepackage[utf8]{inputenc}
\usepackage{amsmath, amsfonts, amssymb, amsthm}
\usepackage[framemethod=TikZ]{mdframed}
\usepackage{mathrsfs}
\usepackage{comment}
\usepackage{enumerate}
\usepackage{xcolor}
\usepackage{titlesec}
\usepackage{setspace}
\usepackage[hidelinks,backref]{hyperref}
\usepackage{cleveref}
\usepackage[most]{tcolorbox}
\usepackage{ragged2e}
\usepackage{todonotes}
\usepackage{cleveref}
\usepackage{mathtools}

%
\titleformat*{\section}{\LARGE \bfseries}
\titleformat*{\subsection}{\Large \bfseries}
\titleformat*{\subsubsection}{\Large \bfseries}
% \titleformat*{\paragraph}{\large \bfseries}
\titleformat*{\subparagraph}{\large \bfseries}



% General 
\newcommand{\nextline}{\hfill\break}
\newcommand{\nl}{\nextline\rm}
% \newcommand{\placeholder}{{\bf\color{red} NOOOOOOT COMPLEEEEEEET! COOOOOOOM BAAAAAACK!!!}}
\newcommand{\placeholder}{\todo{NOOOOOOT COMPLEEEEEEET! COOOOOOOM BAAAAAACK!!!}}

\newcommand{\defeq}{\stackrel{def.}{=}}
% FA and LA
% inner product: \inne{a}{b}
\newcommand{\inne}[2]{\left<{#1},{#2}\right>}

% norm: \norm{a}
\newcommand{\norm}[1]{\left\|{#1}\right\|}

% Curly H
\newcommand{\hbs}{$\mathscr{H}$ }
\newcommand{\hbp}{\mathscr{H}}

% Dual : \dual{x}
\newcommand{\dual}[1]{{#1}^*}

% Sequence from 1  to infty: \sequ{x_n}
\newcommand{\sequ}[1]{\left({#1}\right)_1^\infty}

% f: A-> B \func{f}{A}{B}
\newcommand{\func}[3]{${#1}:{#2}\xrightarrow{}{#3}$}

% interior
\newcommand{\interior}{\textrm{int}}

% Bounded linear funcs
\newcommand{\blf}[2]{\mathcal{L}({#1},{#2})}

\newcommand{\prf}{\textit{proof}:   }



% Fields 
\newcommand{\real}{\mathbb{R}}
\newcommand{\comp}{\mathbb{C}}
\newcommand{\inte}{\mathbb{Z}}
\newcommand{\natu}{\mathbb{N}}





% Theorems
% \newtheorem{example}{Example}[subsection]
% \newtheorem{definition}[example]{Definition}
% \newtheorem{proposition}[example]{Proposition}
% \newtheorem{remark}[example]{Remark}
% \newtheorem{theorem}[example]{Theorem}
% \newtheorem{lemma}[example]{Lemma}
% \newtheorem{corollary}[example]{Corollary}


% for numbering the theorems            
\theoremstyle{plain}
%%%%%%%%%%%%%%%%%%%%%%%%%%%%%%%%%%%%%%%%%%%%%%%%%%%%%%%%%%%
\newtheorem{theorem}{Theorem}[section]
\newtheorem{lemma}[theorem]{Lemma}
\newtheorem{corollary}[theorem]{Corollary}
\newtheorem{proposition}[theorem]{Proposition}
%%%%%%%%%%%%%%%%%%%%%%%%%%%%%%%%%%%%%%%%%%%%%%%%%%%%%%%%%%%
% the following are not in italics
\theoremstyle{definition}
\newtheorem{definition}[theorem]{Definition}
\newtheorem{example}[theorem]{Example}
\newtheorem{remark}[theorem]{Remark}
\newtheorem{claim}[theorem]{Claim}
%%%%%%%%%%%%%%%%%%%%%%%%%%%%%%%%%%%%%%%%%%%%%%%%%%%%%%%%%%%%

% proof box
\newtcbtheorem[no counter]{pf}{Proof}{
  enhanced,
  rounded corners,
  attach boxed title to top,
  colback=white,
  colframe=black!25,
  fonttitle=\bfseries,
  coltitle=black,
  boxed title style={
    rounded corners,
    size=small,
    colback=black!25,
    colframe=black!25,
  } 
}{prf}

% extra content box to put in contents not covered in the lecture notes
% use the command \begin{unexaminable}
\newmdenv[skipabove=7pt, skipbelow=7pt,
    rightline=false, leftline=false, topline=false, bottomline=false,
    backgroundcolor = gray!10,
    innerleftmargin=1in, innerrightmargin=1in, innertopmargin=5pt,
    leftmargin=-1in, rightmargin=-1in, linewidth=4pt,
    innerbottommargin=5pt]{unexamBox}
\newenvironment{unexaminable}{\begin{unexamBox}}{\end{unexamBox}}


% clever ref settings
\crefname{lemma}{lemma}{lemmas}
\Crefname{lemma}{Lemma}{Lemmas}
\crefname{theorem}{theorem}{theorems}
\Crefname{theorem}{Theorem}{Theorems}

% formatting 
% https://tex.stackexchange.com/questions/217497/aligning-stackrel-signs-beneath-each-other-using-split
\newlength{\leftstackrelawd}
\newlength{\leftstackrelbwd}
\def\leftstackrel#1#2{\settowidth{\leftstackrelawd}%
{${{}^{#1}}$}\settowidth{\leftstackrelbwd}{$#2$}%
\addtolength{\leftstackrelawd}{-\leftstackrelbwd}%
\leavevmode\ifthenelse{\lengthtest{\leftstackrelawd>0pt}}%
{\kern-.5\leftstackrelawd}{}\mathrel{\mathop{#2}\limits^{#1}}}

\doublespacing
\RaggedRight
% modify innerrightmargin if floats were lost

\usepackage{amsfonts, amsmath, amssymb, amsthm, thmtools, bm}
\usepackage{avant} % Use the Avantgarde font for headings

% Boxed/framed environments
\newtheoremstyle{royalnumbox}%
{0pt}% Space above
{0pt}% Space below
{\normalfont}% Body font
{}% Indent amount
{\small\bf\sffamily\color{royal}}% Theorem head font
{\;}% Punctuation after theorem head
{0.25em}% Space after theorem head
{\sffamily \color{royal} 
    \thmname{#1} 
    \thmnumber{#2} \thmnote{\bfseries\color{black}---\nobreakspace#3.}} % Optional theorem note
\renewcommand{\qedsymbol}{$\blacksquare$}% Optional qed square

\newtheoremstyle{blacknumex}% Theorem style name
{5pt}% Space above
{5pt}% Space below
{\normalfont}% Body font
{} % Indent amount
{\small\bf\sffamily}% Theorem head font
{\;}% Punctuation after theorem head
{0.25em}% Space after theorem head
{\sffamily
    \thmname{#1}
    \thmnumber{#2}
    \thmnote{---\nobreakspace#3.}}% Optional theorem note

\newtheorem*{notation}{Notation}
\newtheorem*{hint}{Hint}
\newtheorem*{solution}{Solution}

\newcounter{dummy} 
\numberwithin{dummy}{section}

\theoremstyle{royalnumbox}
\newtheorem{definitionT}[dummy]{Definition}
\newtheorem{theoremT}[dummy]{Theorem}
\newtheorem{lemmaT}[dummy]{Lemma}
\newtheorem{corollaryT}[dummy]{Corollary}
\newtheorem{propositionT}[dummy]{Proposition}
\newtheorem{propertyT}[dummy]{Property}
\newtheorem{remarkT}[dummy]{Remark}

\theoremstyle{blacknumex}
\newtheorem{exampleT}[dummy]{Example}
\newtheorem{exerciseT}[dummy]{Exercise}

\numberwithin{equation}{section}

\RequirePackage[framemethod=default]{mdframed}

% Definition box
\newmdenv[skipabove=7pt, skipbelow=7pt,
rightline=false, leftline=true, topline=false, bottomline=false,
backgroundcolor = reddish!10, 
linecolor=reddish,
innerleftmargin=5pt, innerrightmargin=30pt, innertopmargin=5pt,
leftmargin=0cm, rightmargin=0cm, linewidth=4pt,
innerbottommargin=5pt]{dBox}

% Main Theorem box
\newmdenv[skipabove=7pt, skipbelow=7pt,
rightline=false, leftline=true, topline=false, bottomline=false,
backgroundcolor=c0!10, 
linecolor=c0,
innerleftmargin=5pt, innerrightmargin=30pt, innertopmargin=5pt,
leftmargin=0cm, rightmargin=0cm, linewidth=4pt, innerbottommargin=5pt]{tBox}

% Lemma/Corollary/Proposition/Property box
\newmdenv[skipabove=7pt, skipbelow=7pt,
rightline=false, leftline=true, topline=false, bottomline=false,
backgroundcolor = c0!10, 
linecolor=c0!80,
innerleftmargin=5pt, innerrightmargin=30pt, innertopmargin=5pt,
leftmargin=0cm, rightmargin=0cm, linewidth=4pt,
innerbottommargin=5pt]{lBox}

% Example/Remark/Exercise box
\newmdenv[skipabove=7pt, skipbelow=7pt,
rightline=false, leftline=true, topline=false, bottomline=false,
backgroundcolor = mossgreen!10!white,
linecolor = mossgreen,
innerleftmargin=5pt, innerrightmargin=30pt, innertopmargin=5pt,
leftmargin=0cm, rightmargin=0cm, linewidth=4pt,
innerbottommargin=5pt]{exBox}

% Proof box
% \newmdenv[skipabove=7pt, skipbelow=7pt,
% rightline=false, leftline=true, topline=false, bottomline=false,
% linecolor=gray,
% innerleftmargin=5pt, innerrightmargin=30pt, innertopmargin=5pt,
% leftmargin=0cm, rightmargin=0cm, linewidth=4pt,
% innerbottommargin=5pt]{proofBox}



% Creates an environment for each type of theorem and assigns it a theorem text style from the "Theorem Styles" section above and a colored box from above
\newenvironment{definition}{\begin{dBox}\begin{definitionT}}{\end{definitionT}\end{dBox}}
\newenvironment{theorem}{\begin{tBox}\begin{theoremT}}{\end{theoremT}\end{tBox}}
\newenvironment{lemma}{\begin{lBox}\begin{lemmaT}}{\end{lemmaT}\end{lBox}}
\newenvironment{proposition}{\begin{lBox}\begin{propositionT}}{\end{propositionT}\end{lBox}}
\newenvironment{corollary}{\begin{lBox}\begin{corollaryT}}{\end{corollaryT}\end{lBox}}
\newenvironment{property}{\begin{lBox}\begin{propertyT}}{\end{propertyT}\end{lBox}}


% \newenvironment{proof*}{\begin{proofBox}\begin{proof}}{\end{proof}\end{proofBox}}
\newenvironment{exercise}{\begin{exBox}\begin{exerciseT}}{\hfill{\color{royal}}\end{exerciseT}\end{exBox}}
\newenvironment{remark}{\begin{exBox}\begin{remarkT}}{\end{remarkT}\end{exBox}}
\newenvironment{example}{\begin{exBox}\begin{exampleT}}{{}\end{exampleT}\end{exBox}}

\title{Week 5 Hahn-Banach and Consequences}

\begin{document}
% \author{\aut}
\maketitle

\section{Hahn-Banach Theorem} 

\begin{definition}[Sublinear functional]\nl
\label{sublinear map}
	Let $X$ be a vector space. \func{p}{X}{\real} is called \textbf{sublinear} if the following holds
	\begin{enumerate}[i)]
    		\item $p( \alpha x)= \alpha p(x),$ $\forall x\in X$ and $\alpha\geq0$
		\item $p(x+y) \leq p(x)+p(y),\,\,\forall x,y\in X$
	\end{enumerate}
\end{definition}

\begin{example}
    Any linear functional is also sublinear. Also, $p(x)=\norm{x}$ on $X$ is sublinear.
\end{example}  

\begin{theorem}[Hahn-Banach]\nl
\label{Hahn-Banach}
	Let $M\subset X$ be a linear subspace, \func{p}{X}{\real} is sublinear, and \func{f}{M}{\real} is linear with 
	\begin{equation*}
	    f(x)\leq p(x) \qquad \forall x\in M
     \tag{$*$}
	\end{equation*}

	Then, there exists a linear map \func{F}{X}{\real} with $F|_M=f$ and
	
	$$F(x)\leq p(x) \qquad \forall x\in X$$
\end{theorem}

\begin{remark}[Geometric Intuitions]\nl
    Take $x=\real^n$, $0 \in M \subset X$ an open convex subset, if $x_1 \notin <M$, then one can find a $f: X \to \real$ linear "separating" $x_1$ from $M$:

    \textit{i.e.} $$
    \begin{cases}
        f(x) < a & , x\in M \\
        f(x) \geq a & x \notin M
    \end{cases}
    $$
    for some $a\neq 0$. One can assume $a=1$ by replacing $f$ by $\frac{f}{a}$.  

    To find $f$, we introduce  
    $$
    p(x) \defeq \inf \{r > 0: \frac{x}{r} \in M\}
    $$
    \begin{unexaminable}
        Minkowski functional, cf. PS6
    \end{unexaminable}
    (e.g. $p(x)=\norm{x}$ if $M=\{x: \norm{x}<1\}$ in a normed space $(X, \norm{\cdot})$).

    One can check (using continuity) that $p$ is sublinear and  
    $$
    p(x) < 1 \iff x \in M
    $$
    
    "$\implies$": if $p(x)<1$, then $\exists \varepsilon >0$ s.t. $\frac{x}{1-\varepsilon} \in M$, hence  
    $$
    x = (1-\varepsilon) \frac{x}{1-\varepsilon} + \varepsilon \cdot 0 \in M
    $$
    by convexity of $M$.  

    "$\impliedby$": $x \in  M \overset{M \ \text{open}}{\implies} \frac{x}{1-\varepsilon} \in M$ which implies that $p(x) \leq 1-\varepsilon$.  

    To find $f$ with above properties, it is thus enough to ensure that $f(x_1)=1$ and $f(x) \leq p(x)$, for all $x \in X$.
\end{remark}

\begin{proof}
    If $M=X$, take $F=f$. Else, choose $x_1 \notin M$ and set  
    $$
    M_1 = \{x+tx_1: x\in M, t\in \real\}
    $$
    ($M\subset X$ linear). Extend $f$ to $M_1$, such that ($*$) holds on $M_1$, then 'repeat'  

    \underline{\textbf{Step (I)}}
    
    For all $x,y \in M$: 
    $$
    f(x)+f(y) \overset{\text{f \ linear}}{=} f(x+y) \overset{(*)}{\leq} p(x+y) \leq p(x-x_1)+p(x_1+y)
    $$
    hence 
    \begin{equation*}
        \forall x,y\in M: \qquad f(x)-p(x-x_1) \leq p(x_1+y) - f(y) \tag{$**$}
    \end{equation*}
    Take supremum over $x$, with $y$ fixed, we get  
    $$
    \alpha \defeq \sup_{x\in M} (f(x)-p(x-x_1)) < \infty
    $$
    Define $f_1: M\to \real$ by  
    $$
    f_1(x+tx_1) = f(x) + t\alpha \qquad ,x\in M, t\in \real
    $$
    \begin{lemma}
    \phantom{something}
        \begin{enumerate}[1)]
            \item $f_1$ is linear
            \item $f_1 \mid_M=f$ 
            \item $f_1(x) \leq p(x), \forall x \in M_1$, (\textit{i.e.} ($*$) for $(f_1, M_1)$ instead of $(f, M)$)
        \end{enumerate}
    \end{lemma}
    The first two properties are easy to check.  
    
    \underline{Proof of 3):}\nl
    By definition of $\alpha$, $\forall x \in M: f(x)-\alpha \leq f(x)-( f(x)-p(x-x_1)) = p(x-x_1)$. On the other hand, taking supremum over $x$ in $(**)$ yields, for all $y \in M$  
    $$
    f(y) + \alpha \leq p(y+x_1)
    $$  
    Overall, we have  
    $$
    \forall x\in M: \qquad f_1(x\pm x_1) = f(x) \pm \alpha \leq p(x \pm x_1)
    $$  
    Apply with $t^{-1}x(\in M)$ for $t > 0$ in place of $x$ and multiply both sides by $t$ to find  
    $$
    \forall x \in M, t>0: \qquad f_1(x \pm tx_1) \leq tp(t^{-1}x\pm x_1)=p(x\pm tx_1)
    $$
\end{proof}

NB: If $X$ has a countable basis $\{e_1, i\geq 1\}$ [e.g. $\ell^p, p\in [1,\infty)$], then we can take $x_1=e_1$ and proceed by induction. For the general case, we use:

\underline{\textbf{Step (II)}}
\begin{proof}(continued:)
\begin{lemma}[Zorn's lemma]\nl
\label{Zorn's Lemma}
	Let $(P, \leq), P\neq \emptyset$ is a nonempty partially ordered set and every totally ordered subset has an upper bound, then $P$ has a maximal element.
\end{lemma}  
\begin{unexaminable}
    \begin{definition}
[Partial Order]\nl
	A \textbf{partial order} on set $X$, is a binary relation, written generically $\leq$, satisfying following property.
	\begin{itemize}
		\item transitivity: if $a\leq b$ and $b\leq c$ then $a\leq c$
		\item reflexivity: $a\leq a$
		\item anti-symmetry: if $a\leq b$ and $b\leq a$ then $a=b$

	\end{itemize}
	If we also have that for any $a$ and $b$, either $a\leq b$ or $b\leq a$, then we say $\leq$ is a total order.
\end{definition}

\begin{definition}
[Upper bound]\nl
	Let $X$ be a set partially ordered by $\leq$ and $Y\subset X$, we say an element $x\in X$ is an {\bf upper bound} of $Y$ if $y\leq x$ for all $y \in Y$

\end{definition}

\begin{definition}
[Maximal element]\nl
	Let $X$ be a set partially ordered by $\leq$ and $Y\subset X$. say $x\in X$ is a \textbf{maximal element} of $X$ if $x\leq m$ implies $m=x$.

\end{definition}
\end{unexaminable}
Take 
$$
P = \{(N,g): N\subset X \ \text{linear\ subspace}, g: N \to \real \ \text{linear}, \ g\mid_N=f, \ g\mid_N \leq p\}
$$  
and define  
$$
(N,g) \leq (O, h ) \overset{\text{def.}}{\iff} N \subset O, h\mid_N = g
$$
Then $(P, \leq)$ is partially ordered, $(M,f) \in P$ so $P \neq \emptyset$. Assume $(N_i, g_i)_{i \in I}$ is a totally ordered subset. Set $N = \cup_{i\in I} N_i$ and for $x\in N$,  
$$
g(x) = g_i(x) \qquad \text{if\ }x\in N_i
$$
Then $(N,g) \in P$. Indeed $N\subset X$ is linear, and $g$ is well-defined and linear with $g\leq p$ on $N$.
\begin{itemize}
    \item (Well-defined) If $x \in N_i \cap N_k$, $N_i \subset N_k$, then $g_k \mid_{N_i}=g_i$, hence $g_i(x)=g_k(x)$
    \item (Linear) If $x,y \in N$, then $x\in N_i, y\in N_k$, for some $i,k \in I$ and $N_i \subset N_k$ (or vice versa), so $x, y \in N_k$ and  
    $$
    g(x+y) = g_k(x+y) = g_k(x)+g_k(y) = g(x) + g(y)
    $$
    \item (Bounded by $p(x)$) Similarly, one can check $g \leq p$ on $N$ (exercise)
\end{itemize}  

$(N,g)$ is an upper bound for $(N_i, g_i)_{i \in I}$, since $N_i \subset N$ and $g\mid_{N_i} = g_i$ by definition  

So \Cref{Zorn's Lemma} applies and yields that $(P, \leq)$ has a maximal element $(N,g)\in P$. Set $F=g$.  
By definition of $P$, all properties required of $F$ hold and $N=X$. For, otherwise, one can apply \underline{\textbf{Step(I)}} to $(N,g)$ and find $(N_1, g_1) \in P$ with $(N,g) < (N_1, g_1)$, which violates the maximality of $(N,g)$.
\end{proof}


\begin{remark}
There is a version of Hahn-Banach theorem for $X$ over $\comp$ and $p: X\to \real$ is called $\comp$-sublinear if  
\begin{enumerate}[i)]
    \item $p(\alpha x) = |\alpha|p(x), \forall x \in X, \alpha \in \comp$
    \item $p(x+y) \leq p(x)+p(y),\,\,\forall x,y\in X$, same as in over $\real$
\end{enumerate}
Then \Cref{Hahn-Banach} holds for $X$ over $\comp$, if $p$ is $\comp$-sublinear and $(*)$ replaced by  
$$
|f(x)| \leq p(x) \qquad \forall x \in M
$$  
The conclusion is the same with $|F|\leq p$ on $X$ instead.  
\begin{proof}
    Apply Hahn-Banach to $f_1 = \text{Re}(f)$ (linear!) and $f_2 = \text{Im}(f)$.
\end{proof}
\end{remark}  

From now on assume $X$ is over $\real$.

\subsection{Applications of Hahn-Banach (H-B)}  

Let $(X, \norm{\cdot}_X)$ be a normed vector space, we have the following corollaries.  

\begin{corollary}
\label{same norm extension}
Let $M \subset X$ be a linear subspace, $f$ a bounded linear functional on $M$. Then $\exists F \in X^*$ with 
$$F |_M = f \ \text{and} \ \norm{F}_{X^*}=\norm{f}_{M^*}$$
\end{corollary}

\begin{proof}
Define \func{p}{X}{\real} via  

$$p(x) = \norm{x}_X \norm{f}_{M^*}$$  

Note that $p$ is sublinear and $\forall x \in M$ and,  

$$f(x) \leq |f(x)|=\norm{x}_X \frac{|f(x)|}{\hphantom{x} \norm{x}_X} \leq \norm{x}_X \norm{f}_{M^*}=p(x)$$  
Now apply Hahn-Banach to obtain $F: X \to \real$, with  
\begin{unexaminable}
$$\norm{F(x)}_{X} \leq \norm{x}_X \norm{f}_{M^*} \implies \norm{F}_{X^*} \leq \norm{f}_{X^*}$$
and the other direction of the inequality follows as $F |_M = f$
\end{unexaminable}
\end{proof}  

\begin{theorem}
\label{existence of functional equals norm squared}\label{dual charaterization of norm}
	Let $X$ be a normed linear space. $\forall x\in X,\,\exists x^*\in \dual{X}$ s.t. $$\inne{\dual{x}}{x}\equiv\dual{x}(x)=\norm{x}^2_X=\norm{\dual{x}}^2_{\dual{X}}$$
\end{theorem}

\begin{proof}
	Let $M=\text{span}(x)$. Define $f: M \to \real$
	
	$$f(tx)=t\norm{x}^2_X \qquad \forall t\in \real$$
	
	Then $f$ is linear, and 
	
	$$\norm{f}_{\dual{M}} = \sup_{\norm{tx}_X \leq 1} |f(tx)| = \norm{x}_X$$
	
	Then we apply \Cref{same norm extension} to extend $f$ to $\dual{x}=F\in \dual{X}$, with $\norm{\dual{x}}_{\dual{X}}=\norm{f}_M=\norm{x}_X$ and $\inne{\dual{x}|_{M}}{x}=f(x)=\norm{x}_X^2$
\end{proof}

\begin{remark}
    \Cref{existence of functional equals norm squared} gives dual characterisation of the norm later.  
    \begin{unexaminable}
        When the space is a Hilbert space, this theorem becomes Riesz representation theorem. (without changing notation! That's why bracket is a good notation here) In short, this theorem says that you can always find a linear functional such that for its value for a chosen element is precisely the norm of this element.
    \end{unexaminable}
\end{remark}  

Using Hahn-Banach, one can "separate" all sorts of things. Two examples:  

\paragraph{1) Separating points}  

\begin{proposition}
\label{dual elements separate points}
$\forall x,y \in X$, $x\neq y$, there exists $\ell \in \dual{X}$, such that $\ell(x) \neq \ell(y)$
\end{proposition}  

\begin{proof}
    Choose $\ell \in \dual{X}$ according to \Cref{same norm extension} with $y-x$ in place of $x$.  
    Then  
    $$\ell (x-y) = \ell(x) - \ell(y) = \norm{y-x}^2_X >0$$ 
    Thus, $\ell(x) \neq \ell(y)$.
\end{proof}

\paragraph{2) Separating points from closed subspaces (Urysohn-type result)}

\begin{theorem}
\label{analog of Urysohn}
$M \subset X$ linear, closed. Assume $x_0 \in X\setminus M$, such that  
$$d = \rm{dist} (x_0, M) = \inf_{x \in M} \norm{x_0-x}_M >0$$
Then $\exists \ell \in \dual{X}$ with $\ell|_M=0$ and  
$$\norm{\ell}_{\dual{X}}=1, \  \ell(x_0)=d$$
\end{theorem}

\begin{proof}
(Sketch)
Let $M_0=\{x+t x_0: x\in M\}$. Define a linear functional \func{f}{M_0}{\real}, 
$$f(x+tx_0)=td$$

Then $f\mid_M=0$ and $f(x_0)=d$. Check $\norm{f}_{M_0^*}=1$ (exercise).  

Now apply \Cref{same norm extension} to obtain extension $\ell \overset{\text{def.}}{=}F \in X^*$ with $\norm{\ell}_{X^*}=1$.
\end{proof}

\begin{unexaminable}
\begin{remark}
    In the proof above, we utilized the fact $M$ is a linear subspace to construct subspace $M_0$. For a similar result but with convex, closed subset, see Problem Sheet 6.
\end{remark}
\end{unexaminable}

The previous \Cref{analog of Urysohn} has a lot of mileage. For instance, one get  

\paragraph{3) Alternative for \Cref{existence of functional equals norm squared}}  
\begin{proposition}
    Let $X$ be a normed linear space. $\forall x\in X,\,\exists x^*\in \dual{X}$ s.t. $$\inne{\dual{x}}{x}\equiv\dual{x}(x)=\norm{x}_X=\norm{\dual{x}}_{\dual{X}}$$
\end{proposition}
\begin{proof}
    (Sketch) Apply \Cref{analog of Urysohn} with $M = \{0\}$, $x_0=\frac{x}{\norm{x}_X}$ to recover \Cref{existence of functional equals norm squared} with $x^*(x) \defeq \norm{x}_X$.
\end{proof}

\paragraph{4) $X^*$ separable $\implies$ $X$ separable}  
\begin{proof}
    See notes Page 82, which uses \Cref{analog of Urysohn}.
\end{proof}

In particular, this gives  

\begin{corollary}
    $(\ell^{\infty})^* \ncong \ell^1$
\end{corollary}
\begin{proof}
    We know $\ell^1$ is separable (use the set $\{\sum_{i=1}^n x_i e_i: x_i \in \qq, n \in \natu\}$), so if $(\ell^{\infty})^* \cong \ell^1$, then $(\ell^{\infty})^*$ is separable and so is $\ell^{\infty}$ by the theorem above. But $\ell^{\infty}$ is not separable.
\end{proof} 

From \Cref{existence of functional equals norm squared}, one gets a dual characterisation of the norm:  

\begin{corollary}
\phantom{something}
\label{duality of norm}
\label{dual characterisation of norm}
\begin{enumerate}[i)]
    \item $\forall x \in X$: $\norm{x}_X = \underset{{\norm{x^*}_{\dual{X}} \leq 1}}{\sup} |\inne{\dual{x}}{x}|$
    \item $\forall \dual{x} \in \dual{X}$: $\norm{\dual{x}}_{\dual{X}} = \underset{{\norm{x}_X \leq 1}}{\sup} |\inne{\dual{x}}{x}|$
\end{enumerate}
The supremum in i) is always achieved.  
\end{corollary}


\begin{proof}
For $x=0$, the RHS of i) is $0$ by linearity. Let $x \neq 0$, we show two directions of inequality.  
\begin{itemize}
    \item "$\geq$": By homogeneity, we can assume $\norm{x}_X=1$. If $\dual{x} \in \dual{X}$ satisfies $\norm{x^*}_{\dual{X}} \leq 1$, then  
    $$|\inne{\dual{x}}{x}| \leq \norm{x^*}_{\dual{X}} \norm{x}_X \leq \norm{x}_X$$
    
    %\todo{Is this the alternative of Theorem 1.12 without the squared?}
    
    \item "$\leq$": By \Cref{existence of functional equals norm squared}, $\exists \dual{x} \in \dual{X}$, such that $|\inne{\dual{x}}{x}|=\norm{x}_X^2=1$. So the supremum is achieved.
\end{itemize}   

For ii), note that this is the definition of operator norm.   
\end{proof}

Another consequence of \Cref{existence of functional equals norm squared} is (cf. notes Theorem 23 on Page 67 for the special case $X=Y$ Hilbert, in which case the use of Hahn-Banach can be substituted by the Riesz representation theorem.)

\begin{theorem}
Let $X,Y$ be normed linear spaces and $A \in \blf{X}{Y}$. The dual operator \func{A^*}{\dual{Y}}{\dual{X}} is bounded and $\norm{\dual{A}}_{\blf{\dual{Y}}{\dual{X}}}=\norm{A}_{\blf{X}{Y}}$
\end{theorem}  

\begin{proof}
\begin{align*}
    \norm{\dual{A}} &\stackrel{\textrm{def \ of} \ \norm{\cdot}}{=} \sup_{\norm{\dual{y}}_{\dual{Y}}=1} \norm{\dual{A} \dual{y}}_{\dual{X}} \\
    &\stackrel{\textrm{def \ of} \  \norm{\cdot}_{\dual{X}}}{=} \sup_{\norm{\dual{y}}_{\dual{Y}}=1} \sup_{\norm{x}_X=1} |\inne{\dual{A}\dual{y}}{x}| \\
    &\stackrel{\textrm{def \ of} \ \dual{A}}{=} \sup_{\norm{x}_X=1}  \sup_{\norm{\dual{y}}_{\dual{Y}}=1}  |\inne{\dual{y}}{Ax}| \\
    &\stackrel{\phantom{\textrm{def \ of} \ \dual{A}}}{=} \sup_{\norm{x}_X=1} \norm{Ax}_Y
\end{align*}
where in the last step we used the "$\leq$" direction in the proof of \Cref{duality of norm} holds and the supremum over $\dual{y}$ is attained by \Cref{existence of functional equals norm squared}.  
\end{proof}

\end{document}